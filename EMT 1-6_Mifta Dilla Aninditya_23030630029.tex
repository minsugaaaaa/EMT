\documentclass[a4paper,10pt]{article}
\usepackage{eumat}

\begin{document}
\begin{eulernotebook}
\eulerheading{EMT untuk Perhitungan Aljabar}
\begin{eulercomment}
Pada notebook ini Anda belajar menggunakan EMT untuk melakukan
berbagai perhitungan terkait dengan materi atau topik dalam Aljabar.
Kegiatan yang harus Anda lakukan adalah sebagai berikut:

- Membaca secara cermat dan teliti notebook ini;\\
- Menerjemahkan teks bahasa Inggris ke bahasa Indonesia;\\
- Mencoba contoh-contoh perhitungan (perintah EMT) dengan cara
meng-ENTER setiap perintah EMT yang ada (pindahkan kursor ke baris
perintah)\\
- Jika perlu Anda dapat memodifikasi perintah yang ada dan memberikan
keterangan/penjelasan tambahan terkait hasilnya.\\
- Menyisipkan baris-baris perintah baru untuk mengerjakan soal-soal
Aljabar dari file PDF yang saya berikan;\\
- Memberi catatan hasilnya.\\
- Jika perlu tuliskan soalnya pada teks notebook (menggunakan format
LaTeX).\\
- Gunakan tampilan hasil semua perhitungan yang eksak atau simbolik
dengan format LaTeX. (Seperti contoh-contoh pada notebook ini.)

\end{eulercomment}
\eulersubheading{Contoh pertama}
\begin{eulercomment}
Menyederhanakan bentuk aljabar:

\end{eulercomment}
\begin{eulerformula}
\[
6x^{-3}y^5\times -7x^2y^{-9}
\]
\end{eulerformula}
\begin{eulercomment}
\end{eulercomment}
\begin{eulerprompt}
>$&6*x^(-3)*y^5*-7*x^2*y^(-9)
\end{eulerprompt}
\begin{eulerformula}
\[
-\frac{42}{x\,y^4}
\]
\end{eulerformula}
\begin{eulercomment}
Menjabarkan:

\end{eulercomment}
\begin{eulerformula}
\[
(6x^{-3}+y^5)(-7x^2-y^{-9})
\]
\end{eulerformula}
\begin{eulerprompt}
>$&showev('expand((6*x^(-3)+y^5)*(-7*x^2-y^(-9))))
\end{eulerprompt}
\begin{eulerformula}
\[
{\it expand}\left(\left(-\frac{1}{y^9}-7\,x^2\right)\,\left(y^5+
 \frac{6}{x^3}\right)\right)=-7\,x^2\,y^5-\frac{1}{y^4}-\frac{6}{x^3
 \,y^9}-\frac{42}{x}
\]
\end{eulerformula}
\begin{eulercomment}
\end{eulercomment}
\eulersubheading{Baris Perintah}
\begin{eulercomment}
Baris perintah Euler terdiri dari satu atau beberapa perintah Euler
yang diikuti dengan titik koma ";" atau koma ",". Titik koma mencegah
pencetakan hasil. Koma setelah perintah terakhir dapat dihilangkan.

Baris perintah berikut ini hanya akan mencetak hasil dari ekspresi,
bukan penugasan atau perintah fotmat.
\end{eulercomment}
\begin{eulerprompt}
>r:=2; h:=4; pi*r^2*h/3
\end{eulerprompt}
\begin{euleroutput}
  16.7551608191
\end{euleroutput}
\begin{eulercomment}
Perintah harus dipisahkan dengan spasi. Baris perintah berikut ini
mencetak dua hasilnya.
\end{eulercomment}
\begin{eulerprompt}
>pi*2*r*h, %+2*pi*r*h // Ingat tanda % menyatakan hasil perhitungan terakhir sebelumnya
\end{eulerprompt}
\begin{euleroutput}
  50.2654824574
  100.530964915
\end{euleroutput}
\begin{eulercomment}
Baris perintah dieksekusi sesuai urutan pengguna menekan tombol
return. Jadi, anda mendapatkan nilai baru setiap kali mengeksekusi
baris kedua.
\end{eulercomment}
\begin{eulerprompt}
>x := 1;
>x := cos(x) // nilai cosinus (x dalam radian)
\end{eulerprompt}
\begin{euleroutput}
  0.540302305868
\end{euleroutput}
\begin{eulerprompt}
>x := cos(x)
\end{eulerprompt}
\begin{euleroutput}
  0.857553215846
\end{euleroutput}
\begin{eulercomment}
Jika dua baris dihubungkan dengan "...", kedua baris tersebut akan
selalu dieksekusi secara bersamaan.
\end{eulercomment}
\begin{eulerprompt}
>x := 1.5; ...
>x := (x+2/x)/2, x := (x+2/x)/2, x := (x+2/x)/2, 
\end{eulerprompt}
\begin{euleroutput}
  1.41666666667
  1.41421568627
  1.41421356237
\end{euleroutput}
\begin{eulercomment}
Ini juga merupakan cara yang baik untuk membagi perintah yang panjang
menjadi dua baris atau lebih. Anda dapat menekan Ctrl+Return untuk
membagi baris menjadi dua pada posisi kursor saat ini, atau Ctlr+Back
untuk menggabungkan baris-baris tersebut.

Untuk melipat semua garis yang ada tekan Ctrl+L. Maka garis-garis
berikutnya hanya akan terlihat jika salah satu dari garis-garis
tersebut menjadi fokus. Untuk melipat satu baris ganda, mulailah baris
pertama dengan "\%+".
\end{eulercomment}
\begin{eulerprompt}
>%+ x=4+5; ...
\end{eulerprompt}
\begin{eulercomment}
Baris yang dimulai dengan \%\% tidak akan terlihat sama sekali.
\end{eulercomment}
\begin{euleroutput}
  81
\end{euleroutput}
\begin{eulercomment}
Euler mendukung perulangan dalam baris perintah, asalkan dapat
dimasukkan ke dalam satu baris atau beberapa baris. Dalam program,
pembatasan ini tentu saja tidak berlaku. Untuk informasi lebih lanjut,
lihat pengantar berikut.
\end{eulercomment}
\begin{eulerprompt}
>x=1; for i=1 to 5; x := (x+2/x)/2, end; // menghitung akar 2
\end{eulerprompt}
\begin{euleroutput}
  1.5
  1.41666666667
  1.41421568627
  1.41421356237
  1.41421356237
\end{euleroutput}
\begin{eulercomment}
Tidak apa-apa menggunakan beberapa baris. Pastikan baris diakhiri
dengan "...".
\end{eulercomment}
\begin{eulerprompt}
>x := 1.5; // komentar pergi ke sini sebelum ...
>repeat xnew:=(x+2/x)/2; until xnew~=x; ...
>   x := xnew; ...
>end; ...
>x,
\end{eulerprompt}
\begin{euleroutput}
  1.41421356237
\end{euleroutput}
\begin{eulercomment}
Struktur kondisional juga berfungsi.
\end{eulercomment}
\begin{eulerprompt}
>if E^pi>pi^E; then "Thought so!", endif;
\end{eulerprompt}
\begin{euleroutput}
  Thought so!
\end{euleroutput}
\begin{eulercomment}
Saat menjalankan perintah,kursor dapat berada di posisi mana pun di
baris perintah. Anda dapat kembali ke perintah sebelumnya atau
melompat ke perintah berikutnya dengan tombol panah. Atau Anda dapat
mengeklik bagian komentar di atas perintah untuk membuka perintah
tersebut.

Saat menggerakkan kursor di sepanjang baris, pasangan tanda kurung
buka dan tutup akan disorot. Perhatikan juga baris status. Setelah
tanda kurung buka fungsi sqrt(), baris status akan menampilkan teks
bantuan untuk fungsi tersebut. Jalankan perintah dengan tombol return.
\end{eulercomment}
\begin{eulerprompt}
>sqrt(sin(10°)/cos(20°))
\end{eulerprompt}
\begin{euleroutput}
  0.429875017772
\end{euleroutput}
\begin{eulercomment}
Untuk melihat bantuan untuk perintah terbaru, buka jendela bantuan
dengan F1. Di sana, Anda dapat memasukkan teks yang ingin dicari. Pada
baris kosong, bantuan untuk jendela bantuan akan ditampilkan. Anda
dapat menekan escape untuk menghapus baris, atau untuk menutup jendela
bantuan.\\
Anda dapat mengklik dua kali pada perintah apa pun untuk membuka
bantuan untuk perintah ini. Coba klik dua kali perintah expand di
bawah pada baris perintah.
\end{eulercomment}
\begin{eulerprompt}
>exp(log(2.5))
\end{eulerprompt}
\begin{euleroutput}
  2.5
\end{euleroutput}
\begin{eulercomment}
Anda juga dapat menyalin dan menempel di Euler. Gunakan Ctrl-C dan
Ctrl-V untuk ini. Untuk menandai teks, seret mouse atau gunakan shift
bersamaan dengan tombol kursor apa pun. Selain itu, Anda dapat
menyalin tanda kurung yang disorot.
\end{eulercomment}
\begin{eulercomment}

\end{eulercomment}
\eulersubheading{Sintaksis Dasar}
\begin{eulercomment}
Euler mengetahui fungsi matematika yang umum. Seperti yang telah Anda
lihat di atas, fungsi trigonometri bekerja dalam radian atau derajat.
Untuk mengonversi ke derajat, tambahkan simbol derajat (dengan tombol
F7) ke nilai, atau gunakan fungsi rad(x). Fungsi akar kuadrat disebut
sqrt dalam Euler. Tentu saja, x\textasciicircum{}(1/2) juga memungkinkan.

Untuk mengatur variabel, gunakan "=" atau ":=". Demi kejelasan,
pengantar ini menggunakan bentuk yang terakhir. Spasi tidak menjadi
masalah. Namun, spasi di antara perintah diharapkan.

Beberapa perintah dalam satu baris dipisahkan dengan "," atau ";".
Tanda titik koma menghilangkan keluaran perintah. Di akhir baris
perintah, tanda "," diasumsikan, jika ";" tidak ada.
\end{eulercomment}
\begin{eulerprompt}
>g:=9.81; t:=2.5; 1/2*g*t^2
\end{eulerprompt}
\begin{euleroutput}
  30.65625
\end{euleroutput}
\begin{eulercomment}
EMT menggunakan sintaks pemrograman untuk ekspresi. Untuk  memasukkan

\end{eulercomment}
\begin{eulerformula}
\[
e^2 \cdot \left( \frac{1}{3+4 \log(0.6)}+\frac{1}{7} \right)
\]
\end{eulerformula}
\begin{eulercomment}
Anda harus menetapkan tanda kurung yang benar dan menggunakan / untuk
pecahan. Perhatikan tanda kurung yang disorot untuk mendapatkan
bantuan. Perhatikan bahwa konstanta Euler e diberi nama E dalam EMT.
\end{eulercomment}
\begin{eulerprompt}
>E^2*(1/(3+4*log(0.6))+1/7)
\end{eulerprompt}
\begin{euleroutput}
  8.77908249441
\end{euleroutput}
\begin{eulercomment}
Untuk menghitung ekspresi rumit seperti

\end{eulercomment}
\begin{eulerformula}
\[
\left(\frac{\frac17 + \frac18 + 2}{\frac13 + \frac12}\right)^2 \pi
\]
\end{eulerformula}
\begin{eulercomment}
Anda perlu memasukkannya dalam formulir baris.
\end{eulercomment}
\begin{eulerprompt}
>((1/7 + 1/8 + 2) / (1/3 + 1/2))^2 * pi
\end{eulerprompt}
\begin{euleroutput}
  23.2671801626
\end{euleroutput}
\begin{eulercomment}
Letakkan tanda kurung di sekitar sub-ekspresi yang perlu dihitung
terlebih dahulu dengan hati-hati. EMT membantu Anda dengan menyorot
ekspresi yang diakhiri tanda kurung tutup. Anda juga harus memasukkan
nama "pi" untuk huruf Yunani pi.\\
\end{eulercomment}
\begin{eulerttcomment}
 
\end{eulerttcomment}
\begin{eulercomment}
Hasil perhitungan ini adalah angka floating point. Angka ini dicetak
dengan akurasi sekitar 12 digit secara default.\\
Pada baris perintah berikut, kita juga mempelajari cara merujuk ke
hasil sebelumnya dalam baris yang sama.
\end{eulercomment}
\begin{eulerprompt}
>1/3+1/7, fraction %
\end{eulerprompt}
\begin{euleroutput}
  0.47619047619
  10/21
\end{euleroutput}
\begin{eulercomment}
Perintah Euler dapat berupa ekspresi atau perintah primitif. Ekspresi
terdiri dari operator dan fungsi. Jika perlu, ekspresi harus berisi
tanda kurung untuk memaksakan urutan eksekusi yang benar. Jika ragu,
sebaiknya gunakan tanda kurung. Perhatikan bahwa EMT menampilkan tanda
kurung buka dan tutup saat mengedit baris perintah
\end{eulercomment}
\begin{eulerprompt}
>(cos(pi/4)+1)^3*(sin(pi/4)+1)^2
\end{eulerprompt}
\begin{euleroutput}
  14.4978445072
\end{euleroutput}
\begin{eulercomment}
Operator numerik Euler meliputi

\end{eulercomment}
\begin{eulerttcomment}
 + unary or operator plus
 - unary or operator minus
 *, /
 . the matrix product
 a^b power for positive a or integer b (a**b works too)
 n! the factorial operator
\end{eulerttcomment}
\begin{eulercomment}

dan masih banyak lagi.

Berikut ini beberapa fungsi yang mungkin Anda perlukan. Masih banyak
lagi.

\end{eulercomment}
\begin{eulerttcomment}
 sin,cos,tan,atan,asin,acos,rad,deg
 log,exp,log10,sqrt,logbase
 bin,logbin,logfac,mod,floor,ceil,round,abs,sign
 conj,re,im,arg,conj,real,complex
 beta,betai,gamma,complexgamma,ellrf,ellf,ellrd,elle
 bitand,bitor,bitxor,bitnot
\end{eulerttcomment}
\begin{eulercomment}

Beberapa perintah memiliki aliasi, misalnya ln untuk log.
\end{eulercomment}
\begin{eulerprompt}
>ln(E^2), arctan(tan(0.5))
\end{eulerprompt}
\begin{euleroutput}
  2
  0.5
\end{euleroutput}
\begin{eulerprompt}
>sin(30°)
\end{eulerprompt}
\begin{euleroutput}
  0.5
\end{euleroutput}
\begin{eulercomment}
Pastikan untuk menggunakan tanda kurung (kurung bundar), jika ada
keraguan tentang urutan eksekusi! Berikut ini tidak sama dengan
(2\textasciicircum{}3)\textasciicircum{}4, yang merupakan default untuk 2\textasciicircum{}3\textasciicircum{}4 dalam EMT (beberapa sistem
numerik melakukannya dengan cara lain).
\end{eulercomment}
\begin{eulerprompt}
>2^3^4, (2^3)^4, 2^(3^4)
\end{eulerprompt}
\begin{euleroutput}
  2.41785163923e+24
  4096
  2.41785163923e+24
\end{euleroutput}
\eulersubheading{Bilangan Riil}
\begin{eulercomment}
Tipe data utama dalam Euler adalah bilangan riil. Bilangan riil
direpresentasikan dalam format IEEE dengan akurasi sekitar 16 digit
desimal.
\end{eulercomment}
\begin{eulerprompt}
>longest 1/3
\end{eulerprompt}
\begin{euleroutput}
       0.3333333333333333 
\end{euleroutput}
\begin{eulercomment}
Representasi ganda internal membutuhkan 8 byte.
\end{eulercomment}
\begin{eulerprompt}
>printdual(1/3)
\end{eulerprompt}
\begin{euleroutput}
  1.0101010101010101010101010101010101010101010101010101*2^-2
\end{euleroutput}
\begin{eulerprompt}
>printhex(1/3)
\end{eulerprompt}
\begin{euleroutput}
  5.5555555555554*16^-1
\end{euleroutput}
\eulersubheading{String}
\begin{eulercomment}
Suatu string dalam Euler didefinisikan dengan "...".
\end{eulercomment}
\begin{eulerprompt}
>"Suatu string dapat berisi apa saja."
\end{eulerprompt}
\begin{euleroutput}
  Suatu string dapat berisi apa saja.
\end{euleroutput}
\begin{eulercomment}
String dapat dirangkai dengan \textbar{} atau dengan +. Ini juga berlaku untuk
angka, yang dalam kasus tersebut diubah menjadi string.
\end{eulercomment}
\begin{eulerprompt}
>"The area of the circle with radius " + 2 + " cm is " + pi*4 + " cm^2."
\end{eulerprompt}
\begin{euleroutput}
  The area of the circle with radius 2 cm is 12.5663706144 cm^2.
\end{euleroutput}
\begin{eulercomment}
Fungsi cetak juga mengonversi angka menjadi string. Fungsi ini dapat
memuat sejumlah digit dan sejumlah tempat (0 untuk keluaran padat),
dan optimalnya satu unit.
\end{eulercomment}
\begin{eulerprompt}
>"Golden Ratio : " + print((1+sqrt(5))/2,5,0)
\end{eulerprompt}
\begin{euleroutput}
  Golden Ratio : 1.61803
\end{euleroutput}
\begin{eulercomment}
Ada string khusus none, yang tidak dicetak. String ini dikembalikan
oleh beberapa fungsi, ketika hasilnya tidak penting. (Dikembalikan
secara otomatis, jika fungsi tersebut tidak memiliki pernyataan
return.)
\end{eulercomment}
\begin{eulerprompt}
>none
\end{eulerprompt}
\begin{eulercomment}
Untuk mengubah string menjadi angka, cukup evaluasi string tersebut.
Ini juga berlaku untuk ekspresi (lihat di bawah).
\end{eulercomment}
\begin{eulerprompt}
>"1234.5"()
\end{eulerprompt}
\begin{euleroutput}
  1234.5
\end{euleroutput}
\begin{eulercomment}
Untuk mendefinisikan vektor string, gunakan notasi vektor [...]
\end{eulercomment}
\begin{eulerprompt}
>v:=["affe","charlie","bravo"]
\end{eulerprompt}
\begin{euleroutput}
  affe
  charlie
  bravo
\end{euleroutput}
\begin{eulercomment}
Vektor string kosong dilambangkan dengan [none]. Vektor string dapat
dirangkai.
\end{eulercomment}
\begin{eulerprompt}
>w:=[none]; w|v|v
\end{eulerprompt}
\begin{euleroutput}
  affe
  charlie
  bravo
  affe
  charlie
  bravo
\end{euleroutput}
\begin{eulercomment}
String dapat berisi karakter Unicode. Secara internal, string ini
berisi kode UTF-8. Untuk membuat string seperti itu, gunakan u"..."
dan salah satu entitas HTML.

String Unicode dapat dirangkai seperti string lainnya.
\end{eulercomment}
\begin{eulerprompt}
>u"&alpha; = " + 45 + u"&deg;" // pdfLaTeX mungkin gagal menampilkan secara benar
\end{eulerprompt}
\begin{euleroutput}
  α = 45°
\end{euleroutput}
\begin{eulercomment}
I
\end{eulercomment}
\begin{eulercomment}
Dalam komentar, entitas yang sama seperti alpha, beta, dll. dapat
digunakan. Ini mungkin merupakan alternatif cepat untuk Lateks.
(Rincian lebih lanjut pada komentar di bawah).
\end{eulercomment}
\begin{eulercomment}
Ada beberapa fungsi untuk membuat atau menganalisis string unicode.
Fungsi strtochar() akan mengenali string Unicode dan  menerjemahkannya
dengan benar.
\end{eulercomment}
\begin{eulerprompt}
>v=strtochar(u"&Auml; is a German letter")
\end{eulerprompt}
\begin{euleroutput}
  [196,  32,  105,  115,  32,  97,  32,  71,  101,  114,  109,  97,  110,
  32,  108,  101,  116,  116,  101,  114]
\end{euleroutput}
\begin{eulercomment}
Hasilnya adalah vektor angka Unicode. Fungsi kebalikannya adalah
chartoutf().
\end{eulercomment}
\begin{eulerprompt}
>v[1]=strtochar(u"&Uuml;")[1]; chartoutf(v)
\end{eulerprompt}
\begin{euleroutput}
  Ü is a German letter
\end{euleroutput}
\begin{eulercomment}
Fungsi utf() dapat menerjemahkan string dengan entitas dalam variabel
menjadi string Unicode.
\end{eulercomment}
\begin{eulerprompt}
>s="We have &alpha;=&beta;."; utf(s) // pdfLaTeX mungkin gagal menampilkan secara benar
\end{eulerprompt}
\begin{euleroutput}
  We have α=β.
\end{euleroutput}
\begin{eulercomment}
Dimungkinkan juga untuk menggunakan entitas numerik.
\end{eulercomment}
\begin{eulerprompt}
>u"&#196;hnliches"
\end{eulerprompt}
\begin{euleroutput}
  Ähnliches
\end{euleroutput}
\eulersubheading{Nilai Boolean }
\begin{eulercomment}
Nilai Boolean direpresentasikan dengan 1=benar atau 0=salah dalam
Euler. String dapat dibandingkan, seperti halnya angka.
\end{eulercomment}
\begin{eulerprompt}
>2<1, "apel"<"banana"
\end{eulerprompt}
\begin{euleroutput}
  0
  1
\end{euleroutput}
\begin{eulercomment}
"and" adalah operator "\&\&" dan "or" adalah operator "\textbar{}\textbar{}", seperti
dalam bahasa C. (Kata "and" dan "or" hanya dapat digunakan dalam
kondisi "if".)
\end{eulercomment}
\begin{eulerprompt}
>2<E && E<3
\end{eulerprompt}
\begin{euleroutput}
  1
\end{euleroutput}
\begin{eulercomment}
Operator Boolean mematuhi aturan bahasa matriks.
\end{eulercomment}
\begin{eulerprompt}
>(1:10)>5, nonzeros(%)
\end{eulerprompt}
\begin{euleroutput}
  [0,  0,  0,  0,  0,  1,  1,  1,  1,  1]
  [6,  7,  8,  9,  10]
\end{euleroutput}
\begin{eulercomment}
Anda dapat menggunakan fungsi nonzeros() untuk mengekstrak elemen
tertentu dari sebuah vektor. Dalam contoh ini, kami  menggunakan
kondisional isprime(n).
\end{eulercomment}
\begin{eulerprompt}
>N=2|3:2:99 // N berisi elemen 2 dan bilangan2 ganjil dari 3 s.d. 99
\end{eulerprompt}
\begin{euleroutput}
  [2,  3,  5,  7,  9,  11,  13,  15,  17,  19,  21,  23,  25,  27,  29,
  31,  33,  35,  37,  39,  41,  43,  45,  47,  49,  51,  53,  55,  57,
  59,  61,  63,  65,  67,  69,  71,  73,  75,  77,  79,  81,  83,  85,
  87,  89,  91,  93,  95,  97,  99]
\end{euleroutput}
\begin{eulerprompt}
>N[nonzeros(isprime(N))] //pilih anggota2 N yang prima
\end{eulerprompt}
\begin{euleroutput}
  [2,  3,  5,  7,  11,  13,  17,  19,  23,  29,  31,  37,  41,  43,  47,
  53,  59,  61,  67,  71,  73,  79,  83,  89,  97]
\end{euleroutput}
\eulersubheading{Format Keluaran}
\begin{eulercomment}
Format keluaran default EMT mencetak 12 digit. Untuk memastikan bahwa
kita melihat default, kita mengatur ulang formatnya.
\end{eulercomment}
\begin{eulerprompt}
>defformat; pi
\end{eulerprompt}
\begin{euleroutput}
  3.14159265359
\end{euleroutput}
\begin{eulercomment}
Secara internal, EMT menggunakan standar IEEE untuk angka ganda dengan
sekitar 16 digit desimal. Untuk melihat jumlah digit lengkap, gunakan
perintah "longestformat", atau kami menggunakan operator "longest"
untuk menampilkan hasil dalam format terpanjang.
\end{eulercomment}
\begin{eulerprompt}
>longest pi
\end{eulerprompt}
\begin{euleroutput}
        3.141592653589793 
\end{euleroutput}
\begin{eulercomment}
Berikut adalah representasi heksadesimal internal dari angka ganda.
\end{eulercomment}
\begin{eulerprompt}
>printhex(pi)
\end{eulerprompt}
\begin{euleroutput}
  3.243F6A8885A30*16^0
\end{euleroutput}
\begin{eulercomment}
Format keluaran dapat diubah secara permanen dengan perintah format.
\end{eulercomment}
\begin{eulerprompt}
>format(12,5); 1/3, pi, sin(1)
\end{eulerprompt}
\begin{euleroutput}
      0.33333 
      3.14159 
      0.84147 
\end{euleroutput}
\begin{eulercomment}
Format defaulnya adalah (12).
\end{eulercomment}
\begin{eulerprompt}
>format(12); 1/3
\end{eulerprompt}
\begin{euleroutput}
  0.333333333333
\end{euleroutput}
\begin{eulercomment}
Fungsi seperti "shortestformat", "shortformat", "longformat" bekerja
untuk vektor dengan cara berikut.
\end{eulercomment}
\begin{eulerprompt}
>shortestformat; random(3,8)
\end{eulerprompt}
\begin{euleroutput}
    0.66    0.2   0.89   0.28   0.53   0.31   0.44    0.3 
    0.28   0.88   0.27    0.7   0.22   0.45   0.31   0.91 
    0.19   0.46  0.095    0.6   0.43   0.73   0.47   0.32 
\end{euleroutput}
\begin{eulercomment}
Format default untuk skalar adalah format(12). Namun, ini dapat
diubah.
\end{eulercomment}
\begin{eulerprompt}
>setscalarformat(5); pi
\end{eulerprompt}
\begin{euleroutput}
  3.1416
\end{euleroutput}
\begin{eulercomment}
Fungsi "longestformat" juga mengatur format skalar.
\end{eulercomment}
\begin{eulerprompt}
>longestformat; pi
\end{eulerprompt}
\begin{euleroutput}
  3.141592653589793
\end{euleroutput}
\begin{eulercomment}
Sebagai referensi, berikut adalah daftar format keluaran yang paling
penting.

shortestformat  shortformat longformat, longestformat \\
format(length,digits) goodformat(length)\\
fracformat(length)\\
defformat 

Akurasi internal EMT adalah sekitar 16 tempat desimal, yang merupakan
standar IEEE. Angka-angka disimpan dalam format internal ini.

Namun format keluaran EMT dapat diatur secara fleksibel.
\end{eulercomment}
\begin{eulerprompt}
>longestformat; pi,
\end{eulerprompt}
\begin{euleroutput}
  3.141592653589793
\end{euleroutput}
\begin{eulerprompt}
>format(10,5); pi
\end{eulerprompt}
\begin{euleroutput}
    3.14159 
\end{euleroutput}
\begin{eulercomment}
Standarnya adalah defformat().
\end{eulercomment}
\begin{eulerprompt}
>defformat; // default
\end{eulerprompt}
\begin{eulercomment}
Ada operator pendek yang hanya mencetak satu nilai. Operator
"terpanjang" akan mencetak semua digit angka yang valid.
\end{eulercomment}
\begin{eulerprompt}
>longest pi^2/2
\end{eulerprompt}
\begin{euleroutput}
        4.934802200544679 
\end{euleroutput}
\begin{eulercomment}
Ada juga operator pendek untuk mencetak hasil dalam format pecahan.
Kami telah menggunakannya di atas.
\end{eulercomment}
\begin{eulerprompt}
>fraction 1+1/2+1/3+1/4
\end{eulerprompt}
\begin{euleroutput}
  25/12
\end{euleroutput}
\begin{eulercomment}
Karena format internal menggunakan cara biner untuk menyimpan angka,
nilai 0,1 tidak akan terwakili secara tepat. Kesalahannya bertambah
sedikit, seperti yang Anda lihat dalam perhitungan berikut.
\end{eulercomment}
\begin{eulerprompt}
>longest 0.1+0.1+0.1+0.1+0.1+0.1+0.1+0.1+0.1+0.1-1
\end{eulerprompt}
\begin{euleroutput}
   -1.110223024625157e-16 
\end{euleroutput}
\begin{eulercomment}
Namun dengan "longformat" default, Anda tidak akan melihat hal ini.
Demi kenyamanan, output angka yang sangat kecil adalah 0.
\end{eulercomment}
\begin{eulerprompt}
>0.1+0.1+0.1+0.1+0.1+0.1+0.1+0.1+0.1+0.1-1
\end{eulerprompt}
\begin{euleroutput}
  0
\end{euleroutput}
\eulerheading{Ekspresi}
\begin{eulercomment}
String atau nama dapat digunakan untuk menyimpan ekspresi matematika,
yang dapat dievaluasi oleh EMT. Untuk ini, gunakan tanda kurung
setelah ekspresi. Jika Anda ingin menggunakan string sebagai ekspresi,
gunakan konvensi untuk menamainya "fx" atau "fxy", dst. Ekspresi lebih
diutamakan daripada fungsi.

Variabel global dapat digunakan dalam evaluasi.
\end{eulercomment}
\begin{eulerprompt}
>r:=2; fx:="pi*r^2"; longest fx()
\end{eulerprompt}
\begin{euleroutput}
        12.56637061435917 
\end{euleroutput}
\begin{eulercomment}
Parameter ditetapkan ke x, y, dan z dalam urutan tersebut. Parameter
tambahan dapat ditambahkan menggunakan parameter yang ditetapkan.
\end{eulercomment}
\begin{eulerprompt}
>fx:="a*sin(x)^2"; fx(5,a=-1)
\end{eulerprompt}
\begin{euleroutput}
  -0.919535764538
\end{euleroutput}
\begin{eulercomment}
Perhatikan bahwa ekspresi akan selalu menggunakan variabel global,
bahkan jika ada variabel dalam suatu fungsi dengan nama yang sama.
(Jika tidak, evaluasi ekspresi dalam fungsi dapat memberikan hasil
yang sangat membingungkan bagi pengguna yang memanggil fungsi
tersebut.)
\end{eulercomment}
\begin{eulerprompt}
>at:=4; function f(expr,x,at) := expr(x); ...
>f("at*x^2",3,5) // computes 4*3^2 not 5*3^2
\end{eulerprompt}
\begin{euleroutput}
  36
\end{euleroutput}
\begin{eulercomment}
Jika Anda ingin menggunakan nilai lain untuk "at" selain nilai global,
Anda perlu menambahkan "at=value".
\end{eulercomment}
\begin{eulerprompt}
>at:=4; function f(expr,x,a) := expr(x,at=a); ...
>f("at*x^2",3,5)
\end{eulerprompt}
\begin{euleroutput}
  45
\end{euleroutput}
\begin{eulercomment}
Sebagai referensi, kami mencatat bahwa koleksi panggilan (dibahas di
tempat lain) dapat berisi ekspresi. Jadi, kita dapat membuat contoh di
atas sebagai berikut.
\end{eulercomment}
\begin{eulerprompt}
>at:=4; function f(expr,x) := expr(x); ...
>f(\{\{"at*x^2",at=5\}\},3)
\end{eulerprompt}
\begin{euleroutput}
  45
\end{euleroutput}
\begin{eulercomment}
Ekspresi dalam x sering digunakan seperti fungsi.\\
Perhatikan bahwa mendefinisikan fungsi dengan nama yang sama seperti
ekspresi simbolik global akan menghapus variabel ini untuk menghindari
kebingungan antara ekspresi simbolik dan fungsi.
\end{eulercomment}
\begin{eulerprompt}
>f &= 5*x;
>function f(x) := 6*x;
>f(2)
\end{eulerprompt}
\begin{euleroutput}
  12
\end{euleroutput}
\begin{eulercomment}
Berdasarkan konvensi, ekspresi simbolik atau numerik harus diberi nama
fx, fxy, dst. Skema penamaan ini tidak boleh digunakan untuk fungsi.
\end{eulercomment}
\begin{eulerprompt}
>fx &= diff(x^x,x); $&fx
\end{eulerprompt}
\begin{eulerformula}
\[
x^{x}\,\left(\log x+1\right)
\]
\end{eulerformula}
\begin{eulercomment}
Bentuk khusus dari suatu ekspresi memperbolehkan variabel apa pun
sebagai parameter tanpa nama untuk evaluasi ekspresi, bukan  hanya
"x", "y", dst. Untuk ini, awali ekspresi dengan "@(variabel)...".
\end{eulercomment}
\begin{eulerprompt}
>"@(a,b) a^2+b^2", %(4,5)
\end{eulerprompt}
\begin{euleroutput}
  @(a,b) a^2+b^2
  41
\end{euleroutput}
\begin{eulercomment}
Hal ini memungkinkan untuk memanipulasi ekspresi dalam variabel lain
untuk fungsi EMT yang memerlukan ekspresi dalam "x".

Cara paling mendasar untuk mendefinisikan fungsi sederhana adalah
dengan menyimpan rumusnya dalam ekspresi simbolik atau  numerik. Jika
variabel utamanya adalah x, ekspresi tersebut dapat dievaluasi seperti
halnya fungsi.

Seperti yang Anda lihat dalam contoh berikut, variabel global terlihat
selama evaluasi.
\end{eulercomment}
\begin{eulerprompt}
>fx &= x^3-a*x;  ...
>a=1.2; fx(0.5)
\end{eulerprompt}
\begin{euleroutput}
  -0.475
\end{euleroutput}
\begin{eulercomment}
Semua variabel lain dalam ekspresi dapat ditentukan dalam evaluasi
menggunakan parameter yang ditetapkan.
\end{eulercomment}
\begin{eulerprompt}
>fx(0.5,a=1.1)
\end{eulerprompt}
\begin{euleroutput}
  -0.425
\end{euleroutput}
\begin{eulercomment}
Suatu ekspresi tidak harus bersifat simbolis. Hal ini diperlukan jika
ekspresi tersebut mengandung fungsi-fungsi yang hanya diketahui dalam
kernel numerik, bukan dalam Maxima.

\begin{eulercomment}
\eulerheading{Matematika Simbolik}
\begin{eulercomment}
EMT mengerjakan matematika simbolik dengan bantuan Maxima. Untuk
detailnya, mulailah dengan tutorial berikut, atau telusuri referensi
untuk Maxima. Para ahli di Maxima harus memperhatikan bahwa ada
perbedaan dalam sintaksis antara sintaksis asli Maxima dan  sintaksis
default ekspresi simbolik di EMT.

Matematika simbolik terintegrasi dengan lancar ke dalam Euler dengan
\&. Ekspresi apa pun yang dimulai dengan \& adalah ekspresi simbolik.
Ekspresi ini dievaluasi dan dicetak oleh Maxima.

Pertama-tama, Maxima memiliki aritmatika "tak terbatas" yang dapat
menangani angka yang sangat besar.
\end{eulercomment}
\begin{eulerprompt}
>$&44!
\end{eulerprompt}
\begin{eulerformula}
\[
2658271574788448768043625811014615890319638528000000000
\]
\end{eulerformula}
\begin{eulercomment}
Dengan cara ini, Anda dapat menghitung hasil yang besar secara tepat.
Mari kita hitung

\end{eulercomment}
\begin{eulerformula}
\[
C(44,10) = \frac{44!}{34! \cdot 10!}
\]
\end{eulerformula}
\begin{eulerprompt}
>$& 44!/(34!*10!) // nilai C(44,10)
\end{eulerprompt}
\begin{eulerformula}
\[
2481256778
\]
\end{eulerformula}
\begin{eulercomment}
Tentu saja, Maxima memiliki fungsi yang lebih efisien untuk ini
(seperti halnya bagian numerik EMT).
\end{eulercomment}
\begin{eulerprompt}
>$binomial(44,10) //menghitung C(44,10) menggunakan fungsi binomial()
\end{eulerprompt}
\begin{eulerformula}
\[
2481256778
\]
\end{eulerformula}
\begin{eulercomment}
Untuk mempelajari lebih lanjut tentang fungsi tertentu, klik dua kali
pada fungsi tersebut. Misalnya, coba klik dua kali pada "\&binomial"
pada baris perintah sebelumnya. Ini akan membuka dokumentasi Maxima
sebagaimana disediakan oleh penulis program tersebut.

Anda akan mempelajari bahwa hal berikut juga berfungsi

\end{eulercomment}
\begin{eulerformula}
\[
C(x,3)=\frac{x!}{(x-3)!3!}=\frac{(x-2)(x-1)x}{6}
\]
\end{eulerformula}
\begin{eulerprompt}
>$binomial(x,3) // C(x,3)
\end{eulerprompt}
\begin{eulerformula}
\[
\frac{\left(x-2\right)\,\left(x-1\right)\,x}{6}
\]
\end{eulerformula}
\begin{eulercomment}
Jika Anda ingin mengganti x dengan nilai tertentu gunakan "with".
\end{eulercomment}
\begin{eulerprompt}
>$&binomial(x,3) with x=10 // substitusi x=10 ke C(x,3)
\end{eulerprompt}
\begin{eulerformula}
\[
120
\]
\end{eulerformula}
\begin{eulercomment}
Dengan cara itu Anda dapat menggunakan solusi suatu persamaan dalam
persamaan lainnya.

Ekspresi simbolik dicetak oleh Maxima dalam bentuk 2D. Alasannya
adalah adanya bendera simbolik khusus dalam string.

Seperti yang telah Anda lihat pada contoh sebelumnya dan berikut, jika
Anda telah menginstal LaTeX, Anda dapat mencetak ekspresi simbolik
dengan Latex. Jika tidak, perintah berikut akan menampilkan pesan
kesalahan.

Untuk mencetak ekspresi simbolik dengan LaTeX, gunakan \textdollar{} di depan \&
(atau Anda dapat menghilangkan \&) sebelum perintah. Jangan jalankan
perintah Maxima dengan \textdollar{}, jika Anda tidak menginstal LaTeX.
\end{eulercomment}
\begin{eulerprompt}
>$(3+x)/(x^2+1)
\end{eulerprompt}
\begin{eulerformula}
\[
\frac{x+3}{x^2+1}
\]
\end{eulerformula}
\begin{eulercomment}
Ekspresi simbolik diurai oleh Euler. Jika Anda memerlukan sintaksis
yang kompleks dalam satu ekspresi, Anda dapat melampirkan ekspresi
tersebut dalam "...". Menggunakan lebih dari satu ekspresi sederhana
dimungkinkan, tetapi sangat tidak disarankan.
\end{eulercomment}
\begin{eulerprompt}
>&"v := 5; v^2"
\end{eulerprompt}
\begin{euleroutput}
  
                                    25
  
\end{euleroutput}
\begin{eulercomment}
Untuk kelengkapan, kami mencatat bahwa ekspresi simbolik dapat
digunakan dalam program, tetapi harus disertakan dalam tanda  kutip.
Selain itu, akan jauh lebih efektif untuk memanggil Maxima pada waktu
kompilasi jika memungkinkan.
\end{eulercomment}
\begin{eulerprompt}
>$&expand((1+x)^4), $&factor(diff(%,x)) // diff: turunan, factor: faktor
\end{eulerprompt}
\begin{eulerformula}
\[
x^4+4\,x^3+6\,x^2+4\,x+1
\]
\end{eulerformula}
\begin{eulerformula}
\[
4\,\left(x+1\right)^3
\]
\end{eulerformula}
\begin{eulercomment}
Sekali lagi, \% mengacu pada hasil sebelumnya.

Untuk mempermudah, kami menyimpan solusi ke variabel simbolik.
Variabel simbolik didefinisikan dengan "\&=".
\end{eulercomment}
\begin{eulerprompt}
>fx &= (x+1)/(x^4+1); $&fx
\end{eulerprompt}
\begin{eulerformula}
\[
\frac{x+1}{x^4+1}
\]
\end{eulerformula}
\begin{eulercomment}
Ekspresi simbolik dapat digunakan dalam ekspresi simbolik lainnya.
\end{eulercomment}
\begin{eulerprompt}
>$&factor(diff(fx,x))
\end{eulerprompt}
\begin{eulerformula}
\[
\frac{-3\,x^4-4\,x^3+1}{\left(x^4+1\right)^2}
\]
\end{eulerformula}
\begin{eulercomment}
Input langsung perintah Maxima juga tersedia. Awali baris perintah
dengan "::". Sintaks Maxima disesuaikan dengan sintaks EMT (disebut
"mode kompatibilitas").
\end{eulercomment}
\begin{eulerprompt}
>&factor(20!)
\end{eulerprompt}
\begin{euleroutput}
  
                           2432902008176640000
  
\end{euleroutput}
\begin{eulerprompt}
>::: factor(10!)
\end{eulerprompt}
\begin{euleroutput}
  
                                 8  4  2
                                2  3  5  7
  
\end{euleroutput}
\begin{eulerprompt}
>:: factor(20!)
\end{eulerprompt}
\begin{euleroutput}
  
                          18  8  4  2
                         2   3  5  7  11 13 17 19
  
\end{euleroutput}
\begin{eulercomment}
Jika Anda ahli dalam Maxima, Anda mungkin ingin menggunakan sintaksis
asli Maxima. Anda dapat melakukannya dengan ":::".
\end{eulercomment}
\begin{eulerprompt}
>::: av:g$ av^2;
\end{eulerprompt}
\begin{euleroutput}
  
                                     2
                                    g
  
\end{euleroutput}
\begin{eulerprompt}
>fx &= x^3*exp(x), $fx
\end{eulerprompt}
\begin{euleroutput}
  
                                   3  x
                                  x  E
  
\end{euleroutput}
\begin{eulerformula}
\[
x^3\,e^{x}
\]
\end{eulerformula}
\begin{eulercomment}
Variabel tersebut dapat digunakan dalam ekspresi simbolik lainnya.
Perhatikan bahwa dalam perintah berikut sisi kanan \&= dievaluasi
sebelum penugasan ke Fx.
\end{eulercomment}
\begin{eulerprompt}
>&(fx with x=5), $%, &float(%)
\end{eulerprompt}
\begin{euleroutput}
  
                                       5
                                  125 E
  
\end{euleroutput}
\begin{eulerformula}
\[
125\,e^5
\]
\end{eulerformula}
\begin{euleroutput}
  
                            18551.64488782208
  
\end{euleroutput}
\begin{eulerprompt}
>fx(5)
\end{eulerprompt}
\begin{euleroutput}
  18551.6448878
\end{euleroutput}
\begin{eulercomment}
Untuk mengevaluasi suatu ekspresi dengan nilai variabel tertentu, Anda
dapat menggunakan operator "with".

Baris perintah berikut juga menunjukkan bahwa Maxima dapat
mengevaluasi ekspresi secara numerik dengan float().
\end{eulercomment}
\begin{eulerprompt}
>&(fx with x=10)-(fx with x=5), &float(%)
\end{eulerprompt}
\begin{euleroutput}
  
                                  10        5
                            1000 E   - 125 E
  
  
                           2.20079141499189e+7
  
\end{euleroutput}
\begin{eulerprompt}
>$factor(diff(fx,x,2))
\end{eulerprompt}
\begin{eulerformula}
\[
x\,\left(x^2+6\,x+6\right)\,e^{x}
\]
\end{eulerformula}
\begin{eulercomment}
Untuk mendapatkan kode Latex untuk suatu ekspresi, Anda dapat
menggunakan perintah tex.
\end{eulercomment}
\begin{eulerprompt}
>tex(fx)
\end{eulerprompt}
\begin{euleroutput}
  x^3\(\backslash\),e^\{x\}
\end{euleroutput}
\begin{eulercomment}
Ekspresi simbolik dapat dievaluasi seperti halnya ekspresi numerik.
\end{eulercomment}
\begin{eulerprompt}
>fx(0.5)
\end{eulerprompt}
\begin{euleroutput}
  0.206090158838
\end{euleroutput}
\begin{eulercomment}
Dalam ekspresi simbolik, ini tidak berfungsi, karena Maxima tidak
mendukungnya. Sebagai gantinya, gunakan sintaks "with" (bentuk yang
lebih baik dari perintah at(...) dari Maxima).
\end{eulercomment}
\begin{eulerprompt}
>$&fx with x=1/2
\end{eulerprompt}
\begin{eulerformula}
\[
\frac{\sqrt{e}}{8}
\]
\end{eulerformula}
\begin{eulercomment}
Penugasan tersebut juga dapat bersifat simbolis.
\end{eulercomment}
\begin{eulerprompt}
>$&fx with x=1+t
\end{eulerprompt}
\begin{eulerformula}
\[
\left(t+1\right)^3\,e^{t+1}
\]
\end{eulerformula}
\begin{eulercomment}
Perintah solve memecahkan ekspresi simbolik untuk variabel dalam
Maxima. Hasilnya adalah vektor solusi.
\end{eulercomment}
\begin{eulerprompt}
>$&solve(x^2+x=4,x)
\end{eulerprompt}
\begin{eulerformula}
\[
\left[ x=\frac{-\sqrt{17}-1}{2} , x=\frac{\sqrt{17}-1}{2} \right] 
\]
\end{eulerformula}
\begin{eulercomment}
Bandingkan dengan perintah numerik "solve" di Euler, yang memerlukan
nilai awal, dan secara opsional nilai target.
\end{eulercomment}
\begin{eulerprompt}
>solve("x^2+x",1,y=4)
\end{eulerprompt}
\begin{euleroutput}
  1.56155281281
\end{euleroutput}
\begin{eulercomment}
Nilai numerik dari solusi simbolik dapat dihitung dengan mengevaluasi
hasil simbolik. Euler akan membaca ulang penugasan x= dst. Jika Anda
tidak memerlukan hasil numerik untuk perhitungan lebih lanjut, Anda
juga dapat membiarkan Maxima menemukan nilai numeriknya.
\end{eulercomment}
\begin{eulerprompt}
>sol &= solve(x^2+2*x=4,x); $&sol, sol(), $&float(sol)
\end{eulerprompt}
\begin{eulerformula}
\[
\left[ x=-\sqrt{5}-1 , x=\sqrt{5}-1 \right] 
\]
\end{eulerformula}
\begin{euleroutput}
  [-3.23607,  1.23607]
\end{euleroutput}
\begin{eulerformula}
\[
\left[ x=-3.23606797749979 , x=1.23606797749979 \right] 
\]
\end{eulerformula}
\begin{eulercomment}
Untuk mendapatkan solusi simbolis yang spesifik, seseorang dapat
menggunakan "with" dan indeks.
\end{eulercomment}
\begin{eulerprompt}
>$&solve(x^2+x=1,x), x2 &= x with %[2]; $&x2
\end{eulerprompt}
\begin{eulerformula}
\[
\left[ x=\frac{-\sqrt{5}-1}{2} , x=\frac{\sqrt{5}-1}{2} \right] 
\]
\end{eulerformula}
\begin{eulerformula}
\[
\frac{\sqrt{5}-1}{2}
\]
\end{eulerformula}
\begin{eulercomment}
Untuk menyelesaikan sistem persamaan, gunakan vektor persamaan.
Hasilnya adalah vektor solusi.
\end{eulercomment}
\begin{eulerprompt}
>sol &= solve([x+y=3,x^2+y^2=5],[x,y]); $&sol, $&x*y with sol[1]
\end{eulerprompt}
\begin{eulerformula}
\[
\left[ \left[ x=2 , y=1 \right]  , \left[ x=1 , y=2 \right] 
  \right] 
\]
\end{eulerformula}
\begin{eulerformula}
\[
2
\]
\end{eulerformula}
\begin{eulercomment}
Ekspresi simbolik dapat memiliki tanda, yang menunjukkan perlakuan
khusus di Maxima. Beberapa tanda dapat digunakan sebagai perintah
juga, yang lainnya tidak. Tanda ditambahkan dengan "\textbar{}" (bentuk yang
lebih baik dari "ev(...,flags)")
\end{eulercomment}
\begin{eulerprompt}
>$& diff((x^3-1)/(x+1),x) //turunan bentuk pecahan
\end{eulerprompt}
\begin{eulerformula}
\[
\frac{3\,x^2}{x+1}-\frac{x^3-1}{\left(x+1\right)^2}
\]
\end{eulerformula}
\begin{eulerprompt}
>$& diff((x^3-1)/(x+1),x) | ratsimp //menyederhanakan pecahan
\end{eulerprompt}
\begin{eulerformula}
\[
\frac{2\,x^3+3\,x^2+1}{x^2+2\,x+1}
\]
\end{eulerformula}
\begin{eulerprompt}
>$&factor(%)
\end{eulerprompt}
\begin{eulerformula}
\[
\frac{2\,x^3+3\,x^2+1}{\left(x+1\right)^2}
\]
\end{eulerformula}
\eulerheading{Fungsi}
\begin{eulercomment}
Dalam EMT, fungsi adalah program yang ditentukan dengan perintah
“function”. Fungsi dapat berupa fungsi satu baris atau fungsi
multibaris.\\
ungsi satu baris dapat berupa numerik atau simbolik. Fungsi satu baris
numerik didefinisikan dengan “:=”.
\end{eulercomment}
\begin{eulerprompt}
>function f(x) := x*sqrt(x^2+1)
\end{eulerprompt}
\begin{eulercomment}
Sebagai gambaran umum, kami menunjukkan semua definisi yang mungkin
untuk fungsi satu baris. Sebuah fungsi dapat dievaluasi seperti halnya
fungsi Euler bawaan.
\end{eulercomment}
\begin{eulerprompt}
>f(2)
\end{eulerprompt}
\begin{euleroutput}
  4.472135955
\end{euleroutput}
\begin{eulercomment}
Fungsi ini juga dapat digunakan untuk vektor, mengikuti bahasa matriks
Euler, karena ekspresi yang digunakan dalam fungsi ini adalah vektor.
\end{eulercomment}
\begin{eulerprompt}
>f(0:0.1:1)
\end{eulerprompt}
\begin{euleroutput}
  [0,  0.100499,  0.203961,  0.313209,  0.430813,  0.559017,  0.699714,
  0.854459,  1.0245,  1.21083,  1.41421]
\end{euleroutput}
\begin{eulercomment}
Fungsi dapat diplot. Alih-alih ekspresi, kita hanya perlu memberikan
nama fungsi.\\
Berbeda dengan ekspresi simbolik atau numerik, nama fungsi harus
disediakan dalam bentuk string.
\end{eulercomment}
\begin{eulerprompt}
>solve("f",1,y=1)
\end{eulerprompt}
\begin{euleroutput}
  0.786151377757
\end{euleroutput}
\begin{eulercomment}
Secara default, jika Anda perlu menimpa fungsi built-in, Anda harus
menambahkan kata kunci “overwrite”. Menimpa fungsi bawaan berbahaya
dan dapat menyebabkan masalah bagi fungsi lain yang bergantung pada
fungsi tersebut.\\
Anda masih dapat memanggil fungsi bawaan sebagai “\_...”, jika fungsi
tersebut merupakan fungsi dalam inti Euler.
\end{eulercomment}
\begin{eulerprompt}
>function overwrite sin (x) := _sin(x°) // redine sine in degrees
>sin(45)
\end{eulerprompt}
\begin{euleroutput}
  0.707106781187
\end{euleroutput}
\begin{eulercomment}
Sebaiknya kita hilangkan definisi ulang tentang dosa ini.
\end{eulercomment}
\begin{eulerprompt}
>forget sin; sin(pi/4)
\end{eulerprompt}
\begin{euleroutput}
  0.707106781187
\end{euleroutput}
\eulersubheading{Parameter Default}
\begin{eulercomment}
Fungsi numerik dapat memiliki parameter default.
\end{eulercomment}
\begin{eulerprompt}
>function f(x,a=1) := a*x^2
\end{eulerprompt}
\begin{eulercomment}
Menghilangkan parameter ini menggunakan nilai default.
\end{eulercomment}
\begin{eulerprompt}
>f(4)
\end{eulerprompt}
\begin{euleroutput}
  16
\end{euleroutput}
\begin{eulercomment}
Menetapkannya akan menimpa nilai default.
\end{eulercomment}
\begin{eulerprompt}
>f(4,5)
\end{eulerprompt}
\begin{euleroutput}
  80
\end{euleroutput}
\begin{eulercomment}
Parameter yang ditetapkan juga menimpanya. Ini digunakan oleh banyak
fungsi Euler seperti plot2d, plot3d.
\end{eulercomment}
\begin{eulerprompt}
>f(4,a=1)
\end{eulerprompt}
\begin{euleroutput}
  16
\end{euleroutput}
\begin{eulercomment}
Jika sebuah variabel bukan parameter, maka variabel tersebut harus
bersifat global. Fungsi satu baris dapat melihat variabel global.
\end{eulercomment}
\begin{eulerprompt}
>function f(x) := a*x^2
>a=6; f(2)
\end{eulerprompt}
\begin{euleroutput}
  24
\end{euleroutput}
\begin{eulercomment}
Tetapi parameter yang ditetapkan akan menggantikan nilai global.\\
Jika argumen tidak ada dalam daftar parameter yang telah ditetapkan
sebelumnya, argumen tersebut harus dideklarasikan dengan “:=”!
\end{eulercomment}
\begin{eulerprompt}
>f(2,a:=5)
\end{eulerprompt}
\begin{euleroutput}
  20
\end{euleroutput}
\begin{eulercomment}
Fungsi simbolik didefinisikan dengan “\&=”. Fungsi-fungsi ini
didefinisikan dalam Euler dan Maxima, dan dapat digunakan di kedua
bahasa tersebut. Ekspresi pendefinisian dijalankan melalui Maxima
sebelum definisi.
\end{eulercomment}
\begin{eulerprompt}
>function g(x) &= x^3-x*exp(-x); $&g(x)
\end{eulerprompt}
\begin{eulerformula}
\[
x^3-x\,e^ {- x }
\]
\end{eulerformula}
\begin{eulercomment}
Fungsi simbolis dapat digunakan dalam ekspresi simbolis.
\end{eulercomment}
\begin{eulerprompt}
>$&diff(g(x),x), $&% with x=4/3
\end{eulerprompt}
\begin{eulerformula}
\[
x\,e^ {- x }-e^ {- x }+3\,x^2
\]
\end{eulerformula}
\begin{eulerformula}
\[
\frac{e^ {- \frac{4}{3} }}{3}+\frac{16}{3}
\]
\end{eulerformula}
\begin{eulercomment}
Fungsi ini juga dapat digunakan dalam ekspresi numerik. Tentu saja,
ini hanya akan berfungsi jika EMT dapat menginterpretasikan semua yang
ada di dalam fungsi.
\end{eulercomment}
\begin{eulerprompt}
>g(5+g(1))
\end{eulerprompt}
\begin{euleroutput}
  178.635099908
\end{euleroutput}
\begin{eulercomment}
Mereka dapat digunakan untuk mendefinisikan fungsi atau ekspresi
simbolis lainnya.
\end{eulercomment}
\begin{eulerprompt}
>function G(x) &= factor(integrate(g(x),x)); $&G(c) // integrate: mengintegralkan
\end{eulerprompt}
\begin{eulerformula}
\[
\frac{e^ {- c }\,\left(c^4\,e^{c}+4\,c+4\right)}{4}
\]
\end{eulerformula}
\begin{eulerprompt}
>solve(&g(x),0.5)
\end{eulerprompt}
\begin{euleroutput}
  0.703467422498
\end{euleroutput}
\begin{eulercomment}
Hal berikut ini juga dapat digunakan, karena Euler menggunakan
ekspresi simbolik dalam fungsi g, jika tidak menemukan variabel
simbolik g, dan jika ada fungsi simbolik g.
\end{eulercomment}
\begin{eulerprompt}
>solve(&g,0.5)
\end{eulerprompt}
\begin{euleroutput}
  0.703467422498
\end{euleroutput}
\begin{eulerprompt}
>function P(x,n) &= (2*x-1)^n; $&P(x,n)
\end{eulerprompt}
\begin{eulerformula}
\[
\left(2\,x-1\right)^{n}
\]
\end{eulerformula}
\begin{eulerprompt}
>function Q(x,n) &= (x+2)^n; $&Q(x,n)
\end{eulerprompt}
\begin{eulerformula}
\[
\left(x+2\right)^{n}
\]
\end{eulerformula}
\begin{eulerprompt}
>$&P(x,4), $&expand(%)
\end{eulerprompt}
\begin{eulerformula}
\[
\left(2\,x-1\right)^4
\]
\end{eulerformula}
\begin{eulerformula}
\[
16\,x^4-32\,x^3+24\,x^2-8\,x+1
\]
\end{eulerformula}
\begin{eulerprompt}
>P(3,4)
\end{eulerprompt}
\begin{euleroutput}
  625
\end{euleroutput}
\begin{eulerprompt}
>$&P(x,4)+ Q(x,3), $&expand(%)
\end{eulerprompt}
\begin{eulerformula}
\[
\left(2\,x-1\right)^4+\left(x+2\right)^3
\]
\end{eulerformula}
\begin{eulerformula}
\[
16\,x^4-31\,x^3+30\,x^2+4\,x+9
\]
\end{eulerformula}
\begin{eulerprompt}
>$&P(x,4)-Q(x,3), $&expand(%), $&factor(%)
\end{eulerprompt}
\begin{eulerformula}
\[
\left(2\,x-1\right)^4-\left(x+2\right)^3
\]
\end{eulerformula}
\begin{eulerformula}
\[
16\,x^4-33\,x^3+18\,x^2-20\,x-7
\]
\end{eulerformula}
\begin{eulerformula}
\[
16\,x^4-33\,x^3+18\,x^2-20\,x-7
\]
\end{eulerformula}
\begin{eulerprompt}
>$&P(x,4)*Q(x,3), $&expand(%), $&factor(%)
\end{eulerprompt}
\begin{eulerformula}
\[
\left(x+2\right)^3\,\left(2\,x-1\right)^4
\]
\end{eulerformula}
\begin{eulerformula}
\[
16\,x^7+64\,x^6+24\,x^5-120\,x^4-15\,x^3+102\,x^2-52\,x+8
\]
\end{eulerformula}
\begin{eulerformula}
\[
\left(x+2\right)^3\,\left(2\,x-1\right)^4
\]
\end{eulerformula}
\begin{eulerprompt}
>$&P(x,4)/Q(x,1), $&expand(%), $&factor(%)
\end{eulerprompt}
\begin{eulerformula}
\[
\frac{\left(2\,x-1\right)^4}{x+2}
\]
\end{eulerformula}
\begin{eulerformula}
\[
\frac{16\,x^4}{x+2}-\frac{32\,x^3}{x+2}+\frac{24\,x^2}{x+2}-\frac{8
 \,x}{x+2}+\frac{1}{x+2}
\]
\end{eulerformula}
\begin{eulerformula}
\[
\frac{\left(2\,x-1\right)^4}{x+2}
\]
\end{eulerformula}
\begin{eulerprompt}
>function f(x) &= x^3-x; $&f(x)
\end{eulerprompt}
\begin{eulerformula}
\[
x^3-x
\]
\end{eulerformula}
\begin{eulercomment}
Dengan \&=, fungsi ini bersifat simbolis, dan dapat digunakan dalam
ekspresi simbolis lainnya.
\end{eulercomment}
\begin{eulerprompt}
>$&integrate(f(x),x)
\end{eulerprompt}
\begin{eulerformula}
\[
\frac{x^4}{4}-\frac{x^2}{2}
\]
\end{eulerformula}
\begin{eulercomment}
Dengan := fungsi tersebut berupa angka. Contoh yang baik adalah
integral pasti seperti


\end{eulercomment}
\begin{eulerformula}
\[
f(x) = \int_1^x t^t \, dt,
\]
\end{eulerformula}
\begin{eulercomment}
yang tidak dapat dievaluasi secara simbolik.\\
Jika kita mendefinisikan ulang fungsi tersebut dengan kata kunci
“map”, maka fungsi tersebut dapat digunakan untuk vektor x.\\
Secara internal, fungsi tersebut dipanggil untuk semua nilai x satu
kali, dan hasilnya disimpan dalam sebuah vektor.
\end{eulercomment}
\begin{eulerprompt}
>function map f(x) := integrate("x^x",1,x)
>f(0:0.5:2)
\end{eulerprompt}
\begin{euleroutput}
  [-0.783431,  -0.410816,  0,  0.676863,  2.05045]
\end{euleroutput}
\begin{eulercomment}
Fungsi dapat memiliki nilai default untuk parameter.
\end{eulercomment}
\begin{eulerprompt}
>function mylog (x,base=10) := ln(x)/ln(base);
\end{eulerprompt}
\begin{eulercomment}
Sekarang, fungsi ini dapat dipanggil dengan atau tanpa parameter
“base”.
\end{eulercomment}
\begin{eulerprompt}
>mylog(100), mylog(2^6.7,2)
\end{eulerprompt}
\begin{euleroutput}
  2
  6.7
\end{euleroutput}
\begin{eulercomment}
Selain itu, dimungkinkan untuk menggunakan parameter yang ditetapkan.
\end{eulercomment}
\begin{eulerprompt}
>mylog(E^2,base=E)
\end{eulerprompt}
\begin{euleroutput}
  2
\end{euleroutput}
\begin{eulercomment}
Sering kali, kita ingin menggunakan fungsi untuk vektor di satu
tempat, dan untuk masing-masing elemen di tempat lain. Hal ini
dimungkinkan dengan parameter vektor.
\end{eulercomment}
\begin{eulerprompt}
>function f([a,b]) &= a^2+b^2-a*b+b; $&f(a,b), $&f(x,y)
\end{eulerprompt}
\begin{eulerformula}
\[
b^2-a\,b+b+a^2
\]
\end{eulerformula}
\begin{eulerformula}
\[
y^2-x\,y+y+x^2
\]
\end{eulerformula}
\begin{eulercomment}
Fungsi simbolik seperti itu dapat digunakan untuk variabel simbolik.\\
Tetapi fungsi ini juga dapat digunakan untuk vektor numerik.
\end{eulercomment}
\begin{eulerprompt}
>v=[3,4]; f(v)
\end{eulerprompt}
\begin{euleroutput}
  17
\end{euleroutput}
\begin{eulercomment}
Ada juga fungsi yang murni simbolis, yang tidak dapat digunakan secara
numerik.
\end{eulercomment}
\begin{eulerprompt}
>function lapl(expr,x,y) &&= diff(expr,x,2)+diff(expr,y,2)//turunan parsial kedua
\end{eulerprompt}
\begin{euleroutput}
  
                   diff(expr, y, 2) + diff(expr, x, 2)
  
\end{euleroutput}
\begin{eulerprompt}
>$&realpart((x+I*y)^4), $&lapl(%,x,y)
\end{eulerprompt}
\begin{eulerformula}
\[
y^4-6\,x^2\,y^2+x^4
\]
\end{eulerformula}
\begin{eulerformula}
\[
0
\]
\end{eulerformula}
\begin{eulercomment}
Tetapi tentu saja, semua itu bisa digunakan dalam ekspresi simbolis
atau dalam definisi fungsi simbolis.
\end{eulercomment}
\begin{eulerprompt}
>function f(x,y) &= factor(lapl((x+y^2)^5,x,y)); $&f(x,y)
\end{eulerprompt}
\begin{eulerformula}
\[
10\,\left(y^2+x\right)^3\,\left(9\,y^2+x+2\right)
\]
\end{eulerformula}
\begin{eulercomment}
Untuk meringkas

- \&= mendefinisikan fungsi simbolik,

- := mendefinisikan fungsi numerik,

- \&\&= mendefinisikan fungsi simbolik murni.


\begin{eulercomment}
\eulerheading{Memecahkan Ekspresi }
\begin{eulercomment}
Ekspresi dapat diselesaikan secara numerik dan simbolik.\\
Untuk menyelesaikan ekspresi sederhana dari satu variabel, kita dapat
menggunakan fungsi solve(). Fungsi ini membutuhkan nilai awal untuk
memulai pencarian. Secara internal, solve() menggunakan metode secant.
\end{eulercomment}
\begin{eulerprompt}
>solve("x^2-2",1)
\end{eulerprompt}
\begin{euleroutput}
  1.41421356237
\end{euleroutput}
\begin{eulercomment}
Hal ini juga bisa digunakan untuk ekspresi simbolis. Perhatikan fungsi
berikut ini.
\end{eulercomment}
\begin{eulerprompt}
>$&solve(x^2=2,x)
\end{eulerprompt}
\begin{eulerformula}
\[
\left[ x=-\sqrt{2} , x=\sqrt{2} \right] 
\]
\end{eulerformula}
\begin{eulerprompt}
>$&solve(x^2-2,x)
\end{eulerprompt}
\begin{eulerformula}
\[
\left[ x=-\sqrt{2} , x=\sqrt{2} \right] 
\]
\end{eulerformula}
\begin{eulerprompt}
>$&solve(a*x^2+b*x+c=0,x)
\end{eulerprompt}
\begin{eulerformula}
\[
\left[ x=\frac{-\sqrt{b^2-4\,a\,c}-b}{2\,a} , x=\frac{\sqrt{b^2-4\,
 a\,c}-b}{2\,a} \right] 
\]
\end{eulerformula}
\begin{eulerprompt}
>$&solve([a*x+b*y=c,d*x+e*y=f],[x,y])
\end{eulerprompt}
\begin{eulerformula}
\[
\left[ \left[ x=-\frac{c\,e}{b\,\left(d-5\right)-a\,e} , y=\frac{c
 \,\left(d-5\right)}{b\,\left(d-5\right)-a\,e} \right]  \right] 
\]
\end{eulerformula}
\begin{eulerprompt}
>px &= 4*x^8+x^7-x^4-x; $&px
\end{eulerprompt}
\begin{eulerformula}
\[
4\,x^8+x^7-x^4-x
\]
\end{eulerformula}
\begin{eulercomment}
Sekarang kita mencari titik, di mana polinomialnya adalah 2. Dalam
solve(), nilai target default y=0 dapat diubah dengan variabel yang
ditetapkan.\\
ami menggunakan y=2 dan mengeceknya dengan mengevaluasi polinomial
pada hasil sebelumnya.
\end{eulercomment}
\begin{eulerprompt}
>solve(px,1,y=2), px(%)
\end{eulerprompt}
\begin{euleroutput}
  0.966715594851
  2
\end{euleroutput}
\begin{eulercomment}
Memecahkan sebuah ekspresi simbolik dalam bentuk simbolik
mengembalikan sebuah daftar solusi. \\
Kami menggunakan pemecah simbolik solve() yang disediakan oleh Maxima.
\end{eulercomment}
\begin{eulerprompt}
>sol &= solve(x^2-x-1,x); $&sol
\end{eulerprompt}
\begin{eulerformula}
\[
\left[ x=\frac{1-\sqrt{5}}{2} , x=\frac{\sqrt{5}+1}{2} \right] 
\]
\end{eulerformula}
\begin{eulercomment}
Cara termudah untuk mendapatkan nilai numerik adalah dengan
mengevaluasi solusi secara numerik seperti sebuah ekspresi.
\end{eulercomment}
\begin{eulerprompt}
>longest sol()
\end{eulerprompt}
\begin{euleroutput}
      -0.6180339887498949       1.618033988749895 
\end{euleroutput}
\begin{eulercomment}
Untuk menggunakan solusi secara simbolis dalam ekspresi lain, cara
termudah adalah “with”.
\end{eulercomment}
\begin{eulerprompt}
>$&x^2 with sol[1], $&expand(x^2-x-1 with sol[2])
\end{eulerprompt}
\begin{eulerformula}
\[
\frac{\left(\sqrt{5}-1\right)^2}{4}
\]
\end{eulerformula}
\begin{eulerformula}
\[
0
\]
\end{eulerformula}
\begin{eulercomment}
Menyelesaikan sistem persamaan secara simbolik dapat dilakukan dengan
vektor persamaan dan pemecah simbolik solve(). Jawabannya adalah
sebuah daftar daftar persamaan.
\end{eulercomment}
\begin{eulerprompt}
>$&solve([x+y=2,x^3+2*y+x=4],[x,y])
\end{eulerprompt}
\begin{eulerformula}
\[
\left[ \left[ x=-1 , y=3 \right]  , \left[ x=1 , y=1 \right]  , 
 \left[ x=0 , y=2 \right]  \right] 
\]
\end{eulerformula}
\begin{eulercomment}
Fungsi f() dapat melihat variabel global. Tetapi seringkali kita ingin
menggunakan parameter lokal.

\end{eulercomment}
\begin{eulerformula}
\[
f(x) = \int_1^x t^t \, dt,
\]
\end{eulerformula}
\begin{eulercomment}
dengan a = 3.
\end{eulercomment}
\begin{eulerprompt}
>function f(x,a) := x^a-a^x;
\end{eulerprompt}
\begin{eulercomment}
Salah satu cara untuk mengoper parameter tambahan ke f() adalah dengan
menggunakan sebuah daftar yang berisi nama fungsi dan parameternya
(cara lainnya adalah dengan menggunakan parameter titik koma).
\end{eulercomment}
\begin{eulerprompt}
>solve(\{\{"f",3\}\},2,y=0.1)
\end{eulerprompt}
\begin{euleroutput}
  2.54116291558
\end{euleroutput}
\begin{eulercomment}
Hal ini juga dapat dilakukan dengan ekspresi. Namun, elemen daftar
bernama harus digunakan. (Lebih lanjut tentang daftar dalam tutorial
tentang sintaks EMT).
\end{eulercomment}
\begin{eulerprompt}
>solve(\{\{"x^a-a^x",a=3\}\},2,y=0.1)
\end{eulerprompt}
\begin{euleroutput}
  2.54116291558
\end{euleroutput}
\eulerheading{Menyelesaikan Pertidaksamaan}
\begin{eulercomment}
Untuk menyelesaikan pertidaksamaan, EMT tidak akan dapat melakukannya,
melainkan dengan bantuan Maxima, artinya secara eksak (simbolik).
Perintah Maxima yang digunakan adalah fourier\_elim(), yang harus
dipanggil dengan perintah "load(fourier\_elim)" terlebih dahulu.
\end{eulercomment}
\begin{eulerprompt}
>&load(fourier_elim)
\end{eulerprompt}
\begin{euleroutput}
  
          C:/Program Files/Euler x64/maxima/share/maxima/5.35.1/share/f\(\backslash\)
  ourier_elim/fourier_elim.lisp
  
\end{euleroutput}
\begin{eulerprompt}
>$&fourier_elim([x^2 - 1>0],[x]) // x^2-1 > 0
\end{eulerprompt}
\begin{eulerformula}
\[
\left[ 1<x \right] \lor \left[ x<-1 \right] 
\]
\end{eulerformula}
\begin{eulerprompt}
>$&fourier_elim([x^2 - 1<0],[x]) // x^2-1 < 0
\end{eulerprompt}
\begin{eulerformula}
\[
\left[ -1<x , x<1 \right] 
\]
\end{eulerformula}
\begin{eulerprompt}
>$&fourier_elim([x^2 - 1 # 0],[x]) // x^-1 <> 0
\end{eulerprompt}
\begin{eulerformula}
\[
\left[ -1<x , x<1 \right] \lor \left[ 1<x \right] \lor \left[ x<-1
  \right] 
\]
\end{eulerformula}
\begin{eulerprompt}
>$&fourier_elim([x # 6],[x])
\end{eulerprompt}
\begin{eulerformula}
\[
\left[ x<6 \right] \lor \left[ 6<x \right] 
\]
\end{eulerformula}
\begin{eulerprompt}
>$&fourier_elim([x < 1, x > 1],[x]) // tidak memiliki penyelesaian
\end{eulerprompt}
\begin{eulerformula}
\[
{\it emptyset}
\]
\end{eulerformula}
\begin{eulerprompt}
>$&fourier_elim([minf < x, x < inf],[x]) // solusinya R
\end{eulerprompt}
\begin{eulerformula}
\[
{\it universalset}
\]
\end{eulerformula}
\begin{eulerprompt}
>$&fourier_elim([x^3 - 1 > 0],[x])
\end{eulerprompt}
\begin{eulerformula}
\[
\left[ 1<x , x^2+x+1>0 \right] \lor \left[ x<1 , -x^2-x-1>0
  \right] 
\]
\end{eulerformula}
\begin{eulerprompt}
>$&fourier_elim([cos(x) < 1/2],[x]) // ??? gagal
\end{eulerprompt}
\begin{eulerformula}
\[
\left[ 1-2\,\cos x>0 \right] 
\]
\end{eulerformula}
\begin{eulerprompt}
>$&fourier_elim([y-x < 5, x - y < 7, 10 < y],[x,y]) // sistem pertidaksamaan
\end{eulerprompt}
\begin{eulerformula}
\[
\left[ y-5<x , x<y+7 , 10<y \right] 
\]
\end{eulerformula}
\begin{eulerprompt}
>$&fourier_elim([y-x < 5, x - y < 7, 10 < y],[y,x])
\end{eulerprompt}
\begin{eulerformula}
\[
\left[ {\it max}\left(10 , x-7\right)<y , y<x+5 , 5<x \right] 
\]
\end{eulerformula}
\begin{eulerprompt}
>$&fourier_elim((x + y < 5) and (x - y >8),[x,y])
\end{eulerprompt}
\begin{eulerformula}
\[
\left[ y+8<x , x<5-y , y<-\frac{3}{2} \right] 
\]
\end{eulerformula}
\begin{eulerprompt}
>$&fourier_elim(((x + y < 5) and x < 1) or  (x - y >8),[x,y])
\end{eulerprompt}
\begin{eulerformula}
\[
\left[ y+8<x \right] \lor \left[ x<{\it min}\left(1 , 5-y\right)
  \right] 
\]
\end{eulerformula}
\begin{eulerprompt}
>&fourier_elim([max(x,y) > 6, x # 8, abs(y-1) > 12],[x,y])
\end{eulerprompt}
\begin{euleroutput}
  
          [6 < x, x < 8, y < - 11] or [8 < x, y < - 11]
   or [x < 8, 13 < y] or [x = y, 13 < y] or [8 < x, x < y, 13 < y]
   or [y < x, 13 < y]
  
\end{euleroutput}
\begin{eulerprompt}
>$&fourier_elim([(x+6)/(x-9) <= 6],[x])
\end{eulerprompt}
\begin{eulerformula}
\[
\left[ x=12 \right] \lor \left[ 12<x \right] \lor \left[ x<9
  \right] 
\]
\end{eulerformula}
\eulerheading{Bahasa Matriks}
\begin{eulercomment}
Dokumentasi inti EMT berisi diskusi terperinci tentang bahasa matriks
Euler.\\
Vektor dan matriks dimasukkan dengan tanda kurung siku, elemen
dipisahkan dengan koma, baris dipisahkan dengan titik koma.
\end{eulercomment}
\begin{eulerprompt}
>A=[1,2;3,4]
\end{eulerprompt}
\begin{euleroutput}
              1             2 
              3             4 
\end{euleroutput}
\begin{eulercomment}
Hasil kali matriks dilambangkan dengan sebuah titik.
\end{eulercomment}
\begin{eulerprompt}
>b=[3;4]
\end{eulerprompt}
\begin{euleroutput}
              3 
              4 
\end{euleroutput}
\begin{eulerprompt}
>b' // transpose b
\end{eulerprompt}
\begin{euleroutput}
  [3,  4]
\end{euleroutput}
\begin{eulerprompt}
>inv(A) //inverse A
\end{eulerprompt}
\begin{euleroutput}
             -2             1 
            1.5          -0.5 
\end{euleroutput}
\begin{eulerprompt}
>A.b //perkalian matriks
\end{eulerprompt}
\begin{euleroutput}
             11 
             25 
\end{euleroutput}
\begin{eulerprompt}
>A.inv(A)
\end{eulerprompt}
\begin{euleroutput}
              1             0 
              0             1 
\end{euleroutput}
\begin{eulercomment}
Poin utama dari bahasa matriks adalah bahwa semua fungsi dan operator
bekerja elemen demi elemen.
\end{eulercomment}
\begin{eulerprompt}
>A.A
\end{eulerprompt}
\begin{euleroutput}
              7            10 
             15            22 
\end{euleroutput}
\begin{eulerprompt}
>A^2 //perpangkatan elemen2 A
\end{eulerprompt}
\begin{euleroutput}
              1             4 
              9            16 
\end{euleroutput}
\begin{eulerprompt}
>A.A.A
\end{eulerprompt}
\begin{euleroutput}
             37            54 
             81           118 
\end{euleroutput}
\begin{eulerprompt}
>power(A,3) //perpangkatan matriks
\end{eulerprompt}
\begin{euleroutput}
             37            54 
             81           118 
\end{euleroutput}
\begin{eulerprompt}
>A/A //pembagian elemen-elemen matriks yang seletak
\end{eulerprompt}
\begin{euleroutput}
              1             1 
              1             1 
\end{euleroutput}
\begin{eulerprompt}
>A/b //pembagian elemen2 A oleh elemen2 b kolom demi kolom (karena b vektor kolom)
\end{eulerprompt}
\begin{euleroutput}
       0.333333      0.666667 
           0.75             1 
\end{euleroutput}
\begin{eulerprompt}
>A\(\backslash\)b // hasilkali invers A dan b, A^(-1)b 
\end{eulerprompt}
\begin{euleroutput}
             -2 
            2.5 
\end{euleroutput}
\begin{eulerprompt}
>inv(A).b
\end{eulerprompt}
\begin{euleroutput}
             -2 
            2.5 
\end{euleroutput}
\begin{eulerprompt}
>A\(\backslash\)A   //A^(-1)A
\end{eulerprompt}
\begin{euleroutput}
              1             0 
              0             1 
\end{euleroutput}
\begin{eulerprompt}
>inv(A).A
\end{eulerprompt}
\begin{euleroutput}
              1             0 
              0             1 
\end{euleroutput}
\begin{eulerprompt}
>A*A //perkalin elemen-elemen matriks seletak
\end{eulerprompt}
\begin{euleroutput}
              1             4 
              9            16 
\end{euleroutput}
\begin{eulercomment}
Ini bukan hasil kali matriks, tetapi perkalian elemen demi elemen. Hal
yang sama berlaku untuk vektor.
\end{eulercomment}
\begin{eulerprompt}
>b^2 // perpangkatan elemen-elemen matriks/vektor
\end{eulerprompt}
\begin{euleroutput}
              9 
             16 
\end{euleroutput}
\begin{eulercomment}
ika salah satu operan adalah vektor atau skalar, maka operan tersebut
akan diperluas dengan cara alami.
\end{eulercomment}
\begin{eulerprompt}
>2*A
\end{eulerprompt}
\begin{euleroutput}
              2             4 
              6             8 
\end{euleroutput}
\begin{eulercomment}
Misalnya, jika operan adalah vektor kolom, elemen-elemennya diterapkan
ke semua baris A.
\end{eulercomment}
\begin{eulerprompt}
>[1,2]*A
\end{eulerprompt}
\begin{euleroutput}
              1             4 
              3             8 
\end{euleroutput}
\begin{eulercomment}
Jika ini adalah vektor baris, vektor ini diterapkan ke semua kolom A.
\end{eulercomment}
\begin{eulerprompt}
>A*[2,3]
\end{eulerprompt}
\begin{euleroutput}
              2             6 
              6            12 
\end{euleroutput}
\begin{eulercomment}
Kita dapat membayangkan perkalian ini seolah-olah vektor baris v telah
diduplikasi untuk membentuk matriks dengan ukuran yang sama dengan A.
\end{eulercomment}
\begin{eulerprompt}
>dup([1,2],2) // dup: menduplikasi/menggandakan vektor [1,2] sebanyak 2 kali (baris)
\end{eulerprompt}
\begin{euleroutput}
              1             2 
              1             2 
\end{euleroutput}
\begin{eulerprompt}
>A*dup([1,2],2) 
\end{eulerprompt}
\begin{euleroutput}
              1             4 
              3             8 
\end{euleroutput}
\begin{eulercomment}
Hal ini juga berlaku untuk dua vektor di mana satu vektor adalah
vektor baris dan yang lainnya adalah vektor kolom. Kami menghitung i*j
untuk i, j dari 1 sampai 5. Caranya adalah dengan mengalikan 1:5
dengan transposenya. Bahasa matriks Euler secara otomatis menghasilkan
sebuah tabel nilai.
\end{eulercomment}
\begin{eulerprompt}
>(1:5)*(1:5)' // hasilkali elemen-elemen vektor baris dan vektor kolom
\end{eulerprompt}
\begin{euleroutput}
              1             2             3             4             5 
              2             4             6             8            10 
              3             6             9            12            15 
              4             8            12            16            20 
              5            10            15            20            25 
\end{euleroutput}
\begin{eulercomment}
Sekali lagi, ingatlah bahwa ini bukan produk matriks!
\end{eulercomment}
\begin{eulerprompt}
>(1:5).(1:5)' // hasilkali vektor baris dan vektor kolom
\end{eulerprompt}
\begin{euleroutput}
  55
\end{euleroutput}
\begin{eulerprompt}
>sum((1:5)*(1:5)) // sama hasilnya
\end{eulerprompt}
\begin{euleroutput}
  55
\end{euleroutput}
\begin{eulercomment}
Bahkan operator seperti \textless{} atau == bekerja dengan cara yang sama.
\end{eulercomment}
\begin{eulerprompt}
>(1:10)<6 // menguji elemen-elemen yang kurang dari 6
\end{eulerprompt}
\begin{euleroutput}
  [1,  1,  1,  1,  1,  0,  0,  0,  0,  0]
\end{euleroutput}
\begin{eulercomment}
Sebagai contoh, kita dapat menghitung jumlah elemen yang memenuhi
kondisi tertentu dengan fungsi sum().
\end{eulercomment}
\begin{eulerprompt}
>sum((1:10)<6) // banyak elemen yang kurang dari 6
\end{eulerprompt}
\begin{euleroutput}
  5
\end{euleroutput}
\begin{eulercomment}
Euler memiliki operator perbandingan, seperti “==”, yang memeriksa
kesetaraan.\\
Kita mendapatkan vektor 0 dan 1, di mana 1 berarti benar.
\end{eulercomment}
\begin{eulerprompt}
>t=(1:10)^2; t==25 //menguji elemen2 t yang sama dengan 25 (hanya ada 1)
\end{eulerprompt}
\begin{euleroutput}
  [0,  0,  0,  0,  1,  0,  0,  0,  0,  0]
\end{euleroutput}
\begin{eulercomment}
Dari vektor seperti itu, “nonzeros” memilih elemen bukan nol.\\
Dalam hal ini, kita mendapatkan indeks semua elemen yang lebih besar
dari 50.
\end{eulercomment}
\begin{eulerprompt}
>nonzeros(t>50) //indeks elemen2 t yang lebih besar daripada 50
\end{eulerprompt}
\begin{euleroutput}
  [8,  9,  10]
\end{euleroutput}
\begin{eulercomment}
Tentu saja, kita dapat menggunakan vektor indeks ini untuk mendapatkan
nilai yang sesuai dalam t.
\end{eulercomment}
\begin{eulerprompt}
>t[nonzeros(t>50)] //elemen2 t yang lebih besar daripada 50
\end{eulerprompt}
\begin{euleroutput}
  [64,  81,  100]
\end{euleroutput}
\begin{eulercomment}
Sebagai contoh, mari kita cari semua kuadrat dari angka 1 sampai 1000,
yaitu 5 modulo 11 dan 3 modulo 13.
\end{eulercomment}
\begin{eulerprompt}
>t=1:1000; nonzeros(mod(t^2,11)==5 && mod(t^2,13)==3)
\end{eulerprompt}
\begin{euleroutput}
  [4,  48,  95,  139,  147,  191,  238,  282,  290,  334,  381,  425,
  433,  477,  524,  568,  576,  620,  667,  711,  719,  763,  810,  854,
  862,  906,  953,  997]
\end{euleroutput}
\begin{eulercomment}
EMT tidak sepenuhnya efektif untuk komputasi bilangan bulat. EMT
menggunakan floating point presisi ganda secara internal. Akan tetapi,
hal ini sering kali sangat berguna.


Kita dapat memeriksa bilangan prima. Mari kita cari tahu, berapa
banyak kuadrat ditambah 1 yang merupakan bilangan prima.
\end{eulercomment}
\begin{eulerprompt}
>t=1:1000; length(nonzeros(isprime(t^2+1)))
\end{eulerprompt}
\begin{euleroutput}
  112
\end{euleroutput}
\begin{eulercomment}
Fungsi nonzeros() hanya bekerja untuk vektor. Untuk matriks, ada
mnonzeros().
\end{eulercomment}
\begin{eulerprompt}
>seed(2); A=random(3,4)
\end{eulerprompt}
\begin{euleroutput}
       0.765761      0.401188      0.406347      0.267829 
        0.13673      0.390567      0.495975      0.952814 
       0.548138      0.006085      0.444255      0.539246 
\end{euleroutput}
\begin{eulercomment}
Ini mengembalikan indeks elemen, yang bukan nol.
\end{eulercomment}
\begin{eulerprompt}
>k=mnonzeros(A<0.4) //indeks elemen2 A yang kurang dari 0,4
\end{eulerprompt}
\begin{euleroutput}
              1             4 
              2             1 
              2             2 
              3             2 
\end{euleroutput}
\begin{eulercomment}
Indeks ini dapat digunakan untuk menetapkan elemen ke suatu nilai.
\end{eulercomment}
\begin{eulerprompt}
>mset(A,k,0) //mengganti elemen2 suatu matriks pada indeks tertentu
\end{eulerprompt}
\begin{euleroutput}
       0.765761      0.401188      0.406347             0 
              0             0      0.495975      0.952814 
       0.548138             0      0.444255      0.539246 
\end{euleroutput}
\begin{eulercomment}
Fungsi mset() juga dapat mengatur elemen-elemen pada indeks ke
entri-entri matriks lain.
\end{eulercomment}
\begin{eulerprompt}
>mset(A,k,-random(size(A)))
\end{eulerprompt}
\begin{euleroutput}
       0.765761      0.401188      0.406347     -0.126917 
      -0.122404     -0.691673      0.495975      0.952814 
       0.548138     -0.483902      0.444255      0.539246 
\end{euleroutput}
\begin{eulercomment}
Dan dimungkinkan untuk mendapatkan elemen-elemen dalam vektor.
\end{eulercomment}
\begin{eulerprompt}
>mget(A,k)
\end{eulerprompt}
\begin{euleroutput}
  [0.267829,  0.13673,  0.390567,  0.006085]
\end{euleroutput}
\begin{eulercomment}
Fungsi lain yang berguna adalah extrema, yang mengembalikan nilai
minimal dan maksimal di setiap baris matriks dan posisinya.
\end{eulercomment}
\begin{eulerprompt}
>ex=extrema(A)
\end{eulerprompt}
\begin{euleroutput}
       0.267829             4      0.765761             1 
        0.13673             1      0.952814             4 
       0.006085             2      0.548138             1 
\end{euleroutput}
\begin{eulercomment}
Kita bisa menggunakan ini untuk mengekstrak nilai maksimal dalam
setiap baris.
\end{eulercomment}
\begin{eulerprompt}
>ex[,3]'
\end{eulerprompt}
\begin{euleroutput}
  [0.765761,  0.952814,  0.548138]
\end{euleroutput}
\begin{eulercomment}
Ini, tentu saja, sama dengan fungsi max().
\end{eulercomment}
\begin{eulerprompt}
>max(A)'
\end{eulerprompt}
\begin{euleroutput}
  [0.765761,  0.952814,  0.548138]
\end{euleroutput}
\begin{eulercomment}
Tetapi dengan mget(), kita dapat mengekstrak indeks dan menggunakan
informasi ini untuk mengekstrak elemen-elemen pada posisi yang sama
dari matriks yang lain.
\end{eulercomment}
\begin{eulerprompt}
>j=(1:rows(A))'|ex[,4], mget(-A,j)
\end{eulerprompt}
\begin{euleroutput}
              1             1 
              2             4 
              3             1 
  [-0.765761,  -0.952814,  -0.548138]
\end{euleroutput}
\eulerheading{Fungsi Matriks Lainnya (Membangun Matriks)}
\begin{eulercomment}
Untuk membangun sebuah matriks, kita dapat menumpuk satu matriks di
atas matriks lainnya. Jika keduanya tidak memiliki jumlah kolom yang
sama, kolom yang lebih pendek akan diisi dengan 0.
\end{eulercomment}
\begin{eulerprompt}
>v=1:3; v_v
\end{eulerprompt}
\begin{euleroutput}
              1             2             3 
              1             2             3 
\end{euleroutput}
\begin{eulercomment}
Demikian juga, kita dapat melampirkan matriks ke matriks lain secara
berdampingan, jika keduanya memiliki jumlah baris yang sama.
\end{eulercomment}
\begin{eulerprompt}
>A=random(3,4); A|v'
\end{eulerprompt}
\begin{euleroutput}
       0.032444     0.0534171      0.595713      0.564454             1 
        0.83916      0.175552      0.396988       0.83514             2 
      0.0257573      0.658585      0.629832      0.770895             3 
\end{euleroutput}
\begin{eulercomment}
Jika keduanya tidak memiliki jumlah baris yang sama, matriks yang
lebih pendek diisi dengan 0.


Ada pengecualian untuk aturan ini. Bilangan real yang dilampirkan pada
sebuah matriks akan digunakan sebagai kolom yang diisi dengan bilangan
real tersebut.
\end{eulercomment}
\begin{eulerprompt}
>A|1
\end{eulerprompt}
\begin{euleroutput}
       0.032444     0.0534171      0.595713      0.564454             1 
        0.83916      0.175552      0.396988       0.83514             1 
      0.0257573      0.658585      0.629832      0.770895             1 
\end{euleroutput}
\begin{eulercomment}
Dimungkinkan untuk membuat matriks vektor baris dan kolom.
\end{eulercomment}
\begin{eulerprompt}
>[v;v]
\end{eulerprompt}
\begin{euleroutput}
              1             2             3 
              1             2             3 
\end{euleroutput}
\begin{eulerprompt}
>[v',v']
\end{eulerprompt}
\begin{euleroutput}
              1             1 
              2             2 
              3             3 
\end{euleroutput}
\begin{eulercomment}
Tujuan utama dari hal ini adalah untuk menginterpretasikan vektor
ekspresi untuk vektor kolom.
\end{eulercomment}
\begin{eulerprompt}
>"[x,x^2]"(v')
\end{eulerprompt}
\begin{euleroutput}
              1             1 
              2             4 
              3             9 
\end{euleroutput}
\begin{eulercomment}
Untuk mendapatkan ukuran A, kita dapat menggunakan fungsi berikut ini.
\end{eulercomment}
\begin{eulerprompt}
>C=zeros(2,4); rows(C), cols(C), size(C), length(C)
\end{eulerprompt}
\begin{euleroutput}
  2
  4
  [2,  4]
  4
\end{euleroutput}
\begin{eulercomment}
Untuk vektor, ada length().
\end{eulercomment}
\begin{eulerprompt}
>length(2:10)
\end{eulerprompt}
\begin{euleroutput}
  9
\end{euleroutput}
\begin{eulercomment}
Ada banyak fungsi lain yang menghasilkan matriks.
\end{eulercomment}
\begin{eulerprompt}
>ones(2,2)
\end{eulerprompt}
\begin{euleroutput}
              1             1 
              1             1 
\end{euleroutput}
\begin{eulercomment}
Ini juga dapat digunakan dengan satu parameter. Untuk mendapatkan
vektor dengan angka selain 1, gunakan yang berikut ini.
\end{eulercomment}
\begin{eulerprompt}
>ones(5)*6
\end{eulerprompt}
\begin{euleroutput}
  [6,  6,  6,  6,  6]
\end{euleroutput}
\begin{eulercomment}
Matriks angka acak juga dapat dibuat dengan acak (distribusi seragam)
atau normal (distribusi Gauß).
\end{eulercomment}
\begin{eulerprompt}
>random(2,2)
\end{eulerprompt}
\begin{euleroutput}
        0.66566      0.831835 
          0.977      0.544258 
\end{euleroutput}
\begin{eulercomment}
Berikut ini adalah fungsi lain yang berguna, yang merestrukturisasi
elemen-elemen matriks menjadi matriks lain.
\end{eulercomment}
\begin{eulerprompt}
>redim(1:9,3,3) // menyusun elemen2 1, 2, 3, ..., 9 ke bentuk matriks 3x3
\end{eulerprompt}
\begin{euleroutput}
              1             2             3 
              4             5             6 
              7             8             9 
\end{euleroutput}
\begin{eulercomment}
Dengan fungsi berikut, kita dapat menggunakan fungsi ini dan fungsi
dup untuk menulis fungsi rep(), yang mengulang sebuah vektor sebanyak
n kali.
\end{eulercomment}
\begin{eulerprompt}
>function rep(v,n) := redim(dup(v,n),1,n*cols(v))
\end{eulerprompt}
\begin{eulercomment}
Mari kita uji.
\end{eulercomment}
\begin{eulerprompt}
>rep(1:3,5)
\end{eulerprompt}
\begin{euleroutput}
  [1,  2,  3,  1,  2,  3,  1,  2,  3,  1,  2,  3,  1,  2,  3]
\end{euleroutput}
\begin{eulercomment}
Fungsi multdup() menduplikasi elemen-elemen sebuah vektor.
\end{eulercomment}
\begin{eulerprompt}
>multdup(1:3,5), multdup(1:3,[2,3,2])
\end{eulerprompt}
\begin{euleroutput}
  [1,  1,  1,  1,  1,  2,  2,  2,  2,  2,  3,  3,  3,  3,  3]
  [1,  1,  2,  2,  2,  3,  3]
\end{euleroutput}
\begin{eulercomment}
Fungsi flipx() dan flipy() membalik urutan baris atau kolom dari
sebuah matriks. Misalnya, fungsi flipx() membalik secara horizontal.
\end{eulercomment}
\begin{eulerprompt}
>flipx(1:5) //membalik elemen2 vektor baris
\end{eulerprompt}
\begin{euleroutput}
  [5,  4,  3,  2,  1]
\end{euleroutput}
\begin{eulercomment}
Untuk rotasi, Euler memiliki rotleft() dan rotright().
\end{eulercomment}
\begin{eulerprompt}
>rotleft(1:5) // memutar elemen2 vektor baris
\end{eulerprompt}
\begin{euleroutput}
  [2,  3,  4,  5,  1]
\end{euleroutput}
\begin{eulercomment}
Fungsi khusus adalah drop(v,i), yang menghapus elemen dengan indeks di
i dari vektor v.
\end{eulercomment}
\begin{eulerprompt}
>drop(10:20,3)
\end{eulerprompt}
\begin{euleroutput}
  [10,  11,  13,  14,  15,  16,  17,  18,  19,  20]
\end{euleroutput}
\begin{eulercomment}
Perhatikan bahwa vektor i dalam drop(v,i) merujuk pada indeks elemen
dalam v, bukan nilai elemen. Jika Anda ingin menghapus elemen, Anda
harus menemukan elemen-elemen tersebut terlebih dahulu. Fungsi
indexof(v,x) dapat digunakan untuk menemukan elemen x dalam vektor
terurut v.
\end{eulercomment}
\begin{eulerprompt}
>v=primes(50), i=indexof(v,10:20), drop(v,i)
\end{eulerprompt}
\begin{euleroutput}
  [2,  3,  5,  7,  11,  13,  17,  19,  23,  29,  31,  37,  41,  43,  47]
  [0,  5,  0,  6,  0,  0,  0,  7,  0,  8,  0]
  [2,  3,  5,  7,  23,  29,  31,  37,  41,  43,  47]
\end{euleroutput}
\begin{eulercomment}
Seperti yang Anda lihat, tidak ada salahnya menyertakan indeks di luar
jangkauan (seperti 0), indeks ganda, atau indeks yang tidak diurutkan.
\end{eulercomment}
\begin{eulerprompt}
>drop(1:10,shuffle([0,0,5,5,7,12,12]))
\end{eulerprompt}
\begin{euleroutput}
  [1,  2,  3,  4,  6,  8,  9,  10]
\end{euleroutput}
\begin{eulercomment}
Ada beberapa fungsi khusus untuk mengatur diagonal atau menghasilkan
matriks diagonal.


Kita mulai dengan matriks identitas.
\end{eulercomment}
\begin{eulerprompt}
>A=id(5) // matriks identitas 5x5
\end{eulerprompt}
\begin{euleroutput}
              1             0             0             0             0 
              0             1             0             0             0 
              0             0             1             0             0 
              0             0             0             1             0 
              0             0             0             0             1 
\end{euleroutput}
\begin{eulercomment}
Kemudian, kami menetapkan diagonal bawah (-1) ke 1:4.
\end{eulercomment}
\begin{eulerprompt}
>setdiag(A,-1,1:4) //mengganti diagonal di bawah diagonal utama
\end{eulerprompt}
\begin{euleroutput}
              1             0             0             0             0 
              1             1             0             0             0 
              0             2             1             0             0 
              0             0             3             1             0 
              0             0             0             4             1 
\end{euleroutput}
\begin{eulercomment}
Perhatikan bahwa kita tidak mengubah matriks A. Kita mendapatkan
sebuah matriks baru sebagai hasil dari setdiag().


Berikut adalah sebuah fungsi yang mengembalikan sebuah matriks
tri-diagonal.
\end{eulercomment}
\begin{eulerprompt}
>function tridiag (n,a,b,c) := setdiag(setdiag(b*id(n),1,c),-1,a); ...
>tridiag(5,1,2,3)
\end{eulerprompt}
\begin{euleroutput}
              2             3             0             0             0 
              1             2             3             0             0 
              0             1             2             3             0 
              0             0             1             2             3 
              0             0             0             1             2 
\end{euleroutput}
\begin{eulercomment}
Diagonal sebuah matriks juga dapat diekstrak dari matriks. Untuk
mendemonstrasikan hal ini, kami merestrukturisasi vektor 1:9 menjadi
matriks 3x3.
\end{eulercomment}
\begin{eulerprompt}
>A=redim(1:9,3,3)
\end{eulerprompt}
\begin{euleroutput}
              1             2             3 
              4             5             6 
              7             8             9 
\end{euleroutput}
\begin{eulercomment}
Sekarang kita bisa mengekstrak diagonal.
\end{eulercomment}
\begin{eulerprompt}
>d=getdiag(A,0)
\end{eulerprompt}
\begin{euleroutput}
  [1,  5,  9]
\end{euleroutput}
\begin{eulercomment}
Contoh: Kita dapat membagi matriks dengan diagonalnya. Bahasa matriks
memperhatikan bahwa vektor kolom d diterapkan ke matriks baris demi
baris.
\end{eulercomment}
\begin{eulerprompt}
>fraction A/d'
\end{eulerprompt}
\begin{euleroutput}
          1         2         3 
        4/5         1       6/5 
        7/9       8/9         1 
\end{euleroutput}
\eulerheading{Vektorisasi}
\begin{eulercomment}
Hampir semua fungsi di Euler juga dapat digunakan untuk input matriks
dan vektor, jika hal ini masuk akal.

Sebagai contoh, fungsi sqrt() menghitung akar kuadrat dari semua
elemen vektor atau matriks.
\end{eulercomment}
\begin{eulerprompt}
>sqrt(1:3)
\end{eulerprompt}
\begin{euleroutput}
  [1,  1.41421,  1.73205]
\end{euleroutput}
\begin{eulercomment}
Jadi, Anda dapat dengan mudah membuat tabel nilai. Ini adalah salah
satu cara untuk memplot sebuah fungsi (alternatif lainnya menggunakan
ekspresi).
\end{eulercomment}
\begin{eulerprompt}
>x=1:0.01:5; y=log(x)/x^2; // terlalu panjang untuk ditampikan
\end{eulerprompt}
\begin{eulercomment}
Dengan ini dan operator titik dua a:delta:b, vektor nilai fungsi dapat
dihasilkan dengan mudah.


Pada contoh berikut, kita membuat vektor nilai t[i] dengan jarak 0.1
dari -1 hingga 1. Kemudian kita membuat vektor nilai dari fungsi \\
lateks:

\end{eulercomment}
\begin{eulerttcomment}
 s = t^3-t
\end{eulerttcomment}
\begin{eulerprompt}
>t=-1:0.1:1; s=t^3-t
\end{eulerprompt}
\begin{euleroutput}
  [0,  0.171,  0.288,  0.357,  0.384,  0.375,  0.336,  0.273,  0.192,
  0.099,  0,  -0.099,  -0.192,  -0.273,  -0.336,  -0.375,  -0.384,
  -0.357,  -0.288,  -0.171,  0]
\end{euleroutput}
\begin{eulercomment}
EMT memperluas operator untuk skalar, vektor, dan matriks dengan cara
yang jelas.

Misalnya, vektor kolom dikalikan vektor baris diperluas menjadi
matriks, jika operator diterapkan. Berikut ini, v' adalah vektor yang
ditransposisikan (vektor kolom).
\end{eulercomment}
\begin{eulerprompt}
>shortest (1:5)*(1:5)'
\end{eulerprompt}
\begin{euleroutput}
       1      2      3      4      5 
       2      4      6      8     10 
       3      6      9     12     15 
       4      8     12     16     20 
       5     10     15     20     25 
\end{euleroutput}
\begin{eulercomment}
Perhatikan, bahwa ini sangat berbeda dengan hasil kali matriks. Hasil
kali matriks dilambangkan dengan sebuah titik “.” dalam EMT.
\end{eulercomment}
\begin{eulerprompt}
>(1:5).(1:5)'
\end{eulerprompt}
\begin{euleroutput}
  55
\end{euleroutput}
\begin{eulercomment}
Secara default, vektor baris dicetak dalam format ringkas.
\end{eulercomment}
\begin{eulerprompt}
>[1,2,3,4]
\end{eulerprompt}
\begin{euleroutput}
  [1,  2,  3,  4]
\end{euleroutput}
\begin{eulercomment}
Untuk matriks, operator khusus . menyatakan perkalian matriks, dan A'
menyatakan transposisi. Matriks 1x1 dapat digunakan seperti halnya
bilangan real.
\end{eulercomment}
\begin{eulerprompt}
>v:=[1,2]; v.v', %^2
\end{eulerprompt}
\begin{euleroutput}
  5
  25
\end{euleroutput}
\begin{eulercomment}
Untuk mentransposisikan matriks, kita menggunakan apostrof.
\end{eulercomment}
\begin{eulerprompt}
>v=1:4; v'
\end{eulerprompt}
\begin{euleroutput}
              1 
              2 
              3 
              4 
\end{euleroutput}
\begin{eulercomment}
Jadi kita dapat menghitung matriks A dikali vektor b.
\end{eulercomment}
\begin{eulerprompt}
>A=[1,2,3,4;5,6,7,8]; A.v'
\end{eulerprompt}
\begin{euleroutput}
             30 
             70 
\end{euleroutput}
\begin{eulercomment}
Perhatikan bahwa v masih merupakan vektor baris. Jadi v'.v berbeda
dengan v.v'.
\end{eulercomment}
\begin{eulerprompt}
>v'.v
\end{eulerprompt}
\begin{euleroutput}
              1             2             3             4 
              2             4             6             8 
              3             6             9            12 
              4             8            12            16 
\end{euleroutput}
\begin{eulercomment}
v.v' menghitung norma v kuadrat untuk vektor baris v. Hasilnya adalah
vektor 1x1, yang berfungsi seperti bilangan real.
\end{eulercomment}
\begin{eulerprompt}
>v.v'
\end{eulerprompt}
\begin{euleroutput}
  30
\end{euleroutput}
\begin{eulercomment}
Ada juga norma fungsi (bersama dengan banyak fungsi Aljabar Linier
lainnya).
\end{eulercomment}
\begin{eulerprompt}
>norm(v)^2
\end{eulerprompt}
\begin{euleroutput}
  30
\end{euleroutput}
\begin{eulercomment}
Operator dan fungsi mematuhi bahasa matriks Euler.\\
Berikut ini adalah ringkasan aturannya.

- Sebuah fungsi yang diterapkan pada vektor atau matriks diterapkan
pada setiap elemen.

- Operator yang beroperasi pada dua matriks dengan ukuran yang sama
diterapkan secara berpasangan pada elemen-elemen matriks.

- Jika dua matriks memiliki dimensi yang berbeda, keduanya diperluas
dengan cara yang masuk akal, sehingga memiliki ukuran yang sama.


Misalnya, nilai skalar dikalikan vektor mengalikan nilai tersebut
dengan setiap elemen vektor. Atau matriks dikali vektor (dengan *,
bukan .) memperluas vektor ke ukuran matriks dengan menduplikasinya.\\
Berikut ini adalah kasus sederhana dengan operator \textasciicircum{}.
\end{eulercomment}
\begin{eulerprompt}
>[1,2,3]^2
\end{eulerprompt}
\begin{euleroutput}
  [1,  4,  9]
\end{euleroutput}
\begin{eulercomment}
Ini adalah kasus yang lebih rumit. Vektor baris dikalikan vektor kolom
memperluas keduanya dengan menduplikasi.
\end{eulercomment}
\begin{eulerprompt}
>v:=[1,2,3]; v*v'
\end{eulerprompt}
\begin{euleroutput}
              1             2             3 
              2             4             6 
              3             6             9 
\end{euleroutput}
\begin{eulercomment}
Perhatikan bahwa hasil kali skalar menggunakan hasil kali matriks,
bukan tanda *!
\end{eulercomment}
\begin{eulerprompt}
>v.v'
\end{eulerprompt}
\begin{euleroutput}
  14
\end{euleroutput}
\begin{eulercomment}
Ada banyak fungsi untuk matriks. Kami memberikan daftar singkat. Anda
harus membaca dokumentasi untuk informasi lebih lanjut mengenai
perintah-perintah ini.\\
\end{eulercomment}
\begin{eulerttcomment}
  
\end{eulerttcomment}
\begin{eulercomment}

sum,prod menghitung jumlah dan hasil kali dari baris-baris\\
\end{eulercomment}
\begin{eulerttcomment}
 cumsum,cumprod melakukan hal yang sama secara kumulatif
 menghitung nilai ekstrem dari setiap baris
 extrema mengembalikan vektor dengan informasi ekstrem
 diag(A,i) mengembalikan diagonal ke-i
 setdiag(A,i,v) menetapkan diagonal ke-i
 id(n) matriks identitas
 det(A) determinan
 charpoly(A) polinomial karakteristik
 eigenvalues(A) nilai eigen
\end{eulerttcomment}
\begin{eulerprompt}
>v*v, sum(v*v), cumsum(v*v)
\end{eulerprompt}
\begin{euleroutput}
  [1,  4,  9]
  14
  [1,  5,  14]
\end{euleroutput}
\begin{eulercomment}
Operator : menghasilkan vektor baris dengan spasi yang sama, opsional
dengan ukuran langkah.
\end{eulercomment}
\begin{eulerprompt}
>1:4, 1:2:10
\end{eulerprompt}
\begin{euleroutput}
  [1,  2,  3,  4]
  [1,  3,  5,  7,  9]
\end{euleroutput}
\begin{eulercomment}
Untuk menggabungkan matriks dan vektor, terdapat operator “\textbar{}” dan “\_”.
\end{eulercomment}
\begin{eulerprompt}
>[1,2,3]|[4,5], [1,2,3]_1
\end{eulerprompt}
\begin{euleroutput}
  [1,  2,  3,  4,  5]
              1             2             3 
              1             1             1 
\end{euleroutput}
\begin{eulercomment}
Elemen-elemen dari sebuah matriks disebut dengan “A[i,j]”.
\end{eulercomment}
\begin{eulerprompt}
>A:=[1,2,3;4,5,6;7,8,9]; A[2,3]
\end{eulerprompt}
\begin{euleroutput}
  6
\end{euleroutput}
\begin{eulercomment}
Untuk vektor baris atau kolom, v[i] adalah elemen ke-i dari vektor
tersebut. Untuk matriks, ini mengembalikan baris ke-i dari matriks.
\end{eulercomment}
\begin{eulerprompt}
>v:=[2,4,6,8]; v[3], A[3]
\end{eulerprompt}
\begin{euleroutput}
  6
  [7,  8,  9]
\end{euleroutput}
\begin{eulercomment}
Indeks juga dapat berupa vektor baris dari indeks. : menunjukkan semua
indeks.
\end{eulercomment}
\begin{eulerprompt}
>v[1:2], A[:,2]
\end{eulerprompt}
\begin{euleroutput}
  [2,  4]
              2 
              5 
              8 
\end{euleroutput}
\begin{eulercomment}
Bentuk singkat untuk : adalah menghilangkan indeks sepenuhnya.
\end{eulercomment}
\begin{eulerprompt}
>A[,2:3]
\end{eulerprompt}
\begin{euleroutput}
              2             3 
              5             6 
              8             9 
\end{euleroutput}
\begin{eulercomment}
Untuk tujuan vektorisasi, elemen-elemen matriks dapat diakses
seolah-olah mereka adalah vektor.
\end{eulercomment}
\begin{eulerprompt}
>A\{4\}
\end{eulerprompt}
\begin{euleroutput}
  4
\end{euleroutput}
\begin{eulercomment}
Sebuah matriks juga dapat diratakan, dengan menggunakan fungsi
redim(). Hal ini diimplementasikan dalam fungsi flatten().
\end{eulercomment}
\begin{eulerprompt}
>redim(A,1,prod(size(A))), flatten(A)
\end{eulerprompt}
\begin{euleroutput}
  [1,  2,  3,  4,  5,  6,  7,  8,  9]
  [1,  2,  3,  4,  5,  6,  7,  8,  9]
\end{euleroutput}
\begin{eulercomment}
Untuk menggunakan matriks untuk tabel, mari kita atur ulang ke format
default, dan menghitung tabel nilai sinus dan kosinus. Perhatikan
bahwa sudut dalam radian secara default.
\end{eulercomment}
\begin{eulerprompt}
>defformat; w=0°:45°:360°; w=w'; deg(w)
\end{eulerprompt}
\begin{euleroutput}
              0 
             45 
             90 
            135 
            180 
            225 
            270 
            315 
            360 
\end{euleroutput}
\begin{eulercomment}
Sekarang kita menambahkan kolom ke matriks.
\end{eulercomment}
\begin{eulerprompt}
>M = deg(w)|w|cos(w)|sin(w)
\end{eulerprompt}
\begin{euleroutput}
              0             0             1             0 
             45      0.785398      0.707107      0.707107 
             90        1.5708             0             1 
            135       2.35619     -0.707107      0.707107 
            180       3.14159            -1             0 
            225       3.92699     -0.707107     -0.707107 
            270       4.71239             0            -1 
            315       5.49779      0.707107     -0.707107 
            360       6.28319             1             0 
\end{euleroutput}
\begin{eulercomment}
Dengan menggunakan bahasa matriks, kita dapat membuat beberapa tabel
dari beberapa fungsi sekaligus.


Pada contoh berikut, kita menghitung t[j]\textasciicircum{}i untuk i dari 1 hingga n.
Kita mendapatkan sebuah matriks, di mana setiap baris adalah tabel t\textasciicircum{}i
untuk satu i. Dengan kata lain, matriks tersebut memiliki
elemen-elemen lateks: a\_\{i, j\} = t\_j\textasciicircum{}i, \textbackslash{}qad 1 \textbackslash{}le j \textbackslash{}le 101, \textbackslash{}qad 1
\textbackslash{}le i \textbackslash{}le n


Sebuah fungsi yang tidak bekerja untuk input vektor harus
“divektorkan”. Hal ini dapat dicapai dengan kata kunci “map” dalam
definisi fungsi. Kemudian fungsi akan dievaluasi untuk setiap elemen
parameter vektor.

Integrasi numerik integrate() hanya bekerja untuk batas interval
skalar. Jadi kita perlu membuat vektornya.
\end{eulercomment}
\begin{eulerprompt}
>function map f(x) := integrate("x^x",1,x)
\end{eulerprompt}
\begin{eulercomment}
Kata kunci “map” membuat vektor fungsi. Fungsi ini sekarang akan
bekerja\\
ntuk vektor angka.
\end{eulercomment}
\begin{eulerprompt}
>f([1:5])
\end{eulerprompt}
\begin{euleroutput}
  [0,  2.05045,  13.7251,  113.336,  1241.03]
\end{euleroutput}
\eulerheading{Sub-Matriks dan Elemen Matriks}
\begin{eulercomment}
Untuk mengakses elemen mmatriks, gunakan notasi kurung. 

\end{eulercomment}
\begin{eulerprompt}
>A=[1,2,3;4,5,6;7,8,9], A[2,2]
\end{eulerprompt}
\begin{euleroutput}
              1             2             3 
              4             5             6 
              7             8             9 
  5
\end{euleroutput}
\begin{eulercomment}
Kita dapat mengakses baris lengkap dari sebuah matriks.
\end{eulercomment}
\begin{eulerprompt}
>A[2]
\end{eulerprompt}
\begin{euleroutput}
  [4,  5,  6]
\end{euleroutput}
\begin{eulercomment}
Dalam kasus vektor baris atau kolom, ini mengembalikan elemen vektor.
\end{eulercomment}
\begin{eulerprompt}
>v=1:3; v[2]
\end{eulerprompt}
\begin{euleroutput}
  2
\end{euleroutput}
\begin{eulercomment}
Untuk memastikan, Anda mendapatkan baris pertama untuk matriks 1xn dan
mxn, tentukan semua kolom menggunakan indeks kedua yang kosong.
\end{eulercomment}
\begin{eulerprompt}
>A[2,]
\end{eulerprompt}
\begin{euleroutput}
  [4,  5,  6]
\end{euleroutput}
\begin{eulercomment}
Jika indeks adalah vektor indeks, Euler akan memgembalikan baris-baris
yang sesuai dari matriks.

Di sini kita menginginkan baris pertama dan kedua dari A.
\end{eulercomment}
\begin{eulerprompt}
>A[[1,2]]
\end{eulerprompt}
\begin{euleroutput}
              1             2             3 
              4             5             6 
\end{euleroutput}
\begin{eulercomment}
Kita bahkan dapat menyusun ulang A menggunakan vektor indeks.
Tepatnya, kita tidak menubah A di sini tetapi menghitung versi susunan
ulng dari A.
\end{eulercomment}
\begin{eulerprompt}
>A[[3,2,1]]
\end{eulerprompt}
\begin{euleroutput}
              7             8             9 
              4             5             6 
              1             2             3 
\end{euleroutput}
\begin{eulercomment}
Trik indeks juga dapat digunakan pada kolom.

Contoh ini memilih semua baris A dan kolom kedua dan ketiga.
\end{eulercomment}
\begin{eulerprompt}
>A[1:3,2:3]
\end{eulerprompt}
\begin{euleroutput}
              2             3 
              5             6 
              8             9 
\end{euleroutput}
\begin{eulercomment}
Untuk singkatan ":" menunjukkan semua indeks baris atau kolom.
\end{eulercomment}
\begin{eulerprompt}
>A[:,3]
\end{eulerprompt}
\begin{euleroutput}
              3 
              6 
              9 
\end{euleroutput}
\begin{eulercomment}
Sebagai artenatif, biarkan indeks pertama kosong.
\end{eulercomment}
\begin{eulerprompt}
>A[,2:3]
\end{eulerprompt}
\begin{euleroutput}
              2             3 
              5             6 
              8             9 
\end{euleroutput}
\begin{eulercomment}
Kita juga bisa mendapatkan baris terakhir A.
\end{eulercomment}
\begin{eulerprompt}
>A[-1]
\end{eulerprompt}
\begin{euleroutput}
  [7,  8,  9]
\end{euleroutput}
\begin{eulercomment}
Sekarang mari kita ubah elemen-elemen dari A dengan memberikan sebuah
submatriks dari A kesuatu nilai.

Hal ini sebenarnya mengubah matriks A yang tersimpan.
\end{eulercomment}
\begin{eulerprompt}
>A[1,1]=4
\end{eulerprompt}
\begin{euleroutput}
              4             2             3 
              4             5             6 
              7             8             9 
\end{euleroutput}
\begin{eulercomment}
Kita juga dapat menetapkan nilai pada deretan A.
\end{eulercomment}
\begin{eulerprompt}
>A[1]=[-1,-1,-1]
\end{eulerprompt}
\begin{euleroutput}
             -1            -1            -1 
              4             5             6 
              7             8             9 
\end{euleroutput}
\begin{eulercomment}
Kami bahkan dapat menetapkan ke sub-matriks juka memiliki ukuran yang
tepat.
\end{eulercomment}
\begin{eulerprompt}
>A[1:2,1:2]=[5,6;7,8]
\end{eulerprompt}
\begin{euleroutput}
              5             6            -1 
              7             8             6 
              7             8             9 
\end{euleroutput}
\begin{eulercomment}
Selain itu, bebrapa jalan pintas diperbolehkan.
\end{eulercomment}
\begin{eulerprompt}
>A[1:2,1:2]=0
\end{eulerprompt}
\begin{euleroutput}
              0             0            -1 
              0             0             6 
              7             8             9 
\end{euleroutput}
\begin{eulercomment}
Peringatan : Indeks diluar batas akan mengembalikan matriks kosong,
atau pesan kesalahan, tergantung pada pengaturan sistem. Standarnya
adalah pesan kesalahan. Namun, ingatlah bahwa indeks negatif dapat
diginakan untuk mengakses elemen-elemen matriks yang dihitung dari
akhir.
\end{eulercomment}
\begin{eulerprompt}
>A[4]
\end{eulerprompt}
\begin{euleroutput}
  Row index 4 out of bounds!
  Error in:
  A[4] ...
      ^
\end{euleroutput}
\eulerheading{Menyortir dan Mengacak}
\begin{eulercomment}
Fungsi mengurutkan () mengurutkan vektor baris.
\end{eulercomment}
\begin{eulerprompt}
>sort([5,6,4,8,1,9])
\end{eulerprompt}
\begin{euleroutput}
  [1,  4,  5,  6,  8,  9]
\end{euleroutput}
\begin{eulercomment}
Sering kali diperlukan untuk mengetahui indeks vektor yang diurutkan
dalam vektor aslinya. Hal ini dapat digunakan untuk menyusun ulang
vektor lain dengan cara yang sama.

Mari kita acak sebuah vektor.
\end{eulercomment}
\begin{eulerprompt}
>v=shuffle(1:10)
\end{eulerprompt}
\begin{euleroutput}
  [4,  5,  10,  6,  8,  9,  1,  7,  2,  3]
\end{euleroutput}
\begin{eulercomment}
Indeks berisi urutan v yang tepat.
\end{eulercomment}
\begin{eulerprompt}
>\{vs,ind\}=sort(v); v[ind]
\end{eulerprompt}
\begin{euleroutput}
  [1,  2,  3,  4,  5,  6,  7,  8,  9,  10]
\end{euleroutput}
\begin{eulercomment}
Hal ini juga berlaku untuk vektor string.
\end{eulercomment}
\begin{eulerprompt}
>s=["a","d","e","a","aa","e"]
\end{eulerprompt}
\begin{euleroutput}
  a
  d
  e
  a
  aa
  e
\end{euleroutput}
\begin{eulerprompt}
>\{ss,ind\}=sort(s); ss
\end{eulerprompt}
\begin{euleroutput}
  a
  a
  aa
  d
  e
  e
\end{euleroutput}
\begin{eulercomment}
Seperti yang anda lihat, posisi entri ganda agak acak.
\end{eulercomment}
\begin{eulerprompt}
>ind
\end{eulerprompt}
\begin{euleroutput}
  [4,  1,  5,  2,  6,  3]
\end{euleroutput}
\begin{eulercomment}
Fungsi unique mengembalikan daftar terurut dari elemen unik vektor.
\end{eulercomment}
\begin{eulerprompt}
>intrandom(1,10,10), unique(%)
\end{eulerprompt}
\begin{euleroutput}
  [4,  4,  9,  2,  6,  5,  10,  6,  5,  1]
  [1,  2,  4,  5,  6,  9,  10]
\end{euleroutput}
\begin{eulercomment}
Hal ini juga berlaku untuk vektor string.
\end{eulercomment}
\begin{eulerprompt}
>unique(s)
\end{eulerprompt}
\begin{euleroutput}
  a
  aa
  d
  e
\end{euleroutput}
\eulerheading{Aljabar Linier}
\begin{eulercomment}
EMT memiiki banyak fungsi untuk menyelesaikan sistem linier, sistem
jarang, atau masalah regresi. 

Untuk sistem linier Ax=b, anda dapat menggunakan algoritma Gauss,
matriks invers, atau kecocokan linier. Operator A/b menggunakan versi
algoritma Gauss.
\end{eulercomment}
\begin{eulerprompt}
>A=[1,2;3,4]; b=[5;6]; A\(\backslash\)b
\end{eulerprompt}
\begin{euleroutput}
             -4 
            4.5 
\end{euleroutput}
\begin{eulercomment}
Sebagai contoh lain, kita membuat matriks 200x200 dan jumlah barisnya.
Kemudian kita selesaikan Ax=b dengan menggunakan matriks kebalikannya.
Kita mengukur kesalahan sebagai deviasi maksimal dari semua elemen
dari 1, yang tentu saja merupakan solusi yang benar. 
\end{eulercomment}
\begin{eulerprompt}
>A=normal(200,200); b=sum(A); longest totalmax(abs(inv(A).b-1))
\end{eulerprompt}
\begin{euleroutput}
    8.790745908981989e-13 
\end{euleroutput}
\begin{eulercomment}
Jika sistem tidak memiliki solusi, pencocokan linier meminimalkan
norma kesalahan Ax-b.
\end{eulercomment}
\begin{eulerprompt}
>A=[1,2,3;4,5,6;7,8,9]
\end{eulerprompt}
\begin{euleroutput}
              1             2             3 
              4             5             6 
              7             8             9 
\end{euleroutput}
\begin{eulercomment}
Determinan dari matris ini adalah 0.
\end{eulercomment}
\begin{eulerprompt}
>det(A)
\end{eulerprompt}
\begin{euleroutput}
  0
\end{euleroutput}
\eulerheading{Matriks Simbolik}
\begin{eulercomment}
Maxima memiliki matriks simbolik. Tentu saja, Maxima dapat digunakan
untuk masalah aljabar linier sederhana. Kita bisa mendefinisikan
matriks untuk Euler dan Maxima dengan \&:=, lalu menggunakannya dalam
ekspresi simbolik. Bentuk [...] yang biasa untuk mendefinisikan
matriks dapat digunakan dalam Euler untuk mendefinisikan matriks
simbolik.
\end{eulercomment}
\begin{eulerprompt}
>A &= [a,1,1;1,a,1;1,1,a]; $A
\end{eulerprompt}
\begin{eulerformula}
\[
\begin{pmatrix}a & 1 & 1 \\ 1 & a & 1 \\ 1 & 1 & a \\ \end{pmatrix}
\]
\end{eulerformula}
\begin{eulerprompt}
>$&det(A), $&factor(%)
\end{eulerprompt}
\begin{eulerformula}
\[
a\,\left(a^2-1\right)-2\,a+2
\]
\end{eulerformula}
\begin{eulerformula}
\[
\left(a-1\right)^2\,\left(a+2\right)
\]
\end{eulerformula}
\begin{eulerprompt}
>$&invert(A) with a=0
\end{eulerprompt}
\begin{eulerformula}
\[
\begin{pmatrix}-\frac{1}{2} & \frac{1}{2} & \frac{1}{2} \\ \frac{1
 }{2} & -\frac{1}{2} & \frac{1}{2} \\ \frac{1}{2} & \frac{1}{2} & -
 \frac{1}{2} \\ \end{pmatrix}
\]
\end{eulerformula}
\begin{eulerprompt}
>A &= [1,a;b,2]; $A
\end{eulerprompt}
\begin{eulerformula}
\[
\begin{pmatrix}1 & a \\ b & 2 \\ \end{pmatrix}
\]
\end{eulerformula}
\begin{eulercomment}
Seperti semua variabel simbolik, matriks ini dapat digunakan dalam
ekspresi simbolik lainnya.
\end{eulercomment}
\begin{eulerprompt}
>$&det(A-x*ident(2))
\end{eulerprompt}
\begin{eulerformula}
\[
\left(1-x\right)\,\left(2-x\right)-a\,b
\]
\end{eulerformula}
\begin{eulerprompt}
>$&solve(%,x)
\end{eulerprompt}
\begin{eulerformula}
\[
\left[ x=\frac{3-\sqrt{4\,a\,b+1}}{2} , x=\frac{\sqrt{4\,a\,b+1}+3
 }{2} \right] 
\]
\end{eulerformula}
\begin{eulercomment}
Nilai eigen juga dapat dihitung secara otomatis. Hasilnya adalah
sebuah vektor dengan dua vektor nilai eigen dan kelipatannya.
\end{eulercomment}
\begin{eulerprompt}
>$&eigenvalues([a,1;1,a])
\end{eulerprompt}
\begin{eulerformula}
\[
\left[ \left[ a-1 , a+1 \right]  , \left[ 1 , 1 \right]  \right] 
\]
\end{eulerformula}
\begin{eulercomment}
Untuk mengekstrak vektor eigen tertentu, diperlukan pengindeksan yang
cermat.
\end{eulercomment}
\begin{eulerprompt}
>$&eigenvectors([a,1;1,a]), &%[2][1][1]
\end{eulerprompt}
\begin{eulerformula}
\[
\left[ \left[ \left[ a-1 , a+1 \right]  , \left[ 1 , 1 \right] 
  \right]  , \left[ \left[ \left[ 1 , -1 \right]  \right]  , \left[ 
 \left[ 1 , 1 \right]  \right]  \right]  \right] 
\]
\end{eulerformula}
\begin{euleroutput}
  
                                 [1, - 1]
  
\end{euleroutput}
\begin{eulercomment}
Matriks simbolik dapat dievaluasi dalam Euler secara numerik seperti
halnya ekspresi simbolik lainnya.
\end{eulercomment}
\begin{eulerprompt}
>A(a=4,b=5)
\end{eulerprompt}
\begin{euleroutput}
              1             4 
              5             2 
\end{euleroutput}
\begin{eulercomment}
Dalam ekspresi simbolik, gunakan dengan.
\end{eulercomment}
\begin{eulerprompt}
>$&A with [a=4,b=5]
\end{eulerprompt}
\begin{eulerformula}
\[
\begin{pmatrix}1 & 4 \\ 5 & 2 \\ \end{pmatrix}
\]
\end{eulerformula}
\begin{eulercomment}
Akses ke baris matriks simbolik bekerja seperti halnya matriks
numerik.
\end{eulercomment}
\begin{eulerprompt}
>$&A[1]
\end{eulerprompt}
\begin{eulerformula}
\[
\left[ 1 , a \right] 
\]
\end{eulerformula}
\begin{eulercomment}
Ekspresi simbolik dapat berisi sebuah penugasan. Dana itu mengubah
matriks A.
\end{eulercomment}
\begin{eulerprompt}
>&A[1,1]:=t+1; $&A
\end{eulerprompt}
\begin{eulerformula}
\[
\begin{pmatrix}t+1 & a \\ b & 2 \\ \end{pmatrix}
\]
\end{eulerformula}
\begin{eulercomment}
Terdapat fungsi-fungsi simbolik dalam Maxima untuk membuat vektor dan
matriks. Untuk hal ini, lihat dokumentasi Maxima atau tutorial tentang
Maxima di EMT.
\end{eulercomment}
\begin{eulerprompt}
>v &= makelist(1/(i+j),i,1,3); $v
\end{eulerprompt}
\begin{eulerformula}
\[
\left[ \frac{1}{j+1} , \frac{1}{j+2} , \frac{1}{j+3} \right] 
\]
\end{eulerformula}
\begin{eulerttcomment}
 
\end{eulerttcomment}
\begin{eulerprompt}
>B &:= [1,2;3,4]; $B, $&invert(B)
\end{eulerprompt}
\begin{eulerformula}
\[
\begin{pmatrix}1 & 2 \\ 3 & 4 \\ \end{pmatrix}
\]
\end{eulerformula}
\begin{eulerformula}
\[
\begin{pmatrix}-2 & 1 \\ \frac{3}{2} & -\frac{1}{2} \\ 
 \end{pmatrix}
\]
\end{eulerformula}
\begin{eulercomment}
Hasilnya dapat dievaluasi secara numerik dalam Euler. Untuk informasi
lebih lanjut tentang Maxima, lihat pengantar Maxima.
\end{eulercomment}
\begin{eulerprompt}
>$&invert(B)()
\end{eulerprompt}
\begin{euleroutput}
             -2             1 
            1.5          -0.5 
\end{euleroutput}
\begin{eulercomment}
Euler juga memiliki fungsi yang kuat xinv(), yang membuat usaha yang
lebih besar dan mendapatkan hasil yang tepat.

Perhatikan, bahwa dengan \&:= matriks B telah didefinisikan sebagai
simbolik dalam ekspresi numerik. Jadi kita dapat menggunakannya
disini.
\end{eulercomment}
\begin{eulerprompt}
>longest B.xinv(B)
\end{eulerprompt}
\begin{euleroutput}
                        1                       0 
                        0                       1 
\end{euleroutput}
\begin{eulercomment}
Misalnya, nilai eigen dari A dapat dihitung secara numerik.
\end{eulercomment}
\begin{eulerprompt}
>A=[1,2,3;4,5,6;7,8,9]; real(eigenvalues(A))
\end{eulerprompt}
\begin{euleroutput}
  [16.1168,  -1.11684,  0]
\end{euleroutput}
\begin{eulercomment}
Atau secara simbolik. Lihat tutorial tentang Maxima untuk mengetahui
detailnya.
\end{eulercomment}
\begin{eulerprompt}
>$&eigenvalues(@A)
\end{eulerprompt}
\begin{eulerformula}
\[
\left[ \left[ \frac{15-3\,\sqrt{33}}{2} , \frac{3\,\sqrt{33}+15}{2}
  , 0 \right]  , \left[ 1 , 1 , 1 \right]  \right] 
\]
\end{eulerformula}
\eulerheading{Nilai Numerik dalam Ekspresi simbolik}
\begin{eulercomment}
Ekspresi simbolik hanyalah sebuah string yang berisi ekspresi. Jika
kita ingin mendefinisikan nilai untuk ekspresi simbolik dan ekspresi
numerik, kita haris menggunkan "\&:=".
\end{eulercomment}
\begin{eulerprompt}
>A &:= [1,pi;4,5]
\end{eulerprompt}
\begin{euleroutput}
              1       3.14159 
              4             5 
\end{euleroutput}
\begin{eulercomment}
Masih ada perbedaan antara bentuk numerik dan bentuk simbolik. Ketika
mentransfer matriks ke bentuk simbolik, perkiraan pecahan untuk
bilangan real akan digunakan.
\end{eulercomment}
\begin{eulerprompt}
>$&A
\end{eulerprompt}
\begin{eulerformula}
\[
\begin{pmatrix}1 & \frac{1146408}{364913} \\ 4 & 5 \\ \end{pmatrix}
\]
\end{eulerformula}
\begin{eulercomment}
Untuk menghindari hal ini, ada fungsi "mxmset(variabel)".
\end{eulercomment}
\begin{eulerprompt}
>mxmset(A); $&A
\end{eulerprompt}
\begin{eulerformula}
\[
\begin{pmatrix}1 & 3.141592653589793 \\ 4 & 5 \\ \end{pmatrix}
\]
\end{eulerformula}
\begin{eulercomment}
Maxima juga dapat menghitung dengan angka floating point, dan bahkan
dengan angka mengambang yang besar dengan 32 digit. Namun, evaluasinya
jauh lebih lambat.
\end{eulercomment}
\begin{eulerprompt}
>$&bfloat(sqrt(2)), $&float(sqrt(2))
\end{eulerprompt}
\begin{eulerformula}
\[
1.4142135623730950488016887242097_B \times 10^{0}
\]
\end{eulerformula}
\begin{eulerformula}
\[
1.414213562373095
\]
\end{eulerformula}
\begin{eulercomment}
Ketepatan angka floating point yang besar dapat diubah.
\end{eulercomment}
\begin{eulerprompt}
>&fpprec:=100; &bfloat(pi)
\end{eulerprompt}
\begin{euleroutput}
  
          3.14159265358979323846264338327950288419716939937510582097494\(\backslash\)
  4592307816406286208998628034825342117068b0
  
\end{euleroutput}
\begin{eulercomment}
Variabel numerik dapat digunakan dalam ekspresi simbolik apapun dengan
menggunakan "@var".

Perhatikan bahwa hal ini hanya diperlukan, jika variabel telh
didefinisikan dengan ":=" atau "=" sebuah variabel numerik.
\end{eulercomment}
\begin{eulerprompt}
>B:=[1,pi;3,4]; $&det(@B)
\end{eulerprompt}
\begin{eulerformula}
\[
-5.424777960769379
\]
\end{eulerformula}
\begin{eulercomment}
\begin{eulercomment}
\eulerheading{Demo - Suku Bunga}
\begin{eulercomment}
Di bawah ini, kami menggunakan Eular Mtah Toolbox (EMT) untuk
menghitung suku bunga. Kami melakukan secara numerik dan simbolik
untuk menunjukkan kepada anda bagaimana euler dapat digunakan untuk
memecahkan masalah dikehidupan nyata.

Asumsikan anda memiliki modal awal sebesar modal awal sebesar
5000(katakan dalama dolar).
\end{eulercomment}
\begin{eulerprompt}
>K=5000
\end{eulerprompt}
\begin{euleroutput}
  5000
\end{euleroutput}
\begin{eulercomment}
Sekarang kita asumsikan suku bunga 3\% per tahun. Maka kita tambahkan
satu suku bunga sederhana dan hitung hasilnya. 
\end{eulercomment}
\begin{eulerprompt}
>K*1.03
\end{eulerprompt}
\begin{euleroutput}
  5150
\end{euleroutput}
\begin{eulercomment}
Euler juga akan memahami sintaks berikut ini.
\end{eulercomment}
\begin{eulerprompt}
>K+K*3%
\end{eulerprompt}
\begin{euleroutput}
  5150
\end{euleroutput}
\begin{eulercomment}
Tetapi akan lebih mudah untuk menggunakan faktor.
\end{eulercomment}
\begin{eulerprompt}
>q=1+3%, K*q
\end{eulerprompt}
\begin{euleroutput}
  1.03
  5150
\end{euleroutput}
\begin{eulercomment}
Untuk 10 tahun, kita cukup mengalihkan faktor-faktor tersebut dan
mendapatkan nilai akhir dengan suku bunga majemuk. 
\end{eulercomment}
\begin{eulerprompt}
>K*q^10
\end{eulerprompt}
\begin{euleroutput}
  6719.58189672
\end{euleroutput}
\begin{eulercomment}
Untuk tujuan kita, kita bisa menetapkan format ke 2 digit setelah
titik desimal.
\end{eulercomment}
\begin{eulerprompt}
>format(12,2); K*q^10
\end{eulerprompt}
\begin{euleroutput}
      6719.58 
\end{euleroutput}
\begin{eulercomment}
Mari kita cetak angka yang dibulatkan menjadi 2 digit dalam kalimat
lengkap.
\end{eulercomment}
\begin{eulerprompt}
>K=5000
\end{eulerprompt}
\begin{euleroutput}
      5000.00 
\end{euleroutput}
\begin{eulerprompt}
>"Starting from " + K + "$ you get " + round(K*q^10,2) + "$."
\end{eulerprompt}
\begin{euleroutput}
  Starting from 5000$ you get 6719.58$.
\end{euleroutput}
\begin{eulercomment}
Bagaimana jika kita ingin mengetahui hasil antara dari tahun ke-1
hingga tahun ke-9? Untuk hal ini, bahasa matriks euler sangat
membantu. Anda tidak perlu menulis perulangan, tetapi cukup
memasukkan.
\end{eulercomment}
\begin{eulerprompt}
>K*q^(0:10)
\end{eulerprompt}
\begin{euleroutput}
  Real 1 x 11 matrix
  
      5000.00     5150.00     5304.50     5463.64     ...
\end{euleroutput}
\begin{eulerprompt}
>q=1+3%
\end{eulerprompt}
\begin{euleroutput}
         1.03 
\end{euleroutput}
\begin{eulercomment}
Bagaimana keajaiban ini bekerja? Pertama, ekspresi 0:10 mengembalikan
sebuah vektor bilangan bulat
\end{eulercomment}
\begin{eulerprompt}
>short 0:10
\end{eulerprompt}
\begin{euleroutput}
  [0,  1,  2,  3,  4,  5,  6,  7,  8,  9,  10]
\end{euleroutput}
\begin{eulercomment}
Kemudian semua operator dan fungsi dalam Euler dapat diterapkan pada
vektor elemen demi elemen. Jadi
\end{eulercomment}
\begin{eulerprompt}
>short q^(0:10)
\end{eulerprompt}
\begin{euleroutput}
  [1,  1.03,  1.0609,  1.0927,  1.1255,  1.1593,  1.1941,  1.2299,
  1.2668,  1.3048,  1.3439]
\end{euleroutput}
\begin{eulercomment}
adalah vektor faktor q\textasciicircum{}0 hingga q\textasciicircum{}10. Ini dikalikan dengan K, dan kita
mendapatkan vektor nilai.
\end{eulercomment}
\begin{eulerprompt}
>K=5000; VK=K*q^(0:10)
\end{eulerprompt}
\begin{euleroutput}
  Real 1 x 11 matrix
  
      5000.00     5150.00     5304.50     5463.64     ...
\end{euleroutput}
\begin{eulercomment}
Tentu saja, cara yang realistis untuk menghitung suku bunga ini adalah
dengan membulatkan ke sen terdekat setelah setiap tahun. Mari kita
tambahkan fungsi untuk ini.
\end{eulercomment}
\begin{eulerprompt}
>function oneyear (K) := round(K*q,2)
\end{eulerprompt}
\begin{eulercomment}
Mari kita bandingkan kedua hasil tersebut, dengan dan tanpa
pembulatan.
\end{eulercomment}
\begin{eulerprompt}
>longest oneyear(1234.57), longest 1234.57*q
\end{eulerprompt}
\begin{euleroutput}
                  1271.61 
                1271.6071 
\end{euleroutput}
\begin{eulercomment}
Sekarang tidak ada rumus sederhana untuk tahun ke-n, dan kita harus
mengulang selama bertahun-tahun. Euler menyediakan banyak solusi untuk
ini.

Cara termudah adalah iterasi fungsi, yang mengulang fungsi yang
diberikan beberapa kali.
\end{eulercomment}
\begin{eulerprompt}
>VKr=iterate("oneyear",5000,10)
\end{eulerprompt}
\begin{euleroutput}
  Real 1 x 11 matrix
  
      5000.00     5150.00     5304.50     5463.64     ...
\end{euleroutput}
\begin{eulercomment}
Kita bisa mencetaknya dengan cara yang bersahabat, menggunakan format
kita dengan angka desimal yang tetap.
\end{eulercomment}
\begin{eulerprompt}
>VKr'
\end{eulerprompt}
\begin{euleroutput}
      5000.00 
      5150.00 
      5304.50 
      5463.64 
      5627.55 
      5796.38 
      5970.27 
      6149.38 
      6333.86 
      6523.88 
      6719.60 
\end{euleroutput}
\begin{eulercomment}
Untuk mendapatkan elemen tertentu dari vektor, kita menggunakan indeks
dalam tanda kurung siku.
\end{eulercomment}
\begin{eulerprompt}
>VKr[2], VKr[1:3]
\end{eulerprompt}
\begin{euleroutput}
      5150.00 
      5000.00     5150.00     5304.50 
\end{euleroutput}
\begin{eulercomment}
Yang mengejutkan, kita juga dapat menggunakan vektor indeks. Ingatlah
bahwa 1:3 menghasilkan vektor [1,2,3].

Mari kita bandingkan elemen terakhir dari nilai yang dibulatkan dengan
nilai penuh.
\end{eulercomment}
\begin{eulerprompt}
>VKr[-1], VK[-1]
\end{eulerprompt}
\begin{euleroutput}
      6719.60 
      6719.58 
\end{euleroutput}
\begin{eulercomment}
Perbedaannya sangat kecil.

\begin{eulercomment}
\eulerheading{Menyelesaikan Persamaan}
\begin{eulercomment}
Sekarang kita ambil fungsi yang lebih maju, yang menambahkan tingkat
uang tertentu setiap tahun.
\end{eulercomment}
\begin{eulerprompt}
>function onepay (K) := K*q+R
\end{eulerprompt}
\begin{eulercomment}
Kita tidak perlu menentukan q atau R untuk definisi fungsi. Hanya jika
kita menjalankan perintah, kita harus mendefinisikan nilai-nilai ini.
Kami memilih R = 200.
\end{eulercomment}
\begin{eulerprompt}
>R=200; iterate("onepay",5000,10)
\end{eulerprompt}
\begin{euleroutput}
  Real 1 x 11 matrix
  
      5000.00     5350.00     5710.50     6081.82     ...
\end{euleroutput}
\begin{eulercomment}
Bagaimana jika kita menghapus jumlah yang sama setiap tahun?
\end{eulercomment}
\begin{eulerprompt}
>R=-200; iterate("onepay",5000,10)
\end{eulerprompt}
\begin{euleroutput}
  Real 1 x 11 matrix
  
      5000.00     4950.00     4898.50     4845.45     ...
\end{euleroutput}
\begin{eulercomment}
Kami melihat bahwa uangnya berkurang. Jelas, jika kita hanya
mendapatkan 150 bunga di tahun pertama, tetapi menghapus 200, kita
kehilangan uang setiap tahun.

Bagaimana kita dapat menentukan berapa tahun uang itu akan bertahan?
Kita harus menulis perulangan untuk ini. Cara termudah adalah dengan
melakukan perulangan yang cukup lama.
\end{eulercomment}
\begin{eulerprompt}
>VKR=iterate("onepay",5000,50)
\end{eulerprompt}
\begin{euleroutput}
  Real 1 x 51 matrix
  
      5000.00     4950.00     4898.50     4845.45     ...
\end{euleroutput}
\begin{eulercomment}
Dengan menggunakan bahasa matriks, kita dapat menentukan nilai negatif
pertama dengan cara berikut.
\end{eulercomment}
\begin{eulerprompt}
>min(nonzeros(VKR<0))
\end{eulerprompt}
\begin{euleroutput}
        48.00 
\end{euleroutput}
\begin{eulercomment}
Alasannya adalah karena nonzeros(VKR\textless{}0) mengembalikan vektor dengan
indeks i, di mana VKR[i]\textless{}0, dan min menghitung indeks minimal.

Karena vektor selalu dimulai dengan indeks 1, maka jawabannya adalah
47 tahun.

Fungsi iterate() memiliki satu trik lagi. Fungsi ini dapat menerima
kondisi akhir sebagai argumen. Kemudian fungsi ini akan mengembalikan
nilai dan jumlah iterasi.
\end{eulercomment}
\begin{eulerprompt}
>\{x,n\}=iterate("onepay",5000,till="x<0"); x, n,
\end{eulerprompt}
\begin{euleroutput}
       -19.83 
        47.00 
\end{euleroutput}
\begin{eulercomment}
Mari kita coba menjawab pertanyaan yang lebih ambigu. Anggaplah kita
tahu bahwa nilainya adalah 0 setelah 50 tahun. Berapakah tingkat suku
bunganya?


Ini adalah pertanyaan yang hanya bisa dijawab secara numerik. Di bawah
ini, kami akan menurunkan rumus yang diperlukan. Kemudian Anda akan
melihat bahwa tidak ada rumus yang mudah untuk suku bunga. Namun untuk
saat ini, kita akan mencari solusi numerik.


Langkah pertama adalah mendefinisikan sebuah fungsi yang melakukan
iterasi sebanyak n kali. Kita tambahkan semua parameter ke fungsi ini.
\end{eulercomment}
\begin{eulerprompt}
>function f(K,R,P,n) := iterate("x*(1+P/100)+R",K,n;P,R)[-1]
\end{eulerprompt}
\begin{eulercomment}
Perulangannya sama seperti di atas

\end{eulercomment}
\begin{eulerformula}
\[
x_{n+1} = x_n \cdot \left(1+ \frac{P}{100}\right) + R
\]
\end{eulerformula}
\begin{eulercomment}
Tetapi kita tidak lagi menggunakan nilai global R dalam ekspresi kita.
Fungsi-fungsi seperti iterate() memiliki trik khusus dalam Euler. Anda
bisa mengoper nilai variabel dalam ekspresi sebagai parameter titik
koma. Dalam hal ini P dan R.

Selain itu, kita hanya tertarik pada nilai terakhir. Jadi kita
mengambil indeks [-1].

Mari kita coba sebuah tes.
\end{eulercomment}
\begin{eulerprompt}
>f(5000,-200,3,47)
\end{eulerprompt}
\begin{euleroutput}
       -19.83 
\end{euleroutput}
\begin{eulercomment}
Sekarang kita bisa menyelesaikan masalah kita.
\end{eulercomment}
\begin{eulerprompt}
>solve("f(5000,-200,x,50)",3)
\end{eulerprompt}
\begin{euleroutput}
         3.15 
\end{euleroutput}
\begin{eulercomment}
Rutin penyelesaian menyelesaikan ekspresi = 0 untuk variabel x.
Jawabannya adalah 3,15\% per tahun. Kita mengambil nilai awal 3\% untuk
algoritma ini. Fungsi solve() selalu membutuhkan nilai awal.

Kita dapat menggunakan fungsi yang sama untuk menyelesaikan pertanyaan
berikut: Berapa banyak yang dapat kita hapus per tahun sehingga modal
awal habis setelah 20 tahun dengan asumsi tingkat bunga 3\% per tahun.
\end{eulercomment}
\begin{eulerprompt}
>solve("f(5000,x,3,20)",-200)
\end{eulerprompt}
\begin{euleroutput}
      -336.08 
\end{euleroutput}
\begin{eulercomment}
Perhatikan bahwa Anda tidak dapat menyelesaikan jumlah tahun, karena
fungsi kita mengasumsikan n sebagai nilai bilangan bulat.

\end{eulercomment}
\eulersubheading{Solusi Simbolik untuk Masalah Suku Bunga}
\begin{eulercomment}
Kita dapat menggunakan bagian simbolik dari Euler untuk mempelajari
masalah ini. Pertama, kita mendefinisikan fungsi onepay() secara
simbolik.
\end{eulercomment}
\begin{eulerprompt}
>function op(K) &= K*q+R; $&op(K)
\end{eulerprompt}
\begin{eulerformula}
\[
R+q\,K
\]
\end{eulerformula}
\begin{eulercomment}
Sekarang kita bisa mengulangi hal ini.
\end{eulercomment}
\begin{eulerprompt}
>$&op(op(op(op(K)))) 
\end{eulerprompt}
\begin{eulerformula}
\[
q\,\left(q\,\left(q\,\left(R+q\,K\right)+R\right)+R\right)+R
\]
\end{eulerformula}
\begin{eulerprompt}
>$&expand(%)
\end{eulerprompt}
\begin{eulerformula}
\[
q^3\,R+q^2\,R+q\,R+R+q^4\,K
\]
\end{eulerformula}
\begin{eulercomment}
Kita melihat sebuah pola. Setelah n periode, kita memiliki

\end{eulercomment}
\begin{eulerformula}
\[
K_n = q^n K + R (1+q+\ldots+q^{n-1}) = q^n K + \frac{q^n-1}{q-1} R
\]
\end{eulerformula}
\begin{eulercomment}
Rumus tersebut adalah rumus untuk jumlah geometris, yang dikenal
dengan Maxima.
\end{eulercomment}
\begin{eulerprompt}
>&sum(q^k,k,0,n-1); $& % = ev(%,simpsum)
\end{eulerprompt}
\begin{eulerformula}
\[
\sum_{k=0}^{n-1}{q^{k}}=\frac{q^{n}-1}{q-1}
\]
\end{eulerformula}
\begin{eulercomment}
Ini sedikit rumit. Penjumlahan dievaluasi dengan flag “simpsum” untuk
menguranginya menjadi hasil bagi.

Mari kita buat sebuah fungsi untuk ini.
\end{eulercomment}
\begin{eulerprompt}
>function fs(K,R,P,n) &= (1+P/100)^n*K + ((1+P/100)^n-1)/(P/100)*R; $&fs(K,R,P,n)
\end{eulerprompt}
\begin{eulerformula}
\[
\frac{100\,\left(\left(\frac{P}{100}+1\right)^{n}-1\right)\,R}{P}+K
 \,\left(\frac{P}{100}+1\right)^{n}
\]
\end{eulerformula}
\begin{eulercomment}
Fungsi ini melakukan hal yang sama seperti fungsi f kita sebelumnya.
Tetapi fungsi ini lebih efektif.
\end{eulercomment}
\begin{eulerprompt}
>longest f(5000,-200,3,47), longest fs(5000,-200,3,47)
\end{eulerprompt}
\begin{euleroutput}
       -19.82504734650985 
       -19.82504734652684 
\end{euleroutput}
\begin{eulercomment}
Sekarang kita dapat menggunakannya untuk menanyakan waktu n. Kapan
modal kita habis? \\
Perkiraan awal kita adalah 30 tahun.
\end{eulercomment}
\begin{eulerprompt}
>solve("fs(5000,-330,3,x)",30)
\end{eulerprompt}
\begin{euleroutput}
        20.51 
\end{euleroutput}
\begin{eulercomment}
Jawaban ini mengatakan bahwa nilai tersebut akan menjadi negatif
setelah 21 tahun.

Kita juga bisa menggunakan sisi simbolis dari Euler untuk menghitung
rumus pembayaran.

Asumsikan kita mendapatkan pinjaman sebesar K, dan membayar n kali
pembayaran sebesar R (dimulai setelah tahun pertama) sehingga
menyisakan sisa utang sebesar Kn (pada saat pembayaran terakhir).
Rumus untuk hal ini adalah sebagai berikut
\end{eulercomment}
\begin{eulerprompt}
>equ &= fs(K,R,P,n)=Kn; $&equ
\end{eulerprompt}
\begin{eulerformula}
\[
\frac{100\,\left(\left(\frac{P}{100}+1\right)^{n}-1\right)\,R}{P}+K
 \,\left(\frac{P}{100}+1\right)^{n}={\it Kn}
\]
\end{eulerformula}
\begin{eulercomment}
Biasanya rumus ini diberikan dalam bentuk

\end{eulercomment}
\begin{eulerformula}
\[
i = \frac{P}{100}
\]
\end{eulerformula}
\begin{eulerprompt}
>equ &= (equ with P=100*i); $&equ
\end{eulerprompt}
\begin{eulerformula}
\[
\frac{\left(\left(i+1\right)^{n}-1\right)\,R}{i}+\left(i+1\right)^{
 n}\,K={\it Kn}
\]
\end{eulerformula}
\begin{eulercomment}
Kita dapat menyelesaikan laju R secara simbolis.
\end{eulercomment}
\begin{eulerprompt}
>$&solve(equ,R)
\end{eulerprompt}
\begin{eulerformula}
\[
\left[ R=\frac{i\,{\it Kn}-i\,\left(i+1\right)^{n}\,K}{\left(i+1
 \right)^{n}-1} \right] 
\]
\end{eulerformula}
\begin{eulercomment}
Seperti yang dapat Anda lihat dari rumusnya, fungsi ini mengembalikan
kesalahan floating point untuk i = 0. Euler tetap memplotnya.

Tentu saja, kita memiliki batas berikut.
\end{eulercomment}
\begin{eulerprompt}
>$&limit(R(5000,0,x,10),x,0)
\end{eulerprompt}
\begin{eulerformula}
\[
\lim_{x\rightarrow 0}{R\left(5000 , 0 , x , 10\right)}
\]
\end{eulerformula}
\begin{eulercomment}
Jelasnya, tanpa bunga kita harus membayar kembali 10 suku bunga 500.\\
Persamaan ini juga dapat diselesaikan untuk n. Akan terlihat lebih
baik jika kita menerapkan beberapa penyederhanaan.
\end{eulercomment}
\begin{eulerprompt}
>fn &= solve(equ,n) | ratsimp; $&fn
\end{eulerprompt}
\begin{eulerformula}
\[
\left[ n=\frac{\log \left(\frac{R+i\,{\it Kn}}{R+i\,K}\right)}{
 \log \left(i+1\right)} \right] 
\]
\end{eulerformula}
\eulerheading{Contoh Penyelesaian Soal-Soal Aljabar}
\begin{eulercomment}
\end{eulercomment}
\eulersubheading{R.2 Exercise Set}
\begin{eulercomment}
Write an equvalent expression without negative exponents

\end{eulercomment}
\begin{eulerformula}
\[
49. \left(\frac{24a^{10}b^{-8}c^{7}}{12a^{6}b^{-3}c^{5}}\right)^{-5}
\]
\end{eulerformula}
\begin{eulercomment}
\end{eulercomment}
\begin{eulerprompt}
>$&((24*a^10*b^(-8)*c^7)/(12*a^6*b^(-3)*c^5))^(-5)
\end{eulerprompt}
\begin{eulerformula}
\[
\frac{b^{25}}{32\,a^{20}\,c^{10}}
\]
\end{eulerformula}
\begin{eulercomment}
Calculate.

\end{eulercomment}
\begin{eulerformula}
\[
90. \left(2^{6}\times2^{-3} \div 2^{10} \div 2^{-8}\right)
\]
\end{eulerformula}
\begin{eulercomment}
\end{eulercomment}
\begin{eulerprompt}
>$&(2^6*2^(-3)/2^(10)/2^(-8))
\end{eulerprompt}
\begin{eulerformula}
\[
2
\]
\end{eulerformula}
\begin{eulercomment}
Syntesis\\
Saving plan. The formula

\end{eulercomment}
\begin{eulerformula}
\[
S = P \left(\frac{(1 + \frac{r}{12})^{12t} - 1}{\frac{r}{12}}\right)
\]
\end{eulerformula}
\begin{eulercomment}
gives the amount S accumulated in a savings plan when a deposit of P
dollars is made each month for t years in an account with interest
rate r, compounded monthly.
\end{eulercomment}
\begin{eulerprompt}
>function savingPlan(deposit, years, interestRate) ...
\end{eulerprompt}
\begin{eulerudf}
  return deposit * ((1 + (years / 12) ^ (12 * years)) - 1 / (interestRate / 12))
  endfunction
\end{eulerudf}
\begin{eulercomment}
97. James deposits \textdollar{}250 in a retirement account each month beginning
at age 40. If the investment earns 5\% interest, compounded monthly,
how much will have accumulated in the account when he retires 27 years
later?
\end{eulercomment}
\begin{eulerprompt}
>result := savingPlan(250, 27, 0.05)
\end{eulerprompt}
\begin{euleroutput}
  319945404605897893588266627102000547270066125660294774116001581846226212624189363132318609976546568702425419606917120.00 
\end{euleroutput}
\begin{eulerprompt}
>&round(result)
\end{eulerprompt}
\begin{euleroutput}
  
                              round(result)
  
\end{euleroutput}
\begin{eulercomment}
98. Kayla deposits \textdollar{}100 in a retirement account each month beginning
at age 25. If the investment earns 4\% interest, compounded monthly,
how much will have accumulated in the account when she retires at age
65?
\end{eulercomment}
\begin{eulerprompt}
>savingPlan(100, 65-25, 0.04)
\end{eulerprompt}
\begin{euleroutput}
  9589539106493442563690463168368774518044130868590559604647350406189897316476963273422447606108675909616509656962653544423337273924455938515046908928161701597144557155799016165129358337187728921859095030121289319776216730089619795728920020635639173087232.00 
\end{euleroutput}
\begin{eulercomment}
100. Lamont wants to have \textdollar{}200,000 accumulated in a retirement account
by age 70. If he starts making monthly deposits to the plan at age 30
and can count on an interest rate of 4.5\%, compounded monthly, how
much should he deposit each month in order to accomplish this?
\end{eulercomment}
\begin{eulerprompt}
>savingPlan(200000, 70 - 30, 0.045)
\end{eulerprompt}
\begin{euleroutput}
  19179078212986883147814141575530811418632261366807710681944573874844559515314888679396340235107830586307834680182627024539357340423418619656855826789636894961417786275882862395758810256152385799991823930172790048140944294837904628143399217645453900360187904.00 
\end{euleroutput}
\begin{eulercomment}
Simplify. Assume that all exponents are integers, all denominators are
nonzero, and zero is not raised to a nonpositive power

\end{eulercomment}
\begin{eulerformula}
\[
104. \left((m^{x-b}n^{x+b})^{x}(m^{b}n^{-b})^{x}\right)
\]
\end{eulerformula}
\begin{eulerprompt}
>$&((m^(x-b)*n^(x+b))^(x)*(m^(b)*n^(-b))^(x))
\end{eulerprompt}
\begin{eulerformula}
\[
\left(\frac{m^{b}}{n^{b}}\right)^{x}\,\left(m^{x-b}\,n^{x+b}\right)
 ^{x}
\]
\end{eulerformula}
\eulersubheading{R.3 Exercise Set}
\begin{eulercomment}
Determine the terms and the degree of the polynomials

\end{eulercomment}
\begin{eulerformula}
\[
7. \left(2x + 3y + z - 7\right) + \left(4x - 2y - z + 8\right) + \left(-3x + y - 2z - 4\right)
\]
\end{eulerformula}
\begin{eulercomment}
\end{eulercomment}
\begin{eulerprompt}
>$&factor((2*x + 3*y + x - 7) + (4*x - 2*y - z + 8) + (-3*x + y - 2*z -4))
\end{eulerprompt}
\begin{eulerformula}
\[
-3\,z+2\,y+4\,x-3
\]
\end{eulerformula}
\begin{eulerformula}
\[
8. \left(2x^{2} + 12xy - 11\right) + \left(6x^{2} - 2x + 4\right) + \left(-x^{2} - y -2\right)
\]
\end{eulerformula}
\begin{eulerprompt}
>$&factor((2*x^2 + 12*x*y - 11) + (6*x^2 - 2*x + 4) + (-x^2 - y - 2))
\end{eulerprompt}
\begin{eulerformula}
\[
12\,x\,y-y+7\,x^2-2\,x-9
\]
\end{eulerformula}
\begin{eulerprompt}
> 
\end{eulerprompt}
\begin{eulerformula}
\[
9. \left(3x^{2} - 2x - x^{3} + 2\right) - \left(5x^2 - 8x - x^{3} + 4\right)
\]
\end{eulerformula}
\begin{eulerprompt}
>$&factor((3*x^2 - 2*x - x^3 + 2) - (5*x^2 - 8*x - x^3 + 4))
\end{eulerprompt}
\begin{eulerformula}
\[
-2\,\left(x^2-3\,x+1\right)
\]
\end{eulerformula}
\begin{eulerprompt}
> 
\end{eulerprompt}
\begin{eulerformula}
\[
31. \left(5x - 3\right)^{2}
\]
\end{eulerformula}
\begin{eulerprompt}
>$&expand((5*x - 3)^2)
\end{eulerprompt}
\begin{eulerformula}
\[
25\,x^2-30\,x+9
\]
\end{eulerformula}
\begin{eulerprompt}
> 
\end{eulerprompt}
\begin{eulercomment}
Systhesis\\
Multiply. Assume that all exponents are natural numbers.

\end{eulercomment}
\begin{eulerformula}
\[
55. \left(a + b + c\right)^{2}
\]
\end{eulerformula}
\begin{eulerprompt}
>$&expand((a + b + c)^2)
\end{eulerprompt}
\begin{eulerformula}
\[
c^2+2\,b\,c+2\,a\,c+b^2+2\,a\,b+a^2
\]
\end{eulerformula}
\eulersubheading{R.4 Exercise Set}
\begin{eulercomment}
Factor the difference of squares

\end{eulercomment}
\begin{eulerformula}
\[
47. z^{2} - 81
\]
\end{eulerformula}
\begin{eulercomment}
\end{eulercomment}
\begin{eulerprompt}
>$&factor(z^2 - 81)
\end{eulerprompt}
\begin{eulerformula}
\[
\left(z-9\right)\,\left(z+9\right)
\]
\end{eulerformula}
\begin{eulercomment}
Factor the square of a binomial.

\end{eulercomment}
\begin{eulerformula}
\[
66. \left(5a^{2} - 10ab + 5b^2\right)
\]
\end{eulerformula}
\begin{eulercomment}
\end{eulercomment}
\begin{eulerprompt}
>$&factor(5*a^2 - 10*a*b + 5*b^2)
\end{eulerprompt}
\begin{eulerformula}
\[
5\,\left(b-a\right)^2
\]
\end{eulerformula}
\begin{eulercomment}
Factor the sum or difference of cubes.

\end{eulercomment}
\begin{eulerformula}
\[
75. \left(t^{6} + 1\right)
\]
\end{eulerformula}
\begin{eulerprompt}
>$&factor(t^6 + 1)
\end{eulerprompt}
\begin{eulerformula}
\[
\left(t^2+1\right)\,\left(t^4-t^2+1\right)
\]
\end{eulerformula}
\begin{eulercomment}
Factor completely.

\end{eulercomment}
\begin{eulerformula}
\[
78. \left(4x^{2} + 12xy^{2}\right)
\]
\end{eulerformula}
\begin{eulerprompt}
>$&factor(4*x^2 + 12*x*y^2)
\end{eulerprompt}
\begin{eulerformula}
\[
4\,x\,\left(3\,y^2+x\right)
\]
\end{eulerformula}
\begin{eulercomment}
Synthesis\\
Factor.

\end{eulercomment}
\begin{eulerformula}
\[
130. \left(x + 0.01\right)^{2} - x^{2}
\]
\end{eulerformula}
\begin{eulerprompt}
>$&factor((x + 0.01)^2 - x^2)
\end{eulerprompt}
\begin{eulerformula}
\[
\frac{200\,x+1}{10000}
\]
\end{eulerformula}
\begin{eulercomment}
Factor. Assume that variables in exponents represent\\
natural numbers.

\end{eulercomment}
\begin{eulerformula}
\[
139. \left(y - 1\right)^{4} - \left(y - 1\right)^{2}
\]
\end{eulerformula}
\begin{eulerprompt}
>$&factor((y - 1)^4 - (y - 1)^2)
\end{eulerprompt}
\begin{eulerformula}
\[
\left(y-2\right)\,\left(y-1\right)^2\,y
\]
\end{eulerformula}
\eulersubheading{R.5 Exercise Set}
\begin{eulercomment}
\end{eulercomment}
\begin{eulerformula}
\[
31.  7(3x + 6) = 11 - (x + 2)
\]
\end{eulerformula}
\begin{eulercomment}
\end{eulercomment}
\begin{eulerprompt}
>$&solve(7*(3*x + 6) = 11 - (x + 2), x)
\end{eulerprompt}
\begin{eulerformula}
\[
\left[ x=-\frac{3}{2} \right] 
\]
\end{eulerformula}
\begin{eulerformula}
\[
36. y^2 - 4y - 45 = 0
\]
\end{eulerformula}
\begin{eulercomment}
\end{eulercomment}
\begin{eulerprompt}
>$&solve(y^2 - 4*y - 45 = 0, y)
\end{eulerprompt}
\begin{eulerformula}
\[
\left[ y=9 , y=-5 \right] 
\]
\end{eulerformula}
\begin{eulerformula}
\[
65. A = \frac{1}{2}h\left(b_1 + b_2\right)
\]
\end{eulerformula}
\begin{eulercomment}
for: h (Area of a trapezoid)
\end{eulercomment}
\begin{eulerprompt}
>$&(solve(A = (1/2)* h * (b1 + b2), h))
\end{eulerprompt}
\begin{eulerformula}
\[
\left[ \begin{pmatrix}\frac{2-\left({\it b_2}+{\it b_1}\right)\,h}{
 2} & \frac{80143857-\left(12755291\,{\it b_2}+12755291\,{\it b_1}
 \right)\,h}{25510582} \\ \frac{8-\left({\it b_2}+{\it b_1}\right)\,h
 }{2} & \frac{10-\left({\it b_2}+{\it b_1}\right)\,h}{2} \\ 
 \end{pmatrix}=0 \right] 
\]
\end{eulerformula}
\begin{eulerformula}
\[
71. Ax + By = C
\]
\end{eulerformula}
\begin{eulercomment}
for A
\end{eulercomment}
\begin{eulerprompt}
>$&solve(A*x + By = C, A)
\end{eulerprompt}
\begin{eulerformula}
\[
\left[  \right] 
\]
\end{eulerformula}
\begin{eulercomment}
Synthesis\\
Solve

\end{eulercomment}
\begin{eulerformula}
\[
81. 3[5 - 3(4 - t)] - 2 = 5[3(5t - 4) + 8] - 26
\]
\end{eulerformula}
\begin{eulercomment}
\end{eulercomment}
\begin{eulerprompt}
>$&solve(3*(5 - 3*(4 - t)) - 2 = 5*(3*(5 * t - 4) + 8) - 26, t)
\end{eulerprompt}
\begin{eulerformula}
\[
\left[ t=\frac{23}{66} \right] 
\]
\end{eulerformula}
\begin{eulerformula}
\[
87. 3x^{3} + 6x^{2} - 27x - 54 = 0
\]
\end{eulerformula}
\begin{eulercomment}
\end{eulercomment}
\begin{eulerprompt}
>$&solve(3*x^3 + 6*x^2 - 27*x - 54 = 0, x)
\end{eulerprompt}
\begin{eulerformula}
\[
\left[ x=-3 , x=-2 , x=3 \right] 
\]
\end{eulerformula}
\begin{eulerudf}
  
\end{eulerudf}
\begin{euleroutput}
  
\end{euleroutput}
\eulersubheading{R.6 Exercise Set}
\begin{eulercomment}
\end{eulercomment}
\begin{eulerformula}
\[
11). \frac{x^3-6x^2+9x}{x^3-3x^2}
\]
\end{eulerformula}
\begin{eulercomment}
Penyelesaian :\\
mencari faktor pembilang
\end{eulercomment}
\begin{eulerprompt}
>$&factor(x^3-6*x^2+9*x)
\end{eulerprompt}
\begin{eulerformula}
\[
\left(x-3\right)^2\,x
\]
\end{eulerformula}
\begin{eulercomment}
mencari faktor penyebut
\end{eulercomment}
\begin{eulerprompt}
>$&factor(x^3-3*x^2)
\end{eulerprompt}
\begin{eulerformula}
\[
\left(x-3\right)\,x^2
\]
\end{eulerformula}
\begin{eulercomment}
jadi tugas kita menyelesaikan :\\
\end{eulercomment}
\begin{eulerformula}
\[
\frac{(x-3)^2 \cdot x}{(x-3) \cdot x^2}
\]
\end{eulerformula}
\begin{eulerprompt}
>$&((x-3)^2 * x)/((x-3) * x^2)
\end{eulerprompt}
\begin{eulerformula}
\[
\frac{x-3}{x}
\]
\end{eulerformula}
\begin{eulercomment}
atau dengan perintah EMT untuk menyelesaikan soal ini adalah
\end{eulercomment}
\begin{eulerprompt}
>$&ratsimp((x^3-6*x^2+9*x)/(x^3-3*x^2)) ...
>//menyederhanakan pecahan dengan perintah ratsimp
\end{eulerprompt}
\begin{eulerformula}
\[
\frac{x-3}{x}
\]
\end{eulerformula}
\begin{eulercomment}
17. Multiply or divide and, if possible, simplify\\
\end{eulercomment}
\begin{eulerformula}
\[
\frac{r-s}{r+s} \cdot \frac{r^2-s^2}{(r-s)^2}
\]
\end{eulerformula}
\begin{eulercomment}
Penyelesaian:
\end{eulercomment}
\begin{eulerprompt}
>$&ratsimp((r-s)/(r+s))*((r^2-s^2)/((r-s)^2)) ...
>//menyederhanakan pecahan
\end{eulerprompt}
\begin{eulerformula}
\[
\frac{r^2-s^2}{\left(r-s\right)\,\left(s+r\right)}
\]
\end{eulerformula}
\begin{eulerprompt}
>$&(r^2-s^2)/(r-s)*(s+r), $&ratsimp(%) ...
>//menyederhanakan pecahan lalu hasilnya disederhanakan lagi pakai %
\end{eulerprompt}
\begin{eulerformula}
\[
\frac{\left(s+r\right)\,\left(r^2-s^2\right)}{r-s}
\]
\end{eulerformula}
\begin{eulerformula}
\[
s^2+2\,r\,s+r^2
\]
\end{eulerformula}
\begin{eulercomment}
\end{eulercomment}
\begin{eulerformula}
\[
28.  \frac{(c^3+8)}{(c^2-4)} : \frac{(c^2-2c+4)}{(c^2-4c+4)}
\]
\end{eulerformula}
\begin{eulercomment}
Penyelesaian:

\end{eulercomment}
\begin{eulerprompt}
>$&ratsimp((c^3+8)/(c^2-4))/((c^2-2*c+4)/(c^2-4*c+4)), $&ratsimp(%) ...
>//menyederhanakan pecahan lalu hasilnya disederhanakan lagi pakai %
\end{eulerprompt}
\begin{eulerformula}
\[
\frac{c^2-4\,c+4}{c-2}
\]
\end{eulerformula}
\begin{eulerformula}
\[
c-2
\]
\end{eulerformula}
\begin{eulercomment}
40. add or subtract and, if possible, simplify\\
\end{eulercomment}
\begin{eulerformula}
\[
\frac{6}{y^2+6y+9} - \frac{5}{y-3}
\]
\end{eulerformula}
\begin{eulercomment}
Penyelesaian:
\end{eulercomment}
\begin{eulerprompt}
>$&expand(6/y^2+6*y+9)-(5/y-3), $&ratsimp(%) ...
>//menyederhanakan pecahan lalu melakukan operasi pada hasilnya
\end{eulerprompt}
\begin{eulerformula}
\[
6\,y-\frac{5}{y}+\frac{6}{y^2}+12
\]
\end{eulerformula}
\begin{eulerformula}
\[
\frac{6\,y^3+12\,y^2-5\,y+6}{y^2}
\]
\end{eulerformula}
\begin{eulercomment}
62. Simplify\\
\end{eulercomment}
\begin{eulerformula}
\[
\frac{\frac{a^2}{b}+\frac{b^2}{a}}{{a^2-ab+b^2}}
\]
\end{eulerformula}
\begin{eulercomment}
Penyelesaian:

\end{eulercomment}
\begin{eulerprompt}
>$&factor((a^2/b)+(b^2/a))/(a^2-a*b+b^2) ...
>//menyederhanakan bentuk paling sederhana dengan perintah factor
\end{eulerprompt}
\begin{eulerformula}
\[
\frac{b+a}{a\,b}
\]
\end{eulerformula}
\eulersubheading{R.7 Exercise Set}
\begin{eulercomment}
53. Simplify. Assume that no radicands were formed by raising negative
quantities to even powers.

\end{eulercomment}
\begin{eulerformula}
\[
(2\sqrt{3} + \sqrt{5})\cdot (\sqrt{3}-3\sqrt{5})
\]
\end{eulerformula}
\begin{eulercomment}
Penyelesaian:
\end{eulercomment}
\begin{eulerprompt}
>$&expand(2*(3)^1/2 + (5)^1/2)*((3)^1/2 - 3*(5)^1/2)
\end{eulerprompt}
\begin{eulerformula}
\[
-33
\]
\end{eulerformula}
\begin{eulercomment}
\end{eulercomment}
\eulersubheading{Review Exercise}
\begin{eulercomment}
66. the cost of a house is \textdollar{}98,000. The down payment is \%16,000, the
interest rate is 6 1/2\%, and the loan period is 25 years. What is the
monthly mortgage payment?

Penyelesaian :

Rumus angsuran per bulan:\\
\end{eulercomment}
\begin{eulerformula}
\[
M = P \cdot \frac{\frac{r}{12} \cdot (1 +\frac{r}{12})^2}{(1 +\frac{r}{12})^2 - 1}
\]
\end{eulerformula}
\begin{eulercomment}
dengan M : angsuran per bulan\\
P : jumlah pinjaman\\
r : suku bunga bulanan\\
n : jumlah total pembayaran (dalam bulan)

diketahui:\\
Harga rumah : \textdollar{}98,000\\
Uang muka : \textdollar{}16,000\\
suku bunga tahunan: 6 1/2\%\\
6 1/2\% diubah menjadi 6.5\%\\
lama angsur : 25 tahun

ditanya: angsuran per bulan berapa\\
Variabel yang dibutuhkan pada rumus adalah :
\end{eulercomment}
\begin{eulerprompt}
>P = 98000-16000
\end{eulerprompt}
\begin{euleroutput}
     82000.00 
\end{euleroutput}
\begin{eulerprompt}
>r = 6.5/100
\end{eulerprompt}
\begin{euleroutput}
         0.07 
\end{euleroutput}
\begin{eulerprompt}
>n = 25*12
\end{eulerprompt}
\begin{euleroutput}
       300.00 
\end{euleroutput}
\begin{eulercomment}
Substitusi semua variabel yang telah diperoleh ke rumus M
\end{eulercomment}
\begin{eulerprompt}
>M = P * (r/12 * (1 + r/12)^n)/((1+r/12)^n -1 )
\end{eulerprompt}
\begin{euleroutput}
       553.67 
\end{euleroutput}
\begin{eulercomment}
Jadi, biaya angsuran per bulan selama 25 tahun dengan suku bunga
sebesar 6.5\% persen per tahun adalah \textdollar{}553,669872305



\end{eulercomment}
\begin{eulerformula}
\[
70. (x^n + 10)(x^n -4)
\]
\end{eulerformula}
\begin{eulercomment}
Penyelesaian:
\end{eulercomment}
\begin{eulerprompt}
>$&expand((x^n + 10)*(x^n - 4)) ...
>//menjabarkan fungsi
\end{eulerprompt}
\begin{eulerformula}
\[
x^{2\,n}+6\,x^{n}-40
\]
\end{eulerformula}
\begin{eulercomment}
\end{eulercomment}
\begin{eulerformula}
\[
71. (t^a + t^{-a})^2
\]
\end{eulerformula}
\begin{eulercomment}
Penyelesaian:
\end{eulercomment}
\begin{eulerprompt}
>$&expand((t^a + t^(-a))^2) ...
>//menjabarkan fungsi
\end{eulerprompt}
\begin{eulerformula}
\[
t^{2\,a}+\frac{1}{t^{2\,a}}+2
\]
\end{eulerformula}
\begin{eulercomment}
\end{eulercomment}
\begin{eulerformula}
\[
73. (a^n - b^n)^3
\]
\end{eulerformula}
\begin{eulerprompt}
>$&showev('expand((a^n - b^n)^3)), $&expand((a^n - b^n)^3) ...
>//menjabarkan fungsi dengan showev
\end{eulerprompt}
\begin{eulerformula}
\[
{\it expand}\left(\left(a^{n}-b^{n}\right)^3\right)=-b^{3\,n}+3\,a
 ^{n}\,b^{2\,n}-3\,a^{2\,n}\,b^{n}+a^{3\,n}
\]
\end{eulerformula}
\begin{eulerformula}
\[
-b^{3\,n}+3\,a^{n}\,b^{2\,n}-3\,a^{2\,n}\,b^{n}+a^{3\,n}
\]
\end{eulerformula}
\eulersubheading{2.3 Exercise Set}
\begin{eulercomment}
Given that\\
\end{eulercomment}
\begin{eulerformula}
\[
f(x) = 3x + 1
\]
\end{eulerformula}
\begin{eulerformula}
\[
g(x) = x^2 - 2x -6
\]
\end{eulerformula}
\begin{eulerformula}
\[
h(x) = x^3
\]
\end{eulerformula}
\begin{eulercomment}
Find each of the following

Menyimpan semua nilai fx, gx, dan hx.

\end{eulercomment}
\begin{eulerprompt}
>fx &= 3*x + 1; gx &= x^2 - 2*x - 6; hx &= x^3;
\end{eulerprompt}
\begin{eulercomment}
1. Mencari nilai\\
\end{eulercomment}
\begin{eulerformula}
\[
(f \circ g)(-1)
\]
\end{eulerformula}
\begin{eulercomment}
Penyelesaian:

fungsi komposisi di atas juga dapat ditulis sebagai :\\
\end{eulercomment}
\begin{eulerformula}
\[
(f(g(-1))
\]
\end{eulerformula}
\begin{eulercomment}
Mencari nilai g(-1)
\end{eulercomment}
\begin{eulerprompt}
>gx(-1) //cara manual
\end{eulerprompt}
\begin{euleroutput}
        -3.00 
\end{euleroutput}
\begin{eulercomment}
Lalu mencari nilai f(-3)
\end{eulercomment}
\begin{eulerprompt}
>fx(-3)
\end{eulerprompt}
\begin{euleroutput}
        -8.00 
\end{euleroutput}
\begin{eulercomment}
Perintah fungsi komposisi di EMT\\
Akan menghasilkan hasil yang sama dengan cara manual di atas. Cara ini
lebih efisien.
\end{eulercomment}
\begin{eulerprompt}
>fx(gx(-1)) //cara cepat
\end{eulerprompt}
\begin{euleroutput}
        -8.00 
\end{euleroutput}
\begin{eulercomment}
4. Mencari nilai\\
\end{eulercomment}
\begin{eulerformula}
\[
(g \circ h)(\frac{1}{2})
\]
\end{eulerformula}
\begin{eulercomment}
Penyelesaian:

Fungsi komposisi di atas juga dapat ditulis\\
\end{eulercomment}
\begin{eulerformula}
\[
g(h(\frac{1}{2}))
\]
\end{eulerformula}
\begin{eulerprompt}
>gx(hx(1/2))
\end{eulerprompt}
\begin{euleroutput}
        -6.23 
\end{euleroutput}
\begin{eulercomment}
7. Mencari nilai\\
\end{eulercomment}
\begin{eulerformula}
\[
(f \circ h)(-3)
\]
\end{eulerformula}
\begin{eulercomment}
Penyelesaian:\\
Fungsi komposisi di atas bisa ditulis\\
\end{eulercomment}
\begin{eulerformula}
\[
f(g(-3))
\]
\end{eulerformula}
\begin{eulerprompt}
>fx(gx(-3))
\end{eulerprompt}
\begin{euleroutput}
        28.00 
\end{euleroutput}
\begin{eulercomment}
11. Mencari nilai\\
\end{eulercomment}
\begin{eulerformula}
\[
(h \circ h)(2)
\]
\end{eulerformula}
\begin{eulercomment}
Penyelesaian

Fungsi komposisi di atas dapat ditulis\\
\end{eulercomment}
\begin{eulerformula}
\[
(h(h(2)))
\]
\end{eulerformula}
\begin{eulerprompt}
>hx(hx(2))
\end{eulerprompt}
\begin{euleroutput}
       512.00 
\end{euleroutput}
\begin{eulercomment}
13. Mencari nilai\\
\end{eulercomment}
\begin{eulerformula}
\[
(f \circ f)(-4)
\]
\end{eulerformula}
\begin{eulercomment}
Penyelesaian:

Fungsi komposisi di atas dapat ditulis\\
\end{eulercomment}
\begin{eulerformula}
\[
f(f(-4))
\]
\end{eulerformula}
\begin{eulerprompt}
>fx(fx(-4))
\end{eulerprompt}
\begin{euleroutput}
       -32.00 
\end{euleroutput}
\eulersubheading{3.1 Exercise Set}
\begin{eulercomment}
11. Simplify. Where answer in the form a + bi, where a and b are real
numbers.\\
\end{eulercomment}
\begin{eulerformula}
\[
(-5+3i) + (7+8i)
\]
\end{eulerformula}
\begin{eulercomment}
Penyelesaian:
\end{eulercomment}
\begin{eulerprompt}
>bilangan1 = -5 + 3*I, bilangan2 = 7 + 8*I, bilangan1 + bilangan2
\end{eulerprompt}
\begin{euleroutput}
              -5.00+3.00i 
               7.00+8.00i 
              2.00+11.00i 
\end{euleroutput}
\begin{eulercomment}
\end{eulercomment}
\begin{eulerformula}
\[
21. (10+7i)-(5+3i)
\]
\end{eulerformula}
\begin{eulercomment}
Penyelesaian:
\end{eulercomment}
\begin{eulerprompt}
>bilangan1 = 10 + 7*I, bilangan2 = 5 + 3*I, bilangan1 - bilangan2
\end{eulerprompt}
\begin{euleroutput}
              10.00+7.00i 
               5.00+3.00i 
               5.00+4.00i 
\end{euleroutput}
\begin{eulercomment}
\end{eulercomment}
\begin{eulerformula}
\[
35. 7i\cdot (2-5i)
\]
\end{eulerformula}
\begin{eulercomment}
Penyelesaian:
\end{eulercomment}
\begin{eulerprompt}
>bilangan1 = 7*I, bilangan2 = 2-5*I, bilangan1*bilangan2
\end{eulerprompt}
\begin{euleroutput}
               0.00+7.00i 
               2.00-5.00i 
             35.00+14.00i 
\end{euleroutput}
\begin{eulercomment}
\end{eulercomment}
\begin{eulerformula}
\[
36. 3i \cdot (6+4i)
\]
\end{eulerformula}
\begin{eulercomment}
Penyelesaian:

\end{eulercomment}
\begin{eulerprompt}
>bilangan1 = 3*I, bilangan2 = 6 + 4*I, bilangan1*bilangan2
\end{eulerprompt}
\begin{euleroutput}
               0.00+3.00i 
               6.00+4.00i 
            -12.00+18.00i 
\end{euleroutput}
\begin{eulercomment}
\end{eulercomment}
\begin{eulerformula}
\[
37. -2i \cdot (-8+3i)
\]
\end{eulerformula}
\begin{eulercomment}
Penyelesaian:
\end{eulercomment}
\begin{eulerprompt}
>bilangan1 = -2*I, bilangan2 = -8 + 3*I, bilangan1*bilangan2
\end{eulerprompt}
\begin{euleroutput}
              -0.00-2.00i 
              -8.00+3.00i 
              6.00+16.00i 
\end{euleroutput}
\eulersubheading{3.4 Exercise Set}
\begin{eulercomment}
\end{eulercomment}
\begin{eulerformula}
\[
1). \frac{1}{4} + \frac{1}{5} = \frac{1}{t}
\]
\end{eulerformula}
\begin{eulerprompt}
>$&solve((1/4)+(1/5)=(1/t))
\end{eulerprompt}
\begin{eulerformula}
\[
\left[ t=\frac{20}{9} \right] 
\]
\end{eulerformula}
\begin{eulercomment}
\end{eulercomment}
\begin{eulerformula}
\[
3). \frac{x+2}{4} - \frac{x-1}{5} = 15
\]
\end{eulerformula}
\begin{eulerprompt}
>$&solve(((x+2)/4)-((x-1)/5)=15)
\end{eulerprompt}
\begin{eulerformula}
\[
\left[ x=286 \right] 
\]
\end{eulerformula}
\begin{eulercomment}
\end{eulercomment}
\begin{eulerformula}
\[
4). \frac{t+1}{3} - \frac{t-1}{2} = 1
\]
\end{eulerformula}
\begin{eulerprompt}
>$&solve(((t+1)/3)-((t-1)/2))
\end{eulerprompt}
\begin{eulerformula}
\[
\left[ t=5 \right] 
\]
\end{eulerformula}
\begin{eulercomment}
\end{eulercomment}
\begin{eulerformula}
\[
7). \frac{5}{3x+2} = \frac{3}{2x}
\]
\end{eulerformula}
\begin{eulerprompt}
>$&solve((5/(3*x+2))=(3/2*x))
\end{eulerprompt}
\begin{eulerformula}
\[
\left[ x=\frac{-\sqrt{11}-1}{3} , x=\frac{\sqrt{11}-1}{3} \right] 
\]
\end{eulerformula}
\begin{eulercomment}
\end{eulercomment}
\begin{eulerformula}
\[
29). \sqrt{3x-4}=1
\]
\end{eulerformula}
\begin{eulerprompt}
>$&solve((3*x-4)^1/2 = 1)
\end{eulerprompt}
\begin{eulerformula}
\[
\left[ x=2 \right] 
\]
\end{eulerformula}
\eulersubheading{3.5 Exercise Set}
\begin{eulercomment}
\end{eulercomment}
\begin{eulerformula}
\[
23). \left| x+3 \right|-2 = 8
\]
\end{eulerformula}
\begin{eulerprompt}
>$&solve(abs(x+3)-2=8, x)
\end{eulerprompt}
\begin{eulerformula}
\[
\left[ \left| x+3\right| =10 \right] 
\]
\end{eulerformula}
\begin{eulerprompt}
>$&solve(abs(x+3))
\end{eulerprompt}
\begin{eulerformula}
\[
\left[ x=-3 \right] 
\]
\end{eulerformula}
\begin{eulerprompt}
>$&solve(abs(x+3)-2=8)
\end{eulerprompt}
\begin{eulerformula}
\[
\left[ \left| x+3\right| =10 \right] 
\]
\end{eulerformula}
\begin{eulerprompt}
>$&solve((abs(x+3))=10)
\end{eulerprompt}
\begin{eulerformula}
\[
\left[ \left| x+3\right| =10 \right] 
\]
\end{eulerformula}
\begin{eulerprompt}
>$&expand(abs(x+3)=10)
\end{eulerprompt}
\begin{eulerformula}
\[
\left| x+3\right| =10
\]
\end{eulerformula}
\begin{eulerprompt}
>$&solve((x+3)=10)
\end{eulerprompt}
\begin{eulerformula}
\[
\left[ x=7 \right] 
\]
\end{eulerformula}
\begin{eulerprompt}
>$&solve((x+3)=-10)
\end{eulerprompt}
\begin{eulerformula}
\[
\left[ x=-13 \right] 
\]
\end{eulerformula}
\begin{eulercomment}
\end{eulercomment}
\begin{eulerformula}
\[
28). \left| 5x+4 \right|+2=5
\]
\end{eulerformula}
\begin{eulerprompt}
>$&solve(abs(5*x+4)+2=5, x)
\end{eulerprompt}
\begin{eulerformula}
\[
\left[ \left| 5\,x+4\right| =3 \right] 
\]
\end{eulerformula}
\begin{eulerprompt}
>$&solve((5*x+4)=3)
\end{eulerprompt}
\begin{eulerformula}
\[
\left[ x=-\frac{1}{5} \right] 
\]
\end{eulerformula}
\begin{eulerprompt}
>$&solve((5*x+4)=-3)
\end{eulerprompt}
\begin{eulerformula}
\[
\left[ x=-\frac{7}{5} \right] 
\]
\end{eulerformula}
\begin{eulercomment}
\end{eulercomment}
\begin{eulerformula}
\[
32). 5-\left| 4x+3 \right|=2
\]
\end{eulerformula}
\begin{eulerprompt}
>$&solve(5-(abs(4*x+3)=2))
\end{eulerprompt}
\begin{eulerformula}
\[
\left[ \left| 4\,x+3\right| =2 \right] 
\]
\end{eulerformula}
\begin{eulerprompt}
>$&solve((4*x+3)=2)
\end{eulerprompt}
\begin{eulerformula}
\[
\left[ x=-\frac{1}{4} \right] 
\]
\end{eulerformula}
\begin{eulerprompt}
>$&solve((4*x+3)=-2)
\end{eulerprompt}
\begin{eulerformula}
\[
\left[ x=-\frac{5}{4} \right] 
\]
\end{eulerformula}
\begin{eulercomment}
\end{eulercomment}
\begin{eulerformula}
\[
43). \left| 2x \right|\geq 6
\]
\end{eulerformula}
\begin{eulerprompt}
>&load(fourier_elim)
\end{eulerprompt}
\begin{euleroutput}
  
          C:/Program Files/Euler x64/maxima/share/maxima/5.35.1/share/f\(\backslash\)
  ourier_elim/fourier_elim.lisp
  
\end{euleroutput}
\begin{eulerprompt}
>$&fourier_elim([abs(2*x)] >= 6, [x])
\end{eulerprompt}
\begin{eulerformula}
\[
\left[ \left[ 2\,\left(\left| x\right| -3\right) \right] =0
  \right] \lor \left[ \left[ 2\,\left(\left| x\right| -3\right)
  \right] >0 \right] 
\]
\end{eulerformula}
\begin{eulerprompt}
>$&fourier_elim([abs(2*x)] >= 6, [x])
\end{eulerprompt}
\begin{eulerformula}
\[
\left[ \left[ 2\,\left(\left| x\right| -3\right) \right] =0
  \right] \lor \left[ \left[ 2\,\left(\left| x\right| -3\right)
  \right] >0 \right] 
\]
\end{eulerformula}
\eulersubheading{Chapter 3 test}
\begin{eulercomment}
\end{eulercomment}
\begin{eulerformula}
\[
7). x+5\sqrt{x}-36=0
\]
\end{eulerformula}
\begin{eulerprompt}
>$&solve(x+5*(x)^1/2-36=0)
\end{eulerprompt}
\begin{eulerformula}
\[
\left[ x=\frac{72}{7} \right] 
\]
\end{eulerformula}
\begin{eulercomment}
\end{eulercomment}
\begin{eulerformula}
\[
10). \sqrt{x+4}-\sqrt{x-4}=2
\]
\end{eulerformula}
\begin{eulerprompt}
>$&solve(((x+4)^1/2)-((x-4)^1/2=2))
\end{eulerprompt}
\begin{eulerformula}
\[
\left[ x=8 \right] 
\]
\end{eulerformula}
\begin{eulercomment}
\end{eulercomment}
\begin{eulerformula}
\[
9). \sqrt{x+4}-2=1
\]
\end{eulerformula}
\begin{eulerprompt}
>$&solve(((x+4)^1/2)-2=1)
\end{eulerprompt}
\begin{eulerformula}
\[
\left[ x=2 \right] 
\]
\end{eulerformula}
\begin{eulercomment}
\end{eulercomment}
\begin{eulerformula}
\[
13). \left|x+3\right|\leq4
\]
\end{eulerformula}
\begin{eulerprompt}
>$&fourier_elim([abs(x+3)<=4], [x])
\end{eulerprompt}
\begin{eulerformula}
\[
\left[ x=1 \right] \lor \left[ x=-7 \right] \lor \left[ -7<x , x<1
  \right] 
\]
\end{eulerformula}
\begin{eulercomment}
\end{eulercomment}
\begin{eulerformula}
\[
14). \left|2x-1\right|<5
\]
\end{eulerformula}
\begin{eulerprompt}
>$&fourier_elim([abs(2*x-1)<5],[x])
\end{eulerprompt}
\begin{eulerformula}
\[
\left[ -2<x , x<3 \right] 
\]
\end{eulerformula}
\eulersubheading{4.1 Exercise Set}
\begin{eulercomment}
\end{eulercomment}
\begin{eulerformula}
\[
36). f(x) = (x^2-5x+6)^2
\]
\end{eulerformula}
\begin{eulerprompt}
>$&solve((x^2-5*x+6)^2)
\end{eulerprompt}
\begin{eulerformula}
\[
\left[ x=3 , x=2 \right] 
\]
\end{eulerformula}
\begin{eulercomment}
\end{eulercomment}
\begin{eulerformula}
\[
37). f(x) = x^4-4x^2+3
\]
\end{eulerformula}
\begin{eulerprompt}
>$&solve(x^4-4*x^2+3)
\end{eulerprompt}
\begin{eulerformula}
\[
\left[ x=-1 , x=1 , x=-\sqrt{3} , x=\sqrt{3} \right] 
\]
\end{eulerformula}
\begin{eulercomment}
\end{eulercomment}
\begin{eulerformula}
\[
40). f(x) = x^3-x^2-2x+2
\]
\end{eulerformula}
\begin{eulerprompt}
>$&solve(x^3-x^2-2*x+2)
\end{eulerprompt}
\begin{eulerformula}
\[
\left[ x=-\sqrt{2} , x=\sqrt{2} , x=1 \right] 
\]
\end{eulerformula}
\begin{eulercomment}
\end{eulercomment}
\begin{eulerformula}
\[
61). 2y-3\geq1-y+5
\]
\end{eulerformula}
\begin{eulerprompt}
>$&fourier_elim([2*y-3>=1-y+5],[x])
\end{eulerprompt}
\begin{eulerformula}
\[
\left[ y-3 \right] \lor \left[ y-3>0 \right] 
\]
\end{eulerformula}
\begin{eulercomment}
\end{eulercomment}
\begin{eulerformula}
\[
64). \left|x+\frac{1}{4}\right|\leq\frac{2}{3}
\]
\end{eulerformula}
\begin{eulerprompt}
>$&fourier_elim([x+(1/4)<=(2/3)],[x])
\end{eulerprompt}
\begin{eulerformula}
\[
\left[ x=\frac{5}{12} \right] \lor \left[ x<\frac{5}{12} \right] 
\]
\end{eulerformula}
\begin{eulercomment}
\end{eulercomment}
\eulersubheading{Mid-Chapter Mixed Review}
\begin{eulercomment}
\end{eulercomment}
\begin{eulerformula}
\[
18). g(x)=x^3-9x^2+4x-10;
\]
\end{eulerformula}
\begin{eulercomment}
find g(-5)
\end{eulercomment}
\begin{eulerprompt}
>function g(x) := (x^3-9x^2+4x-10)
>g(-5)
\end{eulerprompt}
\begin{euleroutput}
      -380.00 
\end{euleroutput}
\begin{eulercomment}
\end{eulercomment}
\begin{eulerformula}
\[
20). f(x) = 5x^4+x^3-x;
\]
\end{eulerformula}
\begin{eulercomment}
find\\
\end{eulercomment}
\begin{eulerformula}
\[
f(-\sqrt{-2})
\]
\end{eulerformula}
\begin{eulerprompt}
>a = - sqrt(2)
\end{eulerprompt}
\begin{euleroutput}
        -1.41 
\end{euleroutput}
\begin{eulerprompt}
>fx &= 5*x^4 + x^3 - x; fx(a)
\end{eulerprompt}
\begin{euleroutput}
        18.59 
\end{euleroutput}
\begin{eulercomment}
\end{eulercomment}
\begin{eulerformula}
\[
17). \frac{x^5-5}{x+1}
\]
\end{eulerformula}
\begin{eulerprompt}
>$&expand((x^5-5)/(x+1))
\end{eulerprompt}
\begin{eulerformula}
\[
\frac{x^5}{x+1}-\frac{5}{x+1}
\]
\end{eulerformula}
\begin{eulercomment}
\end{eulercomment}
\begin{eulerformula}
\[
19). f(x)=20x^2-40x;
\]
\end{eulerformula}
\begin{eulercomment}
find\\
\end{eulercomment}
\begin{eulerformula}
\[
f(\frac{1}{2})
\]
\end{eulerformula}
\begin{eulerprompt}
>function f(x) :=(20*x^2-40*x)
>f(1/2)
\end{eulerprompt}
\begin{euleroutput}
       -15.00 
\end{euleroutput}
\begin{eulercomment}
\end{eulercomment}
\begin{eulerformula}
\[
23). h(x)=x^3-2x^2-55x+56
\]
\end{eulerformula}
\begin{eulercomment}
Penyelesaian :
\end{eulercomment}
\begin{eulerprompt}
>$&factor(x^3-2*x^2-55*x+56)
\end{eulerprompt}
\begin{eulerformula}
\[
\left(x-8\right)\,\left(x-1\right)\,\left(x+7\right)
\]
\end{eulerformula}
\begin{eulerprompt}
>$&solve(x^3-2*x^2-55*x+56)
\end{eulerprompt}
\begin{eulerformula}
\[
\left[ x=-7 , x=8 , x=1 \right] 
\]
\end{eulerformula}
\begin{eulercomment}
\end{eulercomment}
\begin{eulerformula}
\[
24). g(x) = x^4-2x^3-13x^2+14x+24
\]
\end{eulerformula}
\begin{eulercomment}
Penyelesaian :
\end{eulercomment}
\begin{eulerprompt}
>$&factor(x^4-2*x^3-13*x^2+14*x+24)
\end{eulerprompt}
\begin{eulerformula}
\[
\left(x-4\right)\,\left(x-2\right)\,\left(x+1\right)\,\left(x+3
 \right)
\]
\end{eulerformula}
\begin{eulerprompt}
>$&solve(x^4-2*x^3-13*x^2+14*x+24)
\end{eulerprompt}
\begin{eulerformula}
\[
\left[ x=-3 , x=-1 , x=2 , x=4 \right] 
\]
\end{eulerformula}
\eulersubheading{4.3 Exercise Set}
\begin{eulercomment}
\end{eulercomment}
\begin{eulerformula}
\[
f(x)=x^2+4x^2+x-6
\]
\end{eulerformula}
\begin{eulercomment}
penyelesaian :
\end{eulercomment}
\begin{eulerprompt}
>$&solve(x^3+4*x^2+x-6)
\end{eulerprompt}
\begin{eulerformula}
\[
\left[ x=-3 , x=-2 , x=1 \right] 
\]
\end{eulerformula}
\begin{eulercomment}
\end{eulercomment}
\begin{eulerformula}
\[
50). f(x)=x^4+x^3-3x^2-5x-2
\]
\end{eulerformula}
\begin{eulercomment}
Penyelesaian
\end{eulercomment}
\begin{eulerprompt}
>$&solve(x^4+x^3-3*x^2-5*x-2)
\end{eulerprompt}
\begin{eulerformula}
\[
\left[ x=2 , x=-1 \right] 
\]
\end{eulerformula}
\begin{eulercomment}
\end{eulercomment}
\begin{eulerformula}
\[
51). f(x)=x^3-7x+6
\]
\end{eulerformula}
\begin{eulercomment}
Penyelesaian
\end{eulercomment}
\begin{eulerprompt}
>$&solve(x^3-7*x+6)
\end{eulerprompt}
\begin{eulerformula}
\[
\left[ x=1 , x=2 , x=-3 \right] 
\]
\end{eulerformula}
\begin{eulercomment}
\end{eulercomment}
\begin{eulerformula}
\[
52). f(x)= x^3-12x+16
\]
\end{eulerformula}
\begin{eulercomment}
Penyelesaian :
\end{eulercomment}
\begin{eulerprompt}
>$&solve(x^3-12*x+16)
\end{eulerprompt}
\begin{eulerformula}
\[
\left[ x=-4 , x=2 \right] 
\]
\end{eulerformula}
\begin{eulercomment}
\end{eulercomment}
\begin{eulerformula}
\[
53). f(x)=-x^3+3x^2+6x-8
\]
\end{eulerformula}
\begin{eulerprompt}
>$&solve(-x^3+3*x^2+6*x-8)
\end{eulerprompt}
\begin{eulerformula}
\[
\left[ x=-2 , x=4 , x=1 \right] 
\]
\end{eulerformula}
\eulerheading{Menggambar Grafik 2D dengan EMT}
\begin{eulercomment}
Notebook ini menjelaskan tentang cara menggambar berbagaikurva dan
grafik 2D dengan software EMT. EMT menyediakan fungsi plot2d() untuk
menggambar berbagai kurva dan grafik dua dimensi (2D).\\
\end{eulercomment}
\eulersubheading{Plot Dasar}
\begin{eulercomment}
Ada beberapa fungsi dasar plot. Ada koordinat layar, yang selalu
berkisar dari 0 hingga 1024 di setiap sumbu, tidak peduli apakah
layarnya persegi atau tidak. Selain itu ada koordinat plot, yang dapat
diatur dengan setplot().Pemetaan antara koordinat bergantung pada
jendela plot saat ini. Misalnya, shrinkwindow() default menyisakan
ruang untuk  label sumbu dan judul plot.

Dalam contoh ini, kita hanya menggambar beberapa garis acak dengan
berbagai warna. Untuk detail tentang fungsi-fungsi ini, pelajari
fungsi-fungsi inti EMT.
\end{eulercomment}
\begin{eulerprompt}
>clg; // clear screen
>window(0,0,1024,1024); // use all of the window
>setplot(0,1,0,1); // set plot coordinates
>hold on; // start overwrite mode
>n=100; X=random(n,2); Y=random(n,2);  // get random points
>colors=rgb(random(n),random(n),random(n)); // get random colors
>loop 1 to n; color(colors[#]); plot(X[#],Y[#]); end; // plot
>hold off; // end overwrite mode
>insimg; // insert to notebook
\end{eulerprompt}
\eulerimg{27}{images/EMT 1-6_Mifta Dilla Aninditya_23030630029-186.png}
\begin{eulerprompt}
>reset;
\end{eulerprompt}
\begin{eulercomment}
Perlu untuk menahan grafik, karena perintah plot() akan menghapus
jendela plot.

Untuk menghapus semua yang telah kita lakukan, kita gunakan reset().

Untuk menampilkan gambar hasil plot di layar notebook, perintah
plot2d() dapat diakhiri dengan titik dua (:). Cara lain adalah
perintah plot2d() diakhiri dengan titik koma (;), kemudian menggunakan
perintah insimg() untuk menampilkan gambar hasil plot.

Untuk contoh lain, kita menggambar plot sebagai inset di plot lain.
Ini dilakukan dengan menentukan jendela plot yang lebih kecil.
Perhatikan bahwa jendela ini tidak menyediakan ruang untuk label sumbu
di luar jendela plot. Kita harus menambahkan beberapa margin untuk ini
sesuai kebutuhan. Perhatikan bahwa kita menyimpan dan memulihkan
jendela penuh, dan menahan plot saat ini saat kita memplot inset.
\end{eulercomment}
\begin{eulerprompt}
>plot2d("x^3-x");
>xw=200; yw=100; ww=300; hw=300;
>ow=window();
>window(xw,yw,xw+ww,yw+hw);
>hold on;
>barclear(xw-50,yw-10,ww+60,ww+60);
>plot2d("x^4-x",grid=6):
\end{eulerprompt}
\eulerimg{27}{images/EMT 1-6_Mifta Dilla Aninditya_23030630029-187.png}
\begin{eulerprompt}
>hold off;
>window(ow);
\end{eulerprompt}
\begin{eulercomment}
Plot dengan beberapa gambar dibuat dengan cara yang sama. Ada fungsi
utilitas figure() untuk ini.

\end{eulercomment}
\eulersubheading{Aspek Plot}
\begin{eulercomment}
Plot default menggunakan jendela plot persegi. Anda dapat mengubahnya
dengan fungsi aspect(). Jangan lupa untuk mengatur  ulang aspect
nanti. Anda juga dapat mengubah default ini di menu dengan "Set
Aspect" ke rasio aspek tertentu atau ke ukuran jendela grafik saat
ini.

Namun, Anda juga dapat mengubahnya untuk satu plot. Untuk ini, ukuran
area plot saat ini diubah, dan jendela diatur sehingga  label memiliki
cukup ruang.
\end{eulercomment}
\begin{eulerprompt}
>aspect(2); // rasio panjang dan lebar 2:1
>plot2d(["sin(x)","cos(x)"],0,2pi):
\end{eulerprompt}
\eulerimg{13}{images/EMT 1-6_Mifta Dilla Aninditya_23030630029-188.png}
\begin{eulerprompt}
>aspect();
>reset;
\end{eulerprompt}
\begin{eulercomment}
Fungsi reset() mengembalikan pengaturan plot default termasuk rasio
aspek.\\
Plot 2D dalam Euler

EMT Math Toolbox memiliki plot dalam 2D, baik untuk data maupun
fungsi. EMT menggunakan fungsi plot2d. Fungsi ini dapat  memplot
fungsi dan data.

Dimungkinkan untuk membuat plot di Maxima menggunakan Gnuplot atau di
Python menggunakan Math Plot Lib.

Euler dapat memplot plot 2D dari \\
- ekspresi\\
- fungsi, variabel, atau kurva berparameter,\\
- vektor nilai xy,\\
- titik-titik pada bidang,\\
- kurva implisit dengan level atau daerah level.\\
- Fungsi kompleks

Gaya plot mencakup berbagai gaya untuk garis dan titik, plot batang,
dan plot berbayang.

\begin{eulercomment}
\eulerheading{Plot Ekspresi atau Variabel}
\begin{eulercomment}
Ekspresi tunggal dalam "x" (misalnya "4*x\textasciicircum{}2") atau nama suatu fungsi
(misalnya "f") menghasilkan grafik fungsi tersebut.

Berikut adalah contoh paling dasar, yang menggunakan rentang default
dan menetapkan rentang y yang tepat agar sesuai dengan plot fungsi.

Catatan: Jika Anda mengakhiri baris perintah dengan titik dua ":",
plot akan dimasukkan ke dalam jendela teks. Jika tidak, tekan TAB
untuk melihat plot jika jendela plot tertutup.
\end{eulercomment}
\begin{eulerprompt}
>plot2d("x^2"):
\end{eulerprompt}
\eulerimg{27}{images/EMT 1-6_Mifta Dilla Aninditya_23030630029-189.png}
\begin{eulerprompt}
>aspect(1.5); plot2d("x^3-x"):
\end{eulerprompt}
\eulerimg{17}{images/EMT 1-6_Mifta Dilla Aninditya_23030630029-190.png}
\begin{eulerprompt}
>a:=5.6; plot2d("exp(-a*x^2)/a"); insimg(30); // menampilkan gambar hasil plot setinggi 25 baris
\end{eulerprompt}
\eulerimg{17}{images/EMT 1-6_Mifta Dilla Aninditya_23030630029-191.png}
\begin{eulercomment}
Dari beberapa contoh sebelumnya Anda dapat melihat bahwa aslinya
gambar plot menggunakan sumbu X dengan rentang nilai dari -2 sampai
dengan 2. Untuk mengubah rentang nilai X dan Y, Anda dapat menambahkan
nilai-nilai batas X (dan Y) di belakang ekspresi yang digambar.

Rentang plot ditetapkan dengan parameter yang ditetapkan berikut

- a,b: rentang-x (default -2,2)\\
- c,d: rentang-y (default: skala dengan nilai)\\
- r: alternatifnya radius di sekitar pusat plot\\
- cx,cy: koordinat pusat plot (default 0,0)
\end{eulercomment}
\begin{eulerprompt}
>plot2d("x^3-x",-1,2):
\end{eulerprompt}
\eulerimg{17}{images/EMT 1-6_Mifta Dilla Aninditya_23030630029-192.png}
\begin{eulerprompt}
>plot2d("sin(x)",-2*pi,2*pi): // plot sin(x) pada interval [-2pi, 2pi]
\end{eulerprompt}
\eulerimg{17}{images/EMT 1-6_Mifta Dilla Aninditya_23030630029-193.png}
\begin{eulerprompt}
>plot2d("cos(x)","sin(3*x)",xmin=0,xmax=2pi):
\end{eulerprompt}
\eulerimg{17}{images/EMT 1-6_Mifta Dilla Aninditya_23030630029-194.png}
\begin{eulercomment}
Alternatif untuk titik dua adalah perintah insimg(baris), yang
menyisipkan plot yang menempati sejumlah baris teks tertentu.

Dalam opsi, plot dapat diatur agar muncul

- di jendela terpisah yang dapat diubah ukurannya,\\
- di jendela catatan.


Lebih banyak gaya dapat dicapai dengan perintah plot yang spesifik.

Bagaimanapun, tekan tombol tabulator untuk melihat plot, jika
tersembunyi.

Untuk membagi jendela menjadi beberapa plot, gunakan perintah
figure(). Dalam contoh ini, kita memplot x\textasciicircum{}1 hingga x\textasciicircum{}4 ke dalam 4
bagian jendela. figure(0) akan mengatur ulang jendela default.
\end{eulercomment}
\begin{eulerprompt}
>reset;
>figure(2,2); ...
>for n=1 to 4; figure(n); plot2d("x^"+n); end; ...
>figure(0):
\end{eulerprompt}
\eulerimg{27}{images/EMT 1-6_Mifta Dilla Aninditya_23030630029-195.png}
\begin{eulercomment}
Dalam plot2d(), tersedia gaya alternatif dengan grid=x. Sebagai
gambaran umum, kami menampilkan berbagai gaya grid dalam satu gambar
(lihat di bawah untuk perintah figure()). Gaya grid=0 tidak
disertakan. Gaya ini tidak menampilkan grid dan bingkai.
\end{eulercomment}
\begin{eulerprompt}
>figure(3,3); ...
>for k=1:9; figure(k); plot2d("x^3-x",-2,1,grid=k); end; ...
>figure(0):
\end{eulerprompt}
\eulerimg{27}{images/EMT 1-6_Mifta Dilla Aninditya_23030630029-196.png}
\begin{eulercomment}
Jika argumen untuk plot2d() adalah ekspresi yang diikuti oleh empat
angka, angka-angka ini adalah rentang x dan y untuk plot.

Sebagai alternatif, a, b, c, d dapat ditetapkan sebagai parameter yang
ditetapkan sebagai a=... dst.

Dalam contoh berikut, kami mengubah gaya kisi, menambahkan label, dan
menggunakan label vertikal untuk sumbu y.
\end{eulercomment}
\begin{eulerprompt}
>aspect(1.5); plot2d("sin(x)",0,2pi,-1.2,1.2,grid=3,xl="x",yl="sin(x)"):
\end{eulerprompt}
\eulerimg{17}{images/EMT 1-6_Mifta Dilla Aninditya_23030630029-197.png}
\begin{eulerprompt}
>plot2d("sin(x)+cos(2*x)",0,4pi):
\end{eulerprompt}
\eulerimg{17}{images/EMT 1-6_Mifta Dilla Aninditya_23030630029-198.png}
\begin{eulercomment}
Gambar yang dihasilkan dengan memasukkan plot ke dalam jendela teks
disimpan dalam direktori yang sama dengan buku catatan, secara default
dalam subdirektori bernama "images". Gambar tersebut juga digunakan
oleh ekspor HTML.

Anda cukup menandai gambar apa saja dan menyalinnya ke clipboard
dengan Ctrl-C. Tentu saja, Anda juga dapat mengekspor  grafik saat ini
dengan fungsi-fungsi di menu File.

Fungsi atau ekspresi dalam plot2d dievaluasi secara adaptif. Untuk
kecepatan lebih tinggi, matikan plot adaptif dengan \textless{}adaptive dan
tentukan jumlah subinterval dengan n=... Ini hanya diperlukan dalam
kasus yang jarang terjadi.
\end{eulercomment}
\begin{eulerprompt}
>plot2d("sign(x)*exp(-x^2)",-1,1,<adaptive,n=10000):
\end{eulerprompt}
\eulerimg{17}{images/EMT 1-6_Mifta Dilla Aninditya_23030630029-199.png}
\begin{eulerprompt}
>plot2d("x^x",r=1.2,cx=1,cy=1):
\end{eulerprompt}
\eulerimg{17}{images/EMT 1-6_Mifta Dilla Aninditya_23030630029-200.png}
\begin{eulercomment}
Perhatikan bahwa x\textasciicircum{}x tidak didefinisikan untuk x\textless{}=0. Fungsi plot2d
menangkap kesalahan ini, dan mulai memplot segera setelah fungsi
didefinisikan. Ini berfungsi untuk semua fungsi yang mengembalikan NAN
di luar rentang definisinya.
\end{eulercomment}
\begin{eulerprompt}
>plot2d("log(x)",-0.1,2):
\end{eulerprompt}
\eulerimg{17}{images/EMT 1-6_Mifta Dilla Aninditya_23030630029-201.png}
\begin{eulercomment}
Parameter square=true (atau \textgreater{}square) memilih rentang y secara otomatis
sehingga hasilnya adalah jendela plot persegi.  Perhatikan bahwa
secara default, Euler menggunakan ruang persegi di dalam jendela plot.
\end{eulercomment}
\begin{eulerprompt}
>plot2d("x^3-x",>square):
\end{eulerprompt}
\eulerimg{17}{images/EMT 1-6_Mifta Dilla Aninditya_23030630029-202.png}
\begin{eulerprompt}
>plot2d(''integrate("sin(x)*exp(-x^2)",0,x)'',0,2): // plot integral
\end{eulerprompt}
\eulerimg{17}{images/EMT 1-6_Mifta Dilla Aninditya_23030630029-203.png}
\begin{eulercomment}
Jika Anda memerlukan lebih banyak ruang untuk label-y, panggil
shrinkwindow() dengan parameter yang lebih kecil, atau tetapkan nilai
positif untuk "lebih kecil" di plot2d().
\end{eulercomment}
\begin{eulerprompt}
>plot2d("gamma(x)",1,10,yl="y-values",smaller=6,<vertical):
\end{eulerprompt}
\eulerimg{17}{images/EMT 1-6_Mifta Dilla Aninditya_23030630029-204.png}
\begin{eulercomment}
Ekspresi simbolik juga dapat digunakan, karena disimpan sebagai
ekspresi string sederhana.
\end{eulercomment}
\begin{eulerprompt}
>x=linspace(0,2pi,1000); plot2d(sin(5x),cos(7x)):
\end{eulerprompt}
\eulerimg{17}{images/EMT 1-6_Mifta Dilla Aninditya_23030630029-205.png}
\begin{eulerprompt}
>a:=5.6; expr &= exp(-a*x^2)/a; // define expression
>plot2d(expr,-2,2): // plot from -2 to 2
\end{eulerprompt}
\eulerimg{17}{images/EMT 1-6_Mifta Dilla Aninditya_23030630029-206.png}
\begin{eulerprompt}
>plot2d(expr,r=1,thickness=2): // plot in a square around (0,0)
\end{eulerprompt}
\eulerimg{17}{images/EMT 1-6_Mifta Dilla Aninditya_23030630029-207.png}
\begin{eulerprompt}
>plot2d(&diff(expr,x),>add,style="--",color=red): // add another plot
\end{eulerprompt}
\eulerimg{17}{images/EMT 1-6_Mifta Dilla Aninditya_23030630029-208.png}
\begin{eulerprompt}
>plot2d(&diff(expr,x,2),a=-2,b=2,c=-2,d=1): // plot in rectangle
\end{eulerprompt}
\eulerimg{17}{images/EMT 1-6_Mifta Dilla Aninditya_23030630029-209.png}
\begin{eulerprompt}
>plot2d(&diff(expr,x),a=-2,b=2,>square): // keep plot square
\end{eulerprompt}
\eulerimg{17}{images/EMT 1-6_Mifta Dilla Aninditya_23030630029-210.png}
\begin{eulerprompt}
>plot2d("x^2",0,1,steps=1,color=red,n=10):
\end{eulerprompt}
\eulerimg{17}{images/EMT 1-6_Mifta Dilla Aninditya_23030630029-211.png}
\begin{eulerprompt}
>plot2d("x^2",>add,steps=2,color=blue,n=10):
\end{eulerprompt}
\eulerimg{17}{images/EMT 1-6_Mifta Dilla Aninditya_23030630029-212.png}
\eulerheading{Fungsi dalam Satu Parameter}
\begin{eulercomment}
Fungsi plotting yang paling penting untuk plot planar adalah plot2d().
Fungsi ini diimplementasikan dalam bahasa Euler dalam  file "plot.e",
yang dimuat di awal program.

Berikut ini beberapa contoh penggunaan fungsi. Seperti biasa dalam
EMT, fungsi yang berfungsi untuk fungsi atau ekspresi  lain, Anda
dapat meneruskan parameter tambahan (selain x) yang bukan variabel
global ke fungsi dengan parameter titik koma  atau dengan kumpulan
panggilan.
\end{eulercomment}
\begin{eulerprompt}
>function f(x,a) := x^2/a+a*x^2-x; // define a function
>a=0.3; plot2d("f",0,1;a): // plot with a=0.3
\end{eulerprompt}
\eulerimg{17}{images/EMT 1-6_Mifta Dilla Aninditya_23030630029-213.png}
\begin{eulerprompt}
>plot2d("f",0,1;0.4): // plot with a=0.4
\end{eulerprompt}
\eulerimg{17}{images/EMT 1-6_Mifta Dilla Aninditya_23030630029-214.png}
\begin{eulerprompt}
>plot2d(\{\{"f",0.2\}\},0,1): // plot with a=0.2
\end{eulerprompt}
\eulerimg{17}{images/EMT 1-6_Mifta Dilla Aninditya_23030630029-215.png}
\begin{eulerprompt}
>plot2d(\{\{"f(x,b)",b=0.1\}\},0,1): // plot with 0.1
\end{eulerprompt}
\eulerimg{17}{images/EMT 1-6_Mifta Dilla Aninditya_23030630029-216.png}
\begin{eulerprompt}
>function f(x) := x^3-x; ...
>plot2d("f",r=1):
\end{eulerprompt}
\eulerimg{17}{images/EMT 1-6_Mifta Dilla Aninditya_23030630029-217.png}
\begin{eulercomment}
Berikut ini adalah ringkasan fungsi yang diterima

- ekspresi atau ekspresi simbolik dalam x\\
- fungsi atau fungsi simbolik berdasarkan nama seperti "f"\\
- fungsi simbolik hanya berdasarkan nama f

Fungsi plot2d() juga menerima fungsi simbolik. Untuk fungsi simbolik,
hanya nama yang berfungsi.
\end{eulercomment}
\begin{eulerprompt}
>function f(x) &= diff(x^x,x)
\end{eulerprompt}
\begin{euleroutput}
  
                              x
                             x  (log(x) + 1)
  
\end{euleroutput}
\begin{eulerprompt}
>plot2d(f,0,2):
\end{eulerprompt}
\eulerimg{17}{images/EMT 1-6_Mifta Dilla Aninditya_23030630029-218.png}
\begin{eulercomment}
Tentu saja, untuk ekspresi atau ungkapan simbolik, nama variabel sudah
cukup untuk memplotnya.
\end{eulercomment}
\begin{eulerprompt}
>expr &= sin(x)*exp(-x)
\end{eulerprompt}
\begin{euleroutput}
  
                                - x
                               E    sin(x)
  
\end{euleroutput}
\begin{eulerprompt}
>plot2d(expr,0,3pi):
\end{eulerprompt}
\eulerimg{17}{images/EMT 1-6_Mifta Dilla Aninditya_23030630029-219.png}
\begin{eulerprompt}
>function f(x) &= x^x;
>plot2d(f,r=1,cx=1,cy=1,color=blue,thickness=2);
>plot2d(&diff(f(x),x),>add,color=red,style="-.-"):
\end{eulerprompt}
\eulerimg{17}{images/EMT 1-6_Mifta Dilla Aninditya_23030630029-220.png}
\begin{eulercomment}
Untuk garis gaya, ada berbagai pilihan.

- style="...". Pilih dari "-", "--", "-.", ".", ".-.", "-.-".\\
- warna: Lihat dibawah untuk warna.\\
- ketebalan: Defaultnya adalah 1.

Warna dapat dipilih sebagai salah satu warna default, atau sebagai
warna RGB.

- 0..15: indeks warna default.\\
- warna konstan: putih, hitam, merah, hijau, biru, biru kehijauan,
hijau kekuningan, abu-abu muda, abu-abu tua, oranye, hijau muda,
toska, biru muda, oranye muda, kuning \\
- rgb(merah,hijau,biru): parameternya adalah bilangan riil dalam
[0,1].
\end{eulercomment}
\begin{eulerprompt}
>plot2d("exp(-x^2)",r=2,color=red,thickness=3,style="--"):
\end{eulerprompt}
\eulerimg{17}{images/EMT 1-6_Mifta Dilla Aninditya_23030630029-221.png}
\begin{eulercomment}
Berikut ini tampilan warna EMT yang telah ditetapkan sebelumnya.
\end{eulercomment}
\begin{eulerprompt}
>aspect(2); columnsplot(ones(1,16),lab=0:15,grid=0,color=0:15):
\end{eulerprompt}
\eulerimg{13}{images/EMT 1-6_Mifta Dilla Aninditya_23030630029-222.png}
\begin{eulercomment}
Namun anda dapat menggunakan warna apapun.
\end{eulercomment}
\begin{eulerprompt}
>columnsplot(ones(1,16),grid=0,color=rgb(0,0,linspace(0,1,15))):
\end{eulerprompt}
\eulerimg{13}{images/EMT 1-6_Mifta Dilla Aninditya_23030630029-223.png}
\eulerheading{Menggambar Beberapa Kurva pada bidang koordinat yang sama}
\begin{eulercomment}
Plotting lebih dari satu fungsi (multifungsi) ke dalam satu jendela
dapat dilakukan dengan berbagai cara. Salah satu metodenya adalah
menggunakan \textgreater{}add untuk beberapa panggilan ke plot2d secara
keseluruhan, kecuali panggilan pertama. Kami telah  menggunakan fitur
ini pada contoh di atas.
\end{eulercomment}
\begin{eulerprompt}
>aspect(); plot2d("cos(x)",r=2,grid=6); plot2d("x",style=".",>add):
\end{eulerprompt}
\eulerimg{27}{images/EMT 1-6_Mifta Dilla Aninditya_23030630029-224.png}
\begin{eulerprompt}
>aspect(1.5); plot2d("sin(x)",0,2pi); plot2d("cos(x)",color=blue,style="--",>add):
\end{eulerprompt}
\eulerimg{17}{images/EMT 1-6_Mifta Dilla Aninditya_23030630029-225.png}
\begin{eulercomment}
Salah satu kegunaan \textgreater{}add adalah untuk menambahkan titik pada kurva.
\end{eulercomment}
\begin{eulerprompt}
>plot2d("sin(x)",0,pi); plot2d(2,sin(2),>points,>add):
\end{eulerprompt}
\eulerimg{17}{images/EMT 1-6_Mifta Dilla Aninditya_23030630029-226.png}
\begin{eulercomment}
Kami menambahkan titik persimpangan dengan label (pada posisi "cl"
untuk tengah kiri), dan memasukkan hasilnya ke dalam buku catatan.
Kami juga menambahkan judul pada plot.
\end{eulercomment}
\begin{eulerprompt}
>plot2d(["cos(x)","x"],r=1.1,cx=0.5,cy=0.5, ...
>  color=[black,blue],style=["-","."], ...
>  grid=1);
>x0=solve("cos(x)-x",1);  ...
>  plot2d(x0,x0,>points,>add,title="Intersection Demo");  ...
>  label("cos(x) = x",x0,x0,pos="cl",offset=20):
\end{eulerprompt}
\eulerimg{17}{images/EMT 1-6_Mifta Dilla Aninditya_23030630029-227.png}
\begin{eulercomment}
Dalam demo berikut, kami memplot fungsi sinc(x)=sin(x)/x dan ekspansi
Taylor ke-8 dan ke-16. Kami menghitung ekspansi ini  menggunakan
Maxima melalui ekspresi simbolik.\\
Plot ini dilakukan dalam perintah multi-baris berikut dengan tiga
panggilan ke plot2d(). Yang kedua dan ketiga memiliki tanda \textgreater{}add yang
ditetapkan, yang membuat plot menggunakan rentang sebelumnya.

Kami menambahkan kotak label yang menjelaskan fungsinya.
\end{eulercomment}
\begin{eulerprompt}
>$taylor(sin(x)/x,x,0,4)
\end{eulerprompt}
\begin{eulerformula}
\[
\frac{x^4}{120}-\frac{x^2}{6}+1
\]
\end{eulerformula}
\begin{eulerprompt}
>plot2d("sinc(x)",0,4pi,color=green,thickness=2); ...
>  plot2d(&taylor(sin(x)/x,x,0,8),>add,color=blue,style="--"); ...
>  plot2d(&taylor(sin(x)/x,x,0,16),>add,color=red,style="-.-"); ...
>  labelbox(["sinc","T8","T16"],styles=["-","--","-.-"], ...
>    colors=[black,blue,red]):
\end{eulerprompt}
\eulerimg{17}{images/EMT 1-6_Mifta Dilla Aninditya_23030630029-229.png}
\begin{eulercomment}
Dalam contoh berikut, kami menghasilkan Polinomial Bernstein.

\end{eulercomment}
\begin{eulerformula}
\[
B_i(x) = \binom{n}{i} x^i (1-x)^{n-i}
\]
\end{eulerformula}
\begin{eulerprompt}
>plot2d("(1-x)^10",0,1); // plot first function
>for i=1 to 10; plot2d("bin(10,i)*x^i*(1-x)^(10-i)",>add); end;
>insimg;
\end{eulerprompt}
\eulerimg{17}{images/EMT 1-6_Mifta Dilla Aninditya_23030630029-230.png}
\begin{eulercomment}
Metode kedua menggunakan sepasang matriks nilai-x dan matriks nilai-y
yang berukuran sama.

Kami membuat matriks nilai dengan satu Bernstein-Polinomial di setiap
baris. Untuk ini, kami cukup menggunakan vektor kolom i. Lihat
pengantar tentang bahasa matriks untuk mempelajari lebih detail.
\end{eulercomment}
\begin{eulerprompt}
>x=linspace(0,1,500);
>n=10; k=(0:n)'; // n is row vector, k is column vector
>y=bin(n,k)*x^k*(1-x)^(n-k); // y is a matrix then
>plot2d(x,y):
\end{eulerprompt}
\eulerimg{17}{images/EMT 1-6_Mifta Dilla Aninditya_23030630029-231.png}
\begin{eulercomment}
Perhatikan bahwa parameter warna dapat berupa vektor. Maka setiap
warna digunakan untuk setiap baris matriks.
\end{eulercomment}
\begin{eulerprompt}
>x=linspace(0,1,200); y=x^(1:10)'; plot2d(x,y,color=1:10):
\end{eulerprompt}
\eulerimg{17}{images/EMT 1-6_Mifta Dilla Aninditya_23030630029-232.png}
\begin{eulercomment}
Metode lain adalah menggunakan vektor ekspresi (string). Anda kemudian
dapat menggunakan array warna, array gaya, dan array ketebalan dengan
panjang yang sama.
\end{eulercomment}
\begin{eulerprompt}
>plot2d(["sin(x)","cos(x)"],0,2pi,color=4:5): 
\end{eulerprompt}
\eulerimg{17}{images/EMT 1-6_Mifta Dilla Aninditya_23030630029-233.png}
\begin{eulerprompt}
>plot2d(["sin(x)","cos(x)"],0,2pi): // plot vector of expressions
\end{eulerprompt}
\eulerimg{17}{images/EMT 1-6_Mifta Dilla Aninditya_23030630029-234.png}
\begin{eulercomment}
Kita bisa mendapatkan vektor tersebut dari Maxima menggunakan
makelist() dan mxm2str().
\end{eulercomment}
\begin{eulerprompt}
>v &= makelist(binomial(10,i)*x^i*(1-x)^(10-i),i,0,10) // make list
\end{eulerprompt}
\begin{euleroutput}
  
                  10            9              8  2             7  3
          [(1 - x)  , 10 (1 - x)  x, 45 (1 - x)  x , 120 (1 - x)  x , 
             6  4             5  5             4  6             3  7
  210 (1 - x)  x , 252 (1 - x)  x , 210 (1 - x)  x , 120 (1 - x)  x , 
            2  8              9   10
  45 (1 - x)  x , 10 (1 - x) x , x  ]
  
\end{euleroutput}
\begin{eulerprompt}
>mxm2str(v) // get a vector of strings from the symbolic vector
\end{eulerprompt}
\begin{euleroutput}
  (1-x)^10
  10*(1-x)^9*x
  45*(1-x)^8*x^2
  120*(1-x)^7*x^3
  210*(1-x)^6*x^4
  252*(1-x)^5*x^5
  210*(1-x)^4*x^6
  120*(1-x)^3*x^7
  45*(1-x)^2*x^8
  10*(1-x)*x^9
  x^10
\end{euleroutput}
\begin{eulerprompt}
>plot2d(mxm2str(v),0,1): // plot functions
\end{eulerprompt}
\eulerimg{17}{images/EMT 1-6_Mifta Dilla Aninditya_23030630029-235.png}
\begin{eulercomment}
Alternatif lain adalah menggunakan bahasa matriks Euler.

Jika suatu ekspresi menghasilkan matriks fungsi, dengan satu fungsi
pada setiap baris, semua fungsi ini akan diplot menjadi satu plot.

Untuk ini, gunakan vektor parameter dalam bentuk vektor kolom. Jika
array warna ditambahkan, array tersebut akan digunakan  untuk setiap
baris plot.
\end{eulercomment}
\begin{eulerprompt}
>n=(1:10)'; plot2d("x^n",0,1,color=1:10):
\end{eulerprompt}
\eulerimg{17}{images/EMT 1-6_Mifta Dilla Aninditya_23030630029-236.png}
\begin{eulercomment}
Ekspresi dan fungsi satu baris dapat melihat variabel global.

Jika Anda tidak dapat menggunakan variabel global, Anda perlu
menggunakan fungsi dengan parameter tambahan, dan meneruskan parameter
ini sebagai parameter titik koma.

Berhati-hatilah untuk meletakkan semua parameter yang ditetapkan di
akhir perintah plot2d. Dalam contoh ini, kita meneruskan  a=5 ke
fungsi f, yang kita plot dari -10 hingga 10.
\end{eulercomment}
\begin{eulerprompt}
>function f(x,a) := 1/a*exp(-x^2/a); ...
>plot2d("f",-10,10;5,thickness=2,title="a=5"):
\end{eulerprompt}
\eulerimg{17}{images/EMT 1-6_Mifta Dilla Aninditya_23030630029-237.png}
\begin{eulercomment}
Atau, gunakan koleksi dengan nama fungsi dan semua parameter tambahan.
Daftar khusus ini disebut koleksi panggilan, dan itu  adalah cara yang
lebih disukai untuk meneruskan argumen ke suatu fungsi yang diteruskan
sebagai argumen ke fungsi lain.

Dalam contoh berikut, kami menggunakan loop untuk memplot beberapa
fungsi (lihat tutorial tentang pemrograman untuk loop).
\end{eulercomment}
\begin{eulerprompt}
>plot2d(\{\{"f",1\}\},-10,10); ...
>for a=2:10; plot2d(\{\{"f",a\}\},>add); end:
\end{eulerprompt}
\eulerimg{17}{images/EMT 1-6_Mifta Dilla Aninditya_23030630029-238.png}
\begin{eulercomment}
Kita dapat memperoleh hasil yang sama dengan cara berikut menggunakan
bahasa matriks EMT. Setiap baris matriks f(x,a) adalah satu fungsi.\\
Selain itu, kita dapat mengatur warna untuk setiap baris matriks. Klik
dua kali pada fungsi getspectral() untuk penjelasannya.
\end{eulercomment}
\begin{eulerprompt}
>x=-10:0.01:10; a=(1:10)'; plot2d(x,f(x,a),color=getspectral(a/10)):
\end{eulerprompt}
\eulerimg{17}{images/EMT 1-6_Mifta Dilla Aninditya_23030630029-239.png}
\eulersubheading{Label Teks}
\begin{eulercomment}
Dekorasi sederhana bisa berupa

- judul dengan title="..."\\
- label x dan y dengan xl="...", yl="..."\\
- label teks dengan label("...",x,y)

Perintah label akan memplot ke dalam plot saat ini pada koordinat plot
(x,y). Perintah ini dapat mengambil argumen posisi.
\end{eulercomment}
\begin{eulerprompt}
>plot2d("x^3-x",-1,2,title="y=x^3-x",yl="y",xl="x"):
\end{eulerprompt}
\eulerimg{17}{images/EMT 1-6_Mifta Dilla Aninditya_23030630029-240.png}
\begin{eulerprompt}
>expr := "log(x)/x"; ...
>  plot2d(expr,0.5,5,title="y="+expr,xl="x",yl="y"); ...
>  label("(1,0)",1,0); label("Max",E,expr(E),pos="lc"):
\end{eulerprompt}
\eulerimg{17}{images/EMT 1-6_Mifta Dilla Aninditya_23030630029-241.png}
\begin{eulercomment}
Ada juga fungsi labelbox(), yang dapat menampilkan fungsi dan teks.
Fungsi ini mengambil vektor string dan warna, satu item untuk setiap
fungsi.
\end{eulercomment}
\begin{eulerprompt}
>function f(x) &= x^2*exp(-x^2);  ...
>plot2d(&f(x),a=-3,b=3,c=-1,d=1);  ...
>plot2d(&diff(f(x),x),>add,color=blue,style="--"); ...
>labelbox(["function","derivative"],styles=["-","--"], ...
>   colors=[black,blue],w=0.4):
\end{eulerprompt}
\eulerimg{17}{images/EMT 1-6_Mifta Dilla Aninditya_23030630029-242.png}
\begin{eulercomment}
Kotak tersebut ditambatkan di kanan atas secara default, tetapi \textgreater{}left
menambatkannya di kiri atas. Anda dapat memindahkannya ke  tempat mana
pun yang Anda suka. Posisi jangkar adalah sudut kanan atas kotak, dan
angka-angkanya adalah pecahan dari ukuran  jendela grafis. Lebarnya
otomatis.

Untuk plot titik, kotak label juga berfungsi. Tambahkan parameter
\textgreater{}points, atau vektor bendera, satu untuk setiap label.

Dalam contoh berikut, hanya ada satu fungsi. Jadi, kita dapat
menggunakan string, bukan vektor string. Kita tetapkan warna teks
menjadi hitam untuk contoh ini. 
\end{eulercomment}
\begin{eulerprompt}
>n=10; plot2d(0:n,bin(n,0:n),>addpoints); ...
>labelbox("Binomials",styles="[]",>points,x=0.1,y=0.1, ...
>tcolor=black,>left):
\end{eulerprompt}
\eulerimg{17}{images/EMT 1-6_Mifta Dilla Aninditya_23030630029-243.png}
\begin{eulercomment}
Gaya plot ini juga tersedia di statplot(). Seperti di plot2d(), warna
dapat diatur untuk setiap baris plot. Ada plot yang lebih khusus untuk
keperluan statistik (lihat tutorial tentang statistik).
\end{eulercomment}
\begin{eulerprompt}
>statplot(1:10,random(2,10),color=[red,blue]):
\end{eulerprompt}
\eulerimg{17}{images/EMT 1-6_Mifta Dilla Aninditya_23030630029-244.png}
\begin{eulercomment}
Fitur serupa adalah fungsi textbox().

Lebarnya secara default adalah lebar maksimal baris teks. Namun,
lebarnya juga dapat diatur oleh pengguna.
\end{eulercomment}
\begin{eulerprompt}
>function f(x) &= exp(-x)*sin(2*pi*x); ...
>plot2d("f(x)",0,2pi); ...
>textbox(latex("\(\backslash\)text\{Example of a damped oscillation\}\(\backslash\) f(x)=e^\{-x\}sin(2\(\backslash\)pi x)"),w=0.85):
\end{eulerprompt}
\eulerimg{17}{images/EMT 1-6_Mifta Dilla Aninditya_23030630029-245.png}
\begin{eulercomment}
Label teks, judul, kotak label, dan teks lainnya dapat berisi string
Unicode (lihat sintaksis EMT untuk informasi lebih lanjut tentang
string Unicode).
\end{eulercomment}
\begin{eulerprompt}
>plot2d("x^3-x",title=u"x &rarr; x&sup3; - x"):
\end{eulerprompt}
\eulerimg{17}{images/EMT 1-6_Mifta Dilla Aninditya_23030630029-246.png}
\begin{eulercomment}
Label pada sumbu x dan y dapat vertikal, begitu pula sumbunya. 
\end{eulercomment}
\begin{eulerprompt}
>plot2d("sinc(x)",0,2pi,xl="x",yl=u"x &rarr; sinc(x)",>vertical):
\end{eulerprompt}
\eulerimg{17}{images/EMT 1-6_Mifta Dilla Aninditya_23030630029-247.png}
\eulersubheading{LaTeX}
\begin{eulercomment}
Anda juga dapat memplot rumus LaTeX jika Anda telah menginstal sistem
LaTeX. Saya merekomendasikan MiKTeX. Jalur ke biner "latex" dan
"dvipng" harus berada di jalur sistem, atau Anda harus mengatur LaTeX
di opsi menu.

Perlu dicatat, bahwa penguraian LaTeX berlangsung lambat. Jika Anda
ingin menggunakan LaTeX dalam plot animasi, Anda harus memanggil
latex() sebelum loop sekali dan menggunakan hasilnya (gambar dalam
matriks RGB).

Pada plot berikut, kami menggunakan LaTeX untuk label x dan y, label,
kotak label, dan judul plot.
\end{eulercomment}
\begin{eulerprompt}
>plot2d("exp(-x)*sin(x)/x",a=0,b=2pi,c=0,d=1,grid=6,color=blue, ...
>  title=latex("\(\backslash\)text\{Function $\(\backslash\)Phi$\}"), ...
>  xl=latex("\(\backslash\)phi"),yl=latex("\(\backslash\)Phi(\(\backslash\)phi)")); ...
>textbox( ...
>  latex("\(\backslash\)Phi(\(\backslash\)phi) = e^\{-\(\backslash\)phi\} \(\backslash\)frac\{\(\backslash\)sin(\(\backslash\)phi)\}\{\(\backslash\)phi\}"),x=0.8,y=0.5); ...
>label(latex("\(\backslash\)Phi",color=blue),1,0.4):
\end{eulerprompt}
\eulerimg{17}{images/EMT 1-6_Mifta Dilla Aninditya_23030630029-248.png}
\begin{eulercomment}
Seringkali, kita menginginkan spasi nonkonformal dan label teks pada
sumbu x. Kita dapat menggunakan xaxis() dan yaxis() seperti yang akan
kita tunjukkan nanti.

Cara termudah adalah dengan membuat plot kosong dengan bingkai
menggunakan grid=4, lalu menambahkan grid dengan ygrid() dan xgrid().
Dalam contoh berikut, kami menggunakan tiga string LaTeX untuk label
pada sumbu x dengan xtick().
\end{eulercomment}
\begin{eulerprompt}
>plot2d("sinc(x)",0,2pi,grid=4,<ticks); ...
>ygrid(-2:0.5:2,grid=6); ...
>xgrid([0:2]*pi,<ticks,grid=6);  ...
>xtick([0,pi,2pi],["0","\(\backslash\)pi","2\(\backslash\)pi"],>latex):
\end{eulerprompt}
\eulerimg{17}{images/EMT 1-6_Mifta Dilla Aninditya_23030630029-249.png}
\begin{eulercomment}
Tentu saja, fungsi juga dapat digunakan. 
\end{eulercomment}
\begin{eulerprompt}
>function map f(x) ...
\end{eulerprompt}
\begin{eulerudf}
  if x>0 then return x^4
  else return x^2
  endif
  endfunction
\end{eulerudf}
\begin{eulercomment}
Parameter "map" membantu penggunaan fungsi untuk vektor. Untuk plot,
parameter ini tidak diperlukan. Namun, untuk menunjukkan bahwa
vektorisasi berguna, kami menambahkan beberapa poin penting ke plot
pada x=-1, x=0, dan x=1.

Pada plot berikut, kami juga memasukkan beberapa kode LaTeX. Kami
menggunakannya untuk dua label dan satu kotak teks. Tentu saja, Anda
hanya dapat menggunakan LaTeX jika Anda telah menginstal LaTeX dengan
benar.
\end{eulercomment}
\begin{eulerprompt}
>plot2d("f",-1,1,xl="x",yl="f(x)",grid=6);  ...
>plot2d([-1,0,1],f([-1,0,1]),>points,>add); ...
>label(latex("x^3"),0.72,f(0.72)); ...
>label(latex("x^2"),-0.52,f(-0.52),pos="ll"); ...
>textbox( ...
>  latex("f(x)=\(\backslash\)begin\{cases\} x^3 & x>0 \(\backslash\)\(\backslash\) x^2 & x \(\backslash\)le 0\(\backslash\)end\{cases\}"), ...
>  x=0.7,y=0.2):
\end{eulerprompt}
\eulerimg{17}{images/EMT 1-6_Mifta Dilla Aninditya_23030630029-250.png}
\begin{eulercomment}
\end{eulercomment}
\eulersubheading{Interaksi User}
\begin{eulercomment}
Saat memplot fungsi atau ekspresi, parameter \textgreater{}user memungkinkan
pengguna untuk memperbesar dan menggeser plot dengan tombol kursor
atau mouse. Pengguna dapat

- zoom dengan + or -\\
- memindahkan plot dengan tombol kursor\\
- memilih jendela plot dengan mouse\\
- mengatur ulang tampilan dengan spasi\\
- keluar dengan return

Tombol spasi akan mengatur ulang plot ke jendela plot asli.

Saat memetakan data, tanda \textgreater{}user akan menunggu penekanan tombol.
\end{eulercomment}
\begin{eulerprompt}
>plot2d(\{\{"x^3-a*x",a=1\}\},>user,title="Press any key!"):
\end{eulerprompt}
\eulerimg{17}{images/EMT 1-6_Mifta Dilla Aninditya_23030630029-251.png}
\begin{eulerprompt}
>plot2d("exp(x)*sin(x)",user=true, ...
>  title="+/- or cursor keys (return to exit)"):
\end{eulerprompt}
\eulerimg{17}{images/EMT 1-6_Mifta Dilla Aninditya_23030630029-252.png}
\begin{eulercomment}
Berikut ini menunjukkan cara interaksi pengguna yang canggih (lihat
tutorial tentang pemrograman untuk detailnya).

Fungsi bawaan mousedrag() menungggu kejadian mouse atau keyboard.
Fungsi ini melaporkan gerakan mouse ke atas dan penekanan tombol.
Fungsi dragpoints() memanfaatkan fungsi ini, dan memungkinkan pengguna
menyeret titik m pun dalam plot.

Pertama-tama, kita perlu fungsi plot. Misalnya, kita melakukan
interpolasi pada 5 titik dengan polinomial. Fungsi tersebutharus
diplot ke dalam area plot yang tetap.
\end{eulercomment}
\begin{eulerprompt}
>function plotf(xp,yp,select) ...
\end{eulerprompt}
\begin{eulerudf}
    d=interp(xp,yp);
    plot2d("interpval(xp,d,x)";d,xp,r=2);
    plot2d(xp,yp,>points,>add);
    if select>0 then
      plot2d(xp[select],yp[select],color=red,>points,>add);
    endif;
    title("Drag one point, or press space or return!");
  endfunction
\end{eulerudf}
\begin{eulercomment}
Perhatikan parameter titik koma di plot2d (d dan xp), yang diteruskan
ke evaluasi fungsi interp(). Tanpa ini, kita harus menulis fungsi
plotinterp() terlebih dahulu, yang mengakses nilai secara global.

Sekarang kita buat beberapa nilai acak dan biarkan pengguna menyeret
titik-titiknya.
\end{eulercomment}
\begin{eulerprompt}
>t=-1:0.5:1; dragpoints("plotf",t,random(size(t))-0.5):
\end{eulerprompt}
\eulerimg{17}{images/EMT 1-6_Mifta Dilla Aninditya_23030630029-253.png}
\begin{eulercomment}
Ada juga fungsi yang memplot fungsi lain tergantung pada vektor
parameter, dan memungkinkan pengguna menyesuaikan parameter ini.

Pertama kita butuh fungsi plot.
\end{eulercomment}
\begin{eulerprompt}
>function plotf([a,b]) := plot2d("exp(a*x)*cos(2pi*b*x)",0,2pi;a,b);
\end{eulerprompt}
\begin{eulercomment}
Kemudian kita perlu nama untuk parameter, nilai awal dan matriks
rentang nx2, secara opsional baris judul.\\
Terdapat slider interaktif, yang dapat mengatur nilai oleh pengguna.
Fungsi dragvalues() menyediakannya.
\end{eulercomment}
\begin{eulerprompt}
>dragvalues("plotf",["a","b"],[-1,2],[[-2,2];[1,10]], ...
>  heading="Drag these values:",hcolor=black):
\end{eulerprompt}
\eulerimg{17}{images/EMT 1-6_Mifta Dilla Aninditya_23030630029-254.png}
\begin{eulercomment}
Dimungkinkan untuk membatasi nilai yang diseret ke bilangan bulat.
Misalnya, kita menulis fungsi plot, yang memplot polinomial Taylor
berderajat n ke fungsi kosinus.
\end{eulercomment}
\begin{eulerprompt}
>function plotf(n) ...
\end{eulerprompt}
\begin{eulerudf}
  plot2d("cos(x)",0,2pi,>square,grid=6);
  plot2d(&"taylor(cos(x),x,0,@n)",color=blue,>add);
  textbox("Taylor polynomial of degree "+n,0.1,0.02,style="t",>left);
  endfunction
\end{eulerudf}
\begin{eulercomment}
Sekarang kita mengizinkan derajat n bervariasi dari 0 hingga 20 dalam
20 stop. Hasil dragvalues() digunakan untuk memplot sketsa dengan n
ini, dan untuk memasukkan plot ke dalam buku catatan.
\end{eulercomment}
\begin{eulerprompt}
>nd=dragvalues("plotf","degree",2,[0,20],20,y=0.8, ...
>   heading="Drag the value:"); ...
>plotf(nd):
\end{eulerprompt}
\eulerimg{17}{images/EMT 1-6_Mifta Dilla Aninditya_23030630029-255.png}
\begin{eulercomment}
Berikut ini adalah demonstrasi sederhana dari fungsi tersebut.
Pengguna dapat menggambar di atas jendela plot, meninggalkan jejak
titik-titik.
\end{eulercomment}
\begin{eulerprompt}
>function dragtest ...
\end{eulerprompt}
\begin{eulerudf}
    plot2d(none,r=1,title="Drag with the mouse, or press any key!");
    start=0;
    repeat
      \{flag,m,time\}=mousedrag();
      if flag==0 then return; endif;
      if flag==2 then
        hold on; mark(m[1],m[2]); hold off;
      endif;
    end
  endfunction
\end{eulerudf}
\begin{eulerprompt}
>dragtest // lihat hasilnya dan cobalah lakukan!
\end{eulerprompt}
\eulersubheading{Gaya Plot 2D}
\begin{eulercomment}
Secara default, EMT menghitung tanda centang sumbu otomatis dan
menambahkan label ke setiap tanda centang. Ini dapat diubah dengan
parameter grid. Gaya default sumbu dan label dapat dimodifikasi.
Selain itu, label dan judul dapat ditambahkan secara manual. Untuk
mengatur ulang ke gaya default, gunakan reset().
\end{eulercomment}
\begin{eulerprompt}
>aspect();
>figure(3,4); ...
> figure(1); plot2d("x^3-x",grid=0); ... // no grid, frame or axis
> figure(2); plot2d("x^3-x",grid=1); ... // x-y-axis
> figure(3); plot2d("x^3-x",grid=2); ... // default ticks
> figure(4); plot2d("x^3-x",grid=3); ... // x-y- axis with labels inside
> figure(5); plot2d("x^3-x",grid=4); ... // no ticks, only labels
> figure(6); plot2d("x^3-x",grid=5); ... // default, but no margin
> figure(7); plot2d("x^3-x",grid=6); ... // axes only
> figure(8); plot2d("x^3-x",grid=7); ... // axes only, ticks at axis
> figure(9); plot2d("x^3-x",grid=8); ... // axes only, finer ticks at axis
> figure(10); plot2d("x^3-x",grid=9); ... // default, small ticks inside
> figure(11); plot2d("x^3-x",grid=10); ...// no ticks, axes only
> figure(0):
\end{eulerprompt}
\eulerimg{27}{images/EMT 1-6_Mifta Dilla Aninditya_23030630029-256.png}
\begin{eulercomment}
Parameter \textless{}frame mematikan frame, dan framecolor=blue mengatur frame
menjadi warna biru.

Jika Anda menginginkan tanda centang Anda sendiri, Anda dapat
menggunakan style=0, dan menambahkan semuanya nanti. 
\end{eulercomment}
\begin{eulerprompt}
>aspect(1.5); 
>plot2d("x^3-x",grid=0); // plot
>frame; xgrid([-1,0,1]); ygrid(0): // add frame and grid
\end{eulerprompt}
\eulerimg{17}{images/EMT 1-6_Mifta Dilla Aninditya_23030630029-257.png}
\begin{eulercomment}
Untuk judul plot dan label sumbu, lihat contoh berikut.
\end{eulercomment}
\begin{eulerprompt}
>plot2d("exp(x)",-1,1);
>textcolor(black); // set the text color to black
>title(latex("y=e^x")); // title above the plot
>xlabel(latex("x")); // "x" for x-axis
>ylabel(latex("y"),>vertical); // vertical "y" for y-axis
>label(latex("(0,1)"),0,1,color=blue): // label a point
\end{eulerprompt}
\eulerimg{17}{images/EMT 1-6_Mifta Dilla Aninditya_23030630029-258.png}
\begin{eulercomment}
Sumbu dapat digambar secara terpisah dengan xaxis() dan yaxis().
\end{eulercomment}
\begin{eulerprompt}
>plot2d("x^3-x",<grid,<frame);
>xaxis(0,xx=-2:1,style="->"); yaxis(0,yy=-5:5,style="->"):
\end{eulerprompt}
\eulerimg{17}{images/EMT 1-6_Mifta Dilla Aninditya_23030630029-259.png}
\begin{eulercomment}
Teks pada plot dapat diatur dengan label(). Dalam contoh berikut, "lc"
berarti tengah bawah. Ini mengatur posisi label relatif terhadap
koordinat plot.
\end{eulercomment}
\begin{eulerprompt}
>function f(x) &= x^3-x
\end{eulerprompt}
\begin{euleroutput}
  
                                   3
                                  x  - x
  
\end{euleroutput}
\begin{eulerprompt}
>plot2d(f,-1,1,>square);
>x0=fmin(f,0,1); // compute point of minimum
>label("Rel. Min.",x0,f(x0),pos="lc"): // add a label there
\end{eulerprompt}
\eulerimg{17}{images/EMT 1-6_Mifta Dilla Aninditya_23030630029-260.png}
\begin{eulercomment}
Ada juga kotak teks.
\end{eulercomment}
\begin{eulerprompt}
>plot2d(&f(x),-1,1,-2,2); // function
>plot2d(&diff(f(x),x),>add,style="--",color=red); // derivative
>labelbox(["f","f'"],["-","--"],[black,red]): // label box
\end{eulerprompt}
\eulerimg{17}{images/EMT 1-6_Mifta Dilla Aninditya_23030630029-261.png}
\begin{eulerprompt}
>plot2d(["exp(x)","1+x"],color=[black,blue],style=["-","-.-"]):
\end{eulerprompt}
\eulerimg{17}{images/EMT 1-6_Mifta Dilla Aninditya_23030630029-262.png}
\begin{eulerprompt}
>gridstyle("->",color=gray,textcolor=gray,framecolor=gray);  ...
> plot2d("x^3-x",grid=1);   ...
> settitle("y=x^3-x",color=black); ...
> label("x",2,0,pos="bc",color=gray);  ...
> label("y",0,6,pos="cl",color=gray); ...
> reset():
\end{eulerprompt}
\eulerimg{27}{images/EMT 1-6_Mifta Dilla Aninditya_23030630029-263.png}
\begin{eulercomment}
Untuk kontrol lebih, sumbu x dan sumbu y dapat dilakukan secara
manual.

Perintah fullwindow() memperluas jendela plot karena kita tidak lagi
memerlukan tempat untuk label di luar jendela plot. Gunakan
shrinkwindow() atau reset() untuk mengatur ulang ke default.
\end{eulercomment}
\begin{eulerprompt}
>fullwindow; ...
> gridstyle(color=darkgray,textcolor=darkgray); ...
> plot2d(["2^x","1","2^(-x)"],a=-2,b=2,c=0,d=4,<grid,color=4:6,<frame); ...
> xaxis(0,-2:1,style="->"); xaxis(0,2,"x",<axis); ...
> yaxis(0,4,"y",style="->"); ...
> yaxis(-2,1:4,>left); ...
> yaxis(2,2^(-2:2),style=".",<left); ...
> labelbox(["2^x","1","2^-x"],colors=4:6,x=0.8,y=0.2); ...
> reset:
\end{eulerprompt}
\eulerimg{27}{images/EMT 1-6_Mifta Dilla Aninditya_23030630029-264.png}
\begin{eulercomment}
Berikut contoh lain, dimana string Unicode digunakan dan sumbu berada
di luar area plot.
\end{eulercomment}
\begin{eulerprompt}
>aspect(1.5); 
>plot2d(["sin(x)","cos(x)"],0,2pi,color=[red,green],<grid,<frame); ...
> xaxis(-1.1,(0:2)*pi,xt=["0",u"&pi;",u"2&pi;"],style="-",>ticks,>zero);  ...
> xgrid((0:0.5:2)*pi,<ticks); ...
> yaxis(-0.1*pi,-1:0.2:1,style="-",>zero,>grid); ...
> labelbox(["sin","cos"],colors=[red,green],x=0.5,y=0.2,>left); ...
> xlabel(u"&phi;"); ylabel(u"f(&phi;)"):
\end{eulerprompt}
\eulerimg{17}{images/EMT 1-6_Mifta Dilla Aninditya_23030630029-265.png}
\eulerheading{Memplot Data 2D }
\begin{eulercomment}
Jika x dan y adalah vektor data, data ini akan digunakan sebagai
koordinat x dan y dari sebuah kurva. Dalam kasus ini, a, b, c, dan d,
atau radius r dapat ditentukan, atau jendela plot akan menyesuaikan
secara otomatis dengan data. Atau, \textgreater{}square dapat diatur untuk
mempertahankan rasio aspek persegi.

Memplot ekspresi hanyalah singkatan dari plot data. Untuk plot data,
Anda memerlukan satu atau beberapa baris nilai-x, dan satu atau
beberapa baris nilai-y. Dari rentang dan nilai-x, fungsi plot2d akan
menghitung data yang akan diplot, secara default dengan evaluasi
adaptif fungsi tersebut. Untuk plot titik gunakan "\textgreater{}points", untuk
garis dan titik campuran gunakan "\textgreater{}addpoints".

Namun Anda dapat memasukkan data secara langsung.

- Gunakan vektor baris untuk x dan y untuk satu fungsi.\\
- Matriks untuk x dan y diplot baris demi baris.

Berikut adalah contoh dengan satu baris untuk x dan y.

\end{eulercomment}
\begin{eulerprompt}
>x=-10:0.1:10; y=exp(-x^2)*x; plot2d(x,y):
\end{eulerprompt}
\eulerimg{17}{images/EMT 1-6_Mifta Dilla Aninditya_23030630029-266.png}
\begin{eulercomment}
Data juga dapat diplot sebagai titik. Gunakan points=true untuk ini.
Plot bekerja seperti poligon, tetapi hanya menggambar sudut.

- style="...": Pilih dari "[]", "\textless{}\textgreater{}", "o", ".", "..", "+", "*", "[]#",
"\textless{}\textgreater{}#", "o#", "..#", "#", "\textbar{}".

Untuk memplot kumpulan titik, gunakan \textgreater{}points. Jika warnanya adalah
vektor warna, setiap titik akan mendapatkan warna yang berbeda. Untuk
matriks koordinat dan vektor kolom, warna berlaku untuk baris matriks.\\
Parameter \textgreater{}addpoints menambahkan titik ke segmen garis untuk plot
data.
\end{eulercomment}
\begin{eulerprompt}
>xdata=[1,1.5,2.5,3,4]; ydata=[3,3.1,2.8,2.9,2.7]; // data
>plot2d(xdata,ydata,a=0.5,b=4.5,c=2.5,d=3.5,style="."); // lines
>plot2d(xdata,ydata,>points,>add,style="o"): // add points
\end{eulerprompt}
\eulerimg{17}{images/EMT 1-6_Mifta Dilla Aninditya_23030630029-267.png}
\begin{eulerprompt}
>p=polyfit(xdata,ydata,1); // get regression line
>plot2d("polyval(p,x)",>add,color=red): // add plot of line
\end{eulerprompt}
\eulerimg{17}{images/EMT 1-6_Mifta Dilla Aninditya_23030630029-268.png}
\eulerheading{Menggambar Daerah Yang Dibatasi Kurva}
\begin{eulercomment}
Plot data sebenarnya adalah poligon. Kita juga dapat memplot kurva
atau kurva terisi.

- filled=true mengisi plot.\\
- style="...": Pilih dari "#", "/", "\textbackslash{}", "\textbackslash{}/".\\
- fillcolor: Lihat di atas untuk warna yang tersedia.

Warna isian ditentukan oleh argumen "fillcolor", dan pada \textless{}outline
opsional mencegah penggambaran batas untuk semua gaya kecuali yang
default. 
\end{eulercomment}
\begin{eulerprompt}
>t=linspace(0,2pi,1000); // parameter for curve
>x=sin(t)*exp(t/pi); y=cos(t)*exp(t/pi); // x(t) and y(t)
>figure(1,2); aspect(16/9)
>figure(1); plot2d(x,y,r=10); // plot curve
>figure(2); plot2d(x,y,r=10,>filled,style="/",fillcolor=red); // fill curve
>figure(0):
\end{eulerprompt}
\eulerimg{14}{images/EMT 1-6_Mifta Dilla Aninditya_23030630029-269.png}
\begin{eulercomment}
Pada contoh berikut ini kami memplot elips yang terisi dan dua segi
enam yang terisi menggunakan kurva tertutup dengan 6 titik dengan gaya
isian yang berbeda.
\end{eulercomment}
\begin{eulerprompt}
>x=linspace(0,2pi,1000); plot2d(sin(x),cos(x)*0.5,r=1,>filled,style="/"):
\end{eulerprompt}
\eulerimg{14}{images/EMT 1-6_Mifta Dilla Aninditya_23030630029-270.png}
\begin{eulerprompt}
>t=linspace(0,2pi,6); ...
>plot2d(cos(t),sin(t),>filled,style="/",fillcolor=red,r=1.2):
\end{eulerprompt}
\eulerimg{14}{images/EMT 1-6_Mifta Dilla Aninditya_23030630029-271.png}
\begin{eulerprompt}
>t=linspace(0,2pi,6); plot2d(cos(t),sin(t),>filled,style="#"):
\end{eulerprompt}
\eulerimg{14}{images/EMT 1-6_Mifta Dilla Aninditya_23030630029-272.png}
\begin{eulercomment}
Contoh lain adalah septagon, yang kita buat dengan 7 titik pada
lingkaran satuan.
\end{eulercomment}
\begin{eulerprompt}
>t=linspace(0,2pi,7);  ...
> plot2d(cos(t),sin(t),r=1,>filled,style="/",fillcolor=red):
\end{eulerprompt}
\eulerimg{14}{images/EMT 1-6_Mifta Dilla Aninditya_23030630029-273.png}
\begin{eulercomment}
Berikut ini adalah himpunan nilai maksimal dari empat kondisi linier
yang kurang dari atau sama dengan 3. Ini adalah A[k].v\textless{}=3 untuk semua
baris A. Untuk mendapatkan sudut yang bagus, kita menggunakan n yang
relatif besar.
\end{eulercomment}
\begin{eulerprompt}
>A=[2,1;1,2;-1,0;0,-1];
>function f(x,y) := max([x,y].A');
>plot2d("f",r=4,level=[0;3],color=green,n=111):
\end{eulerprompt}
\eulerimg{14}{images/EMT 1-6_Mifta Dilla Aninditya_23030630029-274.png}
\begin{eulercomment}
Poin utama dari bahasa matriks adalah memungkinkan pembuatan tabel
fungsi dengan mudah.
\end{eulercomment}
\begin{eulerprompt}
>t=linspace(0,2pi,1000); x=cos(3*t); y=sin(4*t);
\end{eulerprompt}
\begin{eulercomment}
Sekarang kita memiliki vektor nilai x dan y. plot2d() dapat memplot
nilai-nilai ini sebagai kurva yang menghubungkan titik-titik. Plot
dapat diisi. Dalam kasus ini, hasilnya akan bagus karena aturan
lilitan, yang digunakan untuk mengisi.
\end{eulercomment}
\begin{eulerprompt}
>plot2d(x,y,<grid,<frame,>filled):
\end{eulerprompt}
\eulerimg{14}{images/EMT 1-6_Mifta Dilla Aninditya_23030630029-275.png}
\begin{eulercomment}
Vektor interval diplot terhadap nilai x sebagai daerah terisi antara
nilai interval bawah dan atas.

Hal ini dapat berguna untuk memetakan kesalahan perhitungan. Namun,
hal ini juga dapat digunakan untuk memetakan kesalahan statistik.
\end{eulercomment}
\begin{eulerprompt}
>t=0:0.1:1; ...
> plot2d(t,interval(t-random(size(t)),t+random(size(t))),style="|");  ...
> plot2d(t,t,add=true):
\end{eulerprompt}
\eulerimg{14}{images/EMT 1-6_Mifta Dilla Aninditya_23030630029-276.png}
\begin{eulercomment}
Jika x adalah vektor yang diurutkan, dan y adalah vektor interval,
maka plot2d akan memplot rentang interval yang terisi pada bidang,
gaya isian sama dengan gaya poligon.
\end{eulercomment}
\begin{eulerprompt}
>t=-1:0.01:1; x=~t-0.01,t+0.01~; y=x^3-x;
>plot2d(t,y):
\end{eulerprompt}
\eulerimg{14}{images/EMT 1-6_Mifta Dilla Aninditya_23030630029-277.png}
\begin{eulercomment}
Dimungkinkan untuk mengisi wilayah nilai untuk fungsi tertentu. Untuk\\
ini, level harus berupa matriks 2xn. Baris pertama adalah batas bawah\\
dan baris kedua berisi batas atas.
\end{eulercomment}
\begin{eulerprompt}
>expr := "2*x^2+x*y+3*y^4+y"; // define an expression f(x,y)
>plot2d(expr,level=[0;1],style="-",color=blue): // 0 <= f(x,y) <= 1
\end{eulerprompt}
\eulerimg{14}{images/EMT 1-6_Mifta Dilla Aninditya_23030630029-278.png}
\begin{eulercomment}
Kita juga dapat mengisi rentang nilai seperti

\end{eulercomment}
\begin{eulerformula}
\[
-1 \le (x^2+y^2)^2-x^2+y^2 \le 0.
\]
\end{eulerformula}
\begin{eulercomment}
\end{eulercomment}
\begin{eulerprompt}
>plot2d("(x^2+y^2)^2-x^2+y^2",r=1.2,level=[-1;0],style="/"):
\end{eulerprompt}
\eulerimg{14}{images/EMT 1-6_Mifta Dilla Aninditya_23030630029-279.png}
\begin{eulerprompt}
>plot2d("cos(x)","sin(x)^3",xmin=0,xmax=2pi,>filled,style="/"):
\end{eulerprompt}
\eulerimg{14}{images/EMT 1-6_Mifta Dilla Aninditya_23030630029-280.png}
\eulerheading{Grafik Fungsi Parametrik}
\begin{eulercomment}
Nilai x tidak perlu diurutkan. (x,y) hanya menggambarkan sebuah kurva.
Jika x diurutkan, kurva tersebut adalah grafik fungsi.

Pada contoh berikut, kita memplot spiral

lateks: \textbackslash{}gamma(t) = t\textbackslash{}cdot (\textbackslash{}cos(2\textbackslash{}pi t),\textbackslash{}sin(2\textbackslash{}pi t))

Kita mungkin perlu menggunakan sangat banyak titik untuk tampilan yang
halus atau fungsi adaptive() untuk mengevaluasi ekspresi (lihat fungsi
adaptive() untuk lebih jelasnya).
\end{eulercomment}
\begin{eulerprompt}
>t=linspace(0,1,1000); ...
>plot2d(t*cos(2*pi*t),t*sin(2*pi*t),r=1):
\end{eulerprompt}
\eulerimg{14}{images/EMT 1-6_Mifta Dilla Aninditya_23030630029-281.png}
\begin{eulercomment}
Sebagai alternatif, Anda dapat menggunakan dua ekspresi untuk kurva.
Berikut ini memplot kurva yang sama seperti di atas.
\end{eulercomment}
\begin{eulerprompt}
>plot2d("x*cos(2*pi*x)","x*sin(2*pi*x)",xmin=0,xmax=1,r=1):
\end{eulerprompt}
\eulerimg{14}{images/EMT 1-6_Mifta Dilla Aninditya_23030630029-282.png}
\begin{eulerprompt}
>t=linspace(0,1,1000); r=exp(-t); x=r*cos(2pi*t); y=r*sin(2pi*t);
>plot2d(x,y,r=1):
\end{eulerprompt}
\eulerimg{14}{images/EMT 1-6_Mifta Dilla Aninditya_23030630029-283.png}
\begin{eulercomment}
Pada contoh berikutnya, kita memplot kurva

\end{eulercomment}
\begin{eulerformula}
\[
\gamma(t) = (r(t) \cos(t), r(t) \sin(t))
\]
\end{eulerformula}
\begin{eulercomment}
dengan

\end{eulercomment}
\begin{eulerformula}
\[
r(t) = 1 + \dfrac{\sin(3t)}{2}.
\]
\end{eulerformula}
\begin{eulerprompt}
>t=linspace(0,2pi,1000); r=1+sin(3*t)/2; x=r*cos(t); y=r*sin(t); ...
>plot2d(x,y,>filled,fillcolor=red,style="/",r=1.5):
\end{eulerprompt}
\eulerimg{14}{images/EMT 1-6_Mifta Dilla Aninditya_23030630029-284.png}
\eulerheading{Menggambar Grafik Bilangan Kompleks}
\begin{eulercomment}
Sebuah deretan bilangan kompleks juga dapat diplot. Kemudian
titik-titik kisi akan dihubungkan. Jika sejumlah garis kisi ditentukan
(atau vektor 1x2 garis kisi) pada argumen cgrid, hanya garis-garis
kisi tersebut yang akan terlihat.

Matriks bilangan kompleks akan secara otomatis diplot sebagai sebuah
grid pada bidang kompleks.

Pada contoh berikut, kita memplot gambar lingkaran satuan di bawah
fungsi eksponensial. Parameter cgrid menyembunyikan beberapa kurva
grid.
\end{eulercomment}
\begin{eulerprompt}
>aspect(); r=linspace(0,1,50); a=linspace(0,2pi,80)'; z=r*exp(I*a);...
>plot2d(z,a=-1.25,b=1.25,c=-1.25,d=1.25,cgrid=10):
\end{eulerprompt}
\eulerimg{27}{images/EMT 1-6_Mifta Dilla Aninditya_23030630029-285.png}
\begin{eulerprompt}
>aspect(1.25); r=linspace(0,1,50); a=linspace(0,2pi,200)'; z=r*exp(I*a);
>plot2d(exp(z),cgrid=[40,10]):
\end{eulerprompt}
\eulerimg{21}{images/EMT 1-6_Mifta Dilla Aninditya_23030630029-286.png}
\begin{eulerprompt}
>r=linspace(0,1,10); a=linspace(0,2pi,40)'; z=r*exp(I*a);
>plot2d(exp(z),>points,>add):
\end{eulerprompt}
\eulerimg{21}{images/EMT 1-6_Mifta Dilla Aninditya_23030630029-287.png}
\begin{eulercomment}
Vektor bilangan kompleks secara otomatis diplot sebagai kurva pada
bidang kompleks dengan bagian nyata dan bagian imajiner.


Pada contoh, kami memplot lingkaran satuan dengan


atex: \textbackslash{}gamma(t) = e\textasciicircum{}\{it\}
\end{eulercomment}
\begin{eulerprompt}
>t=linspace(0,2pi,1000); ...
>plot2d(exp(I*t)+exp(4*I*t),r=2):
\end{eulerprompt}
\eulerimg{21}{images/EMT 1-6_Mifta Dilla Aninditya_23030630029-288.png}
\eulerheading{Plot Statistik}
\begin{eulercomment}
Terdapat banyak fungsi yang dikhususkan untuk plot statistik. Salah
satu plot yang sering digunakan adalah plot kolom.

Jumlah kumulatif dari nilai berdistribusi normal 0-1 menghasilkan
jalan acak.
\end{eulercomment}
\begin{eulerprompt}
>plot2d(cumsum(randnormal(1,1000))):
\end{eulerprompt}
\eulerimg{21}{images/EMT 1-6_Mifta Dilla Aninditya_23030630029-289.png}
\begin{eulercomment}
Menggunakan dua baris menunjukkan jalan dalam dua dimensi.
\end{eulercomment}
\begin{eulerprompt}
>X=cumsum(randnormal(2,1000)); plot2d(X[1],X[2]):
\end{eulerprompt}
\eulerimg{21}{images/EMT 1-6_Mifta Dilla Aninditya_23030630029-290.png}
\begin{eulerprompt}
>columnsplot(cumsum(random(10)),style="/",color=blue):
\end{eulerprompt}
\eulerimg{21}{images/EMT 1-6_Mifta Dilla Aninditya_23030630029-291.png}
\begin{eulercomment}
Ini juga dapat menampilkan string sebagai label.
\end{eulercomment}
\begin{eulerprompt}
>months=["Jan","Feb","Mar","Apr","May","Jun", ...
>  "Jul","Aug","Sep","Oct","Nov","Dec"];
>values=[10,12,12,18,22,28,30,26,22,18,12,8];
>columnsplot(values,lab=months,color=red,style="-");
>title("Temperature"):
\end{eulerprompt}
\eulerimg{21}{images/EMT 1-6_Mifta Dilla Aninditya_23030630029-292.png}
\begin{eulerprompt}
>k=0:10;
>plot2d(k,bin(10,k),>bar):
\end{eulerprompt}
\eulerimg{21}{images/EMT 1-6_Mifta Dilla Aninditya_23030630029-293.png}
\begin{eulerprompt}
>plot2d(k,bin(10,k)); plot2d(k,bin(10,k),>points,>add):
\end{eulerprompt}
\eulerimg{21}{images/EMT 1-6_Mifta Dilla Aninditya_23030630029-294.png}
\begin{eulerprompt}
>plot2d(normal(1000),normal(1000),>points,grid=6,style=".."):
\end{eulerprompt}
\eulerimg{21}{images/EMT 1-6_Mifta Dilla Aninditya_23030630029-295.png}
\begin{eulerprompt}
>plot2d(normal(1,1000),>distribution,style="O"):
\end{eulerprompt}
\eulerimg{21}{images/EMT 1-6_Mifta Dilla Aninditya_23030630029-296.png}
\begin{eulerprompt}
>plot2d("qnormal",0,5;2.5,0.5,>filled):
\end{eulerprompt}
\eulerimg{21}{images/EMT 1-6_Mifta Dilla Aninditya_23030630029-297.png}
\begin{eulercomment}
Untuk memplot distribusi statistik eksperimental, Anda dapat
menggunakan distribution=n dengan plot2d.
\end{eulercomment}
\begin{eulerprompt}
>w=randexponential(1,1000); // exponential distribution
>plot2d(w,>distribution): // or distribution=n with n intervals
\end{eulerprompt}
\eulerimg{21}{images/EMT 1-6_Mifta Dilla Aninditya_23030630029-298.png}
\begin{eulercomment}
Atau Anda dapat menghitung distribusi dari data dan memplot hasilnya
dengan \textgreater{}bar di plot3d, atau dengan plot kolom.
\end{eulercomment}
\begin{eulerprompt}
>w=normal(1000); // 0-1-normal distribution
>\{x,y\}=histo(w,10,v=[-6,-4,-2,-1,0,1,2,4,6]); // interval bounds v
>plot2d(x,y,>bar):
\end{eulerprompt}
\eulerimg{21}{images/EMT 1-6_Mifta Dilla Aninditya_23030630029-299.png}
\begin{eulercomment}
Fungsi statplot() mengatur gaya dengan string sederhana.
\end{eulercomment}
\begin{eulerprompt}
>statplot(1:10,cumsum(random(10)),"b"):
\end{eulerprompt}
\eulerimg{21}{images/EMT 1-6_Mifta Dilla Aninditya_23030630029-300.png}
\begin{eulerprompt}
>n=10; i=0:n; ...
>plot2d(i,bin(n,i)/2^n,a=0,b=10,c=0,d=0.3); ...
>plot2d(i,bin(n,i)/2^n,points=true,style="ow",add=true,color=blue):
\end{eulerprompt}
\eulerimg{21}{images/EMT 1-6_Mifta Dilla Aninditya_23030630029-301.png}
\begin{eulercomment}
Selain itu, data dapat diplot sebagai batang. Dalam hal ini, x harus
diurutkan dan satu elemen lebih panjang dari y. Batang akan memanjang
dari x[i] ke x[i+1] dengan nilai y[i]. Jika x memiliki ukuran yang
sama dengan y, maka x akan diperpanjang satu elemen dengan jarak
terakhir.

Gaya isian dapat digunakan seperti di atas.
\end{eulercomment}
\begin{eulerprompt}
>n=10; k=bin(n,0:n); ...
>plot2d(-0.5:n+0.5,k,bar=true,fillcolor=lightgray):
\end{eulerprompt}
\eulerimg{21}{images/EMT 1-6_Mifta Dilla Aninditya_23030630029-302.png}
\begin{eulercomment}
Data untuk plot batang (bar=1) dan histogram (histogram = 1) dapat
diberikan secara eksplisit dalam xv dan yv, atau dapat dihitung dari
distribusi empiris dalam xv dengan \textgreater{}distribusi (atau distribusi = n).
Histogram dari nilai xv akan dihitung secara otomatis dengan
\textgreater{}histogram. Jika \textgreater{}even ditentukan, nilai xv akan dihitung dalam
interval bilangan bulat.
\end{eulercomment}
\begin{eulerprompt}
>plot2d(normal(10000),distribution=50):
\end{eulerprompt}
\eulerimg{21}{images/EMT 1-6_Mifta Dilla Aninditya_23030630029-303.png}
\begin{eulerprompt}
>k=0:10; m=bin(10,k); x=(0:11)-0.5; plot2d(x,m,>bar):
\end{eulerprompt}
\eulerimg{21}{images/EMT 1-6_Mifta Dilla Aninditya_23030630029-304.png}
\begin{eulerprompt}
>columnsplot(m,k):
\end{eulerprompt}
\eulerimg{21}{images/EMT 1-6_Mifta Dilla Aninditya_23030630029-305.png}
\begin{eulerprompt}
>plot2d(random(600)*6,histogram=6):
\end{eulerprompt}
\eulerimg{21}{images/EMT 1-6_Mifta Dilla Aninditya_23030630029-306.png}
\begin{eulercomment}
Untuk distribusi, ada parameter distribution=n, yang menghitung nilai
secara otomatis dan mencetak distribusi relatif dengan n sub-interval.
\end{eulercomment}
\begin{eulerprompt}
>plot2d(normal(1,1000),distribution=10,style="\(\backslash\)/"):
\end{eulerprompt}
\eulerimg{21}{images/EMT 1-6_Mifta Dilla Aninditya_23030630029-307.png}
\begin{eulercomment}
Dengan parameter even=true, ini akan menggunakan interval integer.
\end{eulercomment}
\begin{eulerprompt}
>plot2d(intrandom(1,1000,10),distribution=10,even=true):
\end{eulerprompt}
\eulerimg{21}{images/EMT 1-6_Mifta Dilla Aninditya_23030630029-308.png}
\begin{eulercomment}
Perhatikan bahwa ada banyak plot statistik yang mungkin berguna.
Lihatlah tutorial tentang statistik.
\end{eulercomment}
\begin{eulerprompt}
>columnsplot(getmultiplicities(1:6,intrandom(1,6000,6))):
\end{eulerprompt}
\eulerimg{21}{images/EMT 1-6_Mifta Dilla Aninditya_23030630029-309.png}
\begin{eulerprompt}
>plot2d(normal(1,1000),>distribution); ...
>  plot2d("qnormal(x)",color=red,thickness=2,>add):
\end{eulerprompt}
\eulerimg{21}{images/EMT 1-6_Mifta Dilla Aninditya_23030630029-310.png}
\begin{eulercomment}
Ada juga banyak plot khusus untuk statistik. Boxplot menunjukkan
kuartil dari distribusi ini dan banyak pencilan. Menurut definisi,
pencilan dalam boxplot adalah data yang melebihi 1,5 kali kisaran 50\%
tengah plot.
\end{eulercomment}
\begin{eulerprompt}
>M=normal(5,1000); boxplot(quartiles(M)):
\end{eulerprompt}
\eulerimg{21}{images/EMT 1-6_Mifta Dilla Aninditya_23030630029-311.png}
\eulerheading{Fungsi Implisit}
\begin{eulercomment}
Plot implisit menunjukkan garis level yang menyelesaikan f(x,y)=level,
di mana “level” dapat berupa nilai tunggal atau vektor nilai. Jika
level = “auto”, akan ada nc garis level, yang akan menyebar di antara
nilai minimum dan maksimum fungsi secara merata. Warna yang lebih
gelap atau lebih terang dapat ditambahkan dengan \textgreater{}hue untuk
mengindikasikan nilai fungsi. Untuk fungsi implisit, xv haruslah
sebuah fungsi atau ekspresi dari parameter x dan y, atau, sebagai
alternatif, xv dapat berupa sebuah matriks nilai.

Euler dapat menandai garis level

\end{eulercomment}
\begin{eulerformula}
\[
f(x,y) = c
\]
\end{eulerformula}
\begin{eulercomment}
dari fungsi apa pun.

Untuk menggambar himpunan f(x,y)=c untuk satu atau lebih konstanta c,
Anda dapat menggunakan plot2d() dengan plot implisitnya pada bidang.
Parameter untuk c adalah level = c, di mana c dapat berupa vektor
garis level. Sebagai tambahan, sebuah skema warna dapat digambar pada
latar belakang untuk mengindikasikan nilai fungsi untuk setiap titik
pada plot. Parameter “n” menentukan kehalusan plot.
\end{eulercomment}
\begin{eulerprompt}
>aspect(1.5); 
>plot2d("x^2+y^2-x*y-x",r=1.5,level=0,contourcolor=red):
\end{eulerprompt}
\eulerimg{17}{images/EMT 1-6_Mifta Dilla Aninditya_23030630029-312.png}
\begin{eulerprompt}
>expr := "2*x^2+x*y+3*y^4+y"; // define an expression f(x,y)
>plot2d(expr,level=0): // Solutions of f(x,y)=0
\end{eulerprompt}
\eulerimg{17}{images/EMT 1-6_Mifta Dilla Aninditya_23030630029-313.png}
\begin{eulerprompt}
>plot2d(expr,level=0:0.5:20,>hue,contourcolor=white,n=200): // nice
\end{eulerprompt}
\eulerimg{17}{images/EMT 1-6_Mifta Dilla Aninditya_23030630029-314.png}
\begin{eulerprompt}
>plot2d(expr,level=0:0.5:20,>hue,>spectral,n=200,grid=4): // nicer
\end{eulerprompt}
\eulerimg{17}{images/EMT 1-6_Mifta Dilla Aninditya_23030630029-315.png}
\begin{eulercomment}
Hal ini juga berlaku untuk plot data. Tetapi Anda harus menentukan
rentang untuk label sumbu.
\end{eulercomment}
\begin{eulerprompt}
>x=-2:0.05:1; y=x'; z=expr(x,y);
>plot2d(z,level=0,a=-1,b=2,c=-2,d=1,>hue):
\end{eulerprompt}
\eulerimg{17}{images/EMT 1-6_Mifta Dilla Aninditya_23030630029-316.png}
\begin{eulerprompt}
>plot2d("x^3-y^2",>contour,>hue,>spectral):
\end{eulerprompt}
\eulerimg{17}{images/EMT 1-6_Mifta Dilla Aninditya_23030630029-317.png}
\begin{eulerprompt}
>plot2d("x^3-y^2",level=0,contourwidth=3,>add,contourcolor=red):
\end{eulerprompt}
\eulerimg{17}{images/EMT 1-6_Mifta Dilla Aninditya_23030630029-318.png}
\begin{eulerprompt}
>z=z+normal(size(z))*0.2;
>plot2d(z,level=0.5,a=-1,b=2,c=-2,d=1):
\end{eulerprompt}
\eulerimg{17}{images/EMT 1-6_Mifta Dilla Aninditya_23030630029-319.png}
\begin{eulerprompt}
>plot2d(expr,level=[0:0.2:5;0.05:0.2:5.05],color=lightgray):
\end{eulerprompt}
\eulerimg{17}{images/EMT 1-6_Mifta Dilla Aninditya_23030630029-320.png}
\begin{eulerprompt}
>plot2d("x^2+y^3+x*y",level=1,r=4,n=100):
\end{eulerprompt}
\eulerimg{17}{images/EMT 1-6_Mifta Dilla Aninditya_23030630029-321.png}
\begin{eulerprompt}
>plot2d("x^2+2*y^2-x*y",level=0:0.1:10,n=100,contourcolor=white,>hue):
\end{eulerprompt}
\eulerimg{17}{images/EMT 1-6_Mifta Dilla Aninditya_23030630029-322.png}
\begin{eulercomment}
Dimungkinkan juga untuk mengisi set

lateks: a \textbackslash{}le f(x,y) \textbackslash{}le b

dengan rentang level.

Dimungkinkan untuk mengisi wilayah nilai untuk fungsi tertentu. Untuk
ini, level harus berupa matriks 2xn. Baris pertama adalah batas bawah
dan baris kedua berisi batas atas.
\end{eulercomment}
\begin{eulerprompt}
>plot2d(expr,level=[0;1],style="-",color=blue): // 0 <= f(x,y) <= 1
\end{eulerprompt}
\eulerimg{17}{images/EMT 1-6_Mifta Dilla Aninditya_23030630029-323.png}
\begin{eulercomment}
Plot implisit juga dapat menunjukkan rentang level. Maka level harus
berupa matriks 2xn interval level, di mana baris pertama berisi awal
dan baris kedua adalah akhir dari setiap interval. Sebagai alternatif,
vektor baris sederhana dapat digunakan untuk level, dan parameter dl
memperluas nilai level ke interval.
\end{eulercomment}
\begin{eulerprompt}
>plot2d("x^4+y^4",r=1.5,level=[0;1],color=blue,style="/"):
\end{eulerprompt}
\eulerimg{17}{images/EMT 1-6_Mifta Dilla Aninditya_23030630029-324.png}
\begin{eulerprompt}
>plot2d("x^2+y^3+x*y",level=[0,2,4;1,3,5],style="/",r=2,n=100):
\end{eulerprompt}
\eulerimg{17}{images/EMT 1-6_Mifta Dilla Aninditya_23030630029-325.png}
\begin{eulerprompt}
>plot2d("x^2+y^3+x*y",level=-10:20,r=2,style="-",dl=0.1,n=100):
\end{eulerprompt}
\eulerimg{17}{images/EMT 1-6_Mifta Dilla Aninditya_23030630029-326.png}
\begin{eulerprompt}
>plot2d("sin(x)*cos(y)",r=pi,>hue,>levels,n=100):
\end{eulerprompt}
\eulerimg{17}{images/EMT 1-6_Mifta Dilla Aninditya_23030630029-327.png}
\begin{eulercomment}
Anda juga dapat menandai suatu wilayah

lateks: a \textbackslash{}le f(x,y) \textbackslash{}le b.

Hal ini dilakukan dengan menambahkan level dengan dua baris.
\end{eulercomment}
\begin{eulerprompt}
>plot2d("(x^2+y^2-1)^3-x^2*y^3",r=1.3, ...
>  style="#",color=red,<outline, ...
>  level=[-2;0],n=100):
\end{eulerprompt}
\eulerimg{17}{images/EMT 1-6_Mifta Dilla Aninditya_23030630029-328.png}
\begin{eulercomment}
Dimungkinkan untuk menentukan level tertentu. Sebagai contoh, kita
dapat memplot solusi dari persamaan seperti

lateks: x\textasciicircum{}3-xy+x\textasciicircum{}2y\textasciicircum{}2=6
\end{eulercomment}
\begin{eulerprompt}
>plot2d("x^3-x*y+x^2*y^2",r=6,level=1,n=100):
\end{eulerprompt}
\eulerimg{17}{images/EMT 1-6_Mifta Dilla Aninditya_23030630029-329.png}
\begin{eulerprompt}
>function starplot1 (v, style="/", color=green, lab=none) ...
\end{eulerprompt}
\begin{eulerudf}
    if !holding() then clg; endif;
    w=window(); window(0,0,1024,1024);
    h=holding(1);
    r=max(abs(v))*1.2;
    setplot(-r,r,-r,r);
    n=cols(v); t=linspace(0,2pi,n);
    v=v|v[1]; c=v*cos(t); s=v*sin(t);
    cl=barcolor(color); st=barstyle(style);
    loop 1 to n
      polygon([0,c[#],c[#+1]],[0,s[#],s[#+1]],1);
      if lab!=none then
        rlab=v[#]+r*0.1;
        \{col,row\}=toscreen(cos(t[#])*rlab,sin(t[#])*rlab);
        ctext(""+lab[#],col,row-textheight()/2);
      endif;
    end;
    barcolor(cl); barstyle(st);
    holding(h);
    window(w);
  endfunction
\end{eulerudf}
\begin{eulercomment}
Tidak ada kisi-kisi atau kutu sumbu di sini. Selain itu, kami
menggunakan jendela penuh untuk plot.

Kami memanggil reset sebelum kami menguji plot ini untuk mengembalikan
default grafis. Hal ini tidak perlu dilakukan, jika Anda yakin bahwa
plot Anda berfungsi.
\end{eulercomment}
\begin{eulerprompt}
>reset; starplot1(normal(1,10)+5,color=red,lab=1:10):
\end{eulerprompt}
\eulerimg{27}{images/EMT 1-6_Mifta Dilla Aninditya_23030630029-330.png}
\begin{eulercomment}
Terkadang, Anda mungkin ingin memplot sesuatu yang tidak dapat
dilakukan oleh plot2d, tetapi hampir.

Pada fungsi berikut ini, kita akan membuat plot impuls logaritmik.
plot2d dapat membuat plot logaritmik, tetapi tidak untuk batang
impuls.
\end{eulercomment}
\begin{eulerprompt}
>function logimpulseplot1 (x,y) ...
\end{eulerprompt}
\begin{eulerudf}
    \{x0,y0\}=makeimpulse(x,log(y)/log(10));
    plot2d(x0,y0,>bar,grid=0);
    h=holding(1);
    frame();
    xgrid(ticks(x));
    p=plot();
    for i=-10 to 10;
      if i<=p[4] and i>=p[3] then
         ygrid(i,yt="10^"+i);
      endif;
    end;
    holding(h);
  endfunction
\end{eulerudf}
\begin{eulercomment}
Let us test it with exponentially distributed values.
\end{eulercomment}
\begin{eulerprompt}
>aspect(1.5); x=1:10; y=-log(random(size(x)))*200; ...
>logimpulseplot1(x,y):
\end{eulerprompt}
\eulerimg{17}{images/EMT 1-6_Mifta Dilla Aninditya_23030630029-331.png}
\begin{eulercomment}
Mari kita menghidupkan kurva 2D dengan menggunakan plot langsung.
Perintah plot(x,y) hanya memplot kurva ke dalam jendela plot.
setplot(a,b,c,d) mengatur jendela ini.

Fungsi wait(0) memaksa plot untuk muncul pada jendela grafik. Jika
tidak, penggambaran ulang akan dilakukan dalam interval waktu yang
jarang.
\end{eulercomment}
\begin{eulerprompt}
>function animliss (n,m) ...
\end{eulerprompt}
\begin{eulerudf}
  t=linspace(0,2pi,500);
  f=0;
  c=framecolor(0);
  l=linewidth(2);
  setplot(-1,1,-1,1);
  repeat
    clg;
    plot(sin(n*t),cos(m*t+f));
    wait(0);
    if testkey() then break; endif;
    f=f+0.02;
  end;
  framecolor(c);
  linewidth(l);
  endfunction
\end{eulerudf}
\begin{eulercomment}
Tekan sembarang tombol untuk menghentikan animasi ini.
\end{eulercomment}
\begin{eulerprompt}
>animliss(2,3); // lihat hasilnya, jika sudah puas, tekan ENTER
\end{eulerprompt}
\eulerheading{Logarithmic Plots}
\begin{eulercomment}
EMT menggunakan parameter “logplot” untuk skala logaritmik.\\
Plot logaritmik dapat diplot menggunakan skala logaritmik dalam y
dengan logplot = 1, atau menggunakan skala logaritmik dalam x dan y
dengan logplot = 2, atau dalam x dengan logplot = 3.

\end{eulercomment}
\begin{eulerttcomment}
 - logplot=1: y-logarithmic
 - logplot=2: x-y-logarithmic
 - logplot=3: x-logarithmic
\end{eulerttcomment}
\begin{eulerprompt}
>plot2d("exp(x^3-x)*x^2",1,5,logplot=1):
\end{eulerprompt}
\eulerimg{17}{images/EMT 1-6_Mifta Dilla Aninditya_23030630029-332.png}
\begin{eulerprompt}
>plot2d("exp(x+sin(x))",0,100,logplot=1):
\end{eulerprompt}
\eulerimg{17}{images/EMT 1-6_Mifta Dilla Aninditya_23030630029-333.png}
\begin{eulerprompt}
>plot2d("exp(x+sin(x))",10,100,logplot=2):
\end{eulerprompt}
\eulerimg{17}{images/EMT 1-6_Mifta Dilla Aninditya_23030630029-334.png}
\begin{eulerprompt}
>plot2d("gamma(x)",1,10,logplot=1):
\end{eulerprompt}
\eulerimg{17}{images/EMT 1-6_Mifta Dilla Aninditya_23030630029-335.png}
\begin{eulerprompt}
>plot2d("log(x*(2+sin(x/100)))",10,1000,logplot=3):
\end{eulerprompt}
\eulerimg{17}{images/EMT 1-6_Mifta Dilla Aninditya_23030630029-336.png}
\begin{eulercomment}
Hal ini juga bisa dilakukan dengan plot data.
\end{eulercomment}
\begin{eulerprompt}
>x=10^(1:20); y=x^2-x;
>plot2d(x,y,logplot=2):
\end{eulerprompt}
\eulerimg{17}{images/EMT 1-6_Mifta Dilla Aninditya_23030630029-337.png}
\eulerheading{Contoh Soal}
\begin{eulercomment}
1. Gambarkan 2 grafik fungsi di bawah ini dalam 1 jendela plot yang
berbeda ukuran \\
\end{eulercomment}
\begin{eulerformula}
\[
5x^2-4x+7
\]
\end{eulerformula}
\begin{eulerformula}
\[
15x^2+9x+3
\]
\end{eulerformula}
\begin{eulerprompt}
>plot2d("5*x^2-4*x+7");
>xw=200; yw=100; ww=300; hw=300;
>ow=window();
>window(xw,yw,xw+ww,yw+hw);
>hold on;
>&barclear(xw-50,yw-10,ww+60,ww+60);
>plot2d("15*x^2+9*x+3");
>hold off;
>window(ow);
\end{eulerprompt}
\begin{eulercomment}
2. Buatlah grafik dari fungsi berikut dengan domain mulai dari 0
hingga 7pi.\\
\end{eulercomment}
\begin{eulerformula}
\[
3 cos (x)
\]
\end{eulerformula}
\begin{eulerformula}
\[
3 sin (x)
\]
\end{eulerformula}
\begin{eulerprompt}
>aspect(2);
>plot2d(["3*cos(x)","3*sin(x)"],0,7pi):
\end{eulerprompt}
\eulerimg{13}{images/EMT 1-6_Mifta Dilla Aninditya_23030630029-338.png}
\begin{eulerprompt}
>reset;
\end{eulerprompt}
\begin{eulercomment}
3. Gambarkan grafik fungsi berikut\\
\end{eulercomment}
\begin{eulerformula}
\[
x^3+2x+7
\]
\end{eulerformula}
\begin{eulerformula}
\[
x^5-3x+8
\]
\end{eulerformula}
\begin{eulerformula}
\[
x+9
\]
\end{eulerformula}
\begin{eulerprompt}
>aspect(2,1);...
>figure(1,2);...
>figure(1); plot2d("x^3+2*x+7", grid=2);...
>figure(2); plot2d("x^5-3*x+8", grid=3);...
>figure(3); plot2d("x+9", grid=5);...
>figure(0):
\end{eulerprompt}
\eulerimg{13}{images/EMT 1-6_Mifta Dilla Aninditya_23030630029-339.png}
\begin{eulercomment}
4. Buatlah grafik fungsi dari\\
\end{eulercomment}
\begin{eulerformula}
\[
x^4x
\]
\end{eulerformula}
\begin{eulerprompt}
>reset;
>plot2d("x^4", a=-1, b=1.5, c=-1, d=2, grid=7, style="--");...
>insimg(10):
\end{eulerprompt}
\eulerimg{9}{images/EMT 1-6_Mifta Dilla Aninditya_23030630029-340.png}
\eulerimg{27}{images/EMT 1-6_Mifta Dilla Aninditya_23030630029-341.png}
\begin{eulercomment}
5. Buatlah grafik dalam 1 koordinat \\
\end{eulercomment}
\begin{eulerformula}
\[
tanx
\]
\end{eulerformula}
\begin{eulerformula}
\[
sinx
\]
\end{eulerformula}
\begin{eulerprompt}
>reset;
>aspect();...
>plot2d("tan(x)", 0, pi); plot2d("sin(x)", 0, pi):
\end{eulerprompt}
\eulerimg{27}{images/EMT 1-6_Mifta Dilla Aninditya_23030630029-342.png}
\eulersubheading{}
\eulerheading{Menggambar Plot 3D dengan EMT}
\begin{eulercomment}
Ini adalah pengenalan plot 3D di Euler. Kita memerlukan plot 3D untuk
memvisualisasikan fungsi dua variabel.

Euler menggambar fungsi tersebut menggunakan algoritma pengurutan
untuk menyembunyikan bagian-bagian di latar belakang. Secara umum,
Euler menggunakan proyeksi pusat. Standarnya adalah dari kuadran x-y
positif ke arah titik asal x=y=z=0, tetapi sudut=0° terlihat dari arah
sumbu y. Sudut pandang dan ketinggian dapat diubah.

Euler dapat merencanakan

- permukaan dengan bayangan dan garis level atau rentang level,\\
- awan titik,\\
- kurva parametrik,\\
- permukaan implisit.

Plot 3D dari sebuah fungsi menggunakan plot3d. Cara termudah adalah
dengan memplot ekspresi dalam x dan y. Parameter r mengatur rentang
plot di sekitar (0,0).
\end{eulercomment}
\begin{eulerprompt}
>aspect(1.5); plot3d("x^2+sin(y)",-5,5,0,6*pi):
\end{eulerprompt}
\eulerimg{17}{images/EMT 1-6_Mifta Dilla Aninditya_23030630029-343.png}
\begin{eulerprompt}
>plot3d("x^2+x*sin(y)",-5,5,0,6*pi):
\end{eulerprompt}
\eulerimg{17}{images/EMT 1-6_Mifta Dilla Aninditya_23030630029-344.png}
\begin{eulercomment}
Silakan lakukan modifikasi agar gambar "talang bergelombang" tersebut tidak lurus melainkan melengkung/melingkar, baik
melingkar secara mendatar maupun melingkar turun/naik (seperti papan peluncur pada kolam renang. Temukan rumusnya.
\end{eulercomment}
\eulerheading{Fungsi dari dua Variabel}
\begin{eulercomment}
Untuk grafik sebuah fungsi, gunakan

- ekspresi sederhana dalam x dan y,\\
- nama fungsi dari dua variabell\\
- atau matriks data.

Standarnya adalah kisi-kisi kawat yang terisi dengan warna yang
berbeda di kedua sisi. Perhatikan bahwa jumlah default interval grid
adalah 10, namun plot menggunakan jumlah default 40x40 persegi panjang
untuk membangun permukaan. Hal ini dapat diubah.

- n=40, n=[40,40]: jumlah garis kisi di setiap arah\\
- grid=10, grid=[10,10]: jumlah garis grid di setiap arah.

Kami menggunakan default n=40 dan grid=10.
\end{eulercomment}
\begin{eulerprompt}
>plot3d("x^2+y^2"):
\end{eulerprompt}
\eulerimg{17}{images/EMT 1-6_Mifta Dilla Aninditya_23030630029-345.png}
\begin{eulercomment}
Interaksi pengguna dapat dilakukan dengan parameter \textgreater{}user. Pengguna
dapat menekan tombol berikut ini.

- kiri, kanan, atas, bawah: memutar sudut pandang\\
- +,-: memperbesar atau memperkecil\\
- a: menghasilkan anaglyph (lihat di bawah)\\
- l: beralih memutar sumber cahaya (lihat di bawah)\\
- spasi: mengatur ulang ke default\\
- kembali: mengakhiri interaksi
\end{eulercomment}
\begin{eulerprompt}
>plot3d("exp(-x^2+y^2)",>user, ...
>  title="Turn with the vector keys (press return to finish)"):
\end{eulerprompt}
\eulerimg{17}{images/EMT 1-6_Mifta Dilla Aninditya_23030630029-346.png}
\begin{eulercomment}
Rentang plot untuk fungsi dapat ditentukan dengan

- a, b: rentang x\\
- c, d: rentang y\\
- r: bujur sangkar simetris di sekitar (0,0).\\
- n: jumlah subinterval untuk plot.

Terdapat beberapa parameter untuk menskalakan fungsi atau mengubah
tampilan grafik.

fscale: skala untuk nilai fungsi (standarnya adalah \textless{}fscale).\\
scale: angka atau vektor 1x2 untuk menskalakan ke arah x dan y.\\
frame: jenis bingkai (default 1).
\end{eulercomment}
\begin{eulerprompt}
>plot3d("exp(-(x^2+y^2)/5)",r=10,n=80,fscale=4,scale=1.2,frame=3,>user):
\end{eulerprompt}
\eulerimg{17}{images/EMT 1-6_Mifta Dilla Aninditya_23030630029-347.png}
\begin{eulercomment}
Tampilan dapat diubah dengan berbagai cara.

- jarak: jarak pandang ke plot.\\
- zoom: nilai zoom.\\
- angle: sudut ke sumbu y negatif dalam radian.\\
- height: ketinggian tampilan dalam radian.

Nilai default dapat diperiksa atau diubah dengan fungsi view(). Fungsi
ini mengembalikan parameter dalam urutan di atas.
\end{eulercomment}
\begin{eulerprompt}
>view
\end{eulerprompt}
\begin{euleroutput}
  [5,  2.6,  2,  0.4]
\end{euleroutput}
\begin{eulercomment}
Jarak yang lebih dekat membutuhkan zoom yang lebih sedikit. Efeknya
lebih seperti lensa sudut lebar.

Dalam contoh berikut ini, sudut = 0 dan tinggi = 0 terlihat dari sumbu
y negatif. Label sumbu untuk y disembunyikan dalam kasus ini.
\end{eulercomment}
\begin{eulerprompt}
>plot3d("x^2+y",distance=3,zoom=1,angle=pi/2,height=0):
\end{eulerprompt}
\eulerimg{17}{images/EMT 1-6_Mifta Dilla Aninditya_23030630029-348.png}
\begin{eulercomment}
Plot terlihat selalu ke bagian tengah kubus plot. Anda dapat
memindahkan bagian tengah dengan parameter center.
\end{eulercomment}
\begin{eulerprompt}
>plot3d("x^4+y^2",a=0,b=1,c=-1,d=1,angle=-20°,height=20°, ...
>  center=[0.4,0,0],zoom=5):
\end{eulerprompt}
\eulerimg{17}{images/EMT 1-6_Mifta Dilla Aninditya_23030630029-349.png}
\begin{eulercomment}
Plot diskalakan agar sesuai dengan kubus satuan untuk dilihat. Jadi,
tidak perlu mengubah jarak atau melakukan zoom, tergantung pada ukuran
plot. Namun demikian, label mengacu ke ukuran yang sesungguhnya.

Jika Anda menonaktifkannya dengan scale=false, Anda harus berhati-hati
agar plot tetap muat di dalam jendela plotting, dengan mengubah jarak
tampilan atau zoom, dan memindahkan bagian tengahnya.
\end{eulercomment}
\begin{eulerprompt}
>plot3d("5*exp(-x^2-y^2)",r=2,<fscale,<scale,distance=13,height=50°, ...
>  center=[0,0,-2],frame=3):
\end{eulerprompt}
\eulerimg{17}{images/EMT 1-6_Mifta Dilla Aninditya_23030630029-350.png}
\begin{eulercomment}
Plot polar juga tersedia. Parameter polar=true menggambar plot polar.
Fungsi harus tetap merupakan fungsi dari x dan y. Parameter “fscale”
menskalakan fungsi dengan skala sendiri. Jika tidak, fungsi akan
diskalakan agar sesuai dengan kubus.
\end{eulercomment}
\begin{eulerprompt}
>plot3d("1/(x^2+y^2+1)",r=5,>polar, ...
>fscale=2,>hue,n=100,zoom=4,>contour,color=blue):
\end{eulerprompt}
\eulerimg{17}{images/EMT 1-6_Mifta Dilla Aninditya_23030630029-351.png}
\begin{eulerprompt}
>\(\backslash\)function f(r) := exp(-r/2)*cos(r); ...
\end{eulerprompt}
\begin{euleroutput}
  Syntax error in expression, or unfinished expression!
  Error in:
  \(\backslash\)function f(r) := exp(-r/2)*cos(r); ... ...
  ^
\end{euleroutput}
\begin{eulerprompt}
>plot3d("f(x^2+y^2)",>polar,scale=[1,1,0.4],r=pi,frame=3,zoom=4):
\end{eulerprompt}
\begin{eulercomment}
Parameter rotate memutar fungsi dalam x di sekitar sumbu x.

- rotate = 1: Menggunakan sumbu x\\
- rotate=2: Menggunakan sumbu z
\end{eulercomment}
\begin{eulerprompt}
>plot3d("x^2+1",a=-1,b=1,rotate=true,grid=5):
>plot3d("x^2+1",a=-1,b=1,rotate=2,grid=5):
>plot3d("sqrt(25-x^2)",a=0,b=5,rotate=1):
>plot3d("x*sin(x)",a=0,b=6pi,rotate=2):
\end{eulerprompt}
\begin{eulercomment}
Berikut ini adalah plot dengan tiga fungsi.
\end{eulercomment}
\begin{eulerprompt}
>plot3d("x","x^2+y^2","y",r=2,zoom=3.5,frame=3):
\end{eulerprompt}
\eulerheading{Plot Kontur}
\begin{eulercomment}
Untuk plot, Euler menambahkan garis kisi-kisi. Sebagai gantinya,
dimungkinkan untuk menggunakan garis level dan rona satu warna atau
rona berwarna spektral. Euler dapat menggambar ketinggian fungsi pada
plot dengan bayangan. Pada semua plot 3D, Euler dapat menghasilkan
anaglyph merah/cyan.

- Rona: Mengaktifkan bayangan cahaya, bukan kabel.\\
- \textgreater{}contour: Memplot garis kontur otomatis pada plot.\\
- level=... (atau level): Vektor nilai untuk garis kontur.

Defaultnya adalah level=“auto”, yang menghitung beberapa garis level
secara otomatis. Seperti yang Anda lihat di plot, level sebenarnya
adalah rentang level.

Gaya default dapat diubah. Untuk plot kontur berikut ini, kami
menggunakan grid yang lebih halus untuk titik 100x100, skala fungsi
dan plot, dan menggunakan sudut pandang yang berbeda.
\end{eulercomment}
\begin{eulerprompt}
>plot3d("exp(-x^2-y^2)",r=2,n=100,level="thin", ...
> >contour,>spectral,fscale=1,scale=1.1,angle=45°,height=20°):
\end{eulerprompt}
\eulerimg{17}{images/EMT 1-6_Mifta Dilla Aninditya_23030630029-352.png}
\begin{eulerprompt}
>plot3d("exp(x*y)",angle=100°,>contour,color=green):
\end{eulerprompt}
\eulerimg{17}{images/EMT 1-6_Mifta Dilla Aninditya_23030630029-353.png}
\begin{eulercomment}
Bayangan default menggunakan warna abu-abu. Tetapi, kisaran warna
spektral juga tersedia.

- \textgreater{}spektral: Menggunakan skema spektral default\\
- color =...: Menggunakan warna khusus atau skema spektral

Untuk plot berikut ini, kami menggunakan skema spektral default dan
menambah jumlah titik untuk mendapatkan tampilan yang sangat mulus.
\end{eulercomment}
\begin{eulerprompt}
>plot3d("x^2+y^2",>spectral,>contour,n=100):
\end{eulerprompt}
\eulerimg{17}{images/EMT 1-6_Mifta Dilla Aninditya_23030630029-354.png}
\begin{eulercomment}
Alih-alih garis level otomatis, kita juga dapat menetapkan nilai garis
level. Hal ini akan menghasilkan garis level yang tipis, alih-alih
rentang level.
\end{eulercomment}
\begin{eulerprompt}
>plot3d("x^2-y^2",0,5,0,5,level=-1:0.1:1,color=redgreen):
\end{eulerprompt}
\eulerimg{17}{images/EMT 1-6_Mifta Dilla Aninditya_23030630029-355.png}
\begin{eulercomment}
Pada plot berikut ini, kami menggunakan dua pita level yang sangat
luas dari -0,1 hingga 1, dan dari 0,9 hingga 1. Ini dimasukkan sebagai
matriks dengan batas-batas level sebagai kolom.

Selain itu, kami menghamparkan grid dengan 10 interval di setiap arah.
\end{eulercomment}
\begin{eulerprompt}
>plot3d("x^2+y^3",level=[-0.1,0.9;0,1], ...
>  >spectral,angle=30°,grid=10,contourcolor=gray):
\end{eulerprompt}
\eulerimg{17}{images/EMT 1-6_Mifta Dilla Aninditya_23030630029-356.png}
\begin{eulercomment}
Pada contoh berikut, kami memplot himpunan, di mana

\end{eulercomment}
\begin{eulerformula}
\[
f(x,y) = x^y-y^x = 0
\]
\end{eulerformula}
\begin{eulercomment}
Kita menggunakan satu garis tipis untuk garis level.
\end{eulercomment}
\begin{eulerprompt}
>plot3d("x^y-y^x",level=0,a=0,b=6,c=0,d=6,contourcolor=red,n=100):
\end{eulerprompt}
\eulerimg{17}{images/EMT 1-6_Mifta Dilla Aninditya_23030630029-357.png}
\begin{eulercomment}
Dimungkinkan untuk menampilkan bidang kontur di bawah plot. Warna dan
jarak ke plot dapat ditentukan.
\end{eulercomment}
\begin{eulerprompt}
>plot3d("x^2+y^4",>cp,cpcolor=green,cpdelta=0.2):
\end{eulerprompt}
\eulerimg{17}{images/EMT 1-6_Mifta Dilla Aninditya_23030630029-358.png}
\begin{eulercomment}
Berikut ini beberapa gaya lainnya. Kami selalu mematikan bingkai, dan
menggunakan berbagai skema warna untuk plot dan kisi-kisi.
\end{eulercomment}
\begin{eulerprompt}
>figure(2,2); ...
>expr="y^3-x^2"; ...
>figure(1);  ...
>  plot3d(expr,<frame,>cp,cpcolor=spectral); ...
>figure(2);  ...
>  plot3d(expr,<frame,>spectral,grid=10,cp=2); ...
>figure(3);  ...
>  plot3d(expr,<frame,>contour,color=gray,nc=5,cp=3,cpcolor=greenred); ...
>figure(4);  ...
>  plot3d(expr,<frame,>hue,grid=10,>transparent,>cp,cpcolor=gray); ...
>figure(0):
\end{eulerprompt}
\eulerimg{17}{images/EMT 1-6_Mifta Dilla Aninditya_23030630029-359.png}
\begin{eulercomment}
Ada beberapa skema spektral lainnya, yang diberi nomor dari 1 hingga
9. Tetapi Anda juga dapat menggunakan color=value, di mana value

- spektral: untuk rentang dari biru ke merah\\
- putih: untuk rentang yang lebih redup\\
- kuningbiru, ungu-hijau, biru-kuning, hijau-merah\\
- biru-kuning, hijau-ungu, kuning-biru, merah-hijau
\end{eulercomment}
\begin{eulerprompt}
>figure(3,3); ...
>for i=1:9;  ...
>  figure(i); plot3d("x^2+y^2",spectral=i,>contour,>cp,<frame,zoom=4);  ...
>end; ...
>figure(0):
\end{eulerprompt}
\eulerimg{17}{images/EMT 1-6_Mifta Dilla Aninditya_23030630029-360.png}
\begin{eulercomment}
Sumber cahaya dapat diubah dengan l dan tombol kursor selama interaksi
pengguna. Ini juga dapat ditetapkan dengan parameter.

- light: arah cahaya\\
- amb: cahaya sekitar antara 0 dan 1

Perhatikan, bahwa program ini tidak membuat perbedaan di antara
sisi-sisi plot. Tidak ada bayangan. Untuk ini Anda akan membutuhkan
Povray.
\end{eulercomment}
\begin{eulerprompt}
>plot3d("-x^2-y^2", ...
>  hue=true,light=[0,1,1],amb=0,user=true, ...
>  title="Press l and cursor keys (return to exit)"):
\end{eulerprompt}
\eulerimg{17}{images/EMT 1-6_Mifta Dilla Aninditya_23030630029-361.png}
\begin{eulercomment}
Parameter warna mengubah warna permukaan. Warna garis level juga dapat
diubah.
\end{eulercomment}
\begin{eulerprompt}
>plot3d("-x^2-y^2",color=rgb(0.2,0.2,0),hue=true,frame=false, ...
>  zoom=3,contourcolor=red,level=-2:0.1:1,dl=0.01):
\end{eulerprompt}
\eulerimg{17}{images/EMT 1-6_Mifta Dilla Aninditya_23030630029-362.png}
\begin{eulercomment}
Warna 0 memberikan efek pelangi yang istimewa.
\end{eulercomment}
\begin{eulerprompt}
>plot3d("x^2/(x^2+y^2+1)",color=0,hue=true,grid=10):
\end{eulerprompt}
\eulerimg{17}{images/EMT 1-6_Mifta Dilla Aninditya_23030630029-363.png}
\begin{eulercomment}
Permukaannya juga bisa transparan.
\end{eulercomment}
\begin{eulerprompt}
>plot3d("x^2+y^2",>transparent,grid=10,wirecolor=red):
\end{eulerprompt}
\eulerimg{17}{images/EMT 1-6_Mifta Dilla Aninditya_23030630029-364.png}
\eulerheading{Plot Implisit}
\begin{eulercomment}
Ada juga plot implisit dalam tiga dimensi. Euler menghasilkan potongan
melalui objek. Fitur plot3d termasuk plot implisit. Plot-plot ini
menunjukkan himpunan nol dari sebuah fungsi dalam tiga variabel.\\
Solusi dari

\end{eulercomment}
\begin{eulerformula}
\[
f(x,y,z) = 0
\]
\end{eulerformula}
\begin{eulercomment}
dapat divisualisasikan dalam potongan yang sejajar dengan bidang x-y,
bidang x-z, dan bidang y-z.

- implisit = 1: potong sejajar dengan bidang-y-z\\
- implicit = 2: memotong sejajar dengan bidang x-z\\
- implicit=4: memotong sejajar dengan bidang x-y

Tambahkan nilai-nilai ini, jika Anda mau. Pada contoh, kami memplot

\end{eulercomment}
\begin{eulerformula}
\[
M = \{ (x,y,z) : x^2+y^3+zy=1 \}
\]
\end{eulerformula}
\begin{eulerprompt}
>plot3d("x^2+y^3+z*y-1",r=5,implicit=3):
\end{eulerprompt}
\eulerimg{17}{images/EMT 1-6_Mifta Dilla Aninditya_23030630029-365.png}
\begin{eulerprompt}
>c=1; d=1;
>plot3d("((x^2+y^2-c^2)^2+(z^2-1)^2)*((y^2+z^2-c^2)^2+(x^2-1)^2)*((z^2+x^2-c^2)^2+(y^2-1)^2)-d",r=2,<frame,>implicit,>user): 
\end{eulerprompt}
\eulerimg{17}{images/EMT 1-6_Mifta Dilla Aninditya_23030630029-366.png}
\begin{eulerprompt}
>plot3d("x^2+y^2+4*x*z+z^3",>implicit,r=2,zoom=2.5):
\end{eulerprompt}
\eulerimg{17}{images/EMT 1-6_Mifta Dilla Aninditya_23030630029-367.png}
\eulerheading{Memplot Data 3D}
\begin{eulercomment}
Sama seperti plot2d, plot3d menerima data. Untuk objek 3D, Anda perlu
menyediakan matriks nilai x, y, dan z, atau tiga fungsi atau ekspresi
fx(x,y), fy(x,y), fz(x,y).

\end{eulercomment}
\begin{eulerformula}
\[
\gamma(t,s) = (x(t,s),y(t,s),z(t,s))
\]
\end{eulerformula}
\begin{eulercomment}
Karena x,y,z adalah matriks, kita mengasumsikan bahwa (t,s) berjalan
melalui kotak persegi. Hasilnya, Anda dapat memplot gambar persegi
panjang dalam ruang.

Anda dapat menggunakan bahasa matriks Euler untuk menghasilkan
koordinat secara efektif.

Pada contoh berikut, kita menggunakan vektor nilai t dan vektor kolom
nilai s untuk memparameterkan permukaan bola. Pada gambar kita dapat
menandai daerah, dalam kasus kita daerah kutub.
\end{eulercomment}
\begin{eulerprompt}
>t=linspace(0,2pi,180); s=linspace(-pi/2,pi/2,90)'; ...
>x=cos(s)*cos(t); y=cos(s)*sin(t); z=sin(s); ...
>plot3d(x,y,z,>hue, ...
>color=blue,<frame,grid=[10,20], ...
>values=s,contourcolor=red,level=[90°-24°;90°-22°], ...
>scale=1.4,height=50°):
\end{eulerprompt}
\eulerimg{17}{images/EMT 1-6_Mifta Dilla Aninditya_23030630029-368.png}
\begin{eulercomment}
Berikut ini adalah contoh, yang merupakan grafik suatu fungsi.
\end{eulercomment}
\begin{eulerprompt}
>t=-1:0.1:1; s=(-1:0.1:1)'; plot3d(t,s,t*s,grid=10):
\end{eulerprompt}
\eulerimg{17}{images/EMT 1-6_Mifta Dilla Aninditya_23030630029-369.png}
\begin{eulercomment}
Namun demikian, kita bisa membuat segala macam permukaan. Berikut ini
adalah permukaan yang sama dengan fungsi

\end{eulercomment}
\begin{eulerformula}
\[
x = y \, z
\]
\end{eulerformula}
\begin{eulerprompt}
>plot3d(t*s,t,s,angle=180°,grid=10):
\end{eulerprompt}
\eulerimg{17}{images/EMT 1-6_Mifta Dilla Aninditya_23030630029-370.png}
\begin{eulercomment}
Dengan lebih banyak upaya, kita bisa menghasilkan banyak permukaan.

Dalam contoh berikut ini, kami membuat tampilan berbayang dari bola
yang terdistorsi. Koordinat yang biasa digunakan untuk bola adalah

\end{eulercomment}
\begin{eulerformula}
\[
\gamma(t,s) = (\cos(t)\cos(s),\sin(t)\sin(s),\cos(s))
\]
\end{eulerformula}
\begin{eulercomment}
dengan

\end{eulercomment}
\begin{eulerformula}
\[
0 \le t \le 2\pi, \quad \frac{-\pi}{2} \le s \le \frac{\pi}{2}.
\]
\end{eulerformula}
\begin{eulercomment}
Kami mengubahnya dengan sebuah faktor

\end{eulercomment}
\begin{eulerformula}
\[
d(t,s) = \frac{\cos(4t)+\cos(8s)}{4}.
\]
\end{eulerformula}
\begin{eulerprompt}
>t=linspace(0,2pi,320); s=linspace(-pi/2,pi/2,160)'; ...
>d=1+0.2*(cos(4*t)+cos(8*s)); ...
>plot3d(cos(t)*cos(s)*d,sin(t)*cos(s)*d,sin(s)*d,hue=1, ...
>  light=[1,0,1],frame=0,zoom=5):
\end{eulerprompt}
\eulerimg{17}{images/EMT 1-6_Mifta Dilla Aninditya_23030630029-371.png}
\begin{eulercomment}
Tentu saja, awan titik juga dimungkinkan. Untuk memplot data titik
dalam ruang, kita membutuhkan tiga vektor untuk koordinat titik.

Gaya-gayanya sama seperti pada plot2d dengan points=true;
\end{eulercomment}
\begin{eulerprompt}
>n=500;  ...
>  plot3d(normal(1,n),normal(1,n),normal(1,n),points=true,style="."):
\end{eulerprompt}
\eulerimg{17}{images/EMT 1-6_Mifta Dilla Aninditya_23030630029-372.png}
\begin{eulercomment}
Anda juga dapat memplot kurva dalam bentuk 3D. Dalam hal ini, akan
lebih mudah untuk menghitung titik-titik kurva. Untuk kurva pada
bidang, kami menggunakan urutan koordinat dan parameter wire = true.
\end{eulercomment}
\begin{eulerprompt}
>t=linspace(0,8pi,500); ...
>plot3d(sin(t),cos(t),t/10,>wire,zoom=3):
\end{eulerprompt}
\eulerimg{17}{images/EMT 1-6_Mifta Dilla Aninditya_23030630029-373.png}
\begin{eulerprompt}
>t=linspace(0,4pi,1000); plot3d(cos(t),sin(t),t/2pi,>wire, ...
>linewidth=3,wirecolor=blue):
\end{eulerprompt}
\eulerimg{17}{images/EMT 1-6_Mifta Dilla Aninditya_23030630029-374.png}
\begin{eulerprompt}
>X=cumsum(normal(3,100)); ...
> plot3d(X[1],X[2],X[3],>anaglyph,>wire):
\end{eulerprompt}
\eulerimg{17}{images/EMT 1-6_Mifta Dilla Aninditya_23030630029-375.png}
\begin{eulercomment}
EMT juga dapat membuat plot dalam mode anaglyph. Untuk melihat plot
semacam itu, Anda memerlukan kacamata merah/cyan.
\end{eulercomment}
\begin{eulerprompt}
> plot3d("x^2+y^3",>anaglyph,>contour,angle=30°):
\end{eulerprompt}
\eulerimg{17}{images/EMT 1-6_Mifta Dilla Aninditya_23030630029-376.png}
\begin{eulercomment}
Sering kali, skema warna spektral digunakan untuk plot. Hal ini
menekankan ketinggian fungsi.
\end{eulercomment}
\begin{eulerprompt}
>plot3d("x^2*y^3-y",>spectral,>contour,zoom=3.2):
\end{eulerprompt}
\eulerimg{17}{images/EMT 1-6_Mifta Dilla Aninditya_23030630029-377.png}
\begin{eulercomment}
Euler juga dapat memplot permukaan yang diparameterkan, ketika
parameternya adalah nilai x, y, dan z dari sebuah gambar kisi-kisi
persegi panjang di dalam ruang.

Untuk demo berikut ini, kami menyiapkan parameter u dan v, dan
menghasilkan koordinat ruang dari parameter ini.
\end{eulercomment}
\begin{eulerprompt}
>u=linspace(-1,1,10); v=linspace(0,2*pi,50)'; ...
>X=(3+u*cos(v/2))*cos(v); Y=(3+u*cos(v/2))*sin(v); Z=u*sin(v/2); ...
>plot3d(X,Y,Z,>anaglyph,<frame,>wire,scale=2.3):
\end{eulerprompt}
\eulerimg{17}{images/EMT 1-6_Mifta Dilla Aninditya_23030630029-378.png}
\begin{eulercomment}
Berikut ini contoh yang lebih rumit, yang tampak megah dengan kacamata
merah/cyan.
\end{eulercomment}
\begin{eulerprompt}
>u:=linspace(-pi,pi,160); v:=linspace(-pi,pi,400)';  ...
>x:=(4*(1+.25*sin(3*v))+cos(u))*cos(2*v); ...
>y:=(4*(1+.25*sin(3*v))+cos(u))*sin(2*v); ...
> z=sin(u)+2*cos(3*v); ...
>plot3d(x,y,z,frame=0,scale=1.5,hue=1,light=[1,0,-1],zoom=2.8,>anaglyph):
\end{eulerprompt}
\eulerimg{17}{images/EMT 1-6_Mifta Dilla Aninditya_23030630029-379.png}
\eulerheading{Plot Statistik}
\begin{eulercomment}
Plot batang juga dapat digunakan. Untuk ini, kita harus menyediakan

- x: vektor baris dengan n+1 elemen\\
- y: vektor kolom dengan n+1 elemen\\
- z: matriks nilai berukuran nxn.

z dapat lebih besar, tetapi hanya nilai nxn yang akan digunakan.

Pada contoh, pertama-tama kita menghitung nilainya. Kemudian kita
sesuaikan x dan y, sehingga vektor berada di tengah-tengah nilai yang
digunakan.
\end{eulercomment}
\begin{eulerprompt}
>x=-1:0.1:1; y=x'; z=x^2+y^2; ...
>xa=(x|1.1)-0.05; ya=(y_1.1)-0.05; ...
>plot3d(xa,ya,z,bar=true):
\end{eulerprompt}
\eulerimg{17}{images/EMT 1-6_Mifta Dilla Aninditya_23030630029-380.png}
\begin{eulercomment}
Dimungkinkan untuk membagi plot permukaan menjadi dua bagian atau
lebih.
\end{eulercomment}
\begin{eulerprompt}
>x=-1:0.1:1; y=x'; z=x+y; d=zeros(size(x)); ...
>plot3d(x,y,z,disconnect=2:2:20):
\end{eulerprompt}
\eulerimg{17}{images/EMT 1-6_Mifta Dilla Aninditya_23030630029-381.png}
\begin{eulercomment}
Jika memuat atau menghasilkan matriks data M dari file dan perlu
memplotnya dalam 3D, Anda dapat menskalakan matriks ke [-1,1] dengan
scale(M), atau menskalakan matriks dengan \textgreater{}zscale. Hal ini dapat
dikombinasikan dengan faktor penskalaan individual yang diterapkan
sebagai tambahan.
\end{eulercomment}
\begin{eulerprompt}
>i=1:20; j=i'; ...
>plot3d(i*j^2+100*normal(20,20),>zscale,scale=[1,1,1.5],angle=-40°,zoom=1.8):
\end{eulerprompt}
\eulerimg{17}{images/EMT 1-6_Mifta Dilla Aninditya_23030630029-382.png}
\begin{eulerprompt}
>Z=intrandom(5,100,6); v=zeros(5,6); ...
>loop 1 to 5; v[#]=getmultiplicities(1:6,Z[#]); end; ...
>columnsplot3d(v',scols=1:5,ccols=[1:5]):
\end{eulerprompt}
\eulerimg{17}{images/EMT 1-6_Mifta Dilla Aninditya_23030630029-383.png}
\eulerheading{Permukaan Benda Putar}
\begin{eulerprompt}
>plot2d("(x^2+y^2-1)^3-x^2*y^3",r=1.3, ...
>style="#",color=red,<outline, ...
>level=[-2;0],n=100):
\end{eulerprompt}
\eulerimg{17}{images/EMT 1-6_Mifta Dilla Aninditya_23030630029-384.png}
\begin{eulerprompt}
>ekspresi &= (x^2+y^2-1)^3-x^2*y^3; $ekspresi
\end{eulerprompt}
\begin{eulerformula}
\[
\left(y^2+x^2-1\right)^3-x^2\,y^3
\]
\end{eulerformula}
\begin{eulercomment}
Kami ingin memutar kurva jantung di sekitar sumbu y. Berikut ini
adalah ekspresi yang mendefinisikan jantung:

\end{eulercomment}
\begin{eulerformula}
\[
f(x,y)=(x^2+y^2-1)^3-x^2.y^3.
\]
\end{eulerformula}
\begin{eulercomment}
Selanjutnya kita atur

\end{eulercomment}
\begin{eulerformula}
\[
x = r.cos(a),\quad y = r.sin(a).
\]
\end{eulerformula}
\begin{eulerprompt}
>function fr(r,a) &= ekspresi with [x=r*cos(a),y=r*sin(a)] | trigreduce; $fr(r,a)
\end{eulerprompt}
\begin{eulerformula}
\[
\left(r^2-1\right)^3+\frac{\left(\sin \left(5\,a\right)-\sin \left(
 3\,a\right)-2\,\sin a\right)\,r^5}{16}
\]
\end{eulerformula}
\begin{eulercomment}
Hal ini memungkinkan untuk mendefinisikan fungsi numerik, yang
menyelesaikan untuk r, jika a diberikan. Dengan fungsi tersebut kita
dapat memplotkan jantung yang diputar sebagai permukaan parametrik.
\end{eulercomment}
\begin{eulerprompt}
>function map f(a) := bisect("fr",0,2;a); ...
>t=linspace(-pi/2,pi/2,100); r=f(t);  ...
>s=linspace(pi,2pi,100)'; ...
>plot3d(r*cos(t)*sin(s),r*cos(t)*cos(s),r*sin(t), ...
>>hue,<frame,color=red,zoom=4,amb=0,max=0.7,grid=12,height=50°):
\end{eulerprompt}
\eulerimg{17}{images/EMT 1-6_Mifta Dilla Aninditya_23030630029-387.png}
\begin{eulercomment}
Berikut ini adalah plot 3D dari gambar di atas yang diputar
mengelilingi sumbu-z. Kami mendefinisikan fungsi, yang menggambarkan
objek.
\end{eulercomment}
\begin{eulerprompt}
>function f(x,y,z) ...
\end{eulerprompt}
\begin{eulerudf}
  r=x^2+y^2;
  return (r+z^2-1)^3-r*z^3;
   endfunction
\end{eulerudf}
\begin{eulerprompt}
>plot3d("f(x,y,z)", ...
>xmin=0,xmax=1.2,ymin=-1.2,ymax=1.2,zmin=-1.2,zmax=1.4, ...
>implicit=1,angle=-30°,zoom=2.5,n=[10,100,60],>anaglyph):
\end{eulerprompt}
\eulerimg{17}{images/EMT 1-6_Mifta Dilla Aninditya_23030630029-388.png}
\eulerheading{Plot 3D Khusus}
\begin{eulercomment}
Fungsi plot3d memang bagus untuk dimiliki, tetapi tidak memenuhi semua
kebutuhan. Selain rutinitas yang lebih mendasar, Anda juga bisa
mendapatkan plot berbingkai dari objek apa pun yang Anda sukai.

Meskipun Euler bukan program 3D, namun dapat menggabungkan beberapa
objek dasar. Kami mencoba memvisualisasikan parabola dan garis
singgungnya.
\end{eulercomment}
\begin{eulerprompt}
>function myplot ...
\end{eulerprompt}
\begin{eulerudf}
    y=-1:0.01:1; x=(-1:0.01:1)';
    plot3d(x,y,0.2*(x-0.1)/2,<scale,<frame,>hue, ..
      hues=0.5,>contour,color=orange);
    h=holding(1);
    plot3d(x,y,(x^2+y^2)/2,<scale,<frame,>contour,>hue);
    holding(h);
  endfunction
\end{eulerudf}
\begin{eulercomment}
Sekarang framedplot() menyediakan frame, dan mengatur tampilan.
\end{eulercomment}
\begin{eulerprompt}
>framedplot("myplot",[-1,1,-1,1,0,1],height=0,angle=-30°, ...
>  center=[0,0,-0.7],zoom=3):
\end{eulerprompt}
\eulerimg{17}{images/EMT 1-6_Mifta Dilla Aninditya_23030630029-389.png}
\begin{eulercomment}
Dengan cara yang sama, Anda dapat memplot bidang kontur secara manual.
Perhatikan bahwa plot3d() mengatur jendela ke fullwindow() secara
default, namun plotcontourplane() mengasumsikannya.
\end{eulercomment}
\begin{eulerprompt}
>x=-1:0.02:1.1; y=x'; z=x^2-y^4;
>function myplot (x,y,z) ...
\end{eulerprompt}
\begin{eulerudf}
    zoom(2);
    wi=fullwindow();
    plotcontourplane(x,y,z,level="auto",<scale);
    plot3d(x,y,z,>hue,<scale,>add,color=white,level="thin");
    window(wi);
    reset();
  endfunction
\end{eulerudf}
\begin{eulerprompt}
>myplot(x,y,z):
\end{eulerprompt}
\eulerimg{27}{images/EMT 1-6_Mifta Dilla Aninditya_23030630029-390.png}
\eulerheading{Animasi}
\begin{eulercomment}
Euler dapat menggunakan frame untuk melakukan pra-komputasi animasi.

Salah satu fungsi yang memanfaatkan teknik ini adalah rotate. Fungsi
ini dapat mengubah sudut pandang dan menggambar ulang plot 3D. Fungsi
ini memanggil addpage() untuk setiap plot baru. Akhirnya fungsi ini
menganimasikan plot tersebut.

Silakan pelajari sumber dari rotate untuk melihat lebih detail.
\end{eulercomment}
\begin{eulerprompt}
>function testplot () := plot3d("x^2+y^3"); ...
>rotate("testplot"); testplot():
\end{eulerprompt}
\eulerimg{27}{images/EMT 1-6_Mifta Dilla Aninditya_23030630029-391.png}
\eulerheading{Menggambar Povray}
\begin{eulercomment}
Dengan bantuan file Euler povray.e, Euler dapat menghasilkan file
Povray. Hasilnya sangat bagus untuk dilihat.

Anda perlu menginstal Povray (32bit atau 64bit) dari
http://www.povray.org/, dan meletakkan sub-direktori “bin” dari Povray ke dalam jalur lingkungan, atau mengatur variabel “defaultpovray” dengan jalur lengkap yang mengarah ke “pvengine.exe”.

Antarmuka Povray dari Euler menghasilkan file Povray di direktori home
pengguna, dan memanggil Povray untuk mengurai file-file ini. Nama file
default adalah current.pov, dan direktori defaultnya adalah
eulerhome(), biasanya c:\textbackslash{}Users\textbackslash{}Username\textbackslash{}Euler. Povray menghasilkan
sebuah file PNG, yang dapat dimuat oleh Euler ke dalam notebook. Untuk
membersihkan berkas-berkas ini, gunakan povclear().

Fungsi pov3d memiliki semangat yang sama dengan plot3d. Fungsi ini
dapat menghasilkan grafik dari sebuah fungsi f(x,y), atau sebuah
permukaan dengan koordinat X,Y,Z dalam bentuk matriks, termasuk
garis-garis level yang bersifat opsional. Fungsi ini memulai raytracer
secara otomatis, dan memuat adegan ke dalam notebook Euler.

Selain pov3d(), ada banyak fungsi yang menghasilkan objek Povray.
Fungsi-fungsi ini mengembalikan string, yang berisi kode Povray untuk
objek. Untuk menggunakan fungsi-fungsi ini, mulai file Povray dengan
povstart(). Kemudian gunakan writeln(...) untuk menulis objek ke file
scene. Terakhir, akhiri file dengan povend(). Secara default,
raytracer akan dimulai, dan PNG akan dimasukkan ke dalam buku catatan
Euler.

Fungsi objek memiliki parameter yang disebut “look”, yang membutuhkan
string dengan kode povray untuk tekstur dan hasil akhir objek. Fungsi
povlook() dapat digunakan untuk menghasilkan string ini. Fungsi ini
memiliki parameter untuk warna, transparansi, Phong Shading, dll.

Perhatikan bahwa Povray universe memiliki sistem koordinat lain.
Antarmuka ini menerjemahkan semua koordinat ke sistem Povray. Jadi
Anda dapat tetap berpikir dalam sistem koordinat Euler dengan z yang
mengarah vertikal ke atas, dan sumbu x, y, z di tangan kanan.\\
Anda perlu memuat file povray.
\end{eulercomment}
\begin{eulerprompt}
>load povray;
\end{eulerprompt}
\begin{eulercomment}
Pastikan direktori bin povray berada di dalam path. Jika tidak, edit
variabel berikut sehingga berisi jalur ke povray yang dapat
dieksekusi.
\end{eulercomment}
\begin{eulerprompt}
>defaultpovray="C:\(\backslash\)Program Files\(\backslash\)POV-Ray\(\backslash\)v3.7\(\backslash\)bin\(\backslash\)pvengine.exe"
\end{eulerprompt}
\begin{euleroutput}
  C:\(\backslash\)Program Files\(\backslash\)POV-Ray\(\backslash\)v3.7\(\backslash\)bin\(\backslash\)pvengine.exe
\end{euleroutput}
\begin{eulercomment}
Sebagai kesan pertama, kita merencanakan sebuah fungsi sederhana.
Perintah berikut ini menghasilkan file povray di direktori pengguna
Anda, dan menjalankan Povray untuk melacak sinar pada file ini.

Jika Anda memulai perintah berikut, GUI Povray akan terbuka,
menjalankan file, dan menutup secara otomatis. Karena alasan keamanan,
Anda akan ditanya, apakah Anda ingin mengizinkan file exe dijalankan.
Anda dapat menekan cancel untuk menghentikan pertanyaan lebih lanjut.
Anda mungkin harus menekan OK pada jendela Povray untuk mengetahui
dialog awal Povray.
\end{eulercomment}
\begin{eulerprompt}
>plot3d("x^2+y^2",zoom=2):
\end{eulerprompt}
\eulerimg{27}{images/EMT 1-6_Mifta Dilla Aninditya_23030630029-392.png}
\begin{eulerprompt}
>pov3d("x^2+y^2",zoom=3);
\end{eulerprompt}
\eulerimg{28}{images/EMT 1-6_Mifta Dilla Aninditya_23030630029-393.png}
\begin{eulercomment}
Kita dapat membuat fungsi menjadi transparan dan menambahkan hasil
akhir lainnya. Kita juga dapat menambahkan garis level ke plot fungsi.
\end{eulercomment}
\begin{eulerprompt}
>pov3d("x^2+y^3",axiscolor=red,angle=-45°,>anaglyph, ...
>  look=povlook(cyan,0.2),level=-1:0.5:1,zoom=3.8);
\end{eulerprompt}
\begin{euleroutput}
  Command was not allowed!
  exec:
      return _exec(program,param,dir,print,hidden,wait);
  povray:
      exec(program,params,defaulthome);
  Try "trace errors" to inspect local variables after errors.
  pov3d:
      if povray then povray(currentfile,w,h,w/h); endif;
\end{euleroutput}
\begin{eulercomment}
Kadang-kadang perlu untuk mencegah penskalaan fungsi, dan menskalakan
fungsi dengan tangan.

Kami memplot kumpulan titik pada bidang kompleks, di mana hasil kali
jarak ke 1 dan -1 sama dengan 1.
\end{eulercomment}
\begin{eulerprompt}
>pov3d("((x-1)^2+y^2)*((x+1)^2+y^2)/40",r=2, ...
>  angle=-120°,level=1/40,dlevel=0.005,light=[-1,1,1],height=10°,n=50, ...
>  <fscale,zoom=3.8);
\end{eulerprompt}
\eulerimg{28}{images/EMT 1-6_Mifta Dilla Aninditya_23030630029-394.png}
\eulerheading{Merencanakan dengan Koordinat}
\begin{eulercomment}
Sebagai pengganti fungsi, kita dapat membuat plot dengan koordinat.
Seperti pada plot3d, kita membutuhkan tiga matriks untuk
mendefinisikan objek.

Pada contoh, kita memutar sebuah fungsi pada sumbu z.
\end{eulercomment}
\begin{eulerprompt}
>function f(x) := x^3-x+1; ...
>x=-1:0.01:1; t=linspace(0,2pi,50)'; ...
>Z=x; X=cos(t)*f(x); Y=sin(t)*f(x); ...
>pov3d(X,Y,Z,angle=40°,look=povlook(red,0.1),height=50°,axis=0,zoom=4,light=[10,5,15]);
\end{eulerprompt}
\eulerimg{28}{images/EMT 1-6_Mifta Dilla Aninditya_23030630029-395.png}
\begin{eulercomment}
Pada contoh berikut, kita memplot gelombang teredam. Kami menghasilkan
gelombang dengan bahasa matriks Euler.

Kami juga menunjukkan, bagaimana objek tambahan dapat ditambahkan ke
adegan pov3d. Untuk pembuatan objek, lihat contoh berikut. Perhatikan
bahwa plot3d menskalakan plot, sehingga sesuai dengan kubus satuan.
\end{eulercomment}
\begin{eulerprompt}
>r=linspace(0,1,80); phi=linspace(0,2pi,80)'; ...
>x=r*cos(phi); y=r*sin(phi); z=exp(-5*r)*cos(8*pi*r)/3;  ...
>pov3d(x,y,z,zoom=6,axis=0,height=30°,add=povsphere([0.5,0,0.25],0.15,povlook(red)), ...
>  w=500,h=300);
\end{eulerprompt}
\eulerimg{16}{images/EMT 1-6_Mifta Dilla Aninditya_23030630029-396.png}
\begin{eulercomment}
Dengan metode bayangan canggih Povray, hanya sedikit titik yang bisa
menghasilkan permukaan yang sangat halus. Hanya pada batas-batas dan
bayangan, trik ini bisa terlihat jelas.

Untuk itu, kita perlu menambahkan vektor normal di setiap titik
matriks.
\end{eulercomment}
\begin{eulerprompt}
>Z &= x^2*y^3
\end{eulerprompt}
\begin{euleroutput}
  
                                   2  3
                                  x  y
  
\end{euleroutput}
\begin{eulercomment}
Persamaan permukaannya adalah [x,y,Z]. Kami menghitung dua turunan
terhadap x dan y dari persamaan ini dan mengambil hasil perkalian
silang sebagai normal.
\end{eulercomment}
\begin{eulerprompt}
>dx &= diff([x,y,Z],x); dy &= diff([x,y,Z],y);
\end{eulerprompt}
\begin{eulercomment}
Kami mendefinisikan normal sebagai hasil kali silang dari turunan ini,
dan mendefinisikan fungsi koordinat.
\end{eulercomment}
\begin{eulerprompt}
>N &= crossproduct(dx,dy); NX &= N[1]; NY &= N[2]; NZ &= N[3]; N,
\end{eulerprompt}
\begin{euleroutput}
  
                                 3       2  2
                         [- 2 x y , - 3 x  y , 1]
  
\end{euleroutput}
\begin{eulercomment}
Kami hanya menggunakan 25 poin.
\end{eulercomment}
\begin{eulerprompt}
>x=-1:0.5:1; y=x';
>pov3d(x,y,Z(x,y),angle=10°, ...
>  xv=NX(x,y),yv=NY(x,y),zv=NZ(x,y),<shadow);
\end{eulerprompt}
\eulerimg{28}{images/EMT 1-6_Mifta Dilla Aninditya_23030630029-397.png}
\begin{eulercomment}
Berikut ini adalah simpul Trefoil yang dibuat oleh A. Busser di
Povray. Ada versi yang lebih baik dari ini dalam contoh.

See: Examples\textbackslash{}Trefoil Knot \textbar{} Trefoil Knot

Untuk tampilan yang bagus dengan tidak terlalu banyak titik, kita
tambahkan vektor normal di sini. Kita menggunakan Maxima untuk
menghitung normal untuk kita. Pertama, tiga fungsi untuk koordinat
sebagai ekspresi simbolik.
\end{eulercomment}
\begin{eulerprompt}
>X &= ((4+sin(3*y))+cos(x))*cos(2*y); ...
>Y &= ((4+sin(3*y))+cos(x))*sin(2*y); ...
>Z &= sin(x)+2*cos(3*y);
\end{eulerprompt}
\begin{eulercomment}
Kemudian dua vektor turunan terhadap x dan y.
\end{eulercomment}
\begin{eulerprompt}
>dx &= diff([X,Y,Z],x); dy &= diff([X,Y,Z],y);
\end{eulerprompt}
\begin{eulercomment}
Sekarang yang normal, yang merupakan produk silang dari dua turunan.
\end{eulercomment}
\begin{eulerprompt}
>dn &= crossproduct(dx,dy);
\end{eulerprompt}
\begin{eulercomment}
Kami sekarang mengevaluasi semua ini secara numerik.
\end{eulercomment}
\begin{eulerprompt}
>x:=linspace(-%pi,%pi,40); y:=linspace(-%pi,%pi,100)';
\end{eulerprompt}
\begin{eulercomment}
Vektor normal adalah evaluasi dari ekspresi simbolik dn[i] untuk
i=1,2,3. Sintaks untuk ini adalah \&“ekspresi”(parameter). Ini adalah
sebuah alternatif dari metode pada contoh sebelumnya, di mana kita
mendefinisikan ekspresi simbolik NX, NY, NZ terlebih dahulu.
\end{eulercomment}
\begin{eulerprompt}
>pov3d(X(x,y),Y(x,y),Z(x,y),>anaglyph,axis=0,zoom=5,w=450,h=350, ...
>  <shadow,look=povlook(blue), ...
>  xv=&"dn[1]"(x,y), yv=&"dn[2]"(x,y), zv=&"dn[3]"(x,y));
\end{eulerprompt}
\eulerimg{21}{images/EMT 1-6_Mifta Dilla Aninditya_23030630029-398.png}
\begin{eulercomment}
Kami juga dapat menghasilkan kisi-kisi dalam bentuk 3D.
\end{eulercomment}
\begin{eulerprompt}
>povstart(zoom=4); ...
>x=-1:0.5:1; r=1-(x+1)^2/6; ...
>t=(0°:30°:360°)'; y=r*cos(t); z=r*sin(t); ...
>writeln(povgrid(x,y,z,d=0.02,dballs=0.05)); ...
>povend();
\end{eulerprompt}
\eulerimg{28}{images/EMT 1-6_Mifta Dilla Aninditya_23030630029-399.png}
\begin{eulercomment}
Dengan povgrid(), kurva dapat dibuat.
\end{eulercomment}
\begin{eulerprompt}
>povstart(center=[0,0,1],zoom=3.6); ...
>t=linspace(0,2,1000); r=exp(-t); ...
>x=cos(2*pi*10*t)*r; y=sin(2*pi*10*t)*r; z=t; ...
>writeln(povgrid(x,y,z,povlook(red))); ...
>writeAxis(0,2,axis=3); ...
>povend();
\end{eulerprompt}
\eulerimg{28}{images/EMT 1-6_Mifta Dilla Aninditya_23030630029-400.png}
\eulerheading{Objek Povray}
\begin{eulercomment}
Di atas, kami menggunakan pov3d untuk memplot permukaan. Antarmuka
povray di Euler juga dapat menghasilkan objek Povray. Objek-objek ini
disimpan sebagai string di Euler, dan perlu ditulis ke file Povray.

Kita memulai output dengan povstart().
\end{eulercomment}
\begin{eulerprompt}
>povstart(zoom=4);
\end{eulerprompt}
\begin{eulercomment}
Pertama, kita mendefinisikan tiga silinder, dan menyimpannya dalam
bentuk string di Euler.

Fungsi povx() dll. hanya mengembalikan vektor [1,0,0], yang dapat
digunakan sebagai gantinya.
\end{eulercomment}
\begin{eulerprompt}
>c1=povcylinder(-povx,povx,1,povlook(red)); ...
>c2=povcylinder(-povy,povy,1,povlook(yellow)); ...
>c3=povcylinder(-povz,povz,1,povlook(blue)); ...
\end{eulerprompt}
\begin{eulercomment}
String berisi kode Povray, yang tidak perlu kita pahami pada saat itu.
\end{eulercomment}
\begin{eulerprompt}
>c2
\end{eulerprompt}
\begin{euleroutput}
  cylinder \{ <0,0,-1>, <0,0,1>, 1
   texture \{ pigment \{ color rgb <0.941176,0.941176,0.392157> \}  \} 
   finish \{ ambient 0.2 \} 
   \}
\end{euleroutput}
\begin{eulercomment}
Seperti yang Anda lihat, kami menambahkan tekstur ke objek dalam tiga
warna berbeda.

Hal ini dilakukan dengan povlook(), yang mengembalikan sebuah string
dengan kode Povray yang relevan. Kita dapat menggunakan warna default
Euler, atau menentukan warna kita sendiri. Kita juga dapat menambahkan
transparansi, atau mengubah cahaya sekitar.
\end{eulercomment}
\begin{eulerprompt}
>povlook(rgb(0.1,0.2,0.3),0.1,0.5)
\end{eulerprompt}
\begin{euleroutput}
   texture \{ pigment \{ color rgbf <0.101961,0.2,0.301961,0.1> \}  \} 
   finish \{ ambient 0.5 \} 
  
\end{euleroutput}
\begin{eulercomment}
Sekarang kita mendefinisikan objek perpotongan, dan menulis hasilnya
ke file.
\end{eulercomment}
\begin{eulerprompt}
>writeln(povintersection([c1,c2,c3]));
\end{eulerprompt}
\begin{eulercomment}
Perpotongan tiga silinder sulit untuk divisualisasikan, jika Anda
belum pernah melihatnya.
\end{eulercomment}
\begin{eulerprompt}
>povend;
\end{eulerprompt}
\eulerimg{28}{images/EMT 1-6_Mifta Dilla Aninditya_23030630029-401.png}
\begin{eulercomment}
Fungsi-fungsi berikut ini menghasilkan fraktal secara rekursif.

Fungsi pertama menunjukkan, bagaimana Euler menangani objek Povray
sederhana. Fungsi povbox() mengembalikan sebuah string, yang berisi
koordinat kotak, tekstur dan hasil akhir.
\end{eulercomment}
\begin{eulerprompt}
>function onebox(x,y,z,d) := povbox([x,y,z],[x+d,y+d,z+d],povlook());
>function fractal (x,y,z,h,n) ...
\end{eulerprompt}
\begin{eulerudf}
   if n==1 then writeln(onebox(x,y,z,h));
   else
     h=h/3;
     fractal(x,y,z,h,n-1);
     fractal(x+2*h,y,z,h,n-1);
     fractal(x,y+2*h,z,h,n-1);
     fractal(x,y,z+2*h,h,n-1);
     fractal(x+2*h,y+2*h,z,h,n-1);
     fractal(x+2*h,y,z+2*h,h,n-1);
     fractal(x,y+2*h,z+2*h,h,n-1);
     fractal(x+2*h,y+2*h,z+2*h,h,n-1);
     fractal(x+h,y+h,z+h,h,n-1);
   endif;
  endfunction
\end{eulerudf}
\begin{eulerprompt}
>povstart(fade=10,<shadow);
>fractal(-1,-1,-1,2,4);
>povend();
\end{eulerprompt}
\eulerimg{28}{images/EMT 1-6_Mifta Dilla Aninditya_23030630029-402.png}
\begin{eulercomment}
Perbedaan memungkinkan pemotongan satu objek dari objek lainnya.
Seperti persimpangan, ada bagian dari objek CSG Povray.
\end{eulercomment}
\begin{eulerprompt}
>povstart(light=[5,-5,5],fade=10);
\end{eulerprompt}
\begin{eulercomment}
Untuk demonstrasi ini, kita akan mendefinisikan sebuah objek di
Povray, alih-alih menggunakan sebuah string di Euler. Definisi akan
langsung dituliskan ke file.

Koordinat kotak -1 berarti [-1,-1,-1].
\end{eulercomment}
\begin{eulerprompt}
>povdefine("mycube",povbox(-1,1));
\end{eulerprompt}
\begin{eulercomment}
Kita dapat menggunakan objek ini dalam povobject(), yang mengembalikan
sebuah string seperti biasa.
\end{eulercomment}
\begin{eulerprompt}
>c1=povobject("mycube",povlook(red));
\end{eulerprompt}
\begin{eulercomment}
Kami menghasilkan kubus kedua, dan memutar serta menskalakannya
sedikit.
\end{eulercomment}
\begin{eulerprompt}
>c2=povobject("mycube",povlook(yellow),translate=[1,1,1], ...
>  rotate=xrotate(10°)+yrotate(10°), scale=1.2);
\end{eulerprompt}
\begin{eulercomment}
Kemudian kita ambil selisih dari kedua objek tersebut.
\end{eulercomment}
\begin{eulerprompt}
>writeln(povdifference(c1,c2));
\end{eulerprompt}
\begin{eulercomment}
Sekarang tambahkan tiga sumbu.
\end{eulercomment}
\begin{eulerprompt}
>writeAxis(-1.2,1.2,axis=1); ...
>writeAxis(-1.2,1.2,axis=2); ...
>writeAxis(-1.2,1.2,axis=4); ...
>povend();
\end{eulerprompt}
\eulerimg{28}{images/EMT 1-6_Mifta Dilla Aninditya_23030630029-403.png}
\eulerheading{Fungsi Implisit}
\begin{eulercomment}
Povray dapat memplot himpunan di mana f(x,y,z)=0, seperti parameter
implisit di plot3d. Namun, hasilnya terlihat jauh lebih baik.

Sintaks untuk fungsi-fungsi tersebut sedikit berbeda. Anda tidak dapat
menggunakan output dari ekspresi Maxima atau Euler.

\end{eulercomment}
\begin{eulerformula}
\[
((x^2+y^2-c^2)^2+(z^2-1)^2)*((y^2+z^2-c^2)^2+(x^2-1)^2)*((z^2+x^2-c^2)^2+(y^2-1)^2)=d
\]
\end{eulerformula}
\begin{eulerprompt}
>povstart(angle=70°,height=50°,zoom=4);
>c=0.1; d=0.1; ...
>writeln(povsurface("(pow(pow(x,2)+pow(y,2)-pow(c,2),2)+pow(pow(z,2)-1,2))*(pow(pow(y,2)+pow(z,2)-pow(c,2),2)+pow(pow(x,2)-1,2))*(pow(pow(z,2)+pow(x,2)-pow(c,2),2)+pow(pow(y,2)-1,2))-d",povlook(red))); ...
>povend();
\end{eulerprompt}
\begin{euleroutput}
  Error : Povray error!
  
  Error generated by error() command
  
  povray:
      error("Povray error!");
  Try "trace errors" to inspect local variables after errors.
  povend:
      povray(file,w,h,aspect,exit); 
\end{euleroutput}
\begin{eulerprompt}
>povstart(angle=25°,height=10°); 
>writeln(povsurface("pow(x,2)+pow(y,2)*pow(z,2)-1",povlook(blue),povbox(-2,2,"")));
>povend();
\end{eulerprompt}
\eulerimg{28}{images/EMT 1-6_Mifta Dilla Aninditya_23030630029-404.png}
\begin{eulerprompt}
>povstart(angle=70°,height=50°,zoom=4);
\end{eulerprompt}
\begin{eulercomment}
Membuat permukaan implisit. Perhatikan sintaks yang berbeda dalam
ekspresi.
\end{eulercomment}
\begin{eulerprompt}
>writeln(povsurface("pow(x,2)*y-pow(y,3)-pow(z,2)",povlook(green))); ...
>writeAxes(); ...
>povend();
\end{eulerprompt}
\eulerimg{28}{images/EMT 1-6_Mifta Dilla Aninditya_23030630029-405.png}
\eulerheading{Objek Jaring}
\begin{eulercomment}
Pada contoh ini, kami menunjukkan cara membuat objek mesh, dan
menggambarnya dengan informasi tambahan.

Kami ingin memaksimalkan xy di bawah kondisi x+y = 1 dan
mendemonstrasikan sentuhan tangensial dari garis level.
\end{eulercomment}
\begin{eulerprompt}
>povstart(angle=-10°,center=[0.5,0.5,0.5],zoom=7);
\end{eulerprompt}
\begin{eulercomment}
Kita tidak dapat menyimpan objek dalam sebuah string seperti
sebelumnya, karena ukurannya terlalu besar. Jadi kita mendefinisikan
objek dalam file Povray menggunakan #declare. Fungsi povtriangle()
melakukan hal ini secara otomatis. Fungsi ini dapat menerima vektor
normal seperti halnya pov3d().

Berikut ini mendefinisikan objek mesh, dan menuliskannya langsung ke
dalam file.
\end{eulercomment}
\begin{eulerprompt}
>x=0:0.02:1; y=x'; z=x*y; vx=-y; vy=-x; vz=1;
>mesh=povtriangles(x,y,z,"",vx,vy,vz);
\end{eulerprompt}
\begin{eulercomment}
Sekarang kita tentukan dua cakram, yang akan berpotongan dengan
permukaan.
\end{eulercomment}
\begin{eulerprompt}
>cl=povdisc([0.5,0.5,0],[1,1,0],2); ...
>ll=povdisc([0,0,1/4],[0,0,1],2);
\end{eulerprompt}
\begin{eulercomment}
Tuliskan permukaan dikurangi kedua cakram.
\end{eulercomment}
\begin{eulerprompt}
>writeln(povdifference(mesh,povunion([cl,ll]),povlook(green)));
\end{eulerprompt}
\begin{eulercomment}
Tuliskan kedua perpotongan tersebut.
\end{eulercomment}
\begin{eulerprompt}
>writeln(povintersection([mesh,cl],povlook(red))); ...
>writeln(povintersection([mesh,ll],povlook(gray)));
\end{eulerprompt}
\begin{eulercomment}
Tulislah satu titik secara maksimal.
\end{eulercomment}
\begin{eulerprompt}
>writeln(povpoint([1/2,1/2,1/4],povlook(gray),size=2*defaultpointsize));
\end{eulerprompt}
\begin{eulercomment}
Tambahkan sumbu dan selesaikan.
\end{eulercomment}
\begin{eulerprompt}
>writeAxes(0,1,0,1,0,1,d=0.015); ...
>povend();
\end{eulerprompt}
\eulerimg{28}{images/EMT 1-6_Mifta Dilla Aninditya_23030630029-406.png}
\eulerheading{Anaglyph dalam Povray}
\begin{eulercomment}
Untuk menghasilkan anaglyph untuk kacamata merah/cyan, Povray harus
dijalankan dua kali dari posisi kamera yang berbeda. Ini menghasilkan
dua file Povray dan dua file PNG, yang dimuat dengan fungsi
loadanaglyph().

Tentu saja, Anda membutuhkan kacamata merah/cyan untuk melihat contoh
berikut dengan benar.

Fungsi pov3d() memiliki tombol sederhana untuk menghasilkan anaglyph.
\end{eulercomment}
\begin{eulerprompt}
>pov3d("-exp(-x^2-y^2)/2",r=2,height=45°,>anaglyph, ...
>  center=[0,0,0.5],zoom=3.5);
\end{eulerprompt}
\begin{eulercomment}
Jika Anda membuat scene dengan objek, Anda harus menempatkan pembuatan
scene ke dalam suatu fungsi, dan menjalankannya dua kali dengan nilai
yang berbeda untuk parameter anaglyph.
\end{eulercomment}
\begin{eulerprompt}
>function myscene ...
\end{eulerprompt}
\begin{eulerudf}
    s=povsphere(povc,1);
    cl=povcylinder(-povz,povz,0.5);
    clx=povobject(cl,rotate=xrotate(90°));
    cly=povobject(cl,rotate=yrotate(90°));
    c=povbox([-1,-1,0],1);
    un=povunion([cl,clx,cly,c]);
    obj=povdifference(s,un,povlook(red));
    writeln(obj);
    writeAxes();
  endfunction
\end{eulerudf}
\begin{eulercomment}
Fungsi povanaglyph() melakukan semua ini. Parameter-parameternya
seperti pada povstart() dan povend() yang digabungkan.
\end{eulercomment}
\begin{eulerprompt}
>povanaglyph("myscene",zoom=4.5);
\end{eulerprompt}
\eulerheading{Mendefinisikan Objek sendiri}
\begin{eulercomment}
Antarmuka povray Euler berisi banyak objek. Namun Anda tidak dibatasi
pada objek-objek tersebut. Anda dapat membuat objek sendiri, yang
menggabungkan objek-objek lain, atau objek yang benar-benar baru.

Kami mendemonstrasikan sebuah torus. Perintah Povray untuk ini adalah
“torus”. Jadi kita mengembalikan sebuah string dengan perintah ini dan
parameternya. Perhatikan bahwa torus selalu berpusat pada titik asal.
\end{eulercomment}
\begin{eulerprompt}
>function povdonat (r1,r2,look="") ...
\end{eulerprompt}
\begin{eulerudf}
    return "torus \{"+r1+","+r2+look+"\}";
  endfunction
\end{eulerudf}
\begin{eulercomment}
Inilah torus pertama kami.
\end{eulercomment}
\begin{eulerprompt}
>t1=povdonat(0.8,0.2)
\end{eulerprompt}
\begin{euleroutput}
  torus \{0.8,0.2\}
\end{euleroutput}
\begin{eulercomment}
Mari kita gunakan objek ini untuk membuat torus kedua, ditranslasikan
dan diputar.
\end{eulercomment}
\begin{eulerprompt}
>t2=povobject(t1,rotate=xrotate(90°),translate=[0.8,0,0])
\end{eulerprompt}
\begin{euleroutput}
  object \{ torus \{0.8,0.2\}
   rotate 90 *x 
   translate <0.8,0,0>
   \}
\end{euleroutput}
\begin{eulercomment}
Sekarang, kita tempatkan semua benda ini ke dalam suatu pemandangan.
Untuk tampilannya, kami menggunakan Phong Shading.
\end{eulercomment}
\begin{eulerprompt}
>povstart(center=[0.4,0,0],angle=0°,zoom=3.8,aspect=1.5); ...
>writeln(povobject(t1,povlook(green,phong=1))); ...
>writeln(povobject(t2,povlook(green,phong=1))); ...
\end{eulerprompt}
\begin{eulerttcomment}
 >povend();
\end{eulerttcomment}
\begin{eulercomment}
memanggil program Povray. Namun, jika terjadi kesalahan, program ini
tidak menampilkan kesalahan. Oleh karena itu, Anda harus menggunakan

\end{eulercomment}
\begin{eulerttcomment}
 >povend(<keluar);
\end{eulerttcomment}
\begin{eulercomment}

jika ada yang tidak berhasil. Ini akan membiarkan jendela Povray tetap
terbuka.
\end{eulercomment}
\begin{eulerprompt}
>povend(h=320,w=480);
\end{eulerprompt}
\begin{eulercomment}
Berikut adalah contoh yang lebih rumit. Kami menyelesaikan

\end{eulercomment}
\begin{eulerformula}
\[
Ax \le b, \quad x \ge 0, \quad c.x \to \text{Max.}
\]
\end{eulerformula}
\begin{eulercomment}
dan menunjukkan titik-titik yang layak dan optimal dalam plot 3D.
\end{eulercomment}
\begin{eulerprompt}
>A=[10,8,4;5,6,8;6,3,2;9,5,6];
>b=[10,10,10,10]';
>c=[1,1,1];
\end{eulerprompt}
\begin{eulercomment}
Pertama, mari kita periksa, apakah contoh ini memiliki solusi atau
tidak.
\end{eulercomment}
\begin{eulerprompt}
>x=simplex(A,b,c,>max,>check)'
\end{eulerprompt}
\begin{euleroutput}
  [0,  1,  0.5]
\end{euleroutput}
\begin{eulercomment}
Ya, itu benar.

Selanjutnya kita mendefinisikan dua objek. Yang pertama adalah bidang
datar.

\end{eulercomment}
\begin{eulerformula}
\[
a \cdot x \le b
\]
\end{eulerformula}
\begin{eulerprompt}
>function oneplane (a,b,look="") ...
\end{eulerprompt}
\begin{eulerudf}
    return povplane(a,b,look)
  endfunction
\end{eulerudf}
\begin{eulercomment}
Kemudian kita mendefinisikan irisan semua ruang setengah dan sebuah
kubus.
\end{eulercomment}
\begin{eulerprompt}
>function adm (A, b, r, look="") ...
\end{eulerprompt}
\begin{eulerudf}
    ol=[];
    loop 1 to rows(A); ol=ol|oneplane(A[#],b[#]); end;
    ol=ol|povbox([0,0,0],[r,r,r]);
    return povintersection(ol,look);
  endfunction
\end{eulerudf}
\begin{eulercomment}
Sekarang, kita dapat merencanakan adegannya.
\end{eulercomment}
\begin{eulerprompt}
>povstart(angle=120°,center=[0.5,0.5,0.5],zoom=3.5); ...
>writeln(adm(A,b,2,povlook(green,0.4))); ...
>writeAxes(0,1.3,0,1.6,0,1.5); ...
\end{eulerprompt}
\begin{eulercomment}
Berikut ini adalah lingkaran di sekitar titik optimum.
\end{eulercomment}
\begin{eulerprompt}
>writeln(povintersection([povsphere(x,0.5),povplane(c,c.x')], ...
>  povlook(red,0.9)));
\end{eulerprompt}
\begin{eulercomment}
Dan kesalahan dalam arah yang optimum.
\end{eulercomment}
\begin{eulerprompt}
>writeln(povarrow(x,c*0.5,povlook(red)));
\end{eulerprompt}
\begin{eulercomment}
Kita menambahkan teks ke layar. Teks hanyalah objek 3D. Kita perlu
menempatkan dan memutarnya sesuai dengan pandangan kita.
\end{eulercomment}
\begin{eulerprompt}
>writeln(povtext("Linear Problem",[0,0.2,1.3],size=0.05,rotate=5°)); ...
>povend();
\end{eulerprompt}
\eulerheading{More Examples}
\begin{eulercomment}
Anda dapat menemukan beberapa contoh lebih lanjut untuk Povray di
Euler dalam berkas berikut.

See: Examples/Dandelin Spheres\\
See: Examples/Donat Math\\
See: Examples/Trefoil Knot\\
See: Examples/Optimization by Affine Scaling


\begin{eulercomment}
\eulerheading{Soal}
\begin{eulerprompt}
>plot3d("x^2+2y+12",a=3,b=5,rotate=3,grid=2):
\end{eulerprompt}
\begin{euleroutput}
  Matrix 61x61 and 51x61 not compatible in size!
  Try "trace errors" to inspect local variables after errors.
  plot3d:
      n=size(xx,yy,zz)-1;
\end{euleroutput}
\begin{eulerprompt}
>function f(x,y) := x+9*y //
>plot3d("f"):
>function h(x,y) &= 7*x+7*y-17;
>plot3d("h"):
>function g(x,y) &= sqrt(5*x^2+y^3);
>plot3d("g"):
>t=linspace(0,2pi,100); s=linspace(-pi/2,pi/2,100)'; ...
>d=1+0.2*(cos(3*t)+cos(8*s)); ...
>plot3d(cos(t)*cos(s)*d,sin(t)*cos(s)*d,sin(s)*d,color=7,hue=2, ...
>  light=[1,3,1],frame=0,zoom=3):
>plot3d("8*x^2/(x^5+y^3+7)",color=-3,hue=true,grid=6):
>u=linspace(-1,1,6); v=linspace(0,2*pi,15)'; ...
>X=(13+u*cos(v/2))*cos(v); Y=(13+u*cos(v/2))*sin(v); Z=u*sin(v/2); ...
>plot3d(X,Y,Z,>anaglyph,<frame,>wire,scale=1.9):
\end{eulerprompt}
\begin{euleroutput}
  
\end{euleroutput}
\eulerheading{Rujukan Lengkap Fungsi plot2d()}
\begin{eulercomment}
\end{eulercomment}
\begin{eulerttcomment}
  function plot2d (xv, yv, btest, a, b, c, d, xmin, xmax, r, n,  ..
  logplot, grid, frame, framecolor, square, color, thickness, style, ..
  auto, add, user, delta, points, addpoints, pointstyle, bar, histogram,  ..
  distribution, even, steps, own, adaptive, hue, level, contour,  ..
  nc, filled, fillcolor, outline, title, xl, yl, maps, contourcolor, ..
  contourwidth, ticks, margin, clipping, cx, cy, insimg, spectral,  ..
  cgrid, vertical, smaller, dl, niveau, levels)
\end{eulerttcomment}
\begin{eulercomment}
Multipurpose plot function for plots in the plane (2D plots). This function can do
plots of functions of one variables, data plots, curves in the plane, bar plots, grids
of complex numbers, and implicit plots of functions of two variables.

Parameters
\\
x,y       : equations, functions or data vectors\\
a,b,c,d   : Plot area (default a=-2,b=2)\\
r         : if r is set, then a=cx-r, b=cx+r, c=cy-r, d=cy+r\\
\end{eulercomment}
\begin{eulerttcomment}
            r can be a vector [rx,ry] or a vector [rx1,rx2,ry1,ry2].
\end{eulerttcomment}
\begin{eulercomment}
xmin,xmax : range of the parameter for curves\\
auto      : Determine y-range automatically (default)\\
square    : if true, try to keep square x-y-ranges\\
n         : number of intervals (default is adaptive)\\
grid      : 0 = no grid and labels,\\
\end{eulercomment}
\begin{eulerttcomment}
            1 = axis only,
            2 = normal grid (see below for the number of grid lines)
            3 = inside axis
            4 = no grid
            5 = full grid including margin
            6 = ticks at the frame
            7 = axis only
            8 = axis only, sub-ticks
\end{eulerttcomment}
\begin{eulercomment}
frame     : 0 = no frame\\
framecolor: color of the frame and the grid\\
margin    : number between 0 and 0.4 for the margin around the plot\\
color     : Color of curves. If this is a vector of colors,\\
\end{eulercomment}
\begin{eulerttcomment}
            it will be used for each row of a matrix of plots. In the case of
            point plots, it should be a column vector. If a row vector or a
            full matrix of colors is used for point plots, it will be used for
            each data point.
\end{eulerttcomment}
\begin{eulercomment}
thickness : line thickness for curves\\
\end{eulercomment}
\begin{eulerttcomment}
            This value can be smaller than 1 for very thin lines.
\end{eulerttcomment}
\begin{eulercomment}
style     : Plot style for lines, markers, and fills.\\
\end{eulercomment}
\begin{eulerttcomment}
            For points use
            "[]", "<>", ".", "..", "...",
            "*", "+", "|", "-", "o"
            "[]#", "<>#", "o#" (filled shapes)
            "[]w", "<>w", "ow" (non-transparent)
            For lines use
            "-", "--", "-.", ".", ".-.", "-.-", "->"
            For filled polygons or bar plots use
            "#", "#O", "O", "/", "\(\backslash\)", "\(\backslash\)/",
            "+", "|", "-", "t"
\end{eulerttcomment}
\begin{eulercomment}
points    : plot single points instead of line segments\\
addpoints : if true, plots line segments and points\\
add       : add the plot to the existing plot\\
user      : enable user interaction for functions\\
delta     : step size for user interaction\\
bar       : bar plot (x are the interval bounds, y the interval values)\\
histogram : plots the frequencies of x in n subintervals\\
distribution=n : plots the distribution of x with n subintervals\\
even      : use inter values for automatic histograms.\\
steps     : plots the function as a step function (steps=1,2)\\
adaptive  : use adaptive plots (n is the minimal number of steps)\\
level     : plot level lines of an implicit function of two variables\\
outline   : draws boundary of level ranges.
\\
If the level value is a 2xn matrix, ranges of levels will be drawn\\
in the color using the given fill style. If outline is true, it\\
will be drawn in the contour color. Using this feature, regions of\\
f(x,y) between limits can be marked.
\\
hue       : add hue color to the level plot to indicate the function\\
\end{eulercomment}
\begin{eulerttcomment}
            value
\end{eulerttcomment}
\begin{eulercomment}
contour   : Use level plot with automatic levels\\
nc        : number of automatic level lines\\
title     : plot title (default "")\\
xl, yl    : labels for the x- and y-axis\\
smaller   : if \textgreater{}0, there will be more space to the left for labels.\\
vertical  :\\
\end{eulercomment}
\begin{eulerttcomment}
  Turns vertical labels on or off. This changes the global variable
  verticallabels locally for one plot. The value 1 sets only vertical
  text, the value 2 uses vertical numerical labels on the y axis.
\end{eulerttcomment}
\begin{eulercomment}
filled    : fill the plot of a curve\\
fillcolor : fill color for bar and filled curves\\
outline   : boundary for filled polygons\\
logplot   : set logarithmic plots\\
\end{eulercomment}
\begin{eulerttcomment}
            1 = logplot in y,
            2 = logplot in xy,
            3 = logplot in x
\end{eulerttcomment}
\begin{eulercomment}
own       :\\
\end{eulercomment}
\begin{eulerttcomment}
  A string, which points to an own plot routine. With >user, you get
  the same user interaction as in plot2d. The range will be set
  before each call to your function.
\end{eulerttcomment}
\begin{eulercomment}
maps      : map expressions (0 is faster), functions are always mapped.\\
contourcolor : color of contour lines\\
contourwidth : width of contour lines\\
clipping  : toggles the clipping (default is true)\\
title     :\\
\end{eulercomment}
\begin{eulerttcomment}
  This can be used to describe the plot. The title will appear above
  the plot. Moreover, a label for the x and y axis can be added with
  xl="string" or yl="string". Other labels can be added with the
  functions label() or labelbox(). The title can be a unicode
  string or an image of a Latex formula.
\end{eulerttcomment}
\begin{eulercomment}
cgrid     :\\
\end{eulercomment}
\begin{eulerttcomment}
  Determines the number of grid lines for plots of complex grids.
  Should be a divisor of the the matrix size minus 1 (number of
  subintervals). cgrid can be a vector [cx,cy].
\end{eulerttcomment}
\begin{eulercomment}

Overview

The function can plot

- expressions, call collections or functions of one variable,\\
- parametric curves,\\
- x data against y data,\\
- implicit functions,\\
- bar plots,\\
- complex grids,\\
- polygons.

If a function or expression for xv is given, plot2d() will compute\\
values in the given range using the function or expression. The\\
expression must be an expression in the variable x. The range must\\
be defined in the parameters a and b unless the default range\\
[-2,2] should be used. The y-range will be computed automatically,\\
unless c and d are specified, or a radius r, which yields the range\\
[-r,r] for x and y. For plots of functions, plot2d will use an\\
adaptive evaluation of the function by default. To speed up the\\
plot for complicated functions, switch this off with \textless{}adaptive, and\\
optionally decrease the number of intervals n. Moreover, plot2d()\\
will by default use mapping. I.e., it will compute the plot element\\
for element. If your expression or your functions can handle a\\
vector x, you can switch that off with \textless{}maps for faster evaluation.

Note that adaptive plots are always computed element for element. \\
If functions or expressions for both xv and for yv are specified,\\
plot2d() will compute a curve with the xv values as x-coordinates\\
and the yv values as y-coordinates. In this case, a range should be\\
defined for the parameter using xmin, xmax. Expressions contained
\end{eulercomment}
\eulersubheading{}
\eulerheading{Kalkulus dengan EMT}
\begin{eulercomment}
Materi Kalkulus mencakup di antaranya:

- Fungsi (fungsi aljabar, trigonometri, eksponensial, logaritma, komposisi fungsi)\\
- Limit Fungsi,\\
- Turunan Fungsi,\\
- Integral Tak Tentu,\\
- Integral Tentu dan Aplikasinya,\\
- Barisan dan Deret (kekonvergenan barisan dan deret).

EMT (bersama Maxima) dapat digunakan untuk melakukan semua perhitungan di dalam kalkulus, baik secara numerik maupun analitik
(eksak).

\end{eulercomment}
\eulersubheading{Mendefinisikan Fungsi}
\begin{eulercomment}
Terdapat beberapa cara mendefinisikan fungsi pada EMT, yakni:

- Menggunakan format nama\_fungsi := rumus fungsi (untuk fungsi numerik),\\
- Menggunakan format nama\_fungsi \&= rumus fungsi (untuk fungsi simbolik, namun dapat dihitung secara numerik),\\
- Menggunakan format nama\_fungsi \&\&= rumus fungsi (untuk fungsi simbolik murni, tidak dapat dihitung langsung),\\
- Fungsi sebagai program EMT.

Setiap format harus diawali dengan perintah function (bukan sebagai ekspresi).

Berikut adalah adalah beberapa contoh cara mendefinisikan fungsi:

\end{eulercomment}
\begin{eulerformula}
\[
f(x)=2x^2+e^{\sin(x)}.
\]
\end{eulerformula}
\begin{eulerprompt}
>function f(x) := 2*x^2+exp(sin(x)) // fungsi numerik
>f(0), f(1), f(pi)
\end{eulerprompt}
\begin{euleroutput}
  1
  4.31977682472
  20.7392088022
\end{euleroutput}
\begin{eulerprompt}
>f(a) // tidak dapat dihitung nilainya
\end{eulerprompt}
\begin{euleroutput}
  Real 41 x 1 matrix
  
              1 
        1.21868 
        1.55948 
        2.01872 
        2.58957 
        3.26182 
        4.02223 
        4.85563 
        5.74672 
        6.68221 
        7.65308 
        8.65613 
        9.69456 
        10.7774 
        11.9179 
        13.1314 
        14.4331 
        15.8362 
        17.3508 
         18.984 
      ...
\end{euleroutput}
\begin{eulercomment}
Silakan Anda plot kurva fungsi di atas!

Berikutnya kita definisikan fungsi:

\end{eulercomment}
\begin{eulerformula}
\[
g(x)=\frac{\sqrt{x^2-3x}}{x+1}.
\]
\end{eulerformula}
\begin{eulerprompt}
>plot2d("f(x)"):
\end{eulerprompt}
\eulerimg{27}{images/EMT 1-6_Mifta Dilla Aninditya_23030630029-407.png}
\begin{eulerprompt}
>function g(x) := sqrt(x^2-3*x)/(x+1)
>g(3)
\end{eulerprompt}
\begin{euleroutput}
  0
\end{euleroutput}
\begin{eulerprompt}
>g(0)
\end{eulerprompt}
\begin{euleroutput}
  0
\end{euleroutput}
\begin{eulerprompt}
>g(1) // kompleks, tidak dapat dihitung oleh fungsi numerik
\end{eulerprompt}
\begin{euleroutput}
  Floating point error!
  Error in sqrt
  Try "trace errors" to inspect local variables after errors.
  g:
      useglobal; return sqrt(x^2-3*x)/(x+1) 
  Error in:
  g(1) // kompleks, tidak dapat dihitung oleh fungsi numerik ...
      ^
\end{euleroutput}
\begin{eulercomment}
Silakan Anda plot kurva fungsi di atas!
\end{eulercomment}
\begin{eulerprompt}
>plot2d("g(x)"):
\end{eulerprompt}
\eulerimg{27}{images/EMT 1-6_Mifta Dilla Aninditya_23030630029-408.png}
\begin{eulerprompt}
>f(g(5)) // komposisi fungsi
\end{eulerprompt}
\begin{euleroutput}
  2.20920171961
\end{euleroutput}
\begin{eulerprompt}
>g(f(5))
\end{eulerprompt}
\begin{euleroutput}
  0.950898070639
\end{euleroutput}
\begin{eulerprompt}
>function h(x) := f(g(x)) // definisi komposisi fungsi 
>h(5) // sama dengan f(g(5))
\end{eulerprompt}
\begin{euleroutput}
  2.20920171961
\end{euleroutput}
\begin{eulercomment}
Silakan Anda plot kurva fungsi komposisi fungsi f dan g:

\end{eulercomment}
\begin{eulerformula}
\[
h(x)=f(g(x))
\]
\end{eulerformula}
\begin{eulercomment}
dan

\end{eulercomment}
\begin{eulerformula}
\[
u(x)=g(f(x))
\]
\end{eulerformula}
\begin{eulercomment}
bersama-sama kurva fungsi f dan g dalam satu bidang koordinat.
\end{eulercomment}
\begin{eulerprompt}
>f(0:10) // nilai-nilai f(0), f(1), f(2), ..., f(10)
\end{eulerprompt}
\begin{euleroutput}
  [1,  4.31978,  10.4826,  19.1516,  32.4692,  50.3833,  72.7562,
  99.929,  130.69,  163.51,  200.58]
\end{euleroutput}
\begin{eulerprompt}
>fmap(0:10) // sama dengan f(0:10), berlaku untuk semua fungsi
\end{eulerprompt}
\begin{euleroutput}
  [1,  4.31978,  10.4826,  19.1516,  32.4692,  50.3833,  72.7562,
  99.929,  130.69,  163.51,  200.58]
\end{euleroutput}
\begin{eulerprompt}
>gmap(200:210)
\end{eulerprompt}
\begin{euleroutput}
  [0.987534,  0.987596,  0.987657,  0.987718,  0.987778,  0.987837,
  0.987896,  0.987954,  0.988012,  0.988069,  0.988126]
\end{euleroutput}
\begin{eulercomment}
Misalkan kita akan mendefinisikan fungsi

\end{eulercomment}
\begin{eulerformula}
\[
f(x) = \begin{cases} x^3 & x>0 \\ x^2 & x\le 0. \end{cases}
\]
\end{eulerformula}
\begin{eulercomment}
Fungsi tersebut tidak dapat didefinisikan sebagai fungsi numerik
secara "inline" menggunakan format :=, melainkan didefinisikan sebagai
program. Perhatikan, kata "map" digunakan agar fungsi dapat menerima
vektor sebagai input, dan hasilnya berupa vektor. Jika tanpa kata
"map" fungsinya hanya dapat menerima input satu nilai.
\end{eulercomment}
\begin{eulerprompt}
>function map f(x) ...
\end{eulerprompt}
\begin{eulerudf}
    if x>0 then return x^3
    else return x^2
    endif;
  endfunction
\end{eulerudf}
\begin{eulerprompt}
>f(1)
\end{eulerprompt}
\begin{euleroutput}
  1
\end{euleroutput}
\begin{eulerprompt}
>f(-2)
\end{eulerprompt}
\begin{euleroutput}
  4
\end{euleroutput}
\begin{eulerprompt}
>f(-5:5)
\end{eulerprompt}
\begin{euleroutput}
  [25,  16,  9,  4,  1,  0,  1,  8,  27,  64,  125]
\end{euleroutput}
\begin{eulerprompt}
>aspect(1.5); plot2d("f(x)",-5,5):
\end{eulerprompt}
\eulerimg{17}{images/EMT 1-6_Mifta Dilla Aninditya_23030630029-409.png}
\begin{eulerprompt}
>function f(x) &= 2*E^x // fungsi simbolik
\end{eulerprompt}
\begin{euleroutput}
  
                                      x
                                   2 E
  
\end{euleroutput}
\begin{eulerprompt}
>$f(a) // nilai fungsi secara simbolik
\end{eulerprompt}
\begin{eulerformula}
\[
2\,e^{a}
\]
\end{eulerformula}
\begin{eulerprompt}
>f(E) // nilai fungsi berupa bilangan desimal
\end{eulerprompt}
\begin{euleroutput}
  30.308524483
\end{euleroutput}
\begin{eulerprompt}
>$f(E), $float(%)
\end{eulerprompt}
\begin{eulerformula}
\[
2\,e^{e}
\]
\end{eulerformula}
\begin{eulerformula}
\[
30.30852448295852
\]
\end{eulerformula}
\begin{eulerprompt}
>function g(x) &= 3*x+1
\end{eulerprompt}
\begin{euleroutput}
  
                                 3 x + 1
  
\end{euleroutput}
\begin{eulerprompt}
>function h(x) &= f(g(x)) // komposisi fungsi
\end{eulerprompt}
\begin{euleroutput}
  
                                   3 x + 1
                                2 E
  
\end{euleroutput}
\begin{eulerprompt}
>plot2d("h(x)",-1,1):
\end{eulerprompt}
\eulerimg{17}{images/EMT 1-6_Mifta Dilla Aninditya_23030630029-413.png}
\eulerheading{Latihan}
\begin{eulercomment}
Bukalah buku Kalkulus. Cari dan pilih beberapa (paling sedikit 5 fungsi berbeda
tipe/bentuk/jenis) fungsi dari buku tersebut, kemudian definisikan fungsi-fungsi tersebut dan
komposisinya di EMT pada baris-baris perintah berikut (jika perlu tambahkan lagi). Untuk setiap
fungsi, hitung beberapa nilainya, baik untuk satu nilai maupun vektor. Gambar grafik
fungsi-fungsi tersebut dan komposisi-komposisi 2 fungsi.

Juga, carilah fungsi beberapa (dua) variabel. Lakukan hal sama seperti di atas.
\end{eulercomment}
\begin{eulerprompt}
>function f(x) := 7*x^2+exp(sin(x))
>f(0), f(2), f(pi)
\end{eulerprompt}
\begin{euleroutput}
  1
  30.482577728
  70.0872308076
\end{euleroutput}
\begin{eulerprompt}
>plot2d("f(x)"):
\end{eulerprompt}
\eulerimg{17}{images/EMT 1-6_Mifta Dilla Aninditya_23030630029-414.png}
\begin{eulerprompt}
>function g(x) := sqrt(x^3-x)/(x-2)
>g(1)
\end{eulerprompt}
\begin{euleroutput}
  0
\end{euleroutput}
\begin{eulerprompt}
>plot2d("g(x)"):
\end{eulerprompt}
\eulerimg{17}{images/EMT 1-6_Mifta Dilla Aninditya_23030630029-415.png}
\begin{eulerprompt}
>function h(x) := f(g(x))
>h(0)
\end{eulerprompt}
\begin{euleroutput}
  1
\end{euleroutput}
\begin{eulerprompt}
>plot2d("h(x)"):
\end{eulerprompt}
\eulerimg{17}{images/EMT 1-6_Mifta Dilla Aninditya_23030630029-416.png}
\begin{eulerprompt}
>function f(x) &= 5*E^x
\end{eulerprompt}
\begin{euleroutput}
  
                                      x
                                   5 E
  
\end{euleroutput}
\begin{eulerprompt}
>$f(a)
\end{eulerprompt}
\begin{eulerformula}
\[
5\,e^{a}
\]
\end{eulerformula}
\begin{eulerprompt}
>f(E)
\end{eulerprompt}
\begin{euleroutput}
  75.7713112074
\end{euleroutput}
\begin{eulerprompt}
>function g(x) &= 2*x+3
\end{eulerprompt}
\begin{euleroutput}
  
                                 2 x + 3
  
\end{euleroutput}
\begin{eulerprompt}
>function h(x) &= f(g(x)) 
\end{eulerprompt}
\begin{euleroutput}
  
                                   2 x + 3
                                5 E
  
\end{euleroutput}
\begin{eulerprompt}
>plot2d("h(x)",-1,0):
\end{eulerprompt}
\eulerimg{17}{images/EMT 1-6_Mifta Dilla Aninditya_23030630029-418.png}
\begin{eulerprompt}
>function f(x) := x^2+exp(sin(x^3))
>f(0), f(2), f(pi)
\end{eulerprompt}
\begin{euleroutput}
  1
  6.68950791761
  10.5410728979
\end{euleroutput}
\begin{eulerprompt}
>plot2d("f(x)"):
\end{eulerprompt}
\eulerimg{17}{images/EMT 1-6_Mifta Dilla Aninditya_23030630029-419.png}
\begin{eulerudf}
  
  
  
  
\end{eulerudf}
\begin{eulercomment}
\begin{eulercomment}
\eulerheading{Menghitung Limit}
\begin{eulercomment}
Perhitungan limit pada EMT dapat dilakukan dengan menggunakan fungsi Maxima, yakni "limit".
Fungsi "limit" dapat digunakan untuk menghitung limit fungsi dalam bentuk ekspresi maupun fungsi
yang sudah didefinisikan sebelumnya. Nilai limit dapat dihitung pada sebarang nilai atau pada tak
hingga (-inf, minf, dan inf). Limit kiri dan limit kanan juga dapat dihitung, dengan cara memberi
opsi "plus" atau "minus". Hasil limit dapat berupa nilai, "und" (tak definisi), "ind" (tak tentu
namun terbatas), "infinity" (kompleks tak hingga).

Perhatikan beberapa contoh berikut. Perhatikan cara menampilkan perhitungan secara lengkap, tidak
hanya menampilkan hasilnya saja.
\end{eulercomment}
\begin{eulerprompt}
>$showev('limit(sqrt(x^2-3*x)/(x+1),x,inf))
\end{eulerprompt}
\begin{eulerformula}
\[
\lim_{x\rightarrow \infty }{\frac{\sqrt{x^2-3\,x}}{x+1}}=1
\]
\end{eulerformula}
\begin{eulerprompt}
>$limit((x^3-13*x^2+51*x-63)/(x^3-4*x^2-3*x+18),x,3)
\end{eulerprompt}
\begin{eulerformula}
\[
-\frac{4}{5}
\]
\end{eulerformula}
\begin{eulercomment}
maxima: 'limit((x\textasciicircum{}3-13*x\textasciicircum{}2+51*x-63)/(x\textasciicircum{}3-4*x\textasciicircum{}2-3*x+18),x,3)=limit((x\textasciicircum{}3-13*x\textasciicircum{}2+51*x-63)/(x\textasciicircum{}3-4*x\textasciicircum{}2-3*x+18),x,3)

Fungsi tersebut diskontinu di titik x=3. Berikut adalah grafik fungsinya.
\end{eulercomment}
\begin{eulerprompt}
>aspect(1.5); plot2d("(x^3-13*x^2+51*x-63)/(x^3-4*x^2-3*x+18)",0,4); plot2d(3,-4/5,>points,style="ow",>add):
\end{eulerprompt}
\eulerimg{17}{images/EMT 1-6_Mifta Dilla Aninditya_23030630029-422.png}
\begin{eulerprompt}
>$limit(2*x*sin(x)/(1-cos(x)),x,0)
\end{eulerprompt}
\begin{eulerformula}
\[
4
\]
\end{eulerformula}
\begin{eulercomment}
maxima: 'limit(2*x*sin(x)/(1-cos(x)),x,0)=limit(2*x*sin(x)/(1-cos(x)),x,0)

Fungsi tersebut diskontinu di titik x=0. Berikut adalah grafik fungsinya.
\end{eulercomment}
\begin{eulerprompt}
>plot2d("2*x*sin(x)/(1-cos(x))",-pi,pi); plot2d(0,4,>points,style="ow",>add):
\end{eulerprompt}
\eulerimg{17}{images/EMT 1-6_Mifta Dilla Aninditya_23030630029-424.png}
\begin{eulerprompt}
>$limit(cot(7*h)/cot(5*h),h,0)
\end{eulerprompt}
\begin{eulerformula}
\[
\frac{5}{7}
\]
\end{eulerformula}
\begin{eulercomment}
maxima: showev('limit(cot(7*h)/cot(5*h),h,0))

Fungsi tersebut juga diskontinu (karena tidak terdefinisi) di x=0. Berikut adalah grafiknya.
\end{eulercomment}
\begin{eulerprompt}
>plot2d("cot(7*x)/cot(5*x)",-0.001,0.001); plot2d(0,5/7,>points,style="ow",>add):
\end{eulerprompt}
\eulerimg{17}{images/EMT 1-6_Mifta Dilla Aninditya_23030630029-426.png}
\begin{eulerprompt}
>$showev('limit(((x/8)^(1/3)-1)/(x-8),x,8))
\end{eulerprompt}
\begin{eulerformula}
\[
\lim_{x\rightarrow 8}{\frac{\frac{x^{\frac{1}{3}}}{2}-1}{x-8}}=
 \frac{1}{24}
\]
\end{eulerformula}
\begin{eulercomment}
Tunjukkan limit tersebut dengan grafik, seperti contoh-contoh sebelumnya.
\end{eulercomment}
\begin{eulerprompt}
>$showev('limit(1/(2*x-1),x,0))
\end{eulerprompt}
\begin{eulerformula}
\[
\lim_{x\rightarrow 0}{\frac{1}{2\,x-1}}=-1
\]
\end{eulerformula}
\begin{eulercomment}
Tunjukkan limit tersebut dengan grafik, seperti contoh-contoh sebelumnya.
\end{eulercomment}
\begin{eulerprompt}
>$showev('limit((x^2-3*x-10)/(x-5),x,5))
\end{eulerprompt}
\begin{eulerformula}
\[
\lim_{x\rightarrow 5}{\frac{x^2-3\,x-10}{x-5}}=7
\]
\end{eulerformula}
\begin{eulercomment}
Tunjukkan limit tersebut dengan grafik, seperti contoh-contoh sebelumnya.
\end{eulercomment}
\begin{eulerprompt}
>$showev('limit(sqrt(x^2+x)-x,x,inf))
\end{eulerprompt}
\begin{eulerformula}
\[
\lim_{x\rightarrow \infty }{\sqrt{x^2+x}-x}=\frac{1}{2}
\]
\end{eulerformula}
\begin{eulercomment}
Tunjukkan limit tersebut dengan grafik, seperti contoh-contoh sebelumnya.
\end{eulercomment}
\begin{eulerprompt}
>$showev('limit(abs(x-1)/(x-1),x,1,minus))
\end{eulerprompt}
\begin{eulerformula}
\[
\lim_{x\uparrow 1}{\frac{\left| x-1\right| }{x-1}}=-1
\]
\end{eulerformula}
\begin{eulercomment}
Hitung limit di atas untuk x menuju 1 dari kanan.\\
Tunjukkan limit tersebut dengan grafik, seperti contoh-contoh sebelumnya.
\end{eulercomment}
\begin{eulerprompt}
>$showev('limit(sin(x)/x,x,0))
\end{eulerprompt}
\begin{eulerformula}
\[
\lim_{x\rightarrow 0}{\frac{\sin x}{x}}=1
\]
\end{eulerformula}
\begin{eulerprompt}
>plot2d("sin(x)/x",-pi,pi); plot2d(0,1,>points,style="ow",>add):
\end{eulerprompt}
\eulerimg{17}{images/EMT 1-6_Mifta Dilla Aninditya_23030630029-433.png}
\begin{eulerprompt}
>$showev('limit(sin(x^3)/x,x,0))
\end{eulerprompt}
\begin{eulerformula}
\[
\lim_{x\rightarrow 0}{\frac{\sin x^3}{x}}=0
\]
\end{eulerformula}
\begin{eulercomment}
Tunjukkan limit tersebut dengan grafik, seperti contoh-contoh sebelumnya.
\end{eulercomment}
\begin{eulerprompt}
>$showev('limit(log(x), x, minf))
\end{eulerprompt}
\begin{eulerformula}
\[
\lim_{x\rightarrow  -\infty }{\log x}={\it infinity}
\]
\end{eulerformula}
\begin{eulerprompt}
>$showev('limit((-2)^x,x, inf))
\end{eulerprompt}
\begin{eulerformula}
\[
\lim_{x\rightarrow \infty }{\left(-2\right)^{x}}={\it infinity}
\]
\end{eulerformula}
\begin{eulerprompt}
>$showev('limit(t-sqrt(2-t),t,2,minus))
\end{eulerprompt}
\begin{eulerformula}
\[
\lim_{t\uparrow 2}{t-\sqrt{2-t}}=2
\]
\end{eulerformula}
\begin{eulerprompt}
>$showev('limit(t-sqrt(2-t),t,2,plus))
\end{eulerprompt}
\begin{eulerformula}
\[
\lim_{t\downarrow 2}{t-\sqrt{2-t}}=2
\]
\end{eulerformula}
\begin{eulerprompt}
>$showev('limit(t-sqrt(2-t),t,5,plus)) // Perhatikan hasilnya
\end{eulerprompt}
\begin{eulerformula}
\[
\lim_{t\downarrow 5}{t-\sqrt{2-t}}=5-\sqrt{3}\,i
\]
\end{eulerformula}
\begin{eulerprompt}
>plot2d("x-sqrt(2-x)",0,2):
\end{eulerprompt}
\eulerimg{17}{images/EMT 1-6_Mifta Dilla Aninditya_23030630029-440.png}
\begin{eulerprompt}
>$showev('limit((x^2-9)/(2*x^2-5*x-3),x,3))
\end{eulerprompt}
\begin{eulerformula}
\[
\lim_{x\rightarrow 3}{\frac{x^2-9}{2\,x^2-5\,x-3}}=\frac{6}{7}
\]
\end{eulerformula}
\begin{eulercomment}
Tunjukkan limit tersebut dengan grafik, seperti contoh-contoh sebelumnya.
\end{eulercomment}
\begin{eulerprompt}
>$showev('limit((1-cos(x))/x,x,0))
\end{eulerprompt}
\begin{eulerformula}
\[
\lim_{x\rightarrow 0}{\frac{1-\cos x}{x}}=0
\]
\end{eulerformula}
\begin{eulercomment}
Tunjukkan limit tersebut dengan grafik, seperti contoh-contoh sebelumnya.
\end{eulercomment}
\begin{eulerprompt}
>$showev('limit((x^2+abs(x))/(x^2-abs(x)),x,0))
\end{eulerprompt}
\begin{eulerformula}
\[
\lim_{x\rightarrow 0}{\frac{\left| x\right| +x^2}{x^2-\left| x
 \right| }}=-1
\]
\end{eulerformula}
\begin{eulercomment}
Tunjukkan limit tersebut dengan grafik, seperti contoh-contoh sebelumnya.
\end{eulercomment}
\begin{eulerprompt}
>$showev('limit((1+1/x)^x,x,inf))
\end{eulerprompt}
\begin{eulerformula}
\[
\lim_{x\rightarrow \infty }{\left(\frac{1}{x}+1\right)^{x}}=e
\]
\end{eulerformula}
\begin{eulerprompt}
>plot2d("(1+1/x)^x",0,1000):
\end{eulerprompt}
\eulerimg{17}{images/EMT 1-6_Mifta Dilla Aninditya_23030630029-445.png}
\begin{eulerprompt}
>$showev('limit((1+k/x)^x,x,inf))
\end{eulerprompt}
\begin{eulerformula}
\[
\lim_{x\rightarrow \infty }{\left(\frac{k}{x}+1\right)^{x}}=e^{k}
\]
\end{eulerformula}
\begin{eulerprompt}
>$showev('limit((1+x)^(1/x),x,0))
\end{eulerprompt}
\begin{eulerformula}
\[
\lim_{x\rightarrow 0}{\left(x+1\right)^{\frac{1}{x}}}=e
\]
\end{eulerformula}
\begin{eulercomment}
Tunjukkan limit tersebut dengan grafik, seperti contoh-contoh sebelumnya.
\end{eulercomment}
\begin{eulerprompt}
>$showev('limit((x/(x+k))^x,x,inf))
\end{eulerprompt}
\begin{eulerformula}
\[
\lim_{x\rightarrow \infty }{\left(\frac{x}{x+k}\right)^{x}}=e^ {- k
  }
\]
\end{eulerformula}
\begin{eulerprompt}
>$showev('limit((E^x-E^2)/(x-2),x,2))
\end{eulerprompt}
\begin{eulerformula}
\[
\lim_{x\rightarrow 2}{\frac{e^{x}-e^2}{x-2}}=e^2
\]
\end{eulerformula}
\begin{eulercomment}
Tunjukkan limit tersebut dengan grafik, seperti contoh-contoh sebelumnya.
\end{eulercomment}
\begin{eulerprompt}
>$showev('limit(sin(1/x),x,0))
\end{eulerprompt}
\begin{eulerformula}
\[
\lim_{x\rightarrow 0}{\sin \left(\frac{1}{x}\right)}={\it ind}
\]
\end{eulerformula}
\begin{eulerprompt}
>$showev('limit(sin(1/x),x,inf))
\end{eulerprompt}
\begin{eulerformula}
\[
\lim_{x\rightarrow \infty }{\sin \left(\frac{1}{x}\right)}=0
\]
\end{eulerformula}
\begin{eulerprompt}
>plot2d("sin(1/x)",-0.001,0.001):
\end{eulerprompt}
\eulerimg{17}{images/EMT 1-6_Mifta Dilla Aninditya_23030630029-452.png}
\eulerheading{Latihan}
\begin{eulercomment}
Bukalah buku Kalkulus. Cari dan pilih beberapa (paling sedikit 5 fungsi berbeda
tipe/bentuk/jenis) fungsi dari buku tersebut, kemudian definisikan di EMT pada
baris-baris perintah berikut (jika perlu tambahkan lagi). Untuk setiap fungsi, hitung
nilai limit fungsi tersebut di beberapa nilai dan di tak hingga. Gambar grafik fungsi
tersebut untuk mengkonfirmasi nilai-nilai limit tersebut.
\end{eulercomment}
\begin{eulerprompt}
>$showev('limit(sqrt(x^2+5*x)/(x+3),x,inf))
\end{eulerprompt}
\begin{eulerformula}
\[
\lim_{x\rightarrow \infty }{\frac{\sqrt{x^2+5\,x}}{x+3}}=1
\]
\end{eulerformula}
\begin{eulerprompt}
>plot2d("sqrt(x^2+5*x)/(x+3)"):
\end{eulerprompt}
\eulerimg{17}{images/EMT 1-6_Mifta Dilla Aninditya_23030630029-454.png}
\begin{eulerprompt}
>$limit(sin(2*h)/sin(3*h),h,0)
\end{eulerprompt}
\begin{eulerformula}
\[
\frac{2}{3}
\]
\end{eulerformula}
\begin{eulerprompt}
>plot2d("cot(2*x)/cot(3*x)",-0.001,0.001); plot2d(0,2/3,>points,style="ow",>add):
\end{eulerprompt}
\eulerimg{17}{images/EMT 1-6_Mifta Dilla Aninditya_23030630029-456.png}
\begin{eulerprompt}
>$showev('limit(sin(2*x)/x,x,0))
\end{eulerprompt}
\begin{eulerformula}
\[
\lim_{x\rightarrow 0}{\frac{\sin \left(2\,x\right)}{x}}=2
\]
\end{eulerformula}
\begin{eulerprompt}
>plot2d("sin(2*x)/x",-pi,pi); plot2d(0,2,>points,style="ow",>add):
\end{eulerprompt}
\eulerimg{17}{images/EMT 1-6_Mifta Dilla Aninditya_23030630029-458.png}
\begin{eulerprompt}
>$showev('limit(t-sqrt(5-t),t,3,plus))
\end{eulerprompt}
\begin{eulerformula}
\[
\lim_{t\downarrow 3}{t-\sqrt{5-t}}=3-\sqrt{2}
\]
\end{eulerformula}
\begin{eulerprompt}
>plot2d("x-sqrt(5-x)",0,3):
\end{eulerprompt}
\eulerimg{17}{images/EMT 1-6_Mifta Dilla Aninditya_23030630029-460.png}
\begin{eulerprompt}
>$showev('limit(sin(x^2),x,inf))
\end{eulerprompt}
\begin{eulerformula}
\[
\lim_{x\rightarrow \infty }{\sin x^2}={\it ind}
\]
\end{eulerformula}
\begin{eulerprompt}
>plot2d("sin(x^2)",-0.001,0.1):
\end{eulerprompt}
\eulerimg{17}{images/EMT 1-6_Mifta Dilla Aninditya_23030630029-462.png}
\begin{eulercomment}
\begin{eulercomment}
\eulerheading{Turunan Fungsi}
\begin{eulercomment}
Definisi turunan:

\end{eulercomment}
\begin{eulerformula}
\[
f'(x) = \lim_{h\to 0} \frac{f(x+h)-f(x)}{h}
\]
\end{eulerformula}
\begin{eulercomment}
Berikut adalah contoh-contoh menentukan turunan fungsi dengan
menggunakan definisi turunan (limit).
\end{eulercomment}
\begin{eulerprompt}
>$showev('limit(((x+h)^2-x^2)/h,h,0)) // turunan x^2
\end{eulerprompt}
\begin{eulerformula}
\[
\lim_{h\rightarrow 0}{\frac{\left(x+h\right)^2-x^2}{h}}=2\,x
\]
\end{eulerformula}
\begin{eulerprompt}
>p &= expand((x+h)^2-x^2)|simplify; $p //pembilang dijabarkan dan disederhanakan
\end{eulerprompt}
\begin{eulerformula}
\[
2\,h\,x+h^2
\]
\end{eulerformula}
\begin{eulerprompt}
>q &=ratsimp(p/h); $q // ekspresi yang akan dihitung limitnya disederhanakan
\end{eulerprompt}
\begin{eulerformula}
\[
2\,x+h
\]
\end{eulerformula}
\begin{eulerprompt}
>$limit(q,h,0) // nilai limit sebagai turunan
\end{eulerprompt}
\begin{eulerformula}
\[
2\,x
\]
\end{eulerformula}
\begin{eulerprompt}
>$showev('limit(((x+h)^n-x^n)/h,h,0)) // turunan x^n
\end{eulerprompt}
\begin{eulerformula}
\[
\lim_{h\rightarrow 0}{\frac{\left(x+h\right)^{n}-x^{n}}{h}}=n\,x^{n
 -1}
\]
\end{eulerformula}
\begin{eulercomment}
Mengapa hasilnya seperti itu? Tuliskan atau tunjukkan bahwa hasil limit tersebut
benar, sehingga benar turunan fungsinya benar.  Tulis penjelasan Anda di komentar
ini.

Sebagai petunjuk, ekspansikan (x+h)\textasciicircum{}n dengan menggunakan teorema binomial.
\end{eulercomment}
\begin{eulerprompt}
>$showev('limit((sin(x+h)-sin(x))/h,h,0)) // turunan sin(x)
\end{eulerprompt}
\begin{eulerformula}
\[
\lim_{h\rightarrow 0}{\frac{\sin \left(x+h\right)-\sin x}{h}}=\cos 
 x
\]
\end{eulerformula}
\begin{eulercomment}
Mengapa hasilnya seperti itu? Tuliskan atau tunjukkan bahwa hasil limit tersebut\\
benar, sehingga benar turunan fungsinya benar.  Tulis penjelasan Anda di komentar
ini.

Sebagai petunjuk, ekspansikan sin(x+h) dengan menggunakan rumus jumlah dua sudut.
\end{eulercomment}
\begin{eulerprompt}
>$showev('limit((log(x+h)-log(x))/h,h,0)) // turunan log(x)
\end{eulerprompt}
\begin{eulerformula}
\[
\lim_{h\rightarrow 0}{\frac{\log \left(x+h\right)-\log x}{h}}=
 \frac{1}{x}
\]
\end{eulerformula}
\begin{eulercomment}
Mengapa hasilnya seperti itu? Tuliskan atau tunjukkan bahwa hasil limit tersebut\\
benar, sehingga benar turunan fungsinya benar.  Tulis penjelasan Anda di komentar
ini.

Sebagai petunjuk, gunakan sifat-sifat logaritma dan hasil limit pada bagian
sebelumnya di atas.
\end{eulercomment}
\begin{eulerprompt}
>$showev('limit((1/(x+h)-1/x)/h,h,0)) // turunan 1/x
\end{eulerprompt}
\begin{eulerformula}
\[
\lim_{h\rightarrow 0}{\frac{\frac{1}{x+h}-\frac{1}{x}}{h}}=-\frac{1
 }{x^2}
\]
\end{eulerformula}
\begin{eulerprompt}
>$showev('limit((E^(x+h)-E^x)/h,h,0)) // turunan f(x)=e^x
\end{eulerprompt}
\begin{euleroutput}
  Answering "Is x an integer?" with "integer"
  Answering "Is x an integer?" with "integer"
  Answering "Is x an integer?" with "integer"
  Answering "Is x an integer?" with "integer"
  Answering "Is x an integer?" with "integer"
  Maxima is asking
  Acceptable answers are: yes, y, Y, no, n, N, unknown, uk
  Is x an integer?
  
  Use assume!
  Error in:
  $showev('limit((E^(x+h)-E^x)/h,h,0)) // turunan f(x)=e^x ...
                                       ^
\end{euleroutput}
\begin{eulercomment}
Maxima bermasalah dengan limit:

\end{eulercomment}
\begin{eulerformula}
\[
\lim_{h\to 0}\frac{e^{x+h}-e^x}{h}.
\]
\end{eulerformula}
\begin{eulercomment}
Oleh karena itu diperlukan trik khusus agar hasilnya benar.
\end{eulercomment}
\begin{eulerprompt}
>$showev('limit((E^h-1)/h,h,0))
\end{eulerprompt}
\begin{eulerformula}
\[
\lim_{h\rightarrow 0}{\frac{e^{h}-1}{h}}=1
\]
\end{eulerformula}
\begin{eulerprompt}
>$showev('factor(E^(x+h)-E^x))
\end{eulerprompt}
\begin{eulerformula}
\[
{\it factor}\left(e^{x+h}-e^{x}\right)=\left(e^{h}-1\right)\,e^{x}
\]
\end{eulerformula}
\begin{eulerprompt}
>$showev('limit(factor((E^(x+h)-E^x)/h),h,0)) // turunan f(x)=e^x
\end{eulerprompt}
\begin{eulerformula}
\[
\left(\lim_{h\rightarrow 0}{\frac{e^{h}-1}{h}}\right)\,e^{x}=e^{x}
\]
\end{eulerformula}
\begin{eulerprompt}
>function f(x) &= x^x
\end{eulerprompt}
\begin{euleroutput}
  
                                     x
                                    x
  
\end{euleroutput}
\begin{eulerprompt}
>$showev('limit(f(x),x,0))
\end{eulerprompt}
\begin{eulerformula}
\[
\lim_{x\rightarrow 0}{x^{x}}=1
\]
\end{eulerformula}
\begin{eulercomment}
Silakan Anda gambar kurva

\end{eulercomment}
\begin{eulerformula}
\[
y=x^x.
\]
\end{eulerformula}
\begin{eulerprompt}
>$showev('limit((f(x+h)-f(x))/h,h,0)) // turunan f(x)=x^x
\end{eulerprompt}
\begin{eulerformula}
\[
\lim_{h\rightarrow 0}{\frac{\left(x+h\right)^{x+h}-x^{x}}{h}}=
 {\it infinity}
\]
\end{eulerformula}
\begin{eulercomment}
Di sini Maxima juga bermasalah terkait limit:

\end{eulercomment}
\begin{eulerformula}
\[
\lim_{h\to 0} \frac{(x+h)^{x+h}-x^x}{h}.
\]
\end{eulerformula}
\begin{eulercomment}
Dalam hal ini diperlukan asumsi nilai x.
\end{eulercomment}
\begin{eulerprompt}
>&assume(x>0); $showev('limit((f(x+h)-f(x))/h,h,0)) // turunan f(x)=x^x
\end{eulerprompt}
\begin{eulerformula}
\[
\lim_{h\rightarrow 0}{\frac{\left(x+h\right)^{x+h}-x^{x}}{h}}=x^{x}
 \,\left(\log x+1\right)
\]
\end{eulerformula}
\begin{eulercomment}
Mengapa hasilnya seperti itu? Tuliskan atau tunjukkan bahwa hasil limit tersebut benar, sehingga benar turunan fungsinya benar.
Tulis penjelasan Anda di komentar ini.
\end{eulercomment}
\begin{eulerprompt}
>&forget(x>0) // jangan lupa, lupakan asumsi untuk kembali ke semula
\end{eulerprompt}
\begin{euleroutput}
  
                                 [x > 0]
  
\end{euleroutput}
\begin{eulerprompt}
>&forget(x<0)
\end{eulerprompt}
\begin{euleroutput}
  
                                 [x < 0]
  
\end{euleroutput}
\begin{eulerprompt}
>&facts()
\end{eulerprompt}
\begin{euleroutput}
  
          [kind(sinh, one_to_one), kind(log, one_to_one), 
                          kind(tanh, one_to_one), kind(log, increasing)]
  
\end{euleroutput}
\begin{eulerprompt}
>$showev('limit((asin(x+h)-asin(x))/h,h,0)) // turunan arcsin(x)
\end{eulerprompt}
\begin{eulerformula}
\[
\lim_{h\rightarrow 0}{\frac{\arcsin \left(x+h\right)-\arcsin x}{h}}=
 \frac{1}{\sqrt{1-x^2}}
\]
\end{eulerformula}
\begin{eulercomment}
Mengapa hasilnya seperti itu? Tuliskan atau tunjukkan bahwa hasil limit tersebut benar, sehingga
benar turunan fungsinya benar. Tulis penjelasan Anda di komentar ini.
\end{eulercomment}
\begin{eulerprompt}
>$showev('limit((tan(x+h)-tan(x))/h,h,0)) // turunan tan(x)
\end{eulerprompt}
\begin{eulerformula}
\[
\lim_{h\rightarrow 0}{\frac{\tan \left(x+h\right)-\tan x}{h}}=
 \frac{1}{\cos ^2x}
\]
\end{eulerformula}
\begin{eulercomment}
Mengapa hasilnya seperti itu? Tuliskan atau tunjukkan bahwa hasil limit tersebut benar, sehingga
benar turunan fungsinya benar. Tulis penjelasan Anda di komentar ini.
\end{eulercomment}
\begin{eulerprompt}
>function f(x) &= sinh(x) // definisikan f(x)=sinh(x)
\end{eulerprompt}
\begin{euleroutput}
  
                                 sinh(x)
  
\end{euleroutput}
\begin{eulerprompt}
>function df(x) &= limit((f(x+h)-f(x))/h,h,0); $df(x) // df(x) = f'(x)
\end{eulerprompt}
\begin{eulerformula}
\[
\frac{e^ {- x }\,\left(e^{2\,x}+1\right)}{2}
\]
\end{eulerformula}
\begin{eulercomment}
Hasilnya adalah cosh(x), karena

\end{eulercomment}
\begin{eulerformula}
\[
\frac{e^x+e^{-x}}{2}=\cosh(x).
\]
\end{eulerformula}
\begin{eulerprompt}
>plot2d(["f(x)","df(x)"],-pi,pi,color=[blue,red]):
\end{eulerprompt}
\eulerimg{17}{images/EMT 1-6_Mifta Dilla Aninditya_23030630029-480.png}
\begin{eulerprompt}
>function f(x) &= sin(3*x^5+7)^2
\end{eulerprompt}
\begin{euleroutput}
  
                                 2    5
                              sin (3 x  + 7)
  
\end{euleroutput}
\begin{eulerprompt}
>diff(f,3), diffc(f,3)
\end{eulerprompt}
\begin{euleroutput}
  1198.32948904
  1198.72863721
\end{euleroutput}
\begin{eulercomment}
Apakah perbedaan diff dan diffc?
\end{eulercomment}
\begin{eulerprompt}
>$showev('diff(f(x),x))
\end{eulerprompt}
\begin{eulerformula}
\[
\frac{d}{d\,x}\,\sin ^2\left(3\,x^5+7\right)=30\,x^4\,\cos \left(3
 \,x^5+7\right)\,\sin \left(3\,x^5+7\right)
\]
\end{eulerformula}
\begin{eulerprompt}
>$% with x=3
\end{eulerprompt}
\begin{eulerformula}
\[
{\it \%at}\left(\frac{d}{d\,x}\,\sin ^2\left(3\,x^5+7\right) , x=3
 \right)=2430\,\cos 736\,\sin 736
\]
\end{eulerformula}
\begin{eulerprompt}
>$float(%)
\end{eulerprompt}
\begin{eulerformula}
\[
{\it \%at}\left(\frac{d^{1.0}}{d\,x^{1.0}}\,\sin ^2\left(3.0\,x^5+
 7.0\right) , x=3.0\right)=1198.728637211748
\]
\end{eulerformula}
\begin{eulerprompt}
>plot2d(f,0,3.1):
\end{eulerprompt}
\eulerimg{17}{images/EMT 1-6_Mifta Dilla Aninditya_23030630029-484.png}
\begin{eulerprompt}
>function f(x) &=5*cos(2*x)-2*x*sin(2*x) // mendifinisikan fungsi f
\end{eulerprompt}
\begin{euleroutput}
  
                        5 cos(2 x) - 2 x sin(2 x)
  
\end{euleroutput}
\begin{eulerprompt}
>function df(x) &=diff(f(x),x) // fd(x) = f'(x)
\end{eulerprompt}
\begin{euleroutput}
  
                       - 12 sin(2 x) - 4 x cos(2 x)
  
\end{euleroutput}
\begin{eulerprompt}
>$'f(1)=f(1), $float(f(1)), $'f(2)=f(2), $float(f(2)) // nilai f(1) dan f(2)
\end{eulerprompt}
\begin{eulerformula}
\[
f\left(1\right)=5\,\cos 2-2\,\sin 2
\]
\end{eulerformula}
\begin{eulerformula}
\[
-3.899329036387075
\]
\end{eulerformula}
\begin{eulerformula}
\[
f\left(2\right)=5\,\cos 4-4\,\sin 4
\]
\end{eulerformula}
\begin{eulerformula}
\[
-0.2410081230863468
\]
\end{eulerformula}
\begin{eulerprompt}
>xp=solve("df(x)",1,2,0) // solusi f'(x)=0 pada interval [1, 2]
\end{eulerprompt}
\begin{euleroutput}
  1.35822987384
\end{euleroutput}
\begin{eulerprompt}
>df(xp), f(xp) // cek bahwa f'(xp)=0 dan nilai ekstrim di titik tersebut
\end{eulerprompt}
\begin{euleroutput}
  0
  -5.67530133759
\end{euleroutput}
\begin{eulerprompt}
>plot2d(["f(x)","df(x)"],0,2*pi,color=[blue,red]): //grafik fungsi dan turunannya
\end{eulerprompt}
\eulerimg{17}{images/EMT 1-6_Mifta Dilla Aninditya_23030630029-489.png}
\begin{eulercomment}
Perhatikan titik-titik "puncak" grafik y=f(x) dan nilai turunan pada saat grafik fungsinya mencapai titik "puncak" tersebut.
\end{eulercomment}
\eulerheading{Latihan}
\begin{eulercomment}
Bukalah buku Kalkulus. Cari dan pilih beberapa (paling sedikit 5 fungsi berbeda tipe/bentuk/jenis) fungsi dari buku tersebut,
kemudian definisikan di EMT pada baris-baris perintah berikut (jika perlu tambahkan lagi). Untuk setiap fungsi, tentukan turunannya
dengan menggunakan definisi turunan (limit), menggunakan perintah diff, dan secara manual (langkah demi langkah yang dihitung dengan
Maxima) seperti contoh-contoh di atas. Gambar grafik fungsi asli dan fungsi turunannya pada sumbu koordinat yang sama.
\end{eulercomment}
\begin{eulerprompt}
>function f(x) := 1/x
>$showev('limit((((x+h)^3-x^3)/h),h,0))
>plot2d("f(x)"):
>function f(x) &= (cot(x+h)-cot(x))/h
\end{eulerprompt}
\begin{euleroutput}
  
                           cot(x + h) - cot(x)
                           -------------------
                                    h
  
\end{euleroutput}
\begin{eulerprompt}
>$showev('limit(f(x),h,0))
>plot2d("h(x)"):
>function f(x) &= sin(x^5-2)^2
\end{eulerprompt}
\begin{euleroutput}
  
                                  2  5
                               sin (x  - 2)
  
\end{euleroutput}
\begin{eulerprompt}
>diff(f,1), diffc(f,1)
\end{eulerprompt}
\begin{euleroutput}
  -4.54648713413
  -4.54647590906
\end{euleroutput}
\begin{eulerprompt}
>$showev('diff(f(x),x))
>$% with x=1
>$float(%)
>plot2d(f,0,2.5):
>function f(x) &= -sin(x)
\end{eulerprompt}
\begin{euleroutput}
  
                                 - sin(x)
  
\end{euleroutput}
\begin{eulerprompt}
>$showev('limit(f(x),x,0))
>plot2d("f(x)"):
>function f(x) &= cosh(x)
\end{eulerprompt}
\begin{euleroutput}
  
                                 cosh(x)
  
\end{euleroutput}
\begin{eulerprompt}
>function df(x) &= limit((f(x+h)-f(x))/h,h,0); $df(x)
>plot2d(["f(x)","df(x)"],-pi,pi,color=[green,yellow]):
\end{eulerprompt}
\eulerheading{Integral}
\begin{eulercomment}
EMT dapat digunakan untuk menghitung integral, baik integral tak tentu
maupun integral tentu. Untuk integral tak tentu (simbolik) sudah tentu
EMT menggunakan Maxima, sedangkan untuk perhitungan integral tentu EMT
sudah menyediakan beberapa fungsi yang mengimplementasikan algoritma
kuadratur (perhitungan integral tentu menggunakan metode numerik).

Pada notebook ini akan ditunjukkan perhitungan integral tentu dengan
menggunakan Teorema Dasar Kalkulus:

\end{eulercomment}
\begin{eulerformula}
\[
\int_a^b f(x)\ dx = F(b)-F(a), \quad \text{ dengan  } F'(x) = f(x).
\]
\end{eulerformula}
\begin{eulercomment}
Fungsi untuk menentukan integral adalah integrate. Fungsi ini dapat
digunakan untuk menentukan, baik integral tentu maupun tak tentu (jika
fungsinya memiliki antiderivatif). Untuk perhitungan integral tentu
fungsi integrate menggunakan metode numerik (kecuali fungsinya tidak
integrabel, kita tidak akan menggunakan metode ini).
\end{eulercomment}
\begin{eulerprompt}
>$showev('integrate(x^n,x))
\end{eulerprompt}
\begin{euleroutput}
  Answering "Is n equal to -1?" with "no"
\end{euleroutput}
\begin{eulerprompt}
>$showev('integrate(1/(1+x),x))
>$showev('integrate(1/(1+x^2),x))
>$showev('integrate(1/sqrt(1-x^2),x))
>$showev('integrate(sin(x),x,0,pi))
>plot2d("sin(x)",0,2*pi):
>$showev('integrate(sin(x),x,a,b))
>$showev('integrate(x^n,x,a,b))
\end{eulerprompt}
\begin{euleroutput}
  Answering "Is n positive, negative or zero?" with "positive"
\end{euleroutput}
\begin{eulerprompt}
>$showev('integrate(x^2*sqrt(2*x+1),x))
>$showev('integrate(x^2*sqrt(2*x+1),x,0,2))
>$ratsimp(%)
>$showev('integrate((sin(sqrt(x)+a)*E^sqrt(x))/sqrt(x),x,0,pi^2))
>$factor(%)
>function map f(x) &= E^(-x^2)
\end{eulerprompt}
\begin{euleroutput}
  
                                      2
                                   - x
                                  E
  
\end{euleroutput}
\begin{eulerprompt}
>$showev('integrate(f(x),x))
\end{eulerprompt}
\begin{eulercomment}
Fungsi f tidak memiliki antiturunan, integralnya masih memuat integral lain.

\end{eulercomment}
\begin{eulerformula}
\[
erf(x) = \int \frac{e^{-x^2}}{\sqrt{\pi}} \ dx.
\]
\end{eulerformula}
\begin{eulercomment}
Kita tidak dapat menggunakan teorema Dasar kalkulus untuk menghitung integral tentu fungsi tersebut jika semua batasnya berhingga.
Dalam hal ini dapat digunakan metode numerik (rumus kuadratur).

Misalkan kita akan menghitung:

maxima: 'integrate(f(x),x,0,pi)
\end{eulercomment}
\begin{eulerprompt}
>x=0:0.1:pi-0.1; plot2d(x,f(x+0.1),>bar); plot2d("f(x)",0,pi,>add):
\end{eulerprompt}
\begin{eulercomment}
Integral tentu

maxima: 'integrate(f(x),x,0,pi)

dapat dihampiri dengan jumlah luas persegi-persegi panjang di bawah kurva y=f(x)
tersebut. Langkah-langkahnya adalah sebagai berikut.
\end{eulercomment}
\begin{eulerprompt}
>t &= makelist(a,a,0,pi-0.1,0.1); // t sebagai list untuk menyimpan nilai-nilai x
>fx &= makelist(f(t[i]+0.1),i,1,length(t)); // simpan nilai-nilai f(x)
>// jangan menggunakan x sebagai list, kecuali Anda pakar Maxima!
\end{eulerprompt}
\begin{eulercomment}
Hasilnya adalah:

maxima: 'integrate(f(x),x,0,pi) = 0.1*sum(fx[i],i,1,length(fx))

Jumlah tersebut diperoleh dari hasil kali lebar sub-subinterval (=0.1) dan jumlah nilai-nilai f(x) untuk
x = 0.1, 0.2, 0.3, ..., 3.2.
\end{eulercomment}
\begin{eulerprompt}
>0.1*sum(f(x+0.1)) // cek langsung dengan perhitungan numerik EMT
\end{eulerprompt}
\begin{euleroutput}
  0.836219610253
\end{euleroutput}
\begin{eulercomment}
Untuk mendapatkan nilai integral tentu yang mendekati nilai sebenarnya, lebar
sub-intervalnya dapat diperkecil lagi, sehingga daerah di bawah kurva tertutup
semuanya, misalnya dapat digunakan lebar subinterval 0.001. (Silakan dicoba!)

Meskipun Maxima tidak dapat menghitung integral tentu fungsi tersebut untuk
batas-batas yang berhingga, namun integral tersebut dapat dihitung secara eksak jika
batas-batasnya tak hingga. Ini adalah salah satu keajaiban di dalam matematika, yang
terbatas tidak dapat dihitung secara eksak, namun yang tak hingga malah dapat
dihitung secara eksak.
\end{eulercomment}
\begin{eulerprompt}
>$showev('integrate(f(x),x,0,inf))
\end{eulerprompt}
\begin{eulercomment}
Tunjukkan kebenaran hasil di atas!

Berikut adalah contoh lain fungsi yang tidak memiliki antiderivatif, sehingga integral tentunya hanya
dapat dihitung dengan metode numerik.
\end{eulercomment}
\begin{eulerprompt}
>function f(x) &= x^x
\end{eulerprompt}
\begin{euleroutput}
  
                                     x
                                    x
  
\end{euleroutput}
\begin{eulerprompt}
>$showev('integrate(f(x),x,0,1))
>x=0:0.1:1-0.01; plot2d(x,f(x+0.01),>bar); plot2d("f(x)",0,1,>add):
\end{eulerprompt}
\begin{eulercomment}
Maxima gagal menghitung integral tentu tersebut secara langsung menggunakan perintah
integrate. Berikut kita lakukan seperti contoh sebelumnya untuk mendapat hasil atau
pendekatan nilai integral tentu tersebut.
\end{eulercomment}
\begin{eulerprompt}
>t &= makelist(a,a,0,1-0.01,0.01);
>fx &= makelist(f(t[i]+0.01),i,1,length(t));
\end{eulerprompt}
\begin{eulercomment}
maxima: 'integrate(f(x),x,0,1) = 0.01*sum(fx[i],i,1,length(fx))

Apakah hasil tersebut cukup baik? perhatikan gambarnya.
\end{eulercomment}
\begin{eulerprompt}
>function f(x) &= sin(3*x^5+7)^2
\end{eulerprompt}
\begin{euleroutput}
  
                                 2    5
                              sin (3 x  + 7)
  
\end{euleroutput}
\begin{eulerprompt}
>integrate(f,0,1)
\end{eulerprompt}
\begin{euleroutput}
  0.542581176074
\end{euleroutput}
\begin{eulerprompt}
>&showev('integrate(f(x),x,0,1))
\end{eulerprompt}
\begin{euleroutput}
  
           1                           1              pi
          /                      gamma(-) sin(14) sin(--)
          [     2    5                 5              10
          I  sin (3 x  + 7) dx = ------------------------
          ]                                  1/5
          /                              10 6
           0
         4/5                  1          4/5                  1
   - (((6    gamma_incomplete(-, 6 I) + 6    gamma_incomplete(-, - 6 I))
                              5                               5
               4/5                    1
   sin(14) + (6    I gamma_incomplete(-, 6 I)
                                      5
      4/5                    1                       pi
   - 6    I gamma_incomplete(-, - 6 I)) cos(14)) sin(--) - 60)/120
                             5                       10
  
\end{euleroutput}
\begin{eulerprompt}
>&float(%)
\end{eulerprompt}
\begin{euleroutput}
  
           1.0
          /
          [       2      5
          I    sin (3.0 x  + 7.0) dx = 
          ]
          /
           0.0
  0.09820784258795788 - 0.008333333333333333
   (0.3090169943749474 (0.1367372182078336
   (4.192962712629476 I gamma__incomplete(0.2, 6.0 I)
   - 4.192962712629476 I gamma__incomplete(0.2, - 6.0 I))
   + 0.9906073556948704 (4.192962712629476 gamma__incomplete(0.2, 6.0 I)
   + 4.192962712629476 gamma__incomplete(0.2, - 6.0 I))) - 60.0)
  
\end{euleroutput}
\begin{eulerprompt}
>$showev('integrate(x*exp(-x),x,0,1)) // Integral tentu (eksak)
\end{eulerprompt}
\begin{eulerformula}
\[
\int_{0}^{1}{x\,e^ {- x }\;dx}=1-2\,e^ {- 1 }
\]
\end{eulerformula}
\eulerheading{Aplikasi Integral Tentu}
\begin{eulerprompt}
>plot2d("x^3-x",-0.1,1.1); plot2d("-x^2",>add);  ...
>b=solve("x^3-x+x^2",0.5); x=linspace(0,b,200); xi=flipx(x); ...
>plot2d(x|xi,x^3-x|-xi^2,>filled,style="|",fillcolor=1,>add): // Plot daerah antara 2 kurva
\end{eulerprompt}
\eulerimg{17}{images/EMT 1-6_Mifta Dilla Aninditya_23030630029-491.png}
\begin{eulerprompt}
>a=solve("x^3-x+x^2",0), b=solve("x^3-x+x^2",1) // absis titik-titik potong kedua kurva
\end{eulerprompt}
\begin{euleroutput}
  0
  0.61803398875
\end{euleroutput}
\begin{eulerprompt}
>integrate("(-x^2)-(x^3-x)",a,b) // luas daerah yang diarsir
\end{eulerprompt}
\begin{euleroutput}
  0.0758191713542
\end{euleroutput}
\begin{eulercomment}
Hasil tersebut akan kita bandingkan dengan perhitungan secara analitik.
\end{eulercomment}
\begin{eulerprompt}
>a &= solve((-x^2)-(x^3-x),x); $a // menentukan absis titik potong kedua kurva secara eksak
\end{eulerprompt}
\begin{eulerformula}
\[
\left[ x=\frac{-\sqrt{5}-1}{2} , x=\frac{\sqrt{5}-1}{2} , x=0
  \right] 
\]
\end{eulerformula}
\begin{eulerprompt}
>$showev('integrate(-x^2-x^3+x,x,0,(sqrt(5)-1)/2)) // Nilai integral secara eksak
\end{eulerprompt}
\begin{eulerformula}
\[
\int_{0}^{\frac{\sqrt{5}-1}{2}}{-x^3-x^2+x\;dx}=\frac{13-5^{\frac{3
 }{2}}}{24}
\]
\end{eulerformula}
\begin{eulerprompt}
>$float(%)
\end{eulerprompt}
\begin{eulerformula}
\[
\int_{0.0}^{0.6180339887498949}{-1.0\,x^3-1.0\,x^2+x\;dx}=
 0.07581917135421037
\]
\end{eulerformula}
\eulersubheading{Panjang Kurva}
\begin{eulercomment}
Hitunglah panjang kurva berikut ini dan luas daerah di dalam kurva tersebut.

\end{eulercomment}
\begin{eulerformula}
\[
\gamma(t) = (r(t) \cos(t), r(t) \sin(t))
\]
\end{eulerformula}
\begin{eulercomment}
dengan

\end{eulercomment}
\begin{eulerformula}
\[
r(t) = 1 + \dfrac{\sin(3t)}{2},\quad 0\le t\le 2\pi.
\]
\end{eulerformula}
\begin{eulerprompt}
>t=linspace(0,2pi,1000); r=1+sin(3*t)/2; x=r*cos(t); y=r*sin(t); ...
>plot2d(x,y,>filled,fillcolor=red,style="/",r=1.5): // Kita gambar kurvanya terlebih dahulu
\end{eulerprompt}
\eulerimg{17}{images/EMT 1-6_Mifta Dilla Aninditya_23030630029-495.png}
\begin{eulerprompt}
>function r(t) &= 1+sin(3*t)/2; $'r(t)=r(t)
\end{eulerprompt}
\begin{eulerformula}
\[
r\left(t\right)=\frac{\sin \left(3\,t\right)}{2}+1
\]
\end{eulerformula}
\begin{eulerprompt}
>function fx(t) &= r(t)*cos(t); $'fx(t)=fx(t)
\end{eulerprompt}
\begin{eulerformula}
\[
{\it fx}\left(t\right)=\cos t\,\left(\frac{\sin \left(3\,t\right)}{
 2}+1\right)
\]
\end{eulerformula}
\begin{eulerprompt}
>function fy(t) &= r(t)*sin(t); $'fy(t)=fy(t)
\end{eulerprompt}
\begin{eulerformula}
\[
{\it fy}\left(t\right)=\sin t\,\left(\frac{\sin \left(3\,t\right)}{
 2}+1\right)
\]
\end{eulerformula}
\begin{eulerprompt}
>function ds(t) &= trigreduce(radcan(sqrt(diff(fx(t),t)^2+diff(fy(t),t)^2))); $'ds(t)=ds(t)
\end{eulerprompt}
\begin{eulerformula}
\[
{\it ds}\left(t\right)=\frac{\sqrt{4\,\cos \left(6\,t\right)+4\,
 \sin \left(3\,t\right)+9}}{2}
\]
\end{eulerformula}
\begin{eulerprompt}
>$integrate(ds(x),x,0,2*pi) //panjang (keliling) kurva
\end{eulerprompt}
\begin{eulerformula}
\[
\frac{\int_{0}^{2\,\pi}{\sqrt{4\,\cos \left(6\,x\right)+4\,\sin 
 \left(3\,x\right)+9}\;dx}}{2}
\]
\end{eulerformula}
\begin{eulercomment}
Maxima gagal melakukan perhitungan eksak integral tersebut.

Berikut kita hitung integralnya secara umerik dengan perintah EMT.
\end{eulercomment}
\begin{eulerprompt}
>integrate("ds(x)",0,2*pi)
\end{eulerprompt}
\begin{euleroutput}
  9.0749467823
\end{euleroutput}
\begin{eulercomment}
Spiral Logaritmik

\end{eulercomment}
\begin{eulerformula}
\[
x=e^{ax}\cos x,\ y=e^{ax}\sin x.
\]
\end{eulerformula}
\begin{eulerprompt}
>a=0.1; plot2d("exp(a*x)*cos(x)","exp(a*x)*sin(x)",r=2,xmin=0,xmax=2*pi):
\end{eulerprompt}
\eulerimg{17}{images/EMT 1-6_Mifta Dilla Aninditya_23030630029-501.png}
\begin{eulerprompt}
>&kill(a) // hapus expresi a
\end{eulerprompt}
\begin{euleroutput}
  
                                   done
  
\end{euleroutput}
\begin{eulerprompt}
>function fx(t) &= exp(a*t)*cos(t); $'fx(t)=fx(t)
\end{eulerprompt}
\begin{eulerformula}
\[
{\it fx}\left(t\right)=e^{a\,t}\,\cos t
\]
\end{eulerformula}
\begin{eulerprompt}
>function fy(t) &= exp(a*t)*sin(t); $'fy(t)=fy(t)
\end{eulerprompt}
\begin{eulerformula}
\[
{\it fy}\left(t\right)=e^{a\,t}\,\sin t
\]
\end{eulerformula}
\begin{eulerprompt}
>function df(t) &= trigreduce(radcan(sqrt(diff(fx(t),t)^2+diff(fy(t),t)^2))); $'df(t)=df(t)
\end{eulerprompt}
\begin{eulerformula}
\[
{\it df}\left(t\right)=\sqrt{a^2+1}\,e^{a\,t}
\]
\end{eulerformula}
\begin{eulerprompt}
>S &=integrate(df(t),t,0,2*%pi); $S // panjang kurva (spiral)
\end{eulerprompt}
\begin{eulerformula}
\[
\sqrt{a^2+1}\,\left(\frac{e^{2\,\pi\,a}}{a}-\frac{1}{a}\right)
\]
\end{eulerformula}
\begin{eulerprompt}
>S(a=0.1) // Panjang kurva untuk a=0.1
\end{eulerprompt}
\begin{euleroutput}
  8.78817491636
\end{euleroutput}
\begin{eulercomment}
Soal:

Tunjukkan bahwa keliling lingkaran dengan jari-jari r adalah K=2.pi.r.

Berikut adalah contoh menghitung panjang parabola.
\end{eulercomment}
\begin{eulerprompt}
>plot2d("x^2",xmin=-1,xmax=1):
\end{eulerprompt}
\eulerimg{17}{images/EMT 1-6_Mifta Dilla Aninditya_23030630029-506.png}
\begin{eulerprompt}
>$showev('integrate(sqrt(1+diff(x^2,x)^2),x,-1,1))
\end{eulerprompt}
\begin{eulerformula}
\[
\int_{-1}^{1}{\sqrt{4\,x^2+1}\;dx}=\frac{{\rm asinh}\; 2+2\,\sqrt{5
 }}{2}
\]
\end{eulerformula}
\begin{eulerprompt}
>$float(%)
\end{eulerprompt}
\begin{eulerformula}
\[
\int_{-1.0}^{1.0}{\sqrt{4.0\,x^2+1.0}\;dx}=2.957885715089195
\]
\end{eulerformula}
\begin{eulerprompt}
>x=-1:0.2:1; y=x^2; plot2d(x,y);  ...
>  plot2d(x,y,points=1,style="o#",add=1):
\end{eulerprompt}
\eulerimg{17}{images/EMT 1-6_Mifta Dilla Aninditya_23030630029-509.png}
\begin{eulercomment}
Panjang tersebut dapat dihampiri dengan menggunakan jumlah panjang ruas-ruas garis yang menghubungkan titik-titik pada parabola
tersebut.
\end{eulercomment}
\begin{eulerprompt}
>i=1:cols(x)-1; sum(sqrt((x[i+1]-x[i])^2+(y[i+1]-y[i])^2))
\end{eulerprompt}
\begin{euleroutput}
  2.95191957027
\end{euleroutput}
\begin{eulercomment}
Hasilnya mendekati panjang yang dihitung secara eksak. Untuk mendapatkan hampiran yang cukup akurat, jarak antar titik dapat
diperkecil, misalnya 0.1, 0.05, 0.01, dan seterusnya. Cobalah Anda ulangi perhitungannya dengan nilai-nilai tersebut.

\end{eulercomment}
\eulersubheading{Koordinat Kartesius}
\begin{eulercomment}
Berikut diberikan contoh perhitungan panjang kurva menggunakan koordinat Kartesius. Kita akan hitung panjang kurva dengan
persamaan implisit:

\end{eulercomment}
\begin{eulerformula}
\[
x^3+y^3-3xy=0.
\]
\end{eulerformula}
\begin{eulerprompt}
>z &= x^3+y^3-3*x*y; $z
\end{eulerprompt}
\begin{eulerformula}
\[
y^3-3\,x\,y+x^3
\]
\end{eulerformula}
\begin{eulerprompt}
>plot2d(z,r=2,level=0,n=100):
\end{eulerprompt}
\eulerimg{17}{images/EMT 1-6_Mifta Dilla Aninditya_23030630029-511.png}
\begin{eulercomment}
Kita tertarik pada kurva di kuadran pertama.
\end{eulercomment}
\begin{eulerprompt}
>plot2d(z,a=0,b=2,c=0,d=2,level=[-10;0],n=100,contourwidth=3,style="/"):
\end{eulerprompt}
\eulerimg{17}{images/EMT 1-6_Mifta Dilla Aninditya_23030630029-512.png}
\begin{eulercomment}
Kita selesaikan persamaannya untuk x.
\end{eulercomment}
\begin{eulerprompt}
>$z with y=l*x, sol &= solve(%,x); $sol
\end{eulerprompt}
\begin{eulerformula}
\[
l^3\,x^3+x^3-3\,l\,x^2
\]
\end{eulerformula}
\begin{eulerformula}
\[
\left[ x=\frac{3\,l}{l^3+1} , x=0 \right] 
\]
\end{eulerformula}
\begin{eulercomment}
Kita gunakan solusi tersebut untuk mendefinisikan fungsi dengan Maxima.
\end{eulercomment}
\begin{eulerprompt}
>function f(l) &= rhs(sol[1]); $'f(l)=f(l)
\end{eulerprompt}
\begin{eulerformula}
\[
f\left(l\right)=\frac{3\,l}{l^3+1}
\]
\end{eulerformula}
\begin{eulercomment}
Fungsi tersebut juga dapat digunaka untuk menggambar kurvanya. Ingat, bahwa fungsi tersebut adalah nilai x dan nilai y=l*x, yakni
x=f(l) dan y=l*f(l).
\end{eulercomment}
\begin{eulerprompt}
>plot2d(&f(x),&x*f(x),xmin=-0.5,xmax=2,a=0,b=2,c=0,d=2,r=1.5):
\end{eulerprompt}
\eulerimg{17}{images/EMT 1-6_Mifta Dilla Aninditya_23030630029-516.png}
\begin{eulercomment}
Elemen panjang kurva adalah:

\end{eulercomment}
\begin{eulerformula}
\[
ds=\sqrt{f'(l)^2+(lf'(l)+f(l))^2}.
\]
\end{eulerformula}
\begin{eulerprompt}
>function ds(l) &= ratsimp(sqrt(diff(f(l),l)^2+diff(l*f(l),l)^2)); $'ds(l)=ds(l)
\end{eulerprompt}
\begin{eulerformula}
\[
{\it ds}\left(l\right)=\frac{\sqrt{9\,l^8+36\,l^6-36\,l^5-36\,l^3+
 36\,l^2+9}}{\sqrt{l^{12}+4\,l^9+6\,l^6+4\,l^3+1}}
\]
\end{eulerformula}
\begin{eulerprompt}
>$integrate(ds(l),l,0,1)
\end{eulerprompt}
\begin{eulerformula}
\[
\int_{0}^{1}{\frac{\sqrt{9\,l^8+36\,l^6-36\,l^5-36\,l^3+36\,l^2+9}
 }{\sqrt{l^{12}+4\,l^9+6\,l^6+4\,l^3+1}}\;dl}
\]
\end{eulerformula}
\begin{eulercomment}
Integral tersebut tidak dapat dihitung secara eksak menggunakan Maxima. Kita hitung integral etrsebut secara numerik dengan Euler.
Karena kurva simetris, kita hitung untuk nilai variabel integrasi dari 0 sampai 1, kemudian hasilnya dikalikan 2. 
\end{eulercomment}
\begin{eulerprompt}
>2*integrate("ds(x)",0,1)
\end{eulerprompt}
\begin{euleroutput}
  4.91748872168
\end{euleroutput}
\begin{eulerprompt}
>2*romberg(&ds(x),0,1)// perintah Euler lain untuk menghitung nilai hampiran integral yang sama
\end{eulerprompt}
\begin{euleroutput}
  4.91748872168
\end{euleroutput}
\begin{eulercomment}
Perhitungan di datas dapat dilakukan untuk sebarang fungsi x dan y dengan mendefinisikan fungsi EMT, misalnya kita beri nama
panjangkurva. Fungsi ini selalu memanggil Maxima untuk menurunkan fungsi yang diberikan.
\end{eulercomment}
\begin{eulerprompt}
>function panjangkurva(fx,fy,a,b) ...
\end{eulerprompt}
\begin{eulerudf}
  ds=mxm("sqrt(diff(@fx,x)^2+diff(@fy,x)^2)");
  return romberg(ds,a,b);
  endfunction
\end{eulerudf}
\begin{eulerprompt}
>panjangkurva("x","x^2",-1,1) // cek untuk menghitung panjang kurva parabola sebelumnya
\end{eulerprompt}
\begin{euleroutput}
  2.95788571509
\end{euleroutput}
\begin{eulercomment}
Bandingkan dengan nilai eksak di atas.
\end{eulercomment}
\begin{eulerprompt}
>2*panjangkurva(mxm("f(x)"),mxm("x*f(x)"),0,1) // cek contoh terakhir, bandingkan hasilnya!
\end{eulerprompt}
\begin{euleroutput}
  4.91748872168
\end{euleroutput}
\begin{eulercomment}
Kita hitung panjang spiral Archimides berikut ini dengan fungsi tersebut.
\end{eulercomment}
\begin{eulerprompt}
>plot2d("x*cos(x)","x*sin(x)",xmin=0,xmax=2*pi,square=1):
\end{eulerprompt}
\eulerimg{17}{images/EMT 1-6_Mifta Dilla Aninditya_23030630029-519.png}
\begin{eulerprompt}
>panjangkurva("x*cos(x)","x*sin(x)",0,2*pi)
\end{eulerprompt}
\begin{euleroutput}
  21.2562941482
\end{euleroutput}
\begin{eulercomment}
Berikut kita definisikan fungsi yang sama namun dengan Maxima, untuk perhitungan eksak.
\end{eulercomment}
\begin{eulerprompt}
>&kill(ds,x,fx,fy)
\end{eulerprompt}
\begin{euleroutput}
  
                                   done
  
\end{euleroutput}
\begin{eulerprompt}
>function ds(fx,fy) &&= sqrt(diff(fx,x)^2+diff(fy,x)^2)
\end{eulerprompt}
\begin{euleroutput}
  
                             2              2
                    sqrt(diff (fy, x) + diff (fx, x))
  
\end{euleroutput}
\begin{eulerprompt}
>sol &= ds(x*cos(x),x*sin(x)); $sol // Kita gunakan untuk menghitung panjang kurva terakhir di atas
\end{eulerprompt}
\begin{eulerformula}
\[
\sqrt{\left(\cos x-x\,\sin x\right)^2+\left(\sin x+x\,\cos x\right)
 ^2}
\]
\end{eulerformula}
\begin{eulerprompt}
>$sol | trigreduce | expand, $integrate(%,x,0,2*pi), %()
\end{eulerprompt}
\begin{eulerformula}
\[
\sqrt{x^2+1}
\]
\end{eulerformula}
\begin{eulerformula}
\[
\frac{{\rm asinh}\; \left(2\,\pi\right)+2\,\pi\,\sqrt{4\,\pi^2+1}}{
 2}
\]
\end{eulerformula}
\begin{euleroutput}
  21.2562941482
\end{euleroutput}
\begin{eulercomment}
Hasilnya sama dengan perhitungan menggunakan fungsi EMT.

Berikut adalah contoh lain penggunaan fungsi Maxima tersebut.
\end{eulercomment}
\begin{eulerprompt}
>plot2d("3*x^2-1","3*x^3-1",xmin=-1/sqrt(3),xmax=1/sqrt(3),square=1):
\end{eulerprompt}
\eulerimg{17}{images/EMT 1-6_Mifta Dilla Aninditya_23030630029-523.png}
\begin{eulerprompt}
>sol &= radcan(ds(3*x^2-1,3*x^3-1)); $sol
\end{eulerprompt}
\begin{eulerformula}
\[
3\,x\,\sqrt{9\,x^2+4}
\]
\end{eulerformula}
\begin{eulerprompt}
>$showev('integrate(sol,x,0,1/sqrt(3))), $2*float(%) // panjang kurva di atas
\end{eulerprompt}
\begin{eulerformula}
\[
3\,\int_{0}^{\frac{1}{\sqrt{3}}}{x\,\sqrt{9\,x^2+4}\;dx}=3\,\left(
 \frac{7^{\frac{3}{2}}}{27}-\frac{8}{27}\right)
\]
\end{eulerformula}
\begin{eulerformula}
\[
6.0\,\int_{0.0}^{0.5773502691896258}{x\,\sqrt{9.0\,x^2+4.0}\;dx}=
 2.337835372767141
\]
\end{eulerformula}
\eulersubheading{Sikloid}
\begin{eulercomment}
Berikut kita akan menghitung panjang kurva lintasan (sikloid) suatu titik pada lingkaran yang berputar ke kanan pada permukaan
datar. Misalkan jari-jari lingkaran tersebut adalah r. Posisi titik pusat lingkaran pada saat t adalah:

\end{eulercomment}
\begin{eulerformula}
\[
(rt,r).
\]
\end{eulerformula}
\begin{eulercomment}
Misalkan posisi titik pada lingkaran tersebut mula-mula (0,0) dan posisinya pada saat t adalah:

\end{eulercomment}
\begin{eulerformula}
\[
(r(t-\sin(t)),r(1-\cos(t))).
\]
\end{eulerformula}
\begin{eulercomment}
Berikut kita plot lintasan tersebut dan beberapa posisi lingkaran ketika t=0, t=pi/2, t=r*pi.
\end{eulercomment}
\begin{eulerprompt}
>x &= r*(t-sin(t))
\end{eulerprompt}
\begin{euleroutput}
  
          [0, 1.66665833335744e-7 r, 1.33330666692022e-6 r, 
  4.499797504338432e-6 r, 1.066581336583994e-5 r, 
  2.083072932167196e-5 r, 3.599352055540239e-5 r, 
  5.71526624672386e-5 r, 8.530603082730626e-5 r, 
  1.214508019889565e-4 r, 1.665833531718508e-4 r, 
  2.216991628251896e-4 r, 2.877927110806339e-4 r, 
  3.658573803051457e-4 r, 4.568853557635201e-4 r, 
  5.618675264007778e-4 r, 6.817933857540259e-4 r, 
  8.176509330039827e-4 r, 9.704265741758145e-4 r, 
  0.001141105023499428 r, 0.001330669204938795 r, 
  0.001540100153900437 r, 0.001770376919130678 r, 
  0.002022476464811601 r, 0.002297373572865413 r, 
  0.002596040745477063 r, 0.002919448107844891 r, 
  0.003268563311168871 r, 0.003644351435886262 r, 
  0.004047774895164447 r, 0.004479793338660443 r, 0.0049413635565565 r, 
  0.005433439383882244 r, 0.005956971605131645 r, 
  0.006512907859185624 r, 0.007102192544548636 r, 
  0.007725766724910044 r, 0.00838456803503801 r, 
  0.009079530587017326 r, 0.009811584876838586 r, 0.0105816576913495 r, 
  0.01139067201557714 r, 0.01223954694042984 r, 0.01312919757078923 r, 
  0.01406053493400045 r, 0.01503446588876983 r, 0.01605189303448024 r, 
  0.01711371462093175 r, 0.01822082445851714 r, 0.01937411182884202 r, 
  0.02057446139579705 r, 0.02182275311709253 r, 0.02311986215626333 r, 
  0.02446665879515308 r, 0.02586400834688696 r, 0.02731277106934082 r, 
  0.02881380207911666 r, 0.03036795126603076 r, 0.03197606320812652 r, 
  0.0336389770872163 r, 0.03535752660496472 r, 0.03713253989951881 r, 
  0.03896483946269502 r, 0.0408552420577305 r, 0.04280455863760801 r, 
  0.04481359426396048 r, 0.04688314802656623 r, 0.04901401296344043 r, 
  0.05120697598153157 r, 0.05346281777803219 r, 0.05578231276230905 r, 
  0.05816622897846346 r, 0.06061532802852698 r, 0.0631303649963022 r, 
  0.06571208837185505 r, 0.06836123997666599 r, 0.07107855488944881 r, 
  0.07386476137264342 r, 0.07672058079958999 r, 0.07964672758239233 r, 
  0.08264390910047736 r, 0.0857128256298576 r, 0.08885417027310427 r, 
  0.09206862889003742 r, 0.09535688002914089 r, 0.0987195948597075 r, 
  0.1021574371047232 r, 0.1056710629744951 r, 0.1092611211010309 r, 
  0.1129282524731764 r, 0.1166730903725168 r, 0.1204962603100498 r, 
  0.1243983799636342 r, 0.1283800591162231 r, 0.1324418995948859 r, 
  0.1365844952106265 r, 0.140808431699002 r, 0.1451142866615502 r, 
  0.1495026295080298 r, 0.1539740213994798 r]
  
\end{euleroutput}
\begin{eulerprompt}
>y &= r*(1-cos(t))
\end{eulerprompt}
\begin{euleroutput}
  
          [0, 4.999958333473664e-5 r, 1.999933334222437e-4 r, 
  4.499662510124569e-4 r, 7.998933390220841e-4 r, 
  0.001249739605033717 r, 0.00179946006479581 r, 
  0.002448999746720415 r, 0.003198293697380561 r, 
  0.004047266988005727 r, 0.004995834721974179 r, 
  0.006043902043303184 r, 0.00719136414613375 r, 0.00843810628521191 r, 
  0.009784003787362772 r, 0.01122892206395776 r, 0.01277271662437307 r, 
  0.01441523309043924 r, 0.01615630721187855 r, 0.01799576488272969 r, 
  0.01993342215875837 r, 0.02196908527585173 r, 0.02410255066939448 r, 
  0.02633360499462523 r, 0.02866202514797045 r, 0.03108757828935527 r, 
  0.03361002186548678 r, 0.03622910363410947 r, 0.03894456168922911 r, 
  0.04175612448730281 r, 0.04466351087439402 r, 0.04766643011428662 r, 
  0.05076458191755917 r, 0.0539576564716131 r, 0.05724533447165381 r, 
  0.06062728715262111 r, 0.06410317632206519 r, 0.06767265439396564 r, 
  0.07133536442348987 r, 0.07509094014268702 r, 0.07893900599711501 r, 
  0.08287917718339499 r, 0.08691105968769186 r, 0.09103425032511492 r, 
  0.09524833678003664 r, 0.09955289764732322 r, 0.1039475024744748 r, 
  0.1084317118046711 r, 0.113005077220716 r, 0.1176671413898787 r, 
  0.1224174381096274 r, 0.1272554923542488 r, 0.1321808203223502 r, 
  0.1371929294852391 r, 0.1422913186361759 r, 0.1474754779404944 r, 
  0.152744888986584 r, 0.1580990248377314 r, 0.1635373500848132 r, 
  0.1690593208998367 r, 0.1746643850903219 r, 0.1803519821545206 r, 
  0.1861215433374662 r, 0.1919724916878484 r, 0.1979042421157076 r, 
  0.2039162014509444 r, 0.2100077685026351 r, 0.216178334119151 r, 
  0.2224272812490723 r, 0.2287539850028937 r, 0.2351578127155118 r, 
  0.2416381240094921 r, 0.2481942708591053 r, 0.2548255976551299 r, 
  0.2615314412704124 r, 0.2683111311261794 r, 0.2751639892590951 r, 
  0.2820893303890569 r, 0.2890864619877229 r, 0.2961546843477643 r, 
  0.3032932906528349 r, 0.3105015670482534 r, 0.3177787927123868 r, 
  0.3251242399287333 r, 0.3325371741586922 r, 0.3400168541150183 r, 
  0.3475625318359485 r, 0.3551734527599992 r, 0.3628488558014202 r, 
  0.3705879734263036 r, 0.3783900317293359 r, 0.3862542505111889 r, 
  0.3941798433565377 r, 0.4021660177127022 r, 0.4102119749689023 r, 
  0.418316910536117 r, 0.4264800139275439 r, 0.4347004688396462 r, 
  0.4429774532337832 r, 0.451310139418413 r]
  
\end{euleroutput}
\begin{eulercomment}
Berikut kita gambar sikloid untuk r=1.
\end{eulercomment}
\begin{eulerprompt}
>ex &= x-sin(x); ey &= 1-cos(x); aspect(1);
>plot2d(ex,ey,xmin=0,xmax=4pi,square=1); ...
>  plot2d("2+cos(x)","1+sin(x)",xmin=0,xmax=2pi,>add,color=blue); ...
>  plot2d([2,ex(2)],[1,ey(2)],color=red,>add); ...
>  plot2d(ex(2),ey(2),>points,>add,color=red); ...
>  plot2d("2pi+cos(x)","1+sin(x)",xmin=0,xmax=2pi,>add,color=blue); ...
>  plot2d([2pi,ex(2pi)],[1,ey(2pi)],color=red,>add);  ...
>  plot2d(ex(2pi),ey(2pi),>points,>add,color=red):
\end{eulerprompt}
\eulerimg{27}{images/EMT 1-6_Mifta Dilla Aninditya_23030630029-527.png}
\begin{eulercomment}
Berikut dihitung panjang lintasan untuk 1 putaran penuh. (Jangan salah menduga bahwa panjang lintasan 1 putaran penuh sama dengan
keliling lingkaran!)
\end{eulercomment}
\begin{eulerprompt}
>ds &= radcan(sqrt(diff(ex,x)^2+diff(ey,x)^2)); $ds=trigsimp(ds) // elemen panjang kurva sikloid
\end{eulerprompt}
\begin{eulerformula}
\[
\sqrt{\sin ^2x+\cos ^2x-2\,\cos x+1}=\sqrt{2-2\,\cos x}
\]
\end{eulerformula}
\begin{eulerprompt}
>ds &= trigsimp(ds); $ds
\end{eulerprompt}
\begin{eulerformula}
\[
\sqrt{2-2\,\cos x}
\]
\end{eulerformula}
\begin{eulerprompt}
>$showev('integrate(ds,x,0,2*pi)) // hitung panjang sikloid satu putaran penuh
\end{eulerprompt}
\begin{eulerformula}
\[
\int_{0}^{2\,\pi}{\sqrt{2-2\,\cos x}\;dx}=8
\]
\end{eulerformula}
\begin{eulerprompt}
>integrate(mxm("ds"),0,2*pi) // hitung secara numerik
\end{eulerprompt}
\begin{euleroutput}
  8
\end{euleroutput}
\begin{eulerprompt}
>romberg(mxm("ds"),0,2*pi) // cara lain hitung secara numerik
\end{eulerprompt}
\begin{euleroutput}
  8
\end{euleroutput}
\begin{eulercomment}
Perhatikan, seperti terlihat pada gambar, panjang sikloid lebih besar daripada keliling lingkarannya, yakni:

\end{eulercomment}
\begin{eulerformula}
\[
2\pi.
\]
\end{eulerformula}
\eulersubheading{Kurvatur (Kelengkungan) Kurva}
\begin{eulercomment}
image: Osculating.png

Aslinya, kelengkungan kurva diferensiabel (yakni, kurva mulus yang
tidak lancip) di titik P didefinisikan melalui lingkaran oskulasi
(yaitu, lingkaran yang melalui titik P dan terbaik memperkirakan,
paling banyak menyinggung kurva di sekitar P). Pusat dan radius
kelengkungan kurva di P adalah pusat dan radius lingkaran oskulasi.
Kelengkungan adalah kebalikan dari radius kelengkungan:

\end{eulercomment}
\begin{eulerformula}
\[
\kappa =\frac {1}{R}
\]
\end{eulerformula}
\begin{eulercomment}
dengan R adalah radius kelengkungan. (Setiap lingkaran memiliki
kelengkungan ini pada setiap titiknya, dapat diartikan, setiap
lingkaran berputar 2pi sejauh 2piR.)\\
Definisi ini sulit dimanipulasi dan dinyatakan ke dalam rumus untuk
kurva umum. Oleh karena itu digunakan definisi lain yang ekivalen.

\end{eulercomment}
\eulersubheading{Definisi Kurvatur dengan Fungsi Parametrik Panjang Kurva}
\begin{eulercomment}
Setiap kurva diferensiabel dapat dinyatakan dengan persamaan
parametrik terhadap panjang kurva s:

\end{eulercomment}
\begin{eulerformula}
\[
\gamma(s) = (x(s),\ y(s)),
\]
\end{eulerformula}
\begin{eulercomment}
dengan x dan y adalah fungsi riil yang diferensiabel, yang memenuhi:

\end{eulercomment}
\begin{eulerformula}
\[
\|\gamma'(s)\|=\sqrt{x'(s)^2+y'(s)^2}=1.
\]
\end{eulerformula}
\begin{eulercomment}
Ini berarti bahwa vektor singgung


\end{eulercomment}
\begin{eulerformula}
\[
\mathbf{T}(s)=(x'(s),\ y'(s))
\]
\end{eulerformula}
\begin{eulercomment}
memiliki norm 1 dan merupakan vektor singgung satuan.

Apabila kurvanya memiliki turunan kedua, artinya turunan kedua x dan y
ada, maka T'(s) ada. Vektor ini merupakan normal kurva yang arahnya
menuju pusat kurvatur, norm-nya merupakan nilai kurvatur
(kelengkungan):

\end{eulercomment}
\begin{eulerformula}
\[
 \begin{aligned}\mathbf{T}(s) &= \mathbf{\gamma}'(s),\\ \mathbf{T}^{2}(s) &=1\ \text{(konstanta)}\Rightarrow \mathbf{T}'(s)\cdot \mathbf{T}(s)=0\\ \kappa(s) &=\|\mathbf {T}'(s)\|= \|\mathbf{\gamma}''(s)\|=\sqrt{x''(s)^{2}+y''(s)^{2}}.\end{aligned}
\]
\end{eulerformula}
\begin{eulercomment}
Nilai

\end{eulercomment}
\begin{eulerformula}
\[
R(s)=\frac{1}{\kappa(s)}
\]
\end{eulerformula}
\begin{eulercomment}
disebut jari-jari (radius) kelengkungan kurva.

Bilangan riil

\end{eulercomment}
\begin{eulerformula}
\[
 k(s) = \pm\kappa(s)
\]
\end{eulerformula}
\begin{eulercomment}
disebut nilai kelengkungan bertanda.

Contoh:\\
Akan ditentukan kurvatur lingkaran

\end{eulercomment}
\begin{eulerformula}
\[
x=r\cos t,\ y= r\sin t.
\]
\end{eulerformula}
\begin{eulerprompt}
>fx &= r*cos(t); fy &=r*sin(t);
>&assume(t>0,r>0); s &=integrate(sqrt(diff(fx,t)^2+diff(fy,t)^2),t,0,t); s // elemen panjang kurva, panjang busur lingkaran (s)
\end{eulerprompt}
\begin{euleroutput}
  
                                   r t
  
\end{euleroutput}
\begin{eulerprompt}
>&kill(s); fx &= r*cos(s/r); fy &=r*sin(s/r); // definisi ulang persamaan parametrik terhadap s dengan substitusi t=s/r
>k &= trigsimp(sqrt(diff(fx,s,2)^2+diff(fy,s,2)^2)); $k // nilai kurvatur lingkaran dengan menggunakan definisi di atas
\end{eulerprompt}
\begin{eulerformula}
\[
\frac{1}{r}
\]
\end{eulerformula}
\begin{eulercomment}
Untuk representasi parametrik umum, misalkan

\end{eulercomment}
\begin{eulerformula}
\[
x = x(t),\ y= y(t)
\]
\end{eulerformula}
\begin{eulercomment}
merupakan persamaan parametrik untuk kurva bidang yang
terdiferensialkan dua kali. Kurvatur untuk kurva tersebut
didefinisikan sebagai

\end{eulercomment}
\begin{eulerformula}
\[
\begin{aligned}\kappa &= \frac{d\phi}{ds}=\frac{\frac{d\phi}{dt}}{\frac{ds}{dt}}\quad (\phi \text{ adalah sudut kemiringan garis singgung dan }s \text{ adalah panjang kurva})\\ &=\frac{\frac{d\phi}{dt}}{\sqrt{(\frac{dx}{dt})^2+(\frac{dy}{dt})^2}}= \frac{\frac{d\phi}{dt}}{\sqrt{x'(t)^2+y'(t)^2}}.\end{aligned}.
\]
\end{eulerformula}
\begin{eulercomment}
Selanjutnya, pembilang pada persamaan di atas dapat dicari sebagai
berikut.

\end{eulercomment}
\begin{eulerformula}
\[
\begin{aligned}\sec^2\phi\frac{d\phi}{dt} &= \frac{d}{dt}\left(\tan\phi\right)= \frac{d}{dt}\left(\frac{dy}{dx}\right)= \frac{d}{dt}\left(\frac{dy/dt}{dx/dt}\right)= \frac{d}{dt}\left(\frac{y'(t)}{x'(t)}\right)=\frac{x'(t)y''(t)-x''(t)y'(t)}{x'(t)^2}.\\ & \\ \frac{d\phi}{dt} &= \frac{1}{\sec^2\phi}\frac{x'(t)y''(t)-x''(t)y'(t)}{x'(t)^2}\\ &= \frac{1}{1+\tan^2\phi}\frac{x'(t)y''(t)-x''(t)y'(t)}{x'(t)^2}\\ &= \frac{1}{1+\left(\frac{y'(t)}{x'(t)}\right)^2}\frac{x'(t)y''(t)-x''(t)y'(t)}{x'(t)^2}\\ &= \frac{x'(t)y''(t)-x''(t)y'(t)}{x'(t)^2+y'(t)^2}.\end{aligned}
\]
\end{eulerformula}
\begin{eulercomment}
Jadi, rumus kurvatur untuk kurva parametrik

\end{eulercomment}
\begin{eulerformula}
\[
x=x(t),\ y=y(t)
\]
\end{eulerformula}
\begin{eulercomment}
adalah

\end{eulercomment}
\begin{eulerformula}
\[
\kappa(t) = \frac{x'(t)y''(t)-x''(t)y'(t)}{\left(x'(t)^2+y'(t)^2\right)^{3/2}}.
\]
\end{eulerformula}
\begin{eulercomment}
Jika kurvanya dinyatakan dengan persamaan parametrik pada koordinat
kutub

\end{eulercomment}
\begin{eulerformula}
\[
x=r(\theta)\cos\theta,\ y=r(\theta)\sin\theta,
\]
\end{eulerformula}
\begin{eulercomment}
maka rumus kurvaturnya adalah

\end{eulercomment}
\begin{eulerformula}
\[
\kappa(\theta) = \frac{r(\theta)^2+2r'(\theta)^2-r(\theta)r''(\theta)}{\left(r'(\theta)^2+r'(\theta)^2\right)^{3/2}}.
\]
\end{eulerformula}
\begin{eulercomment}
(Silakan Anda turunkan rumus tersebut!)

Contoh:\\
Lingkaran dengan pusat (0,0) dan jari-jari r dapat dinyatakan dengan
persamaan parametrik

\end{eulercomment}
\begin{eulerformula}
\[
x=r\cos t,\ y=r\sin t.
\]
\end{eulerformula}
\begin{eulercomment}
Nilai kelengkungan lingkaran tersebut adalah

\end{eulercomment}
\begin{eulerformula}
\[
\kappa(t)=\frac{x'(t)y''(t)-x''(t)y'(t)}{\left(x'(t)^2+y'(t)^2\right)^{3/2}}=\frac{r^2}{r^3}=\frac 1 r.
\]
\end{eulerformula}
\begin{eulercomment}
Hasil cocok dengan definisi kurvatur suatu kelengkungan.
\end{eulercomment}
\begin{eulercomment}
Kurva

\end{eulercomment}
\begin{eulerformula}
\[
y=f(x)
\]
\end{eulerformula}
\begin{eulercomment}
dapat dinyatakan ke dalam persamaan parametrik

\end{eulercomment}
\begin{eulerformula}
\[
x=t,\ y=f(t),\ \text{ dengan } x'(t)=1,\ x''(t)=0,
\]
\end{eulerformula}
\begin{eulercomment}
sehingga kurvaturnya adalah

\end{eulercomment}
\begin{eulerformula}
\[
\kappa(t) = \frac{y''(t)}{\left(1+y'(t)^2\right)^{3/2}}.
\]
\end{eulerformula}
\begin{eulercomment}
Contoh:\\
Akan ditentukan kurvatur parabola

\end{eulercomment}
\begin{eulerformula}
\[
y=ax^2+bx+c.
\]
\end{eulerformula}
\begin{eulerprompt}
>function f(x) &= a*x^2+b*x+c; $y=f(x)
\end{eulerprompt}
\begin{eulerformula}
\[
y=a\,x^2+b\,x+c
\]
\end{eulerformula}
\begin{eulerprompt}
>function k(x) &= (diff(f(x),x,2))/(1+diff(f(x),x)^2)^(3/2); $'k(x)=k(x) // kelengkungan parabola 
\end{eulerprompt}
\begin{eulerformula}
\[
k\left(x\right)=\frac{2\,a}{\left(\left(2\,a\,x+b\right)^2+1\right)
 ^{\frac{3}{2}}}
\]
\end{eulerformula}
\begin{eulerprompt}
>function f(x) &= x^2+x+1; $y=f(x) // akan kita plot kelengkungan parabola untuk a=b=c=1
\end{eulerprompt}
\begin{eulerformula}
\[
y=x^2+x+1
\]
\end{eulerformula}
\begin{eulerprompt}
>function k(x) &= (diff(f(x),x,2))/(1+diff(f(x),x)^2)^(3/2); $'k(x)=k(x) // kelengkungan parabola 
\end{eulerprompt}
\begin{eulerformula}
\[
k\left(x\right)=\frac{2}{\left(\left(2\,x+1\right)^2+1\right)^{
 \frac{3}{2}}}
\]
\end{eulerformula}
\begin{eulercomment}
Berikut kita gambar parabola tersebut beserta kurva kelengkungan,
kurva jari-jari kelengkungan dan salah satu lingkaran oskulasi di
titik puncak parabola. Perhatikan, puncak parabola dan jari-jari
lingkaran oskulasi di puncak parabola adalah

\end{eulercomment}
\begin{eulerformula}
\[
(-1/2,3/4),\ 1/k(2)=1/2,
\]
\end{eulerformula}
\begin{eulercomment}
sehingga pusat lingkaran oskulasi adalah (-1/2, 5/4).
\end{eulercomment}
\begin{eulerprompt}
>plot2d(["f(x)", "k(x)"],-2,1, color=[blue,red]); plot2d("1/k(x)",-1.5,1,color=green,>add); ...
>plot2d("-1/2+1/k(-1/2)*cos(x)","5/4+1/k(-1/2)*sin(x)",xmin=0,xmax=2pi,>add,color=blue):
\end{eulerprompt}
\eulerimg{27}{images/EMT 1-6_Mifta Dilla Aninditya_23030630029-536.png}
\begin{eulercomment}
Untuk kurva yang dinyatakan dengan fungsi implisit

\end{eulercomment}
\begin{eulerformula}
\[
F(x,y)=0
\]
\end{eulerformula}
\begin{eulercomment}
dengan turunan-turunan parsial

\end{eulercomment}
\begin{eulerformula}
\[
F_x=\frac{\partial F}{\partial x},\ F_y=\frac{\partial F}{\partial y},\ F_{xy}=\frac{\partial}{\partial y}\left(\frac{\partial F}{\partial x}\right),\ F_{xx}=\frac{\partial}{\partial x}\left(\frac{\partial F}{\partial x}\right),\ F_{yy}=\frac{\partial}{\partial y}\left(\frac{\partial F}{\partial y}\right),
\]
\end{eulerformula}
\begin{eulercomment}
berlaku

\end{eulercomment}
\begin{eulerformula}
\[
F_x dx+ F_y dy = 0\text{ atau } \frac{dy}{dx}=-\frac{F_x}{F_y},
\]
\end{eulerformula}
\begin{eulercomment}
sehingga kurvaturnya adalah

\end{eulercomment}
\begin{eulerformula}
\[
\kappa =\frac {F_y^2F_{xx}-2F_xF_yF_{xy}+F_x^2F_{yy}}{\left(F_x^2+F_y^2\right)^{3/2}}.
\]
\end{eulerformula}
\begin{eulercomment}
(Silakan Anda turunkan sendiri!)

Contoh 1:\\
Parabola

\end{eulercomment}
\begin{eulerformula}
\[
y=ax^2+bx+c
\]
\end{eulerformula}
\begin{eulercomment}
dapat dinyatakan ke dalam persamaan implisit

\end{eulercomment}
\begin{eulerformula}
\[
ax^2+bx+c-y=0.
\]
\end{eulerformula}
\begin{eulerprompt}
>function F(x,y) &=a*x^2+b*x+c-y; $F(x,y)
\end{eulerprompt}
\begin{eulerformula}
\[
-y+a\,x^2+b\,x+c
\]
\end{eulerformula}
\begin{eulerprompt}
>Fx &= diff(F(x,y),x), Fxx &=diff(F(x,y),x,2), Fy &=diff(F(x,y),y), Fxy &=diff(diff(F(x,y),x),y), Fyy &=diff(F(x,y),y,2)  
\end{eulerprompt}
\begin{euleroutput}
  
                                2 a x + b
  
  
                                   2 a
  
  
                                   - 1
  
  
                                    0
  
  
                                    0
  
\end{euleroutput}
\begin{eulerprompt}
>function k(x) &= (Fy^2*Fxx-2*Fx*Fy*Fxy+Fx^2*Fyy)/(Fx^2+Fy^2)^(3/2); $'k(x)=k(x) // kurvatur parabola tersebut
\end{eulerprompt}
\begin{eulerformula}
\[
k\left(x\right)=\frac{2\,a}{\left(\left(2\,a\,x+b\right)^2+1\right)
 ^{\frac{3}{2}}}
\]
\end{eulerformula}
\begin{eulercomment}
Hasilnya sama dengan sebelumnya yang menggunakan persamaan parabola biasa.
\end{eulercomment}
\eulerheading{Latihan}
\begin{eulercomment}
- Bukalah buku Kalkulus.\\
- Cari dan pilih beberapa (paling sedikit 5 fungsi berbeda
tipe/bentuk/jenis) fungsi dari buku tersebut, kemudian definisikan di
EMT pada baris-baris perintah berikut (jika perlu tambahkan lagi).\\
- Untuk setiap fungsi, tentukan anti turunannya (jika ada), hitunglah
integral tentu dengan batas-batas yang menarik (Anda tentukan
sendiri), seperti contoh-contoh tersebut.\\
- Lakukan hal yang sama untuk fungsi-fungsi yang tidak dapat
diintegralkan (cari sedikitnya 3 fungsi).\\
- Gambar grafik fungsi dan daerah integrasinya pada sumbu koordinat
yang sama.\\
- Gunakan integral tentu untuk mencari luas daerah yang dibatasi oleh
dua kurva yang berpotongan di dua titik. (Cari dan gambar kedua kurva
dan arsir (warnai) daerah yang dibatasi oleh keduanya.)\\
- Gunakan integral tentu untuk menghitung volume benda putar kurva y=
f(x) yang diputar mengelilingi sumbu x dari x=a sampai x=b, yakni

\end{eulercomment}
\begin{eulerformula}
\[
V = \int_a^b \pi (f(x)^2\ dx.
\]
\end{eulerformula}
\begin{eulercomment}
(Pilih fungsinya dan gambar kurva dan benda putar yang dihasilkan.
Anda dapat mencari contoh-contoh bagaimana cara menggambar benda hasil
perputaran suatu kurva.)\\
- Gunakan integral tentu untuk menghitung panjang kurva y=f(x) dari
x=a sampai x=b dengan menggunakan rumus:

\end{eulercomment}
\begin{eulerformula}
\[
S = \int_a^b \sqrt{1+(f'(x))^2} \ dx.
\]
\end{eulerformula}
\begin{eulercomment}
(Pilih fungsi dan gambar kurvanya.)\\
- Apabila fungsi dinyatakan dalam koordinat kutub x=f(r,t), y=g(r,t),
r=h(t), x=a bersesuaian dengan t=t0 dan x=b bersesuian dengan t=t1,
maka rumus di atas akan menjadi:

\end{eulercomment}
\begin{eulerformula}
\[
S=\int_{t_0}^{t_1} \sqrt{x'(t)^2+y'(t)^2}\ dt.
\]
\end{eulerformula}
\begin{eulercomment}
- Pilih beberapa kurva menarik (selain lingkaran dan parabola) dari
buku  kalkulus. Nyatakan setiap kurva tersebut dalam bentuk:\\
\end{eulercomment}
\begin{eulerttcomment}
  a. koordinat Kartesius (persamaan y=f(x))
  b. koordinat kutub ( r=r(theta))
  c. persamaan parametrik x=x(t), y=y(t)
  d. persamaan implit F(x,y)=0
\end{eulerttcomment}
\begin{eulercomment}
- Tentukan kurvatur masing-masing kurva dengan menggunakan keempat
representasi tersebut (hasilnya harus sama).\\
- Gambarlah kurva asli, kurva kurvatur, kurva jari-jari lingkaran
oskulasi, dan salah satu lingkaran oskulasinya.
\end{eulercomment}
\begin{eulerprompt}
>$showev('integrate(1/t^4-sqrt(t),t,1,2))
\end{eulerprompt}
\begin{eulerformula}
\[
\int_{1}^{2}{\frac{1}{t^4}-\sqrt{t}\;dt}=\frac{23-2^{\frac{11}{2}}
 }{24}
\]
\end{eulerformula}
\begin{eulerprompt}
>$showev('integrate(t-3-sqrt(1/t),t,0,2))
\end{eulerprompt}
\begin{eulerformula}
\[
\int_{0}^{2}{t-\frac{1}{\sqrt{t}}-3\;dt}=-2^{\frac{3}{2}}-4
\]
\end{eulerformula}
\begin{eulerprompt}
>$showev('integrate(sin(t)-(3*t),t,0,5))
\end{eulerprompt}
\begin{eulerformula}
\[
\int_{0}^{5}{\sin t-3\,t\;dt}=\frac{-2\,\cos 5-73}{2}
\]
\end{eulerformula}
\begin{eulerprompt}
>$showev('integrate(tan(t^2)-sqrt(3/t),t,1,2))
\end{eulerprompt}
\begin{eulerformula}
\[
\int_{1}^{2}{\tan t^2-\frac{\sqrt{3}}{\sqrt{t}}\;dt}=\int_{1}^{2}{
 \tan t^2-\frac{\sqrt{3}}{\sqrt{t}}\;dt}
\]
\end{eulerformula}
\begin{eulerprompt}
>$showev('integrate(1/cos(2*x)^3,x,0,4))
\end{eulerprompt}
\begin{eulerformula}
\[
\int_{0}^{4}{\frac{1}{\cos ^3\left(2\,x\right)}\;dx}=\frac{\sin ^28
 \,\log \left(\sin 8+1\right)}{8\,\sin ^28-8}-\frac{\log \left(\sin 8
 +1\right)}{8\,\sin ^28-8}-\frac{\sin ^28\,\log \left(1-\sin 8\right)
 }{8\,\sin ^28-8}+\frac{\log \left(1-\sin 8\right)}{8\,\sin ^28-8}-
 \frac{2\,\sin 8}{8\,\sin ^28-8}
\]
\end{eulerformula}
\begin{eulerprompt}
>$showev('integrate(cos(x)/x,x))
\end{eulerprompt}
\begin{eulerformula}
\[
\int {\frac{\cos x}{x}}{\;dx}=\frac{-{\it gamma\_\_incomplete}
 \left(0 , i\,x\right)-{\it gamma\_\_incomplete}\left(0 , -i\,x
 \right)}{2}
\]
\end{eulerformula}
\begin{eulerprompt}
>$showev('integrate(tan(x)/log(x),x))
\end{eulerprompt}
\begin{eulerformula}
\[
\int {\frac{\tan x}{\log x}}{\;dx}=\frac{4\,\int {\frac{\sin \left(
 2\,x\right)}{\log x\,\sin ^2\left(2\,x\right)+\log x\,\cos ^2\left(2
 \,x\right)+2\,\log x\,\cos \left(2\,x\right)+\log x}}{\;dx}-i\,
 {\it gamma\_\_incomplete}\left(0 , -\log x\right)^\star+i\,
 {\it gamma\_\_incomplete}\left(0 , -\log x\right)}{2}
\]
\end{eulerformula}
\begin{eulerprompt}
>function f(x) &= (x^2-2+x)/2
\end{eulerprompt}
\begin{euleroutput}
  
                                 2
                                x  + x - 2
                                ----------
                                    2
  
\end{euleroutput}
\begin{eulerprompt}
>$showev('integrate(pi*f(x)^2,x,0,3))
\end{eulerprompt}
\begin{eulerformula}
\[
\frac{\pi\,\int_{0}^{3}{\left(x^2+x-2\right)^2\;dx}}{4}=\frac{561\,
 \pi}{40}
\]
\end{eulerformula}
\begin{eulerprompt}
>x=1:0.1:4-0.01; plot2d(x,f(x+0.01),>bar); plot2d("f(x)",0,3,>add):
\end{eulerprompt}
\eulerimg{27}{images/EMT 1-6_Mifta Dilla Aninditya_23030630029-547.png}
\eulerheading{Barisan dan Deret}
\begin{eulercomment}
(Catatan: bagian ini belum lengkap. Anda dapat membaca contoh-contoh pengguanaan EMT dan
Maxima untuk menghitung limit barisan, rumus jumlah parsial suatu deret, jumlah tak hingga
suatu deret konvergen, dan sebagainya. Anda dapat mengeksplor contoh-contoh di EMT atau
perbagai panduan penggunaan Maxima di software Maxima atau dari Internet.)

Barisan dapat didefinisikan dengan beberapa cara di dalam EMT, di antaranya:

- dengan cara yang sama seperti mendefinisikan vektor dengan elemen-elemen beraturan
(menggunakan titik dua ":");\\
- menggunakan perintah "sequence" dan rumus barisan (suku ke -n);\\
- menggunakan perintah "iterate" atau "niterate";\\
- menggunakan fungsi Maxima "create\_list" atau "makelist" untuk menghasilkan barisan
simbolik;\\
- menggunakan fungsi biasa yang inputnya vektor atau barisan;\\
- menggunakan fungsi rekursif.

EMT menyediakan beberapa perintah (fungsi) terkait barisan, yakni:

- sum: menghitung jumlah semua elemen suatu barisan\\
- cumsum: jumlah kumulatif suatu barisan\\
- differences: selisih antar elemen-elemen berturutan

EMT juga dapat digunakan untuk menghitung jumlah deret berhingga maupun deret tak hingga,
dengan menggunakan perintah (fungsi) "sum". Perhitungan dapat dilakukan secara numerik
maupun simbolik dan eksak.

Berikut adalah beberapa contoh perhitungan barisan dan deret menggunakan EMT.
\end{eulercomment}
\begin{eulerprompt}
>1:10 // barisan sederhana
\end{eulerprompt}
\begin{euleroutput}
  [1,  2,  3,  4,  5,  6,  7,  8,  9,  10]
\end{euleroutput}
\begin{eulerprompt}
>1:2:30
\end{eulerprompt}
\begin{euleroutput}
  [1,  3,  5,  7,  9,  11,  13,  15,  17,  19,  21,  23,  25,  27,  29]
\end{euleroutput}
\eulerheading{Iterasi dan Barisan}
\begin{eulercomment}
EMT menyediakan fungsi iterate("g(x)", x0, n) untuk melakukan iterasi

\end{eulercomment}
\begin{eulerformula}
\[
x_{k+1}=g(x_k), \ x_0=x_0, k= 1, 2, 3, ..., n.
\]
\end{eulerformula}
\begin{eulercomment}
Berikut ini disajikan contoh-contoh penggunaan iterasi dan rekursi dengan EMT. Contoh
pertama menunjukkan pertumbuhan dari nilai awal 1000 dengan laju pertambahan 5\%, selama 10
periode.
\end{eulercomment}
\begin{eulerprompt}
>q=1.05; iterate("x*q",1000,n=10)'
\end{eulerprompt}
\begin{euleroutput}
           1000 
           1050 
         1102.5 
        1157.63 
        1215.51 
        1276.28 
         1340.1 
         1407.1 
        1477.46 
        1551.33 
        1628.89 
\end{euleroutput}
\begin{eulercomment}
Contoh berikutnya memperlihatkan bahaya menabung di bank pada masa sekarang! Dengan bunga
tabungan sebesar 6\% per tahun atau 0.5\% per bulan dipotong pajak 20\%, dan biaya administrasi
10000 per bulan, tabungan sebesar 1 juta tanpa diambil selama sekitar 10 tahunan akan habis
diambil oleh bank!
\end{eulercomment}
\begin{eulerprompt}
>r=0.005; plot2d(iterate("(1+0.8*r)*x-10000",1000000,n=130)):
\end{eulerprompt}
\begin{eulercomment}
Silakan Anda coba-coba, dengan tabungan minimal berapa agar tidak akan habis diambil oleh
bank dengan ketentuan bunga dan biaya administrasi seperti di atas.

Berikut adalah perhitungan minimal tabungan agar aman di bank dengan bunga sebesar r dan
biaya administrasi a, pajak bunga 20\%.
\end{eulercomment}
\begin{eulerprompt}
>$solve(0.8*r*A-a,A), $% with [r=0.005, a=10] 
\end{eulerprompt}
\begin{eulercomment}
Berikut didefinisikan fungsi untuk menghitung saldo tabungan, kemudian dilakukan iterasi.
\end{eulercomment}
\begin{eulerprompt}
>function saldo(x,r,a) := round((1+0.8*r)*x-a,2);
>iterate(\{\{"saldo",0.005,10\}\},1000,n=6)
\end{eulerprompt}
\begin{euleroutput}
  [1000,  994,  987.98,  981.93,  975.86,  969.76,  963.64]
\end{euleroutput}
\begin{eulerprompt}
>iterate(\{\{"saldo",0.005,10\}\},2000,n=6)
\end{eulerprompt}
\begin{euleroutput}
  [2000,  1998,  1995.99,  1993.97,  1991.95,  1989.92,  1987.88]
\end{euleroutput}
\begin{eulerprompt}
>iterate(\{\{"saldo",0.005,10\}\},2500,n=6)
\end{eulerprompt}
\begin{euleroutput}
  [2500,  2500,  2500,  2500,  2500,  2500,  2500]
\end{euleroutput}
\begin{eulercomment}
Tabungan senilai 2,5 juta akan aman dan tidak akan berubah nilai (jika tidak ada penarikan),
sedangkan jika tabungan awal kurang dari 2,5 juta, lama kelamaan akan berkurang meskipun
tidak pernah dilakukan penarikan uang tabungan.
\end{eulercomment}
\begin{eulerprompt}
>iterate(\{\{"saldo",0.005,10\}\},3000,n=6)
\end{eulerprompt}
\begin{euleroutput}
  [3000,  3002,  3004.01,  3006.03,  3008.05,  3010.08,  3012.12]
\end{euleroutput}
\begin{eulercomment}
Tabungan yang lebih dari 2,5 juta baru akan bertambah jika tidak ada penarikan.

Untuk barisan yang lebih kompleks dapat digunakan fungsi "sequence()". Fungsi ini menghitung
nilai-nilai x[n] dari semua nilai sebelumnya, x[1],...,x[n-1] yang diketahui.\\
Berikut adalah contoh barisan Fibonacci.

\end{eulercomment}
\begin{eulerformula}
\[
x_n = x_{n-1}+x_{n-2}, \quad x_1=1, \quad x_2 =1
\]
\end{eulerformula}
\begin{eulerprompt}
>sequence("x[n-1]+x[n-2]",[1,1],15)
\end{eulerprompt}
\begin{euleroutput}
  [1,  1,  2,  3,  5,  8,  13,  21,  34,  55,  89,  144,  233,  377,  610]
\end{euleroutput}
\begin{eulercomment}
Barisan Fibonacci memiliki banyak sifat menarik, salah satunya adalah akar pangkat ke-n suku
ke-n akan konvergen ke pecahan emas:
\end{eulercomment}
\begin{eulerprompt}
>$'(1+sqrt(5))/2=float((1+sqrt(5))/2)
>plot2d(sequence("x[n-1]+x[n-2]",[1,1],250)^(1/(1:250))):
\end{eulerprompt}
\begin{eulercomment}
Barisan yang sama juga dapat dihasilkan dengan menggunakan loop.
\end{eulercomment}
\begin{eulerprompt}
>x=ones(500); for k=3 to 500; x[k]=x[k-1]+x[k-2]; end;
\end{eulerprompt}
\begin{eulercomment}
Rekursi dapat dilakukan dengan menggunakan rumus yang tergantung pada semua elemen
sebelumnya. Pada contoh berikut, elemen ke-n merupakan jumlah (n-1) elemen sebelumnya,
dimulai dengan 1 (elemen ke-1). Jelas, nilai elemen ke-n adalah 2\textasciicircum{}(n-2), untuk n=2, 4, 5,
....
\end{eulercomment}
\begin{eulerprompt}
>sequence("sum(x)",1,10)
\end{eulerprompt}
\begin{euleroutput}
  [1,  1,  2,  4,  8,  16,  32,  64,  128,  256]
\end{euleroutput}
\begin{eulercomment}
Selain menggunakan ekspresi dalam x dan n, kita juga dapat menggunakan fungsi.

Pada contoh berikut, digunakan iterasi

\end{eulercomment}
\begin{eulerformula}
\[
x_n =A \cdot x_{n-1},
\]
\end{eulerformula}
\begin{eulercomment}
dengan A suatu matriks 2x2, dan setiap x[n] merupakan matriks/vektor 2x1.
\end{eulercomment}
\begin{eulerprompt}
>A=[1,1;1,2]; function suku(x,n) := A.x[,n-1]
>sequence("suku",[1;1],6)
\end{eulerprompt}
\begin{euleroutput}
  Real 2 x 6 matrix
  
              1             2             5            13     ...
              1             3             8            21     ...
\end{euleroutput}
\begin{eulercomment}
Hasil yang sama juga dapat diperoleh dengan menggunakan fungsi perpangkatan matriks
"matrixpower()". Cara ini lebih cepat, karena hanya menggunakan perkalian matriks sebanyak
log\_2(n).

\end{eulercomment}
\begin{eulerformula}
\[
x_n=A.x_{n-1}=A^2.x_{n-2}=A^3.x_{n-3}= ... = A^{n-1}.x_1.
\]
\end{eulerformula}
\begin{eulerprompt}
>sequence("matrixpower(A,n).[1;1]",1,6)
\end{eulerprompt}
\begin{euleroutput}
  Real 2 x 6 matrix
  
              1             5            13            34     ...
              1             8            21            55     ...
\end{euleroutput}
\eulerheading{Spiral Theodorus}
\begin{eulercomment}
image: Spiral\_of\_Theodorus.png\\
Spiral Theodorus (spiral segitiga siku-siku) dapat digambar secara rekursif. Rumus
rekursifnya adalah:

\end{eulercomment}
\begin{eulerformula}
\[
x_n = \left( 1 + \frac{i}{\sqrt{n-1}} \right) \, x_{n-1}, \quad x_1=1,
\]
\end{eulerformula}
\begin{eulercomment}
yang menghasilkan barisan bilangan kompleks.
\end{eulercomment}
\begin{eulerprompt}
>function g(n) := 1+I/sqrt(n)
\end{eulerprompt}
\begin{eulercomment}
Rekursinya dapat dijalankan sebanyak 17 untuk menghasilkan barisan 17 bilangan kompleks,
kemudian digambar bilangan-bilangan kompleksnya.
\end{eulercomment}
\begin{eulerprompt}
>x=sequence("g(n-1)*x[n-1]",1,17); plot2d(x,r=3.5); textbox(latex("Spiral\(\backslash\) Theodorus"),0.4):
\end{eulerprompt}
\begin{eulercomment}
Selanjutnya dihubungan titik 0 dengan titik-titik kompleks tersebut menggunakan loop.
\end{eulercomment}
\begin{eulerprompt}
>for i=1:cols(x); plot2d([0,x[i]],>add); end:
> 
\end{eulerprompt}
\begin{eulercomment}
Spiral tersebut juga dapat didefinisikan menggunakan fungsi rekursif, yang tidak memmerlukan
indeks dan bilangan kompleks. Dalam hal ini diigunakan vektor kolom pada bidang.
\end{eulercomment}
\begin{eulerprompt}
>function gstep (v) ...
\end{eulerprompt}
\begin{eulerudf}
  w=[-v[2];v[1]];
  return v+w/norm(w);
  endfunction
\end{eulerudf}
\begin{eulercomment}
Jika dilakukan iterasi 16 kali dimulai dari [1;0] akan didapatkan matriks yang memuat
vektor-vektor dari setiap iterasi.
\end{eulercomment}
\begin{eulerprompt}
>x=iterate("gstep",[1;0],16); plot2d(x[1],x[2],r=3.5,>points):
\end{eulerprompt}
\begin{eulercomment}
\begin{eulercomment}
\eulerheading{Kekonvergenan}
\begin{eulercomment}
Terkadang kita ingin melakukan iterasi sampai konvergen. Apabila iterasinya tidak konvergen
setelah ditunggu lama, Anda dapat menghentikannya dengan menekan tombol [ESC].
\end{eulercomment}
\begin{eulerprompt}
>iterate("cos(x)",1) // iterasi x(n+1)=cos(x(n)), dengan x(0)=1.
\end{eulerprompt}
\begin{euleroutput}
  0.739085133216
\end{euleroutput}
\begin{eulercomment}
Iterasi tersebut konvergen ke penyelesaian persamaan

\end{eulercomment}
\begin{eulerformula}
\[
x = \cos(x).
\]
\end{eulerformula}
\begin{eulercomment}
Iterasi ini juga dapat dilakukan pada interval, hasilnya adalah barisan interval yang memuat
akar tersebut.
\end{eulercomment}
\begin{eulerprompt}
>hasil := iterate("cos(x)",~1,2~) //iterasi x(n+1)=cos(x(n)), dengan interval awal (1, 2)
\end{eulerprompt}
\begin{euleroutput}
  ~0.739085133211,0.7390851332133~
\end{euleroutput}
\begin{eulercomment}
Jika interval hasil tersebut sedikit diperlebar, akan terlihat bahwa interval tersebut
memuat akar persamaan x=cos(x).
\end{eulercomment}
\begin{eulerprompt}
>h=expand(hasil,100), cos(h) << h
\end{eulerprompt}
\begin{euleroutput}
  ~0.73908513309,0.73908513333~
  1
\end{euleroutput}
\begin{eulercomment}
Iterasi juga dapat digunakan pada fungsi yang didefinisikan.
\end{eulercomment}
\begin{eulerprompt}
>function f(x) := (x+2/x)/2
\end{eulerprompt}
\begin{eulercomment}
Iterasi x(n+1)=f(x(n)) akan konvergen ke akar kuadrat 2.
\end{eulercomment}
\begin{eulerprompt}
>iterate("f",2), sqrt(2)
\end{eulerprompt}
\begin{euleroutput}
  1.41421356237
  1.41421356237
\end{euleroutput}
\begin{eulercomment}
Jika pada perintah iterate diberikan tambahan parameter n, maka hasil iterasinya akan
ditampilkan mulai dari iterasi pertama sampai ke-n.
\end{eulercomment}
\begin{eulerprompt}
>iterate("f",2,5)
\end{eulerprompt}
\begin{euleroutput}
  [2,  1.5,  1.41667,  1.41422,  1.41421,  1.41421]
\end{euleroutput}
\begin{eulercomment}
Untuk iterasi ini tidak dapat dilakukan terhadap interval.
\end{eulercomment}
\begin{eulerprompt}
>niterate("f",~1,2~,5)
\end{eulerprompt}
\begin{euleroutput}
  [ ~1,2~,  ~1,2~,  ~1,2~,  ~1,2~,  ~1,2~,  ~1,2~ ]
\end{euleroutput}
\begin{eulercomment}
Perhatikan, hasil iterasinya sama dengan interval awal. Alasannya adalah perhitungan dengan
interval bersifat terlalu longgar. Untuk meingkatkan perhitungan pada ekspresi dapat
digunakan pembagian intervalnya, menggunakan fungsi ieval().
\end{eulercomment}
\begin{eulerprompt}
>function s(x) := ieval("(x+2/x)/2",x,10)
\end{eulerprompt}
\begin{eulercomment}
Selanjutnya dapat dilakukan iterasi hingga diperoleh hasil optimal, dan intervalnya tidak
semakin mengecil. Hasilnya berupa interval yang memuat akar persamaan:

\end{eulercomment}
\begin{eulerformula}
\[
x = \frac{1}{2} \left( x + \frac{2}{x} \right).
\]
\end{eulerformula}
\begin{eulercomment}
Satu-satunya solusi adalah\\
\end{eulercomment}
\begin{eulerformula}
\[
x = \sqrt2.
\]
\end{eulerformula}
\begin{eulerprompt}
>iterate("s",~1,2~)
\end{eulerprompt}
\begin{euleroutput}
  ~1.41421356236,1.41421356239~
\end{euleroutput}
\begin{eulercomment}
Fungsi "iterate()" juga dapat bekerja pada vektor. Berikut adalah contoh fungsi vektor, yang
menghasilkan rata-rata aritmetika dan rata-rata geometri.

\end{eulercomment}
\begin{eulerformula}
\[
(a_{n+1},b_{n+1}) = \left( \frac{a_n+b_n}{2}, \sqrt{a_nb_n} \right)
\]
\end{eulerformula}
\begin{eulercomment}
Iterasi ke-n disimpan pada vektor kolom x[n].
\end{eulercomment}
\begin{eulerprompt}
>function g(x) := [(x[1]+x[2])/2;sqrt(x[1]*x[2])]
\end{eulerprompt}
\begin{eulercomment}
Iterasi dengan menggunakan fungsi tersebut akan konvergen ke rata-rata aritmetika dan
geometri dari nilai-nilai awal. 
\end{eulercomment}
\begin{eulerprompt}
>iterate("g",[1;5])
\end{eulerprompt}
\begin{euleroutput}
        2.60401 
        2.60401 
\end{euleroutput}
\begin{eulercomment}
Hasil tersebut konvergen agak cepat, seperti kita cek sebagai berikut.
\end{eulercomment}
\begin{eulerprompt}
>iterate("g",[1;5],4)
\end{eulerprompt}
\begin{euleroutput}
              1             3       2.61803       2.60403       2.60401 
              5       2.23607       2.59002       2.60399       2.60401 
\end{euleroutput}
\begin{eulercomment}
Iterasi pada interval dapat dilakukan dan stabil, namun tidak menunjukkan bahwa limitnya
pada batas-batas yang dihitung.
\end{eulercomment}
\begin{eulerprompt}
>iterate("g",[~1~;~5~],4)
\end{eulerprompt}
\begin{euleroutput}
  Interval 2 x 5 matrix
  
  ~0.999999999999999778,1.00000000000000022~     ...
  ~4.99999999999999911,5.00000000000000089~     ...
\end{euleroutput}
\begin{eulercomment}
Iterasi berikut konvergen sangat lambat.

\end{eulercomment}
\begin{eulerformula}
\[
x_{n+1} = \sqrt{x_n}.
\]
\end{eulerformula}
\begin{eulerprompt}
>iterate("sqrt(x)",2,10)
\end{eulerprompt}
\begin{euleroutput}
  [2,  1.41421,  1.18921,  1.09051,  1.04427,  1.0219,  1.01089,
  1.00543,  1.00271,  1.00135,  1.00068]
\end{euleroutput}
\begin{eulercomment}
Kekonvergenan iterasi tersebut dapat dipercepatdengan percepatan Steffenson:
\end{eulercomment}
\begin{eulerprompt}
>steffenson("sqrt(x)",2,10)
\end{eulerprompt}
\begin{euleroutput}
  [1.04888,  1.00028,  1,  1]
\end{euleroutput}
\eulerheading{Iterasi menggunakan Loop yang ditulis Langsung}
\begin{eulercomment}
Berikut adalah beberapa contoh penggunaan loop untuk melakukan iterasi yang ditulis langsung
pada baris perintah.
\end{eulercomment}
\begin{eulerprompt}
>x=2; repeat x=(x+2/x)/2; until x^2~=2; end; x,
\end{eulerprompt}
\begin{euleroutput}
  1.41421356237
\end{euleroutput}
\begin{eulercomment}
Penggabungan matriks menggunakan tanda "\textbar{}" dapat digunakan untuk menyimpan semua hasil
iterasi.
\end{eulercomment}
\begin{eulerprompt}
>v=[1]; for i=2 to 8; v=v|(v[i-1]*i); end; v,
\end{eulerprompt}
\begin{euleroutput}
  [1,  2,  6,  24,  120,  720,  5040,  40320]
\end{euleroutput}
\begin{eulercomment}
hasil iterasi juga dapat disimpan pada vektor yang sudah ada.
\end{eulercomment}
\begin{eulerprompt}
>v=ones(1,100); for i=2 to cols(v); v[i]=v[i-1]*i; end; ...
>plot2d(v,logplot=1); textbox(latex(&log(n)),x=0.5):
\end{eulerprompt}
\eulerimg{27}{images/EMT 1-6_Mifta Dilla Aninditya_23030630029-548.png}
\begin{eulerprompt}
>A =[0.5,0.2;0.7,0.1]; b=[2;2]; ...
>x=[1;1]; repeat xnew=A.x-b; until all(xnew~=x); x=xnew; end; ...
>x,
\end{eulerprompt}
\begin{euleroutput}
       -7.09677 
       -7.74194 
\end{euleroutput}
\eulerheading{Iterasi di dalam Fungsi}
\begin{eulercomment}
Fungsi atau program juga dapat menggunakan iterasi dan dapat digunakan untuk melakukan iterasi. Berikut adalah beberapa contoh
iterasi di dalam fungsi.

Contoh berikut adalah suatu fungsi untuk menghitung berapa lama suatu iterasi konvergen. Nilai fungsi tersebut adalah hasil akhir
iterasi dan banyak iterasi sampai konvergen.
\end{eulercomment}
\begin{eulerprompt}
>function map hiter(f$,x0) ...
\end{eulerprompt}
\begin{eulerudf}
  x=x0;
  maxiter=0;
  repeat
    xnew=f$(x);
    maxiter=maxiter+1;
    until xnew~=x;
    x=xnew;
  end;
  return maxiter;
  endfunction
\end{eulerudf}
\begin{eulercomment}
Misalnya, berikut adalah iterasi untuk mendapatkan hampiran akar kuadrat 2, cukup cepat,
konvergen pada iterasi ke-5, jika dimulai dari hampiran awal 2.
\end{eulercomment}
\begin{eulerprompt}
>hiter("(x+2/x)/2",2)
\end{eulerprompt}
\begin{euleroutput}
  5
\end{euleroutput}
\begin{eulercomment}
Karena fungsinya didefinisikan menggunakan "map". maka nilai awalnya dapat berupa vektor.
\end{eulercomment}
\begin{eulerprompt}
>x=1.5:0.1:10; hasil=hiter("(x+2/x)/2",x); ...
>  plot2d(x,hasil):
\end{eulerprompt}
\begin{eulercomment}
Dari gambar di atas terlihat bahwa kekonvergenan iterasinya semakin lambat, untuk nilai awal
semakin besar, namun penambahnnya tidak kontinu. Kita dapat menemukan kapan maksimum
iterasinya bertambah.
\end{eulercomment}
\begin{eulerprompt}
>hasil[1:10]
\end{eulerprompt}
\begin{euleroutput}
  hasil is not a variable!
  Error in:
  hasil[1:10] ...
             ^
\end{euleroutput}
\begin{eulerprompt}
>x[nonzeros(differences(hasil))]
\end{eulerprompt}
\begin{euleroutput}
  Variable or function hasil not found.
  Error in:
  x[nonzeros(differences(hasil))] ...
                              ^
\end{euleroutput}
\begin{eulercomment}
maksimum iterasi sampai konvergen meningkat pada saat nilai awalnya 1.5, 2, 3.4, dan 6.6.

Contoh berikutnya adalah metode Newton pada polinomial kompleks berderajat 3.
\end{eulercomment}
\begin{eulerprompt}
>p &= x^3-1; newton &= x-p/diff(p,x); $newton
\end{eulerprompt}
\begin{eulerformula}
\[
x-\frac{x^3-1}{3\,x^2}
\]
\end{eulerformula}
\begin{eulercomment}
Selanjutnya didefinisikan fungsi untuk melakukan iterasi (aslinya 10 kali).
\end{eulercomment}
\begin{eulerprompt}
>function iterasi(f$,x,n=10) ...
\end{eulerprompt}
\begin{eulerudf}
  loop 1 to n; x=f$(x); end;
  return x;
  endfunction
\end{eulerudf}
\begin{eulercomment}
Kita mulai dengan menentukan titik-titik grid pada bidang kompleksnya.
\end{eulercomment}
\begin{eulerprompt}
>r=1.5; x=linspace(-r,r,501); Z=x+I*x'; W=iterasi(newton,Z);
\end{eulerprompt}
\begin{eulercomment}
Berikut adalah akar-akar polinomial di atas.
\end{eulercomment}
\begin{eulerprompt}
>z=&solve(p)()
\end{eulerprompt}
\begin{euleroutput}
  [ -0.5+0.866025i,  -0.5-0.866025i,  1+0i  ]
\end{euleroutput}
\begin{eulercomment}
Untuk menggambar hasil iterasinya, dihitung jarak dari hasil iterasi ke-10 ke masing-masing
akar, kemudian digunakan untuk menghitung warna yang akan digambar, yang menunjukkan limit
untuk masing-masing nilai awal. 

Fungsi plotrgb() menggunakan jendela gambar terkini untuk menggambar warna RGB sebagai
matriks.
\end{eulercomment}
\begin{eulerprompt}
>C=rgb(max(abs(W-z[1]),1),max(abs(W-z[2]),1),max(abs(W-z[3]),1)); ...
>  plot2d(none,-r,r,-r,r); plotrgb(C):
\end{eulerprompt}
\eulerimg{27}{images/EMT 1-6_Mifta Dilla Aninditya_23030630029-550.png}
\eulerheading{Iterasi Simbolik}
\begin{eulercomment}
Seperti sudah dibahas sebelumnya, untuk menghasilkan barisan ekspresi simbolik dengan Maxima
dapat digunakan fungsi makelist().
\end{eulercomment}
\begin{eulerprompt}
>&powerdisp:true // untuk menampilkan deret pangkat mulai dari suku berpangkat terkecil
\end{eulerprompt}
\begin{euleroutput}
  
                                   true
  
\end{euleroutput}
\begin{eulerprompt}
>deret &= makelist(taylor(exp(x),x,0,k),k,1,3); $deret // barisan deret Taylor untuk e^x
\end{eulerprompt}
\begin{eulerformula}
\[
\left[ 1+x , 1+x+\frac{x^2}{2} , 1+x+\frac{x^2}{2}+\frac{x^3}{6}
  \right] 
\]
\end{eulerformula}
\begin{eulercomment}
Untuk mengubah barisan deret tersebut menjadi vektor string di EMT digunakan fungsi
mxm2str(). Selanjutnya, vektor string/ekspresi hasilnya dapat digambar seperti menggambar
vektor eskpresi pada EMT.
\end{eulercomment}
\begin{eulerprompt}
>plot2d("exp(x)",0,3); // plot fungsi aslinya, e^x
>plot2d(mxm2str("deret"),>add,color=4:6): // plot ketiga deret taylor hampiran fungsi tersebut
\end{eulerprompt}
\eulerimg{27}{images/EMT 1-6_Mifta Dilla Aninditya_23030630029-552.png}
\begin{eulercomment}
Selain cara di atas dapat juga dengan cara menggunakan indeks pada vektor/list yang
dihasilkan.
\end{eulercomment}
\begin{eulerprompt}
>$deret[3]
\end{eulerprompt}
\begin{eulerformula}
\[
1+x+\frac{x^2}{2}+\frac{x^3}{6}
\]
\end{eulerformula}
\begin{eulerprompt}
>plot2d(["exp(x)",&deret[1],&deret[2],&deret[3]],0,3,color=1:4):
\end{eulerprompt}
\eulerimg{27}{images/EMT 1-6_Mifta Dilla Aninditya_23030630029-554.png}
\begin{eulerprompt}
>$sum(sin(k*x)/k,k,1,5)
\end{eulerprompt}
\begin{eulerformula}
\[
\sin x+\frac{\sin \left(2\,x\right)}{2}+\frac{\sin \left(3\,x
 \right)}{3}+\frac{\sin \left(4\,x\right)}{4}+\frac{\sin \left(5\,x
 \right)}{5}
\]
\end{eulerformula}
\begin{eulercomment}
Berikut adalah cara menggambar kurva

\end{eulercomment}
\begin{eulerformula}
\[
y=\sin(x) + \dfrac{\sin 3x}{3} + \dfrac{\sin 5x}{5} + \ldots.
\]
\end{eulerformula}
\begin{eulerprompt}
>plot2d(&sum(sin((2*k+1)*x)/(2*k+1),k,0,20),0,2pi):
\end{eulerprompt}
\eulerimg{27}{images/EMT 1-6_Mifta Dilla Aninditya_23030630029-556.png}
\begin{eulercomment}
Hal serupa juga dapat dilakukan dengan menggunakan matriks, misalkan kita akan menggambar
kurva

\end{eulercomment}
\begin{eulerformula}
\[
y = \sum_{k=1}^{100} \dfrac{\sin(kx)}{k},\quad 0\le x\le 2\pi.
\]
\end{eulerformula}
\begin{eulercomment}
\end{eulercomment}
\begin{eulerprompt}
>x=linspace(0,2pi,1000); k=1:100; y=sum(sin(k*x')/k)'; plot2d(x,y):
\end{eulerprompt}
\eulerimg{27}{images/EMT 1-6_Mifta Dilla Aninditya_23030630029-557.png}
\eulerheading{Tabel Fungsi}
\begin{eulercomment}
Terdapat cara menarik untuk menghasilkan barisan dengan ekspresi Maxima. Perintah
mxmtable() berguna untuk menampilkan dan menggambar barisan dan menghasilkan barisan sebagai
vektor kolom. 

Sebagai contoh berikut adalah barisan turunan ke-n x\textasciicircum{}x di x=1.
\end{eulercomment}
\begin{eulerprompt}
>mxmtable("diffat(x^x,x=1,n)","n",1,8,frac=1);
\end{eulerprompt}
\begin{euleroutput}
          1 
          2 
          3 
          8 
         10 
         54 
        -42 
        944 
\end{euleroutput}
\eulerimg{27}{images/EMT 1-6_Mifta Dilla Aninditya_23030630029-558.png}
\begin{eulerprompt}
>$'sum(k, k, 1, n) = factor(ev(sum(k, k, 1, n),simpsum=true)) // simpsum:menghitung deret secara simbolik
\end{eulerprompt}
\begin{eulerformula}
\[
\sum_{k=1}^{n}{k}=\frac{n\,\left(1+n\right)}{2}
\]
\end{eulerformula}
\begin{eulerprompt}
>$'sum(1/(3^k+k), k, 0, inf) = factor(ev(sum(1/(3^k+k), k, 0, inf),simpsum=true))
\end{eulerprompt}
\begin{eulerformula}
\[
\sum_{k=0}^{\infty }{\frac{1}{k+3^{k}}}=\sum_{k=0}^{\infty }{\frac{
 1}{k+3^{k}}}
\]
\end{eulerformula}
\begin{eulercomment}
Di sini masih gagal, hasilnya tidak dihitung.
\end{eulercomment}
\begin{eulerprompt}
>$'sum(1/x^2, x, 1, inf)= ev(sum(1/x^2, x, 1, inf),simpsum=true) // ev: menghitung nilai ekspresi
\end{eulerprompt}
\begin{eulerformula}
\[
\sum_{x=1}^{\infty }{\frac{1}{x^2}}=\frac{\pi^2}{6}
\]
\end{eulerformula}
\begin{eulerprompt}
>$'sum((-1)^(k-1)/k, k, 1, inf) = factor(ev(sum((-1)^(x-1)/x, x, 1, inf),simpsum=true))
\end{eulerprompt}
\begin{eulerformula}
\[
\sum_{k=1}^{\infty }{\frac{\left(-1\right)^{-1+k}}{k}}=-\sum_{x=1
 }^{\infty }{\frac{\left(-1\right)^{x}}{x}}
\]
\end{eulerformula}
\begin{eulercomment}
Di sini masih gagal, hasilnya tidak dihitung.
\end{eulercomment}
\begin{eulerprompt}
>$'sum((-1)^k/(2*k-1), k, 1, inf) = factor(ev(sum((-1)^k/(2*k-1), k, 1, inf),simpsum=true))
\end{eulerprompt}
\begin{eulerformula}
\[
\sum_{k=1}^{\infty }{\frac{\left(-1\right)^{k}}{-1+2\,k}}=\sum_{k=1
 }^{\infty }{\frac{\left(-1\right)^{k}}{-1+2\,k}}
\]
\end{eulerformula}
\begin{eulerprompt}
>$ev(sum(1/n!, n, 0, inf),simpsum=true)
\end{eulerprompt}
\begin{eulerformula}
\[
\sum_{n=0}^{\infty }{\frac{1}{n!}}
\]
\end{eulerformula}
\begin{eulercomment}
Di sini masih gagal, hasilnya tidak dihitung, harusnya hasilnya e.
\end{eulercomment}
\begin{eulerprompt}
>&assume(abs(x)<1); $'sum(a*x^k, k, 0, inf)=ev(sum(a*x^k, k, 0, inf),simpsum=true), &forget(abs(x)<1);
\end{eulerprompt}
\begin{eulerformula}
\[
a\,\sum_{k=0}^{\infty }{x^{k}}=\frac{a}{1-x}
\]
\end{eulerformula}
\begin{eulercomment}
Deret geometri tak hingga, dengan asumsi rasional antara -1 dan 1.
\end{eulercomment}
\begin{eulerprompt}
>$'sum(x^k/k!,k,0,inf)=ev(sum(x^k/k!,k,0,inf),simpsum=true)
\end{eulerprompt}
\begin{eulerformula}
\[
\sum_{k=0}^{\infty }{\frac{x^{k}}{k!}}=\sum_{k=0}^{\infty }{\frac{x
 ^{k}}{k!}}
\]
\end{eulerformula}
\begin{eulerprompt}
>$limit(sum(x^k/k!,k,0,n),n,inf)
\end{eulerprompt}
\begin{eulerformula}
\[
\lim_{n\rightarrow \infty }{\sum_{k=0}^{n}{\frac{x^{k}}{k!}}}
\]
\end{eulerformula}
\begin{eulerprompt}
>function d(n) &= sum(1/(k^2-k),k,2,n); $'d(n)=d(n)
\end{eulerprompt}
\begin{eulerformula}
\[
d\left(n\right)=\sum_{k=2}^{n}{\frac{1}{-k+k^2}}
\]
\end{eulerformula}
\begin{eulerprompt}
>$d(10)=ev(d(10),simpsum=true)
\end{eulerprompt}
\begin{eulerformula}
\[
\sum_{k=2}^{10}{\frac{1}{-k+k^2}}=\frac{9}{10}
\]
\end{eulerformula}
\begin{eulerprompt}
>$d(100)=ev(d(100),simpsum=true)
\end{eulerprompt}
\begin{eulerformula}
\[
\sum_{k=2}^{100}{\frac{1}{-k+k^2}}=\frac{99}{100}
\]
\end{eulerformula}
\eulerheading{Deret Taylor}
\begin{eulercomment}
Deret Taylor suatu fungsi f yang diferensiabel sampai tak hingga di sekitar x=a adalah:

\end{eulercomment}
\begin{eulerformula}
\[
f(x) = \sum_{k=0}^\infty \frac{(x-a)^k f^{(k)}(a)}{k!}.
\]
\end{eulerformula}
\begin{eulerprompt}
>$'e^x =taylor(exp(x),x,0,10) // deret Taylor e^x di sekitar x=0, sampai suku ke-11
\end{eulerprompt}
\begin{eulerformula}
\[
e^{x}=1+x+\frac{x^2}{2}+\frac{x^3}{6}+\frac{x^4}{24}+\frac{x^5}{120
 }+\frac{x^6}{720}+\frac{x^7}{5040}+\frac{x^8}{40320}+\frac{x^9}{
 362880}+\frac{x^{10}}{3628800}
\]
\end{eulerformula}
\begin{eulerprompt}
>$'log(x)=taylor(log(x),x,1,10)// deret log(x) di sekitar x=1
\end{eulerprompt}
\begin{eulerformula}
\[
\log x=-1-\frac{\left(-1+x\right)^2}{2}+\frac{\left(-1+x\right)^3}{
 3}-\frac{\left(-1+x\right)^4}{4}+\frac{\left(-1+x\right)^5}{5}-
 \frac{\left(-1+x\right)^6}{6}+\frac{\left(-1+x\right)^7}{7}-\frac{
 \left(-1+x\right)^8}{8}+\frac{\left(-1+x\right)^9}{9}-\frac{\left(-1
 +x\right)^{10}}{10}+x
\]
\end{eulerformula}
\eulerheading{Soal}
\begin{eulerprompt}
>&load(fourier_elim)
\end{eulerprompt}
\begin{euleroutput}
  
          C:/Program Files/Euler x64/maxima/share/maxima/5.35.1/share/f\(\backslash\)
  ourier_elim/fourier_elim.lisp
  
\end{euleroutput}
\begin{eulerprompt}
>$showev('limit(sin(2*x)/x,x,0))
\end{eulerprompt}
\begin{euleroutput}
  Maxima said:
  expt: undefined: 0 to a negative exponent.
   -- an error. To debug this try: debugmode(true);
  
  Error in:
                                 ^
\end{euleroutput}
\begin{eulerprompt}
>plot2d("sin(2*x)/x",-pi,pi); plot2d(0,2,>points,style="ow",>add):
\end{eulerprompt}
\begin{euleroutput}
  
\end{euleroutput}
\eulersubheading{}
\begin{eulercomment}
\begin{eulercomment}
\eulerheading{Visualisasi dan Perhitungan Geometri dengan EMT}
\begin{eulercomment}
Euler menyediakan beberapa fungsi untuk melakukan visualisasi dan
perhitungan geometri, baik secara numerik maupun analitik (seperti
biasanya tentunya, menggunakan Maxima). Fungsi-fungsi untuk
visualisasi dan perhitungan geometeri tersebut disimpan di dalam file
program "geometry.e", sehingga file tersebut harus dipanggil sebelum
menggunakan fungsi-fungsi atau perintah-perintah untuk geometri.
\end{eulercomment}
\begin{eulerprompt}
>load geometry
\end{eulerprompt}
\begin{euleroutput}
  Numerical and symbolic geometry.
\end{euleroutput}
\eulersubheading{Fungsi-fungsi Geometri}
\begin{eulercomment}
Fungsi-fungsi untuk Menggambar Objek Geometri:

\end{eulercomment}
\begin{eulerttcomment}
  defaultd:=textheight()*1.5: nilai asli untuk parameter d
  setPlotrange(x1,x2,y1,y2): menentukan rentang x dan y pada bidang
\end{eulerttcomment}
\begin{eulercomment}
koordinat\\
\end{eulercomment}
\begin{eulerttcomment}
  setPlotRange(r): pusat bidang koordinat (0,0) dan batas-batas
\end{eulerttcomment}
\begin{eulercomment}
sumbu-x dan y adalah -r sd r\\
\end{eulercomment}
\begin{eulerttcomment}
  plotPoint (P, "P"): menggambar titik P dan diberi label "P"
  plotSegment (A,B, "AB", d): menggambar ruas garis AB, diberi label
\end{eulerttcomment}
\begin{eulercomment}
"AB" sejauh d\\
\end{eulercomment}
\begin{eulerttcomment}
  plotLine (g, "g", d): menggambar garis g diberi label "g" sejauh d
  plotCircle (c,"c",v,d): Menggambar lingkaran c dan diberi label "c"
  plotLabel (label, P, V, d): menuliskan label pada posisi P
\end{eulerttcomment}
\begin{eulercomment}

Fungsi-fungsi Geometri Analitik (numerik maupun simbolik):

\end{eulercomment}
\begin{eulerttcomment}
  turn(v, phi): memutar vektor v sejauh phi
  turnLeft(v):   memutar vektor v ke kiri
  turnRight(v):  memutar vektor v ke kanan
  normalize(v): normal vektor v
  crossProduct(v, w): hasil kali silang vektorv dan w.
  lineThrough(A, B): garis melalui A dan B, hasilnya [a,b,c] sdh.
\end{eulerttcomment}
\begin{eulercomment}
ax+by=c.\\
\end{eulercomment}
\begin{eulerttcomment}
  lineWithDirection(A,v): garis melalui A searah vektor v
  getLineDirection(g): vektor arah (gradien) garis g
  getNormal(g): vektor normal (tegak lurus) garis g
  getPointOnLine(g):  titik pada garis g
  perpendicular(A, g):  garis melalui A tegak lurus garis g
  parallel (A, g):  garis melalui A sejajar garis g
  lineIntersection(g, h):  titik potong garis g dan h
  projectToLine(A, g):   proyeksi titik A pada garis g
  distance(A, B):  jarak titik A dan B
  distanceSquared(A, B):  kuadrat jarak A dan B
  quadrance(A, B): kuadrat jarak A dan B
  areaTriangle(A, B, C):  luas segitiga ABC
  computeAngle(A, B, C):   besar sudut <ABC
  angleBisector(A, B, C): garis bagi sudut <ABC
  circleWithCenter (A, r): lingkaran dengan pusat A dan jari-jari r
  getCircleCenter(c):  pusat lingkaran c
  getCircleRadius(c):  jari-jari lingkaran c
  circleThrough(A,B,C):  lingkaran melalui A, B, C
  middlePerpendicular(A, B): titik tengah AB
  lineCircleIntersections(g, c): titik potong garis g dan lingkran c
  circleCircleIntersections (c1, c2):  titik potong lingkaran c1 dan
\end{eulerttcomment}
\begin{eulercomment}
c2\\
\end{eulercomment}
\begin{eulerttcomment}
  planeThrough(A, B, C):  bidang melalui titik A, B, C
\end{eulerttcomment}
\begin{eulercomment}

Fungsi-fungsi Khusus Untuk Geometri Simbolik:

\end{eulercomment}
\begin{eulerttcomment}
  getLineEquation (g,x,y): persamaan garis g dinyatakan dalam x dan y
  getHesseForm (g,x,y,A): bentuk Hesse garis g dinyatakan dalam x dan
\end{eulerttcomment}
\begin{eulercomment}
y dengan titik A pada\\
\end{eulercomment}
\begin{eulerttcomment}
  sisi positif (kanan/atas) garis
  quad(A,B): kuadrat jarak AB
  spread(a,b,c): Spread segitiga dengan panjang sisi-sisi a,b,c, yakni
\end{eulerttcomment}
\begin{eulercomment}
sin(alpha)\textasciicircum{}2 dengan\\
\end{eulercomment}
\begin{eulerttcomment}
  alpha sudut yang menghadap sisi a.
  crosslaw(a,b,c,sa): persamaan 3 quads dan 1 spread pada segitiga
\end{eulerttcomment}
\begin{eulercomment}
dengan panjang sisi a, b, c.\\
\end{eulercomment}
\begin{eulerttcomment}
  triplespread(sa,sb,sc): persamaan 3 spread sa,sb,sc yang memebntuk
\end{eulerttcomment}
\begin{eulercomment}
suatu segitiga\\
\end{eulercomment}
\begin{eulerttcomment}
  doublespread(sa): Spread sudut rangkap Spread 2*phi, dengan
\end{eulerttcomment}
\begin{eulercomment}
sa=sin(phi)\textasciicircum{}2 spread a.

\end{eulercomment}
\eulersubheading{Contoh 1: Luas, Lingkaran Luar, Lingkaran Dalam Segitiga}
\begin{eulercomment}
Untuk menggambar objek-objek geometri, langkah pertama adalah
menentukan rentang sumbu-sumbu koordinat. Semua objek geometri akan
digambar pada satu bidang koordinat, sampai didefinisikan bidang
koordinat yang baru.
\end{eulercomment}
\begin{eulerprompt}
>setPlotRange(-0.5,2.5,-0.5,2.5); // mendefinisikan bidang koordinat baru 
\end{eulerprompt}
\begin{eulercomment}
Sekarang, tetapkan tiga titik dan plotlah.
\end{eulercomment}
\begin{eulerprompt}
>A=[1,0]; plotPoint(A,"A"); // definisi dan gambar tiga titik
>B=[0,1]; plotPoint(B,"B");
>C=[2,2]; plotPoint(C,"C");
\end{eulerprompt}
\begin{eulercomment}
Kemudian tiga segmen.
\end{eulercomment}
\begin{eulerprompt}
>plotSegment(A,B,"c"); // c=AB
>plotSegment(B,C,"a"); // a=BC
>plotSegment(A,C,"b"); // b=AC
\end{eulerprompt}
\begin{eulercomment}
Fungsi geometri mencakup fungsi untuk membuat garis dan lingkaran.
Format untuk garis adalah [a,b,c], yang merepresentasikan garis dengan
persamaan ax+by=c.
\end{eulercomment}
\begin{eulerprompt}
>lineThrough(B,C) // garis yang melalui B dan C
\end{eulerprompt}
\begin{euleroutput}
  [-1,  2,  2]
\end{euleroutput}
\begin{eulercomment}
Hitung garis tegak lurus yang melalui A pada BC.
\end{eulercomment}
\begin{eulerprompt}
>h=perpendicular(A,lineThrough(B,C)); // garis h tegak lurus BC melalui A
\end{eulerprompt}
\begin{eulercomment}
Dan persimpangannya dengan BC.
\end{eulercomment}
\begin{eulerprompt}
>D=lineIntersection(h,lineThrough(B,C)); // D adalah titik potong h dan BC
\end{eulerprompt}
\begin{eulercomment}
Plot that.
\end{eulercomment}
\begin{eulerprompt}
>plotPoint(D,value=1); // koordinat D ditampilkan
>aspect(1); plotSegment(A,D): // tampilkan semua gambar hasil plot...()
\end{eulerprompt}
\begin{eulercomment}
Hitung luas ABC:

\end{eulercomment}
\begin{eulerformula}
\[
L_{\triangle ABC}= \frac{1}{2}AD.BC.
\]
\end{eulerformula}
\begin{eulerprompt}
>norm(A-D)*norm(B-C)/2 // AD=norm(A-D), BC=norm(B-C)
\end{eulerprompt}
\begin{euleroutput}
  1.5
\end{euleroutput}
\begin{eulercomment}
Bandingkan dengan rumus determinan.
\end{eulercomment}
\begin{eulerprompt}
>areaTriangle(A,B,C) // hitung luas segitiga langusng dengan fungsi
\end{eulerprompt}
\begin{euleroutput}
  1.5
\end{euleroutput}
\begin{eulercomment}
Cara lain menghitung luas segitigas ABC:
\end{eulercomment}
\begin{eulerprompt}
>distance(A,D)*distance(B,C)/2
\end{eulerprompt}
\begin{euleroutput}
  1.5
\end{euleroutput}
\begin{eulercomment}
Sudut pada C.
\end{eulercomment}
\begin{eulerprompt}
>degprint(computeAngle(B,C,A))
\end{eulerprompt}
\begin{euleroutput}
  36°52'11.63''
\end{euleroutput}
\begin{eulercomment}
Sekarang, lingkarilah segitiga tersebut.
\end{eulercomment}
\begin{eulerprompt}
>c=circleThrough(A,B,C); // lingkaran luar segitiga ABC
>R=getCircleRadius(c); // jari2 lingkaran luar 
>O=getCircleCenter(c); // titik pusat lingkaran c 
>plotPoint(O,"O"); // gambar titik "O"
>plotCircle(c,"Lingkaran luar segitiga ABC"):
\end{eulerprompt}
\begin{eulercomment}
Tampilkan koordinat titik pusat dan jari-jari lingkaran luar.
\end{eulercomment}
\begin{eulerprompt}
>O, R
\end{eulerprompt}
\begin{euleroutput}
  [1.16667,  1.16667]
  1.17851130198
\end{euleroutput}
\begin{eulercomment}
Sekarang akan digambar lingkaran dalam segitiga ABC. Titik pusat lingkaran dalam adalah
titik potong garis-garis bagi sudut.
\end{eulercomment}
\begin{eulerprompt}
>l=angleBisector(A,C,B); // garis bagi <ACB
>g=angleBisector(C,A,B); // garis bagi <CAB
>P=lineIntersection(l,g) // titik potong kedua garis bagi sudut
\end{eulerprompt}
\begin{euleroutput}
  [0.86038,  0.86038]
\end{euleroutput}
\begin{eulercomment}
Tambahkan semuanya ke plot.
\end{eulercomment}
\begin{eulerprompt}
>color(5); plotLine(l); plotLine(g); color(1); // gambar kedua garis bagi sudut
>plotPoint(P,"P"); // gambar titik potongnya
>r=norm(P-projectToLine(P,lineThrough(A,B))) // jari-jari lingkaran dalam
\end{eulerprompt}
\begin{euleroutput}
  0.509653732104
\end{euleroutput}
\begin{eulerprompt}
>plotCircle(circleWithCenter(P,r),"Lingkaran dalam segitiga ABC"): // gambar lingkaran dalam
\end{eulerprompt}
\eulersubheading{Latihan}
\begin{eulercomment}
1. Tentukan ketiga titik singgung lingkaran dalam dengan sisi-sisi
segitiga ABC.
\end{eulercomment}
\begin{eulerprompt}
>setPlotRange(-2.5,4.5,-2.5,4.5)
\end{eulerprompt}
\begin{euleroutput}
  [-2.5,  4.5,  -2.5,  4.5]
\end{euleroutput}
\begin{eulerprompt}
>A=[-2,4]; plotPoint(A,"A");
>B=[1,2]; plotPoint(B,"B");
>C=[3,4]; plotPoint(C,"C");
\end{eulerprompt}
\begin{eulercomment}
2. Gambar segitiga dengan titik-titik sudut ketiga titik singgung
tersebut. Merupakan segitiga apakah itu?
\end{eulercomment}
\begin{eulerprompt}
>plotSegment(A,B,"c")
>plotSegment(B,C,"a")
>plotSegment(A,C,"b")
\end{eulerprompt}
\begin{eulercomment}
3. Hitung luas segitiga tersebut.
\end{eulercomment}
\begin{eulerprompt}
>l=angleBisector(A,C,B);
>g=angleBisector(C,A,B);
>P=lineIntersection(l,g)
\end{eulerprompt}
\begin{euleroutput}
  [0.888562,  3.12541]
\end{euleroutput}
\begin{eulerprompt}
>color(5); plotLine(l); plotLine(g); color(1);
>plotPoint(P,"P");
>r=norm(P-projectToLine(P,lineThrough(A,B)))
\end{eulerprompt}
\begin{euleroutput}
  0.874586224495
\end{euleroutput}
\begin{eulerprompt}
>plotCircle(circleWithCenter(P,r),"Lingkaran dalam segitiga ABC"):
\end{eulerprompt}
\begin{eulercomment}
4. Tunjukkan bahwa garis bagi sudut yang ke tiga juga melalui titik
pusat lingkaran dalam.
\end{eulercomment}
\begin{eulerprompt}
>r=norm(P-projectToLine(P,lineThrough(A,B)))
\end{eulerprompt}
\begin{euleroutput}
  0.874586224495
\end{euleroutput}
\begin{eulerprompt}
>plotCircle(circleWithCenter(P,r),"Lingkaran dalam segitiga ABC"):
\end{eulerprompt}
\begin{eulercomment}
5. Gambar jari-jari lingkaran dalam.
\end{eulercomment}
\begin{eulercomment}
6. Hitung luas lingkaran luar dan luas lingkaran dalam segitiga ABC.
Adakah hubungan antara luas kedua lingkaran tersebut dengan luas
segitiga ABC?
\end{eulercomment}
\eulerheading{Contoh 2: Geometri Smbolik}
\begin{eulercomment}
Kita dapat menghitung geometri eksak dan simbolik menggunakan Maxima.


File geometry.e menyediakan fungsi-fungsi yang sama (dan lebih banyak
lagi) di Maxima. Namun, kita dapat menggunakan komputasi simbolik
sekarang.
\end{eulercomment}
\begin{eulerprompt}
>A &= [1,0]; B &= [0,1]; C &= [2,2]; // menentukan tiga titik A, B, C
\end{eulerprompt}
\begin{eulercomment}
Fungsi untuk garis dan lingkaran bekerja seperti fungsi Euler, tetapi
menyediakan komputasi simbolis.
\end{eulercomment}
\begin{eulerprompt}
>c &= lineThrough(B,C) // c=BC
\end{eulerprompt}
\begin{euleroutput}
  
                               [- 1, 2, 2]
  
\end{euleroutput}
\begin{eulercomment}
Kita bisa mendapatkan persamaan untuk sebuah garis dengan mudah.
\end{eulercomment}
\begin{eulerprompt}
>$getLineEquation(c,x,y), $solve(%,y) | expand // persamaan garis c
>$getLineEquation(lineThrough([x1,y1],[x2,y2]),x,y), $solve(%,y) // persamaan garis melalui(x1, y1) dan (x2, y2)
>$getLineEquation(lineThrough(A,[x1,y1]),x,y) // persamaan garis melalui A dan (x1, y1)
>h &= perpendicular(A,lineThrough(B,C)) // h melalui A tegak lurus BC
\end{eulerprompt}
\begin{euleroutput}
  
                                [2, 1, 2]
  
\end{euleroutput}
\begin{eulerprompt}
>Q &= lineIntersection(c,h) // Q titik potong garis c=BC dan h
\end{eulerprompt}
\begin{euleroutput}
  
                                   2  6
                                  [-, -]
                                   5  5
  
\end{euleroutput}
\begin{eulerprompt}
>$projectToLine(A,lineThrough(B,C)) // proyeksi A pada BC
>$distance(A,Q) // jarak AQ
>cc &= circleThrough(A,B,C); $cc // (titik pusat dan jari-jari) lingkaran melalui A, B, C
>r&=getCircleRadius(cc); $r , $float(r) // tampilkan nilai jari-jari
>$computeAngle(A,C,B) // nilai <ACB
>$solve(getLineEquation(angleBisector(A,C,B),x,y),y)[1] // persamaan garis bagi <ACB
>P &= lineIntersection(angleBisector(A,C,B),angleBisector(C,B,A)); $P // titik potong 2 garis bagi sudut
>P() // hasilnya sama dengan perhitungan sebelumnya
\end{eulerprompt}
\begin{euleroutput}
  [0.86038,  0.86038]
\end{euleroutput}
\eulersubheading{Garis dan Lingkaran yang Berpotongan}
\begin{eulercomment}
Tentu saja, kita juga bisa memotong garis dengan lingkaran, dan
lingkaran dengan lingkaran.
\end{eulercomment}
\begin{eulerprompt}
>A &:= [1,0]; c=circleWithCenter(A,4);
>B &:= [1,2]; C &:= [2,1]; l=lineThrough(B,C);
>setPlotRange(5); plotCircle(c); plotLine(l);
\end{eulerprompt}
\begin{eulercomment}
Perpotongan garis dengan lingkaran menghasilkan dua titik dan jumlah
titik perpotongan.
\end{eulercomment}
\begin{eulerprompt}
>\{P1,P2,f\}=lineCircleIntersections(l,c);
>P1, P2, f
\end{eulerprompt}
\begin{euleroutput}
  [4.64575,  -1.64575]
  [-0.645751,  3.64575]
  2
\end{euleroutput}
\begin{eulerprompt}
>plotPoint(P1); plotPoint(P2):
\end{eulerprompt}
\begin{eulercomment}
Hal yang sama pada Maxima.
\end{eulercomment}
\begin{eulerprompt}
>c &= circleWithCenter(A,4) // lingkaran dengan pusat A jari-jari 4
\end{eulerprompt}
\begin{euleroutput}
  
                                [1, 0, 4]
  
\end{euleroutput}
\begin{eulerprompt}
>l &= lineThrough(B,C) // garis l melalui B dan C
\end{eulerprompt}
\begin{euleroutput}
  
                                [1, 1, 3]
  
\end{euleroutput}
\begin{eulerprompt}
>$lineCircleIntersections(l,c) | radcan, // titik potong lingkaran c dan garis l
\end{eulerprompt}
\begin{eulercomment}
Akan ditunjukkan bahwa sudut-sudut yang menghadap bsuusr yang sama adalah sama besar.
\end{eulercomment}
\begin{eulerprompt}
>C=A+normalize([-2,-3])*4; plotPoint(C); plotSegment(P1,C); plotSegment(P2,C);
>degprint(computeAngle(P1,C,P2))
\end{eulerprompt}
\begin{euleroutput}
  69°17'42.68''
\end{euleroutput}
\begin{eulerprompt}
>C=A+normalize([-4,-3])*4; plotPoint(C); plotSegment(P1,C); plotSegment(P2,C);
>degprint(computeAngle(P1,C,P2))
\end{eulerprompt}
\begin{euleroutput}
  69°17'42.68''
\end{euleroutput}
\begin{eulerprompt}
>insimg;
\end{eulerprompt}
\eulersubheading{Garis Sumbu}
\begin{eulercomment}
Berikut adalah langkah-langkah menggambar garis sumbu ruas garis AB:

1. Gambar lingkaran dengan pusat A melalui B.\\
2. Gambar lingkaran dengan pusat B melalui A.\\
3. Tarik garis melallui kedua titik potong kedua lingkaran tersebut. Garis ini merupakan
garis sumbu (melalui titik tengah dan tegak lurus) AB.
\end{eulercomment}
\begin{eulerprompt}
>A=[2,2]; B=[-1,-2];
>c1=circleWithCenter(A,distance(A,B));
>c2=circleWithCenter(B,distance(A,B));
>\{P1,P2,f\}=circleCircleIntersections(c1,c2);
>l=lineThrough(P1,P2);
>setPlotRange(5); plotCircle(c1); plotCircle(c2);
>plotPoint(A); plotPoint(B); plotSegment(A,B); plotLine(l):
\end{eulerprompt}
\begin{eulercomment}
Selanjutnya, kami melakukan hal yang sama di Maxima dengan koordinat
umum.
\end{eulercomment}
\begin{eulerprompt}
>A &= [a1,a2]; B &= [b1,b2];
>c1 &= circleWithCenter(A,distance(A,B));
>c2 &= circleWithCenter(B,distance(A,B));
>P &= circleCircleIntersections(c1,c2); P1 &= P[1]; P2 &= P[2];
\end{eulerprompt}
\begin{eulercomment}
Persamaan untuk persimpangan cukup rumit. Tetapi kita dapat
menyederhanakannya, jika kita menyelesaikan untuk y.
\end{eulercomment}
\begin{eulerprompt}
>g &= getLineEquation(lineThrough(P1,P2),x,y);
>$solve(g,y)
\end{eulerprompt}
\begin{eulercomment}
Ini memang sama dengan tegak lurus tengah, yang dihitung dengan cara
yang sama sekali berbeda.
\end{eulercomment}
\begin{eulerprompt}
>$solve(getLineEquation(middlePerpendicular(A,B),x,y),y)
>h &=getLineEquation(lineThrough(A,B),x,y);
>$solve(h,y)
\end{eulerprompt}
\begin{eulercomment}
Perhatikan hasil kali gradien garis g dan h adalah:

\end{eulercomment}
\begin{eulerformula}
\[
\frac{-(b_1-a_1)}{(b_2-a_2)}\times \frac{(b_2-a_2)}{(b_1-a_1)} = -1.
\]
\end{eulerformula}
\begin{eulercomment}
Artinya kedua garis tegak lurus.
\end{eulercomment}
\eulerheading{Contoh 3: Rumus Heron}
\begin{eulercomment}
Rumus Heron menyatakan bahwa luas segitiga dengan panjang sisi-sisi a,
b dan c adalah:

\end{eulercomment}
\begin{eulerformula}
\[
L = \sqrt{s(s-a)(s-b)(s-c)}\quad \text{ dengan } s=(a+b+c)/2,
\]
\end{eulerformula}
\begin{eulercomment}
atau bisa ditulis dalam bentuk lain:

\end{eulercomment}
\begin{eulerformula}
\[
L = \frac{1}{4}\sqrt{(a+b+c)(b+c-a)(a+c-b)(a+b-c)}
\]
\end{eulerformula}
\begin{eulercomment}
Untuk membuktikan hal ini kita misalkan C(0,0), B(a,0) dan A(x,y),
b=AC, c=AB. Luas segitiga ABC adalah

\end{eulercomment}
\begin{eulerformula}
\[
L_{\triangle ABC}=\frac{1}{2}a\times y.
\]
\end{eulerformula}
\begin{eulercomment}
Nilai y didapat dengan menyelesaikan sistem persamaan:

\end{eulercomment}
\begin{eulerformula}
\[
x^2+y^2=b^2, \quad (x-a)^2+y^2=c^2.
\]
\end{eulerformula}
\begin{eulerprompt}
>setPlotRange(-1,10,-1,8); plotPoint([0,0], "C(0,0)"); plotPoint([5.5,0], "B(a,0)");  ...
> plotPoint([7.5,6], "A(x,y)");
\end{eulerprompt}
\begin{euleroutput}
  Function setPlotRange not found.
  Try list ... to find functions!
  Error in:
  setPlotRange(-1,10,-1,8); plotPoint([0,0], "C(0,0)"); plotPoin ...
                          ^
\end{euleroutput}
\begin{eulerprompt}
>plotSegment([0,0],[5.5,0], "a",25); plotSegment([5.5,0],[7.5,6],"c",15);  ...
>plotSegment([0,0],[7.5,6],"b",25); 
\end{eulerprompt}
\begin{euleroutput}
  Function plotSegment not found.
  Try list ... to find functions!
  Error in:
  plotSegment([0,0],[5.5,0], "a",25); plotSegment([5.5,0],[7.5,6 ...
                                    ^
\end{euleroutput}
\begin{eulerprompt}
>plotSegment([7.5,6],[7.5,0],"t=y",25):
>&assume(a>0); sol &= solve([x^2+y^2=b^2,(x-a)^2+y^2=c^2],[x,y])
\end{eulerprompt}
\begin{euleroutput}
  
                                    []
  
\end{euleroutput}
\begin{eulercomment}
Extract the solution y.
\end{eulercomment}
\begin{eulerprompt}
>ysol &= y with sol[2][2]; $'y=sqrt(factor(ysol^2))
\end{eulerprompt}
\begin{euleroutput}
  Maxima said:
  part: invalid index of list or matrix.
   -- an error. To debug this try: debugmode(true);
  
  Error in:
  ysol &= y with sol[2][2]; $'y=sqrt(factor(ysol^2)) ...
                          ^
\end{euleroutput}
\begin{eulercomment}
We get the Heron formula.
\end{eulercomment}
\begin{eulerprompt}
>function H(a,b,c) &= sqrt(factor((ysol*a/2)^2)); $'H(a,b,c)=H(a,b,c)
>$'Luas=H(2,5,6) // luas segitiga dengan panjang sisi-sisi 2, 5, 6
\end{eulerprompt}
\begin{eulercomment}
Tentu saja, setiap segitiga persegi panjang adalah kasus yang
terkenal.
\end{eulercomment}
\begin{eulerprompt}
>$H(3,4,5) //luas segitiga siku-siku dengan panjang sisi 3, 4, 5
\end{eulerprompt}
\begin{eulercomment}
Dan juga jelas, bahwa ini adalah segitiga dengan luas maksimal dan
kedua sisi 3 dan 4.
\end{eulercomment}
\begin{eulerprompt}
>aspect (1.5); plot2d(&H(3,4,x),1,7): // Kurva luas segitiga sengan panjang sisi 3, 4, x (1<= x <=7)
\end{eulerprompt}
\begin{euleroutput}
  Variable or function ysol not found.
  Error in expression: 3*abs(ysol)/2
   %ploteval:
      y0=f$(x[1],args());
  adaptiveevalone:
      s=%ploteval(g$,t;args());
  Try "trace errors" to inspect local variables after errors.
  plot2d:
      dw/n,dw/n^2,dw/n,auto;args());
\end{euleroutput}
\begin{eulercomment}
Kasus umum juga bisa digunakan.
\end{eulercomment}
\begin{eulerprompt}
>$solve(diff(H(a,b,c)^2,c)=0,c)
\end{eulerprompt}
\begin{euleroutput}
  Maxima said:
  diff: second argument must be a variable; found [1,0,4]
   -- an error. To debug this try: debugmode(true);
  
  Error in:
                                ^
\end{euleroutput}
\begin{eulercomment}
Sekarang mari kita cari himpunan semua titik di mana b+c=d untuk suatu
konstanta d. Sudah diketahui bahwa ini adalah sebuah elips.
\end{eulercomment}
\begin{eulerprompt}
>s1 &= subst(d-c,b,sol[2]); $s1
\end{eulerprompt}
\begin{euleroutput}
  Maxima said:
  part: invalid index of list or matrix.
   -- an error. To debug this try: debugmode(true);
  
  Error in:
  s1 &= subst(d-c,b,sol[2]); $s1 ...
                           ^
\end{euleroutput}
\begin{eulercomment}
Dan membuat fungsi-fungsi ini.
\end{eulercomment}
\begin{eulerprompt}
>function fx(a,c,d) &= rhs(s1[1]); $fx(a,c,d), function fy(a,c,d) &= rhs(s1[2]); $fy(a,c,d)
\end{eulerprompt}
\begin{eulercomment}
Sekarang kita dapat menggambar himpunan tersebut. Sisi b bervariasi
dari 1 hingga 4. Sudah diketahui bahwa kita mendapatkan sebuah elips.
\end{eulercomment}
\begin{eulerprompt}
>aspect(1); plot2d(&fx(3,x,5),&fy(3,x,5),xmin=1,xmax=4,square=1):
\end{eulerprompt}
\begin{eulercomment}
Kita dapat memeriksa persamaan umum untuk elips ini, yaitu

\end{eulercomment}
\begin{eulerformula}
\[
\frac{(x-x_m)^2}{u^2}+\frac{(y-y_m)}{v^2}=1,
\]
\end{eulerformula}
\begin{eulercomment}
di mana (xm, ym) adalah pusat, dan u serta v adalah setengah sumbu.
\end{eulercomment}
\begin{eulerprompt}
>$ratsimp((fx(a,c,d)-a/2)^2/u^2+fy(a,c,d)^2/v^2 with [u=d/2,v=sqrt(d^2-a^2)/2])
\end{eulerprompt}
\begin{eulercomment}
Kita melihat bahwa tinggi dan luas segitiga adalah maksimal untuk x=0.
Dengan demikian, luas segitiga dengan a+b+c=d adalah maksimal, jika
segitiga tersebut sama sisi. Kita ingin membuktikannya secara
analitis.
\end{eulercomment}
\begin{eulerprompt}
>eqns &= [diff(H(a,b,d-(a+b))^2,a)=0,diff(H(a,b,d-(a+b))^2,b)=0]; $eqns
\end{eulerprompt}
\begin{eulercomment}
Kita mendapatkan beberapa minima, yang termasuk dalam segitiga dengan
satu sisi 0, dan solusi a = b = c = d / 3.
\end{eulercomment}
\begin{eulerprompt}
>$solve(eqns,[a,b])
\end{eulerprompt}
\begin{eulercomment}
Ada juga metode Lagrange, yang memaksimalkan H(a,b,c)\textasciicircum{}2 sehubungan
dengan a+b+d=d.
\end{eulercomment}
\begin{eulerprompt}
>&solve([diff(H(a,b,c)^2,a)=la,diff(H(a,b,c)^2,b)=la, ...
>   diff(H(a,b,c)^2,c)=la,a+b+c=d],[a,b,c,la])
\end{eulerprompt}
\begin{euleroutput}
  Maxima said:
  diff: second argument must be a variable; found [1,0,4]
   -- an error. To debug this try: debugmode(true);
  
  Error in:
  ... la,    diff(H(a,b,c)^2,c)=la,a+b+c=d],[a,b,c,la]) ...
                                                       ^
\end{euleroutput}
\begin{eulercomment}
Kita bisa membuat plot situasi
\end{eulercomment}
\begin{eulercomment}
Pertama-tama, tetapkan titik-titik di Maxima.
\end{eulercomment}
\begin{eulerprompt}
>A &= at([x,y],sol[2]); $A
\end{eulerprompt}
\begin{euleroutput}
  Maxima said:
  part: invalid index of list or matrix.
   -- an error. To debug this try: debugmode(true);
  
  Error in:
  A &= at([x,y],sol[2]); $A ...
                       ^
\end{euleroutput}
\begin{eulerprompt}
>B &= [0,0]; $B, C &= [a,0]; $C
\end{eulerprompt}
\begin{eulercomment}
Kemudian, tetapkan kisaran plot, dan plot titik-titiknya.
\end{eulercomment}
\begin{eulerprompt}
>setPlotRange(0,5,-2,3); ...
>a=4; b=3; c=2; ...
>plotPoint(mxmeval("B"),"B"); plotPoint(mxmeval("C"),"C"); ...
>plotPoint(mxmeval("A"),"A"):
\end{eulerprompt}
\begin{euleroutput}
  Variable a1 not found!
  Use global variables or parameters for string evaluation.
  Error in Evaluate, superfluous characters found.
  Try "trace errors" to inspect local variables after errors.
  mxmeval:
      return evaluate(mxm(s));
  Error in:
  ... otPoint(mxmeval("C"),"C"); plotPoint(mxmeval("A"),"A"): ...
                                                       ^
\end{euleroutput}
\begin{eulercomment}
Plot the segments.
\end{eulercomment}
\begin{eulerprompt}
>plotSegment(mxmeval("A"),mxmeval("C")); ...
>plotSegment(mxmeval("B"),mxmeval("C")); ...
>plotSegment(mxmeval("B"),mxmeval("A")):
\end{eulerprompt}
\begin{euleroutput}
  Variable a1 not found!
  Use global variables or parameters for string evaluation.
  Error in Evaluate, superfluous characters found.
  Try "trace errors" to inspect local variables after errors.
  mxmeval:
      return evaluate(mxm(s));
  Error in:
  plotSegment(mxmeval("A"),mxmeval("C")); plotSegment(mxmeval("B ...
                          ^
\end{euleroutput}
\begin{eulercomment}
Hitung garis tegak lurus tengah dalam Maxima.
\end{eulercomment}
\begin{eulerprompt}
>h &= middlePerpendicular(A,B); g &= middlePerpendicular(B,C);
\end{eulerprompt}
\begin{eulercomment}
Dan bagian tengah lingkar.
\end{eulercomment}
\begin{eulerprompt}
>U &= lineIntersection(h,g);
\end{eulerprompt}
\begin{eulercomment}
Kita mendapatkan rumus untuk jari-jari lingkaran.
\end{eulercomment}
\begin{eulerprompt}
>&assume(a>0,b>0,c>0); $distance(U,B) | radcan
\end{eulerprompt}
\begin{eulercomment}
Mari kita tambahkan ini ke dalam plot.
\end{eulercomment}
\begin{eulerprompt}
>plotPoint(U()); ...
>plotCircle(circleWithCenter(mxmeval("U"),mxmeval("distance(U,C)"))):
\end{eulerprompt}
\begin{euleroutput}
  Variable a2 not found!
  Use global variables or parameters for string evaluation.
  Error in ^
  Error in expression: [a/2,(a2^2+a1^2-a*a1)/(2*a2)]
  Error in:
  plotPoint(U()); plotCircle(circleWithCenter(mxmeval("U"),mxmev ...
               ^
\end{euleroutput}
\begin{eulercomment}
Dengan menggunakan geometri, kami memperoleh rumus sederhana

\end{eulercomment}
\begin{eulerformula}
\[
\frac{a}{\sin(\alpha)}=2r
\]
\end{eulerformula}
\begin{eulercomment}
untuk radius. Kita bisa mengecek, apakah hal ini benar dengan Maxima.
Maxima akan memperhitungkannya hanya jika kita mengkuadratkannya.
\end{eulercomment}
\begin{eulerprompt}
>$c^2/sin(computeAngle(A,B,C))^2  | factor
\end{eulerprompt}
\eulerheading{Contoh 4: Garis Euler dan Parabola}
\begin{eulercomment}
Garis Euler adalah garis yang ditentukan dari segitiga apa pun yang
tidak sama sisi. Garis ini merupakan garis tengah segitiga, dan
melewati beberapa titik penting yang ditentukan dari segitiga,
termasuk ortosentrum, circumcentrum, centroid, titik Exeter, dan pusat
lingkaran sembilan titik segitiga.


Sebagai demonstrasi, kami menghitung dan memplot garis Euler dalam
sebuah segitiga.


Pertama, kita mendefinisikan sudut-sudut segitiga dalam Euler. Kami
menggunakan definisi, yang terlihat dalam ekspresi simbolis.
\end{eulercomment}
\begin{eulerprompt}
>A::=[-1,-1]; B::=[2,0]; C::=[1,2];
\end{eulerprompt}
\begin{eulercomment}
Untuk memplot objek geometris, kita menyiapkan area plot, dan
menambahkan titik-titiknya. Semua plot objek geometris ditambahkan ke
plot saat ini.
\end{eulercomment}
\begin{eulerprompt}
>setPlotRange(3); plotPoint(A,"A"); plotPoint(B,"B"); plotPoint(C,"C");
\end{eulerprompt}
\begin{eulercomment}
We can also add the sides of the triangle.
\end{eulercomment}
\begin{eulerprompt}
>plotSegment(A,B,""); plotSegment(B,C,""); plotSegment(C,A,""):
\end{eulerprompt}
\begin{eulercomment}
Berikut ini adalah luas area segitiga, dengan menggunakan rumus
determinan. Tentu saja, kita harus mengambil nilai absolut dari hasil
ini.
\end{eulercomment}
\begin{eulerprompt}
>$areaTriangle(A,B,C)
\end{eulerprompt}
\begin{eulercomment}
Kita dapat menghitung koefisien dari sisi c.
\end{eulercomment}
\begin{eulerprompt}
>c &= lineThrough(A,B)
\end{eulerprompt}
\begin{euleroutput}
  
                              [- 1, 3, - 2]
  
\end{euleroutput}
\begin{eulercomment}
Dan juga mendapatkan formula untuk baris ini.
\end{eulercomment}
\begin{eulerprompt}
>$getLineEquation(c,x,y)
\end{eulerprompt}
\begin{eulercomment}
Untuk bentuk Hesse, kita perlu menentukan sebuah titik, sehingga titik
tersebut berada di sisi positif dari bentuk Hesse. Memasukkan titik
tersebut akan menghasilkan jarak positif ke garis.
\end{eulercomment}
\begin{eulerprompt}
>$getHesseForm(c,x,y,C), $at(%,[x=C[1],y=C[2]])
\end{eulerprompt}
\begin{eulercomment}
Sekarang kita menghitung keliling ABC.
\end{eulercomment}
\begin{eulerprompt}
>LL &= circleThrough(A,B,C); $getCircleEquation(LL,x,y)
>O &= getCircleCenter(LL); $O
\end{eulerprompt}
\begin{eulercomment}
Plot lingkaran dan pusatnya. Cu dan U adalah simbolik. Kami
mengevaluasi ekspresi ini untuk Euler.
\end{eulercomment}
\begin{eulerprompt}
>plotCircle(LL()); plotPoint(O(),"O"):
\end{eulerprompt}
\begin{eulercomment}
Kita dapat menghitung perpotongan ketinggian di ABC (pusat
ortosentrum) secara numerik dengan perintah berikut ini.
\end{eulercomment}
\begin{eulerprompt}
>H &= lineIntersection(perpendicular(A,lineThrough(C,B)),...
>  perpendicular(B,lineThrough(A,C))); $H
\end{eulerprompt}
\begin{eulercomment}
Sekarang kita dapat menghitung garis Euler dari segitiga tersebut.
\end{eulercomment}
\begin{eulerprompt}
>el &= lineThrough(H,O); $getLineEquation(el,x,y)
\end{eulerprompt}
\begin{eulercomment}
Add it to our plot.
\end{eulercomment}
\begin{eulerprompt}
>plotPoint(H(),"H"); plotLine(el(),"Garis Euler"):
\end{eulerprompt}
\begin{eulercomment}
Pusat gravitasi harus berada pada garis ini.
\end{eulercomment}
\begin{eulerprompt}
>M &= (A+B+C)/3; $getLineEquation(el,x,y) with [x=M[1],y=M[2]]
>plotPoint(M(),"M"): // titik berat
\end{eulerprompt}
\begin{eulercomment}
Teori mengatakan bahwa MH = 2*MO. Kita perlu menyederhanakan dengan
radcan untuk mencapai hal ini.
\end{eulercomment}
\begin{eulerprompt}
>$distance(M,H)/distance(M,O)|radcan
\end{eulerprompt}
\begin{eulercomment}
Fungsi-fungsi ini juga mencakup fungsi untuk sudut.
\end{eulercomment}
\begin{eulerprompt}
>$computeAngle(A,C,B), degprint(%())
\end{eulerprompt}
\begin{euleroutput}
  60°15'18.43''
\end{euleroutput}
\begin{eulercomment}
Persamaan untuk bagian tengah lingkaran tidak terlalu bagus.
\end{eulercomment}
\begin{eulerprompt}
>Q &= lineIntersection(angleBisector(A,C,B),angleBisector(C,B,A))|radcan; $Q
\end{eulerprompt}
\begin{eulercomment}
Mari kita hitung juga ekspresi untuk jari-jari lingkaran yang
tertulis.
\end{eulercomment}
\begin{eulerprompt}
>r &= distance(Q,projectToLine(Q,lineThrough(A,B)))|ratsimp; $r
>LD &=  circleWithCenter(Q,r); // Lingkaran dalam
\end{eulerprompt}
\begin{eulercomment}
Let us add this to the plot.
\end{eulercomment}
\begin{eulerprompt}
>color(5); plotCircle(LD()):
\end{eulerprompt}
\eulersubheading{Parabola}
\begin{eulercomment}
Selanjutnya akan dicari persamaan tempat kedudukan titik-titik yang berjarak sama ke titik C
dan ke garis AB.
\end{eulercomment}
\begin{eulerprompt}
>p &= getHesseForm(lineThrough(A,B),x,y,C)-distance([x,y],C); $p='0
\end{eulerprompt}
\begin{eulerformula}
\[
{\it getHesseForm}\left({\it lineThrough}\left(\begin{pmatrix}1 & 
 3.141592653589793 \\ 4 & 5 \\ \end{pmatrix} , \begin{pmatrix}1 & 2
  \\ 3 & 4 \\ \end{pmatrix}\right) , x , y , C\right)-{\it distance}
 \left(\left[ x , y \right]  , C\right)=0
\]
\end{eulerformula}
\begin{eulercomment}
Persamaan tersebut dapat digambar menjadi satu dengan gambar sebelumnya.
\end{eulercomment}
\begin{eulerprompt}
>plot2d(p,level=0,add=1,contourcolor=6):
\end{eulerprompt}
\begin{euleroutput}
  Function lineThrough not found.
  Try list ... to find functions!
  Error in expression: getHesseForm(lineThrough(matrix([1,3.141592653589793],[4,5]),matrix([1,2],[3,4])),x,y,C)-distance([x,y],C)
  %ploteval2:
      if maps then return %mapexpression2(x,y,f$;args());
  fcontour:
      Z=%ploteval2(f$,X,Y,maps;args());
  Try "trace errors" to inspect local variables after errors.
  plot2d:
      =style,=outline,=frame);
\end{euleroutput}
\begin{eulercomment}
Ini seharusnya merupakan suatu fungsi, tetapi solver default Maxima
hanya bisa menemukan solusinya jika kita mengkuadratkan persamaannya.
Akibatnya, kita mendapatkan solusi palsu.
\end{eulercomment}
\begin{eulerprompt}
>akar &= solve(getHesseForm(lineThrough(A,B),x,y,C)^2-distance([x,y],C)^2,y)
\end{eulerprompt}
\begin{euleroutput}
  
           [distance([x, y], C) = 
                             [    80143857 ]
                             [ 1  -------- ]  [ 1  2 ]
  - getHesseForm(lineThrough([    25510582 ], [      ]), x, y, C), 
                             [             ]  [ 3  4 ]
                             [ 4     5     ]
                                                 [    80143857 ]
                                                 [ 1  -------- ]
  distance([x, y], C) = getHesseForm(lineThrough([    25510582 ], 
                                                 [             ]
                                                 [ 4     5     ]
  [ 1  2 ]
  [      ]), x, y, C)]
  [ 3  4 ]
  
\end{euleroutput}
\begin{eulercomment}
Solusi pertama adalah


maxima: akar[1]


Menambahkan solusi pertama ke dalam plot menunjukkan, bahwa ini memang
jalur yang kita cari. Teori mengatakan bahwa ini adalah sebuah
parabola yang diputar.
\end{eulercomment}
\begin{eulerprompt}
>plot2d(&rhs(akar[1]),add=1):
\end{eulerprompt}
\begin{euleroutput}
  Function lineThrough not found.
  Try list ... to find functions!
  Error in expression: -getHesseForm(lineThrough(matrix([1,80143857/25510582],[4,5]),matrix([1,2],[3,4])),x,y,C)
  %ploteval:
      y0=f$(x[1],args());
  adaptiveevalone:
      s=%ploteval(g$,t;args());
  Try "trace errors" to inspect local variables after errors.
  plot2d:
      dw/n,dw/n^2,dw/n,auto;args());
\end{euleroutput}
\begin{eulerprompt}
>function g(x) &= rhs(akar[1]); $'g(x)= g(x)// fungsi yang mendefinisikan kurva di atas
\end{eulerprompt}
\begin{eulerformula}
\[
g\left(x\right)=-{\it getHesseForm}\left({\it lineThrough}\left(
 \begin{pmatrix}1 & \frac{80143857}{25510582} \\ 4 & 5 \\ 
 \end{pmatrix} , \begin{pmatrix}1 & 2 \\ 3 & 4 \\ \end{pmatrix}
 \right) , x , y , C\right)
\]
\end{eulerformula}
\begin{eulerprompt}
>T &=[-1, g(-1)]; // ambil sebarang titik pada kurva tersebut
>dTC &= distance(T,C); $fullratsimp(dTC), $float(%) // jarak T ke C
\end{eulerprompt}
\begin{eulerformula}
\[
{\it distance}\left(\left[ -1 , -{\it getHesseForm}\left(
 {\it lineThrough}\left(\begin{pmatrix}1 & \frac{80143857}{25510582}
  \\ 4 & 5 \\ \end{pmatrix} , \begin{pmatrix}1 & 2 \\ 3 & 4 \\ 
 \end{pmatrix}\right) , -1 , y , C\right) \right]  , C\right)
\]
\end{eulerformula}
\begin{eulerformula}
\[
{\it distance}\left(\left[ -1.0 , -1.0\,{\it getHesseForm}\left(
 {\it lineThrough}\left(\begin{pmatrix}1.0 & 3.141592653589793 \\ 4.0
  & 5.0 \\ \end{pmatrix} , \begin{pmatrix}1.0 & 2.0 \\ 3.0 & 4.0 \\ 
 \end{pmatrix}\right) , -1.0 , y , C\right) \right]  , C\right)
\]
\end{eulerformula}
\begin{eulerprompt}
>U &= projectToLine(T,lineThrough(A,B)); $U // proyeksi T pada garis AB 
\end{eulerprompt}
\begin{eulerformula}
\[
{\it projectToLine}\left(\left[ -1 , -{\it getHesseForm}\left(
 {\it lineThrough}\left(\begin{pmatrix}1 & \frac{80143857}{25510582}
  \\ 4 & 5 \\ \end{pmatrix} , \begin{pmatrix}1 & 2 \\ 3 & 4 \\ 
 \end{pmatrix}\right) , -1 , y , C\right) \right]  , 
 {\it lineThrough}\left(\begin{pmatrix}1 & 3.141592653589793 \\ 4 & 5
  \\ \end{pmatrix} , \begin{pmatrix}1 & 2 \\ 3 & 4 \\ \end{pmatrix}
 \right)\right)
\]
\end{eulerformula}
\begin{eulerprompt}
>dU2AB &= distance(T,U); $fullratsimp(dU2AB), $float(%) // jatak T ke AB
\end{eulerprompt}
\begin{eulercomment}
Ternyata jarak T ke C sama dengan jarak T ke AB. Coba Anda pilih titik T yang lain dan
ulangi perhitungan-perhitungan di atas untuk menunjukkan bahwa hasilnya juga sama.
\end{eulercomment}
\begin{eulercomment}

\begin{eulercomment}
\eulerheading{Contoh 5: Trigonometri Rasional}
\begin{eulercomment}
Hal ini terinspirasi dari sebuah ceramah N.J. Wildberger. Dalam
bukunya “Proporsi Ilahi”, Wildberger mengusulkan untuk mengganti
gagasan klasik tentang jarak dan sudut dengan kuadransi dan
penyebaran. Dengan menggunakan ini, memang memungkinkan untuk
menghindari fungsi trigonometri dalam banyak contoh, dan tetap
“rasional”.


Berikut ini, saya akan memperkenalkan konsep-konsep tersebut, dan
memecahkan beberapa masalah. Saya menggunakan komputasi simbolik
Maxima di sini, yang menyembunyikan keuntungan utama dari trigonometri
rasional yaitu komputasi dapat dilakukan dengan kertas dan pensil
saja. Anda dipersilakan untuk memeriksa hasilnya tanpa komputer.


Intinya adalah bahwa komputasi rasional simbolik sering kali
memberikan hasil yang sederhana. Sebaliknya, trigonometri klasik
menghasilkan hasil trigonometri yang rumit, yang dievaluasi dengan
pendekatan numerik saja.
\end{eulercomment}
\begin{eulerprompt}
>load geometry;
\end{eulerprompt}
\begin{eulercomment}
Untuk pengenalan pertama, kita menggunakan segitiga persegi panjang
dengan proporsi Mesir yang terkenal 3, 4, dan 5. Perintah berikut ini
adalah perintah Euler untuk memplot geometri bidang yang terdapat pada
file Euler “geometry.e”.
\end{eulercomment}
\begin{eulerprompt}
>C&:=[0,0]; A&:=[4,0]; B&:=[0,3]; ...
>setPlotRange(-1,5,-1,5); ...
>plotPoint(A,"A"); plotPoint(B,"B"); plotPoint(C,"C"); ...
>plotSegment(B,A,"c"); plotSegment(A,C,"b"); plotSegment(C,B,"a"); ...
>insimg(30);
\end{eulerprompt}
\begin{eulercomment}
Tentu saja,


\end{eulercomment}
\begin{eulerformula}
\[
\sin(w_a)=\frac{a}{c},
\]
\end{eulerformula}
\begin{eulercomment}
di mana wa adalah sudut di A. Cara biasa untuk menghitung sudut ini,
adalah dengan mengambil kebalikan dari fungsi sinus. Hasilnya adalah
sudut yang tidak dapat dicerna, yang hanya dapat dicetak kira-kira.
\end{eulercomment}
\begin{eulerprompt}
>wa := arcsin(3/5); degprint(wa)
\end{eulerprompt}
\begin{euleroutput}
  36°52'11.63''
\end{euleroutput}
\begin{eulercomment}
Trigonometri rasional mencoba menghindari hal ini.


Gagasan pertama trigonometri rasional adalah kuadrat, yang
menggantikan jarak. Sebenarnya, ini hanyalah jarak yang dikuadratkan.
Berikut ini, a, b, dan c menunjukkan kuadran sisi-sisinya.


Teorema Pythogoras menjadi a + b = c
\end{eulercomment}
\begin{eulerprompt}
>a &= 3^2; b &= 4^2; c &= 5^2; &a+b=c
\end{eulerprompt}
\begin{euleroutput}
  
                                 25 = 25
  
\end{euleroutput}
\begin{eulercomment}
Gagasan kedua dari trigonometri rasional adalah penyebaran. Penyebaran
mengukur bukaan di antara garis-garis. Ini adalah 0, jika
garis-garisnya sejajar, dan 1, jika garis-garisnya persegi panjang.
Ini adalah kuadrat dari sinus sudut antara dua garis.


Penyebaran garis AB dan AC pada gambar di atas didefinisikan sebagai


\end{eulercomment}
\begin{eulerformula}
\[
s_a = \sin(\alpha)^2 = \frac{a}{c},
\]
\end{eulerformula}
\begin{eulercomment}
di mana a dan c adalah kuadran dari segitiga persegi panjang dengan
satu sudut di A.
\end{eulercomment}
\begin{eulerprompt}
>sa &= a/c; $sa
\end{eulerprompt}
\begin{eulercomment}
Tentu saja, hal ini lebih mudah dihitung daripada sudut. Tetapi Anda
kehilangan sifat bahwa sudut dapat ditambahkan dengan mudah.


Tentu saja, kita bisa mengonversi nilai perkiraan kita untuk sudut wa
ke sprad, dan mencetaknya sebagai pecahan.
\end{eulercomment}
\begin{eulerprompt}
>fracprint(sin(wa)^2)
\end{eulerprompt}
\begin{euleroutput}
  9/25
\end{euleroutput}
\begin{eulercomment}
Hukum kosinus trgonometri klasik diterjemahkan ke dalam “hukum silang”
berikut ini.


\end{eulercomment}
\begin{eulerformula}
\[
(c+b-a)^2 = 4 b c \, (1-s_a)
\]
\end{eulerformula}
\begin{eulercomment}
Di sini a, b, dan c adalah kuadran dari sisi-sisi segitiga, dan sa
adalah penyebaran di sudut A. Sisi a, seperti biasa, berlawanan dengan
sudut A.


Hukum-hukum ini diimplementasikan dalam file geometri.e yang kita
masukkan ke dalam Euler.
\end{eulercomment}
\begin{eulerprompt}
>$crosslaw(aa,bb,cc,saa)
\end{eulerprompt}
\begin{eulercomment}
Dalam kasus kami, kami mendapatkan
\end{eulercomment}
\begin{eulerprompt}
>$crosslaw(a,b,c,sa)
\end{eulerprompt}
\begin{eulercomment}
Mari kita gunakan crosslaw ini untuk mencari sebaran di A. Untuk
melakukannya, kita buat crosslaw untuk kuadran a, b, dan c, dan
selesaikan untuk sebaran sa yang tidak diketahui.


Anda bisa melakukan ini dengan tangan dengan mudah, tapi saya
menggunakan Maxima. Tentu saja, kita mendapatkan hasil yang sudah kita
dapatkan.
\end{eulercomment}
\begin{eulerprompt}
>$crosslaw(a,b,c,x), $solve(%,x)
\end{eulerprompt}
\begin{eulercomment}
Kita sudah mengetahui hal ini. Definisi penyebaran adalah kasus khusus
dari crosslaw.


Kita juga dapat menyelesaikannya untuk a, b, c secara umum. Hasilnya
adalah sebuah rumus yang menghitung penyebaran sudut sebuah segitiga
dengan kuadran ketiga sisinya.
\end{eulercomment}
\begin{eulerprompt}
>$solve(crosslaw(aa,bb,cc,x),x)
\end{eulerprompt}
\begin{eulercomment}
Kita dapat membuat sebuah fungsi dari hasil tersebut. Fungsi seperti
itu sudah didefinisikan dalam file geometry.e dari Euler.
\end{eulercomment}
\begin{eulerprompt}
>$spread(a,b,c)
\end{eulerprompt}
\begin{eulercomment}
Sebagai contoh, kita dapat menggunakannya untuk menghitung sudut
segitiga dengan sisi

lateks: a, \textbackslash{}quad a, \textbackslash{}quad \textbackslash{}frac\{4a\}\{7\}

Hasilnya adalah rasional, yang tidak mudah didapat jika kita
menggunakan trigonometri klasik.
\end{eulercomment}
\begin{eulerprompt}
>$spread(a,a,4*a/7)
\end{eulerprompt}
\begin{eulercomment}
Ini adalah sudut dalam derajat.
\end{eulercomment}
\begin{eulerprompt}
>degprint(arcsin(sqrt(6/7)))
\end{eulerprompt}
\begin{euleroutput}
  67°47'32.44''
\end{euleroutput}
\eulersubheading{Contoh Lain}
\begin{eulercomment}
Sekarang, mari kita coba contoh yang lebih lanjut.


Kita tentukan tiga sudut segitiga sebagai berikut.
\end{eulercomment}
\begin{eulerprompt}
>A&:=[1,2]; B&:=[4,3]; C&:=[0,4]; ...
>setPlotRange(-1,5,1,7); ...
>plotPoint(A,"A"); plotPoint(B,"B"); plotPoint(C,"C"); ...
>plotSegment(B,A,"c"); plotSegment(A,C,"b"); plotSegment(C,B,"a"); ...
>insimg;
\end{eulerprompt}
\begin{eulercomment}
Dengan menggunakan Pythogoras, mudah untuk menghitung jarak antara dua
titik. Pertama-tama saya menggunakan jarak fungsi dari file Euler
untuk geometri. Jarak fungsi menggunakan geometri klasik.
\end{eulercomment}
\begin{eulerprompt}
>$distance(A,B)
\end{eulerprompt}
\begin{eulercomment}
Euler juga memiliki fungsi untuk kuadranan antara dua titik.


Pada contoh berikut, karena c+b bukan a, maka segitiga tersebut tidak
berbentuk persegi panjang.
\end{eulercomment}
\begin{eulerprompt}
>c &= quad(A,B); $c, b &= quad(A,C); $b, a &= quad(B,C); $a,
\end{eulerprompt}
\begin{eulercomment}
Pertama, mari kita menghitung sudut tradisional. Fungsi computeAngle
menggunakan metode yang biasa berdasarkan hasil kali titik dari dua
vektor. Hasilnya adalah beberapa perkiraan titik mengambang.

\end{eulercomment}
\begin{eulerformula}
\[
A=<1,2>\quad B=<4,3>,\quad C=<0,4>
\]
\end{eulerformula}
\begin{eulerformula}
\[
\mathbf{a}=C-B=<-4,1>,\quad \mathbf{c}=A-B=<-3,-1>,\quad \beta=\angle ABC
\]
\end{eulerformula}
\begin{eulerformula}
\[
\mathbf{a}.\mathbf{c}=|\mathbf{a}|.|\mathbf{c}|\cos \beta
\]
\end{eulerformula}
\begin{eulerformula}
\[
\cos \angle ABC =\cos\beta=\frac{\mathbf{a}.\mathbf{c}}{|\mathbf{a}|.|\mathbf{c}|}=\frac{12-1}{\sqrt{17}\sqrt{10}}=\frac{11}{\sqrt{17}\sqrt{10}}
\]
\end{eulerformula}
\begin{eulerprompt}
>wb &= computeAngle(A,B,C); $wb, $(wb/pi*180)()
\end{eulerprompt}
\begin{euleroutput}
  32.4711922908
\end{euleroutput}
\begin{eulercomment}
Using pencil and paper, we can do the same with the cross law. We
insert the quadrances a, b, and c into the cross law and solve for x.
\end{eulercomment}
\begin{eulerprompt}
>$crosslaw(a,b,c,x), $solve(%,x), //(b+c-a)^=4b.c(1-x)
\end{eulerprompt}
\begin{eulercomment}
Itulah yang dilakukan oleh fungsi spread yang didefinisikan dalam
“geometry.e”.
\end{eulercomment}
\begin{eulerprompt}
>sb &= spread(b,a,c); $sb
\end{eulerprompt}
\begin{eulercomment}
Maxima mendapatkan hasil yang sama dengan menggunakan trigonometri
biasa, jika kita memaksakannya. Ia menyelesaikan suku sin(arccos(...))
menjadi hasil pecahan. Sebagian besar siswa tidak dapat melakukan ini.
\end{eulercomment}
\begin{eulerprompt}
>$sin(computeAngle(A,B,C))^2
\end{eulerprompt}
\begin{eulercomment}
Setelah kita memiliki penyebaran di B, kita dapat menghitung tinggi ha
di sisi a. Ingatlah bahwa


\end{eulercomment}
\begin{eulerformula}
\[
s_b = \frac{h_a}{c}
\]
\end{eulerformula}
\begin{eulercomment}
menurut definisi.
\end{eulercomment}
\begin{eulerprompt}
>ha &= c*sb; $ha
\end{eulerprompt}
\begin{eulercomment}
Gambar berikut ini dibuat dengan program geometri C.a.R., yang dapat
menggambar kuadran dan penyebaran.


image: (20) Rational\_Geometry\_CaR.png


Menurut definisi, panjang ha adalah akar kuadrat dari kuadrannya.
\end{eulercomment}
\begin{eulerprompt}
>$sqrt(ha)
\end{eulerprompt}
\begin{eulercomment}
Sekarang kita dapat menghitung luas segitiga. Jangan lupa, bahwa kita
berurusan dengan kuadran!
\end{eulercomment}
\begin{eulerprompt}
>$sqrt(ha)*sqrt(a)/2
\end{eulerprompt}
\begin{eulercomment}
Rumus penentu yang biasa menghasilkan hasil yang sama.
\end{eulercomment}
\begin{eulerprompt}
>$areaTriangle(B,A,C)
\end{eulerprompt}
\eulersubheading{The Heron Formula}
\begin{eulercomment}
Sekarang, mari kita selesaikan masalah ini secara umum!
\end{eulercomment}
\begin{eulerprompt}
>&remvalue(a,b,c,sb,ha);
\end{eulerprompt}
\begin{eulercomment}
Pertama-tama kita menghitung penyebaran di B untuk segitiga dengan
sisi a, b, dan c. Kemudian kita menghitung luas kuadrat (“quadrea”?),
memfaktorkannya dengan Maxima, dan kita mendapatkan rumus Heron yang
terkenal.


Memang, hal ini sulit dilakukan dengan pensil dan kertas.
\end{eulercomment}
\begin{eulerprompt}
>$spread(b^2,c^2,a^2), $factor(%*c^2*a^2/4)
\end{eulerprompt}
\eulersubheading{Aturan Triple Spread}
\begin{eulercomment}
Kerugian dari spread adalah bahwa mereka tidak lagi hanya menambahkan
sudut seperti.


Namun, tiga spread dari sebuah segitiga memenuhi aturan “triple
spread” berikut ini.
\end{eulercomment}
\begin{eulerprompt}
>&remvalue(sa,sb,sc); $triplespread(sa,sb,sc)
\end{eulerprompt}
\begin{eulercomment}
Aturan ini berlaku untuk tiga sudut yang berjumlah 180°.

\end{eulercomment}
\begin{eulerformula}
\[
\alpha+\beta+\gamma=\pi
\]
\end{eulerformula}
\begin{eulercomment}
Karena spreads dari

\end{eulercomment}
\begin{eulerformula}
\[
\alpha, \pi-\alpha
\]
\end{eulerformula}
\begin{eulercomment}
adalah sama, aturan triple spread juga benar, jika

\end{eulercomment}
\begin{eulerformula}
\[
\alpha+\beta=\gamma
\]
\end{eulerformula}
\begin{eulercomment}
Karena penyebaran sudut negatif adalah sama, aturan triple spread juga
berlaku, jika


atex: \textbackslash{}alpha+\textbackslash{}beta+\textbackslash{}gamma=0

Sebagai contoh, kita dapat menghitung penyebaran sudut 60°. Ini adalah
3/4. Namun, persamaan ini memiliki solusi kedua, di mana semua
penyebarannya adalah 0.
\end{eulercomment}
\begin{eulerprompt}
>$solve(triplespread(x,x,x),x)
\end{eulerprompt}
\begin{eulercomment}
Penyebaran 90° jelas adalah 1. Jika dua sudut ditambahkan ke 90°,
penyebarannya akan menyelesaikan persamaan penyebaran tiga dengan a,
b, 1. Dengan perhitungan berikut, kita mendapatkan a + b = 1.
\end{eulercomment}
\begin{eulerprompt}
>$triplespread(x,y,1), $solve(%,x)
\end{eulerprompt}
\begin{eulercomment}
Karena penyebaran 180°-t sama dengan penyebaran t, rumus penyebaran
tiga kali lipat juga berlaku, jika satu sudut adalah jumlah atau
selisih dari dua sudut lainnya.


Jadi kita dapat menemukan penyebaran sudut dua kali lipat. Perhatikan
bahwa ada dua solusi lagi. Kita jadikan ini sebuah fungsi.
\end{eulercomment}
\begin{eulerprompt}
>$solve(triplespread(a,a,x),x), function doublespread(a) &= factor(rhs(%[1]))
\end{eulerprompt}
\begin{euleroutput}
  
                              - 4 (a - 1) a
  
\end{euleroutput}
\eulersubheading{Pembagi Sudut}
\begin{eulercomment}
Ini adalah situasi yang sudah kita ketahui.
\end{eulercomment}
\begin{eulerprompt}
>C&:=[0,0]; A&:=[4,0]; B&:=[0,3]; ...
>setPlotRange(-1,5,-1,5); ...
>plotPoint(A,"A"); plotPoint(B,"B"); plotPoint(C,"C"); ...
>plotSegment(B,A,"c"); plotSegment(A,C,"b"); plotSegment(C,B,"a"); ...
>insimg;
\end{eulerprompt}
\begin{eulercomment}
Mari kita hitung panjang garis bagi sudut di A. Tetapi kita ingin
menyelesaikannya untuk a, b, c secara umum.
\end{eulercomment}
\begin{eulerprompt}
>&remvalue(a,b,c);
\end{eulerprompt}
\begin{eulercomment}
Jadi, pertama-tama kita menghitung penyebaran sudut yang dibelah dua
di A, menggunakan rumus penyebaran tiga.


Masalah dengan rumus ini muncul lagi. Rumus ini memiliki dua solusi.
Kita harus memilih salah satu yang benar. Solusi lainnya mengacu pada
sudut terbagi dua 180°-wa.
\end{eulercomment}
\begin{eulerprompt}
>$triplespread(x,x,a/(a+b)), $solve(%,x), sa2 &= rhs(%[1]); $sa2
\end{eulerprompt}
\begin{eulercomment}
Mari kita periksa persegi panjang Mesir.
\end{eulercomment}
\begin{eulerprompt}
>$sa2 with [a=3^2,b=4^2]
\end{eulerprompt}
\begin{eulercomment}
Kita bisa mencetak sudut dalam Euler, setelah mentransfer penyebaran
ke radian.
\end{eulercomment}
\begin{eulerprompt}
>wa2 := arcsin(sqrt(1/10)); degprint(wa2)
\end{eulerprompt}
\begin{euleroutput}
  18°26'5.82''
\end{euleroutput}
\begin{eulercomment}
Titik P adalah perpotongan garis bagi sudut dengan sumbu y.
\end{eulercomment}
\begin{eulerprompt}
>P := [0,tan(wa2)*4]
\end{eulerprompt}
\begin{euleroutput}
  [0,  1.33333]
\end{euleroutput}
\begin{eulerprompt}
>plotPoint(P,"P"); plotSegment(A,P):
\end{eulerprompt}
\begin{eulercomment}
Mari kita periksa sudut-sudutnya dalam contoh spesifik kita.
\end{eulercomment}
\begin{eulerprompt}
>computeAngle(C,A,P), computeAngle(P,A,B)
\end{eulerprompt}
\begin{euleroutput}
  0.321750554397
  0.321750554397
\end{euleroutput}
\begin{eulercomment}
Sekarang kita hitung panjang garis bagi AP.


ita menggunakan teorema sinus dalam segitiga APC. Teorema ini
menyatakan bahwa

\end{eulercomment}
\begin{eulerformula}
\[
\frac{BC}{\sin(w_a)} = \frac{AC}{\sin(w_b)} = \frac{AB}{\sin(w_c)}
\]
\end{eulerformula}
\begin{eulercomment}
berlaku untuk semua segitiga. Kuadratkan, ini diterjemahkan ke dalam
apa yang disebut "spread law"

\end{eulercomment}
\begin{eulerformula}
\[
\frac{a}{s_a} = \frac{b}{s_b} = \frac{c}{s_b}
\]
\end{eulerformula}
\begin{eulercomment}
di mana a, b, c menunjukkan qudrance.


Karena spread CPA adalah 1-sa2, kita mendapatkan bisa/1 = b/(1-sa2)
dan bisa menghitung bisa (kuadransi dari garis-bagi sudut).
\end{eulercomment}
\begin{eulerprompt}
>&factor(ratsimp(b/(1-sa2))); bisa &= %; $bisa
\end{eulerprompt}
\begin{eulercomment}
Mari kita periksa rumus ini untuk nilai-nilai Egyptian kita.
\end{eulercomment}
\begin{eulerprompt}
>sqrt(mxmeval("at(bisa,[a=3^2,b=4^2])")), distance(A,P)
\end{eulerprompt}
\begin{euleroutput}
  4.21637021356
  4.21637021356
\end{euleroutput}
\begin{eulercomment}
Kita juga dapat menghitung P dengan menggunakan rumus penyebaran.
\end{eulercomment}
\begin{eulerprompt}
>py&=factor(ratsimp(sa2*bisa)); $py
\end{eulerprompt}
\begin{eulercomment}
Nilainya sama dengan yang kita dapatkan dengan rumus trigonometri.
\end{eulercomment}
\begin{eulerprompt}
>sqrt(mxmeval("at(py,[a=3^2,b=4^2])"))
\end{eulerprompt}
\begin{euleroutput}
  1.33333333333
\end{euleroutput}
\eulersubheading{Sudut Akor}
\begin{eulercomment}
Lihatlah situasi berikut ini.
\end{eulercomment}
\begin{eulerprompt}
>setPlotRange(1.2); ...
>color(1); plotCircle(circleWithCenter([0,0],1)); ...
>A:=[cos(1),sin(1)]; B:=[cos(2),sin(2)]; C:=[cos(6),sin(6)]; ...
>plotPoint(A,"A"); plotPoint(B,"B"); plotPoint(C,"C"); ...
>color(3); plotSegment(A,B,"c"); plotSegment(A,C,"b"); plotSegment(C,B,"a"); ...
>color(1); O:=[0,0];  plotPoint(O,"0"); ...
>plotSegment(A,O); plotSegment(B,O); plotSegment(C,O,"r"); ...
>insimg;
\end{eulerprompt}
\begin{eulercomment}
Kita dapat menggunakan Maxima untuk menyelesaikan rumus penyebaran
tiga untuk sudut-sudut di pusat O untuk r. Dengan demikian kita
mendapatkan rumus untuk jari-jari kuadrat dari pericircle dalam hal
kuadran sisi-sisinya.


Kali ini, Maxima menghasilkan beberapa angka nol yang rumit, yang kita
abaikan.
\end{eulercomment}
\begin{eulerprompt}
>&remvalue(a,b,c,r); // hapus nilai-nilai sebelumnya untuk perhitungan baru
>rabc &= rhs(solve(triplespread(spread(b,r,r),spread(a,r,r),spread(c,r,r)),r)[4]); $rabc
\end{eulerprompt}
\begin{eulercomment}
Kita dapat menjadikannya sebuah fungsi Euler.
\end{eulercomment}
\begin{eulerprompt}
>function periradius(a,b,c) &= rabc;
\end{eulerprompt}
\begin{eulercomment}
Mari kita periksa hasilnya untuk poin A, B, C.
\end{eulercomment}
\begin{eulerprompt}
>a:=quadrance(B,C); b:=quadrance(A,C); c:=quadrance(A,B);
\end{eulerprompt}
\begin{eulercomment}
Radiusnya memang 1.
\end{eulercomment}
\begin{eulerprompt}
>periradius(a,b,c)
\end{eulerprompt}
\begin{euleroutput}
  1
\end{euleroutput}
\begin{eulercomment}
Faktanya adalah, bahwa penyebaran CBA hanya bergantung pada b dan c.
Ini adalah teorema sudut akor.
\end{eulercomment}
\begin{eulerprompt}
>$spread(b,a,c)*rabc | ratsimp
\end{eulerprompt}
\begin{eulercomment}
Faktanya, penyebarannya adalah b/(4r), dan kita melihat bahwa sudut
chord b adalah setengah dari sudut tengah.
\end{eulercomment}
\begin{eulerprompt}
>$doublespread(b/(4*r))-spread(b,r,r) | ratsimp
\end{eulerprompt}
\begin{eulercomment}
\begin{eulercomment}
\eulerheading{Contoh 6: Jarak Minimal pada Bidang}
\begin{eulercomment}
\end{eulercomment}
\eulersubheading{Keterangan awal}
\begin{eulercomment}
Fungsi yang, pada sebuah titik M pada bidang, menetapkan jarak AM
antara titik tetap A dan M, memiliki garis-garis tingkat yang cukup
sederhana: lingkaran yang berpusat di A.
\end{eulercomment}
\begin{eulerprompt}
>&remvalue();
>A=[-1,-1];
>function d1(x,y):=sqrt((x-A[1])^2+(y-A[2])^2)
>fcontour("d1",xmin=-2,xmax=0,ymin=-2,ymax=0,hue=1, ...
>title="If you see ellipses, please set your window square"):
\end{eulerprompt}
\begin{eulercomment}
dan grafiknya juga cukup sederhana: bagian atas kerucut:
\end{eulercomment}
\begin{eulerprompt}
>plot3d("d1",xmin=-2,xmax=0,ymin=-2,ymax=0):
\end{eulerprompt}
\begin{eulercomment}
Tentu saja nilai minimum 0 diperoleh dalam A.


\end{eulercomment}
\eulersubheading{Dua titik}
\begin{eulercomment}
Sekarang kita lihat fungsi MA+MB di mana A dan B adalah dua titik
(tetap). Ini adalah “fakta yang terkenal” bahwa kurva level adalah
elips, titik fokusnya adalah A dan B; kecuali AB minimum yang konstan
pada segmen [AB]:
\end{eulercomment}
\begin{eulerprompt}
>B=[1,-1];
>function d2(x,y):=d1(x,y)+sqrt((x-B[1])^2+(y-B[2])^2)
>fcontour("d2",xmin=-2,xmax=2,ymin=-3,ymax=1,hue=1):
\end{eulerprompt}
\begin{eulercomment}
Grafiknya lebih menarik:
\end{eulercomment}
\begin{eulerprompt}
>plot3d("d2",xmin=-2,xmax=2,ymin=-3,ymax=1):
\end{eulerprompt}
\begin{eulercomment}
Pembatasan pada garis (AB) lebih terkenal:
\end{eulercomment}
\begin{eulerprompt}
>plot2d("abs(x+1)+abs(x-1)",xmin=-3,xmax=3):
\end{eulerprompt}
\begin{eulercomment}
\begin{eulercomment}
\eulerheading{Tiga poin}
\begin{eulercomment}
Sekarang, hal-hal menjadi kurang sederhana: Hal ini sedikit kurang
dikenal bahwa MA+MB+MC mencapai minimumnya pada satu titik di bidang,
tetapi untuk menentukannya tidak sesederhana itu:


1) Jika salah satu sudut segitiga ABC lebih dari 120° (katakanlah di
A), maka minimum dicapai pada titik ini (katakanlah AB+AC).


Contoh:
\end{eulercomment}
\begin{eulerprompt}
>C=[-4,1];
>function d3(x,y):=d2(x,y)+sqrt((x-C[1])^2+(y-C[2])^2)
>plot3d("d3",xmin=-5,xmax=3,ymin=-4,ymax=4);
>insimg;
>fcontour("d3",xmin=-4,xmax=1,ymin=-2,ymax=2,hue=1,title="The minimum is on A");
>P=(A_B_C_A)'; plot2d(P[1],P[2],add=1,color=12);
>insimg;
\end{eulerprompt}
\begin{eulercomment}
2) Tetapi jika semua sudut segitiga ABC kurang dari 120°, minimumnya
adalah pada titik F di bagian dalam segitiga, yang merupakan
satu-satunya titik yang melihat sisi-sisi ABC dengan sudut yang sama
(masing-masing 120°):
\end{eulercomment}
\begin{eulerprompt}
>C=[-0.5,1];
>plot3d("d3",xmin=-2,xmax=2,ymin=-2,ymax=2):
>fcontour("d3",xmin=-2,xmax=2,ymin=-2,ymax=2,hue=1,title="The Fermat point");
>P=(A_B_C_A)'; plot2d(P[1],P[2],add=1,color=12);
>insimg;
\end{eulerprompt}
\begin{eulercomment}
Merupakan kegiatan yang menarik untuk merealisasikan gambar di atas
dengan perangkat lunak geometri; sebagai contoh, saya tahu sebuah
perangkat lunak yang ditulis dalam bahasa Java yang memiliki instruksi
“garis kontur”...


Semua hal di atas telah ditemukan oleh seorang hakim Perancis bernama
Pierre de Fermat; dia menulis surat kepada para ahli lain seperti
pendeta Marin Mersenne dan Blaise Pascal yang bekerja di bagian pajak
penghasilan. Jadi titik unik F sedemikian rupa sehingga FA+FB+FC
minimal, disebut titik Fermat dari segitiga. Namun tampaknya beberapa
tahun sebelumnya, Torriccelli dari Italia telah menemukan titik ini
sebelum Fermat menemukannya! Bagaimanapun juga, tradisinya adalah
mencatat titik F ini...


\end{eulercomment}
\eulersubheading{Empat poin}
\begin{eulercomment}
Langkah selanjutnya adalah menambahkan titik ke-4 D dan mencoba
meminimumkan MA+MB+MC+MD; misalkan Anda adalah operator TV kabel dan
ingin menemukan di bidang mana Anda harus meletakkan antena sehingga
Anda dapat memberi makan empat desa dan menggunakan panjang kabel
sesedikit mungkin!
\end{eulercomment}
\begin{eulerprompt}
>D=[1,1];
>function d4(x,y):=d3(x,y)+sqrt((x-D[1])^2+(y-D[2])^2)
>plot3d("d4",xmin=-1.5,xmax=1.5,ymin=-1.5,ymax=1.5):
>fcontour("d4",xmin=-1.5,xmax=1.5,ymin=-1.5,ymax=1.5,hue=1);
>P=(A_B_C_D)'; plot2d(P[1],P[2],points=1,add=1,color=12);
>insimg;
\end{eulerprompt}
\begin{eulercomment}
Masih ada nilai minimum dan tidak ada simpul A, B, C, maupun D:
\end{eulercomment}
\begin{eulerprompt}
>function f(x):=d4(x[1],x[2])
>neldermin("f",[0.2,0.2])
\end{eulerprompt}
\begin{euleroutput}
  [0.142858,  0.142857]
\end{euleroutput}
\begin{eulercomment}
Tampaknya dalam kasus ini, koordinat titik optimal adalah rasional
atau mendekati rasional...


Karena ABCD adalah sebuah bujur sangkar, kita berharap bahwa titik
optimalnya adalah pusat dari ABCD:
\end{eulercomment}
\begin{eulerprompt}
>C=[-1,1];
>plot3d("d4",xmin=-1,xmax=1,ymin=-1,ymax=1):
>fcontour("d4",xmin=-1.5,xmax=1.5,ymin=-1.5,ymax=1.5,hue=1);
>P=(A_B_C_D)'; plot2d(P[1],P[2],add=1,color=12,points=1);
>insimg;
\end{eulerprompt}
\eulerheading{Contoh 7: Bola Dandelin dengan Povray}
\begin{eulercomment}
Anda dapat menjalankan demonstrasi ini, jika Anda telah menginstal
Povray, dan pvengine.exe pada path program.


Pertama, kita menghitung jari-jari bola.


Jika Anda melihat gambar di bawah ini, Anda dapat melihat bahwa kita
membutuhkan dua lingkaran yang menyentuh dua garis yang membentuk
kerucut, dan satu garis yang membentuk bidang yang memotong kerucut.


Kita menggunakan file geometri.e dari Euler untuk hal ini.
\end{eulercomment}
\begin{eulerprompt}
>load geometry;
\end{eulerprompt}
\begin{eulercomment}
Pertama, dua garis yang membentuk kerucut.
\end{eulercomment}
\begin{eulerprompt}
>g1 &= lineThrough([0,0],[1,a])
\end{eulerprompt}
\begin{euleroutput}
  
                               [- a, 1, 0]
  
\end{euleroutput}
\begin{eulerprompt}
>g2 &= lineThrough([0,0],[-1,a])
\end{eulerprompt}
\begin{euleroutput}
  
                              [- a, - 1, 0]
  
\end{euleroutput}
\begin{eulercomment}
Kemudianm baris ketiga.
\end{eulercomment}
\begin{eulerprompt}
>g &= lineThrough([-1,0],[1,1])
\end{eulerprompt}
\begin{euleroutput}
  
                               [- 1, 2, 1]
  
\end{euleroutput}
\begin{eulercomment}
Kami merencanakan semuanya sejauh ini.
\end{eulercomment}
\begin{eulerprompt}
>setPlotRange(-1,1,0,2);
>color(black); plotLine(g(),"")
>a:=2; color(blue); plotLine(g1(),""), plotLine(g2(),""):
\end{eulerprompt}
\begin{eulercomment}
Sekarang, kita ambil titik umum pada sumbu y.
\end{eulercomment}
\begin{eulerprompt}
>P &= [0,u]
\end{eulerprompt}
\begin{euleroutput}
  
                                  [0, u]
  
\end{euleroutput}
\begin{eulercomment}
Hitung jarak ke g1.
\end{eulercomment}
\begin{eulerprompt}
>d1 &= distance(P,projectToLine(P,g1)); $d1
\end{eulerprompt}
\begin{eulercomment}
Hitung jarak ke g.
\end{eulercomment}
\begin{eulerprompt}
>d &= distance(P,projectToLine(P,g)); $d
\end{eulerprompt}
\begin{eulercomment}
Dan temukan pusat kedua lingkaran, yang jaraknya sama.
\end{eulercomment}
\begin{eulerprompt}
>sol &= solve(d1^2=d^2,u); $sol
\end{eulerprompt}
\begin{eulercomment}
Ada dua solusi.


Kami mengevaluasi solusi simbolis, dan menemukan kedua pusat, dan
kedua jarak.
\end{eulercomment}
\begin{eulerprompt}
>u := sol()
\end{eulerprompt}
\begin{euleroutput}
  [0.333333,  1]
\end{euleroutput}
\begin{eulerprompt}
>dd := d()
\end{eulerprompt}
\begin{euleroutput}
  [0.149071,  0.447214]
\end{euleroutput}
\begin{eulercomment}
Plot lingkaran-lingkaran tersebut ke dalam gambar.
\end{eulercomment}
\begin{eulerprompt}
>color(red);
>plotCircle(circleWithCenter([0,u[1]],dd[1]),"");
>plotCircle(circleWithCenter([0,u[2]],dd[2]),"");
>insimg;
\end{eulerprompt}
\eulersubheading{Plot dengan Povray}
\begin{eulercomment}
Selanjutnya kita plot semuanya dengan Povray. Perhatikan bahwa Anda
mengubah perintah apapun pada urutan perintah Povray berikut ini, dan
jalankan kembali semua perintah dengan Shift-Return.


Pertama kita memuat fungsi povray.
\end{eulercomment}
\begin{eulerprompt}
>load povray;
>defaultpovray="C:\(\backslash\)Program Files\(\backslash\)POV-Ray\(\backslash\)v3.7\(\backslash\)bin\(\backslash\)pvengine.exe"
\end{eulerprompt}
\begin{euleroutput}
  C:\(\backslash\)Program Files\(\backslash\)POV-Ray\(\backslash\)v3.7\(\backslash\)bin\(\backslash\)pvengine.exe
\end{euleroutput}
\begin{eulercomment}
Kami menyiapkan pemandangan dengan tepat.
\end{eulercomment}
\begin{eulerprompt}
>povstart(zoom=11,center=[0,0,0.5],height=10°,angle=140°);
\end{eulerprompt}
\begin{eulercomment}
Selanjutnya kita tulis kedua bola tersebut ke file Povray.
\end{eulercomment}
\begin{eulerprompt}
>writeln(povsphere([0,0,u[1]],dd[1],povlook(red)));
>writeln(povsphere([0,0,u[2]],dd[2],povlook(red)));
\end{eulerprompt}
\begin{eulercomment}
Dan kerucutnya, transparan.
\end{eulercomment}
\begin{eulerprompt}
>writeln(povcone([0,0,0],0,[0,0,a],1,povlook(lightgray,1)));
\end{eulerprompt}
\begin{eulercomment}
Kami menghasilkan bidang yang terbatas pada kerucut.
\end{eulercomment}
\begin{eulerprompt}
>gp=g();
>pc=povcone([0,0,0],0,[0,0,a],1,"");
>vp=[gp[1],0,gp[2]]; dp=gp[3];
>writeln(povplane(vp,dp,povlook(blue,0.5),pc));
\end{eulerprompt}
\begin{eulercomment}
Sekarang kita menghasilkan dua titik pada lingkaran, di mana bola
menyentuh kerucut.
\end{eulercomment}
\begin{eulerprompt}
>function turnz(v) := return [-v[2],v[1],v[3]]
>P1=projectToLine([0,u[1]],g1()); P1=turnz([P1[1],0,P1[2]]);
>writeln(povpoint(P1,povlook(yellow)));
>P2=projectToLine([0,u[2]],g1()); P2=turnz([P2[1],0,P2[2]]);
>writeln(povpoint(P2,povlook(yellow)));
\end{eulerprompt}
\begin{eulercomment}
Kemudian, kita menghasilkan dua titik di mana bola-bola tersebut
menyentuh bidang. Ini adalah fokus elips.
\end{eulercomment}
\begin{eulerprompt}
>P3=projectToLine([0,u[1]],g()); P3=[P3[1],0,P3[2]];
>writeln(povpoint(P3,povlook(yellow)));
>P4=projectToLine([0,u[2]],g()); P4=[P4[1],0,P4[2]];
>writeln(povpoint(P4,povlook(yellow)));
\end{eulerprompt}
\begin{eulercomment}
Selanjutnya kita menghitung perpotongan P1P2 dengan bidang.
\end{eulercomment}
\begin{eulerprompt}
>t1=scalp(vp,P1)-dp; t2=scalp(vp,P2)-dp; P5=P1+t1/(t1-t2)*(P2-P1);
>writeln(povpoint(P5,povlook(yellow)));
\end{eulerprompt}
\begin{eulercomment}
Kami menghubungkan titik-titik dengan segmen garis.
\end{eulercomment}
\begin{eulerprompt}
>writeln(povsegment(P1,P2,povlook(yellow)));
>writeln(povsegment(P5,P3,povlook(yellow)));
>writeln(povsegment(P5,P4,povlook(yellow)));
\end{eulerprompt}
\begin{eulercomment}
Sekarang, kita menghasilkan pita abu-abu, di mana bola-bola menyentuh
kerucut.
\end{eulercomment}
\begin{eulerprompt}
>pcw=povcone([0,0,0],0,[0,0,a],1.01);
>pc1=povcylinder([0,0,P1[3]-defaultpointsize/2],[0,0,P1[3]+defaultpointsize/2],1);
>writeln(povintersection([pcw,pc1],povlook(gray)));
>pc2=povcylinder([0,0,P2[3]-defaultpointsize/2],[0,0,P2[3]+defaultpointsize/2],1);
>writeln(povintersection([pcw,pc2],povlook(gray)));
\end{eulerprompt}
\begin{eulercomment}
Mulai program Povray.
\end{eulercomment}
\begin{eulerprompt}
>povend();
\end{eulerprompt}
\begin{eulercomment}
Untuk mendapatkan Anaglyph ini, kita harus memasukkan semuanya ke
dalam fungsi scene. Fungsi ini akan digunakan dua kali nanti.
\end{eulercomment}
\begin{eulerprompt}
>function scene () ...
\end{eulerprompt}
\begin{eulerudf}
  global a,u,dd,g,g1,defaultpointsize;
  writeln(povsphere([0,0,u[1]],dd[1],povlook(red)));
  writeln(povsphere([0,0,u[2]],dd[2],povlook(red)));
  writeln(povcone([0,0,0],0,[0,0,a],1,povlook(lightgray,1)));
  gp=g();
  pc=povcone([0,0,0],0,[0,0,a],1,"");
  vp=[gp[1],0,gp[2]]; dp=gp[3];
  writeln(povplane(vp,dp,povlook(blue,0.5),pc));
  P1=projectToLine([0,u[1]],g1()); P1=turnz([P1[1],0,P1[2]]);
  writeln(povpoint(P1,povlook(yellow)));
  P2=projectToLine([0,u[2]],g1()); P2=turnz([P2[1],0,P2[2]]);
  writeln(povpoint(P2,povlook(yellow)));
  P3=projectToLine([0,u[1]],g()); P3=[P3[1],0,P3[2]];
  writeln(povpoint(P3,povlook(yellow)));
  P4=projectToLine([0,u[2]],g()); P4=[P4[1],0,P4[2]];
  writeln(povpoint(P4,povlook(yellow)));
  t1=scalp(vp,P1)-dp; t2=scalp(vp,P2)-dp; P5=P1+t1/(t1-t2)*(P2-P1);
  writeln(povpoint(P5,povlook(yellow)));
  writeln(povsegment(P1,P2,povlook(yellow)));
  writeln(povsegment(P5,P3,povlook(yellow)));
  writeln(povsegment(P5,P4,povlook(yellow)));
  pcw=povcone([0,0,0],0,[0,0,a],1.01);
  pc1=povcylinder([0,0,P1[3]-defaultpointsize/2],[0,0,P1[3]+defaultpointsize/2],1);
  writeln(povintersection([pcw,pc1],povlook(gray)));
  pc2=povcylinder([0,0,P2[3]-defaultpointsize/2],[0,0,P2[3]+defaultpointsize/2],1);
  writeln(povintersection([pcw,pc2],povlook(gray)));
  endfunction
\end{eulerudf}
\begin{eulercomment}
Anda memerlukan kacamata merah/sian untuk mengapresiasi efek berikut
ini.
\end{eulercomment}
\begin{eulerprompt}
>povanaglyph("scene",zoom=11,center=[0,0,0.5],height=10°,angle=140°);
\end{eulerprompt}
\eulerheading{Contoh 8: Geometri Bumi}
\begin{eulercomment}
Pada buku catatan ini, kita ingin melakukan beberapa komputasi bola.
Fungsi-fungsi tersebut terdapat pada file “spherical.e” pada folder
contoh. Kita perlu memuat file tersebut terlebih dahulu.
\end{eulercomment}
\begin{eulerprompt}
>load "spherical.e";
\end{eulerprompt}
\begin{eulercomment}
Untuk memasukkan posisi geografis, kita menggunakan vektor dengan dua
koordinat dalam radian (utara dan timur, nilai negatif untuk selatan
dan barat). Berikut ini adalah koordinat untuk Kampus FMIPA UNY.
\end{eulercomment}
\begin{eulerprompt}
>FMIPA=[rad(-7,-46.467),rad(110,23.05)]
\end{eulerprompt}
\begin{euleroutput}
  [-0.13569,  1.92657]
\end{euleroutput}
\begin{eulercomment}
Anda dapat mencetak posisi ini dengan sposprint (cetak posisi bola).
\end{eulercomment}
\begin{eulerprompt}
>sposprint(FMIPA) // posisi garis lintang dan garis bujur FMIPA UNY
\end{eulerprompt}
\begin{euleroutput}
  S 7°46.467' E 110°23.050'
\end{euleroutput}
\begin{eulercomment}
Mari kita tambahkan dua kota lagi, Solo dan Semarang.
\end{eulercomment}
\begin{eulerprompt}
>Solo=[rad(-7,-34.333),rad(110,49.683)]; Semarang=[rad(-6,-59.05),rad(110,24.533)];
>sposprint(Solo), sposprint(Semarang),
\end{eulerprompt}
\begin{euleroutput}
  S 7°34.333' E 110°49.683'
  S 6°59.050' E 110°24.533'
\end{euleroutput}
\begin{eulercomment}
Pertama, kita menghitung vektor dari satu titik ke titik lainnya pada
bola ideal. Vektor ini adalah [heading, jarak] dalam radian. Untuk
menghitung jarak di bumi, kita kalikan dengan jari-jari bumi pada
garis lintang 7°.
\end{eulercomment}
\begin{eulerprompt}
>br=svector(FMIPA,Solo); degprint(br[1]), br[2]*rearth(7°)->km // perkiraan jarak FMIPA-Solo
\end{eulerprompt}
\begin{euleroutput}
  65°20'26.60''
  53.8945384608
\end{euleroutput}
\begin{eulercomment}
Ini adalah perkiraan yang baik. Rutinitas berikut ini menggunakan
perkiraan yang lebih baik lagi. Pada jarak yang pendek, hasilnya
hampir sama.
\end{eulercomment}
\begin{eulerprompt}
>esdist(FMIPA,Semarang)->" km" // perkiraan jarak FMIPA-Semarang
\end{eulerprompt}
\begin{euleroutput}
  Commands must be separated by semicolon or comma!
  Found:  // perkiraan jarak FMIPA-Semarang (character 32)
  You can disable this in the Options menu.
  Error in:
  esdist(FMIPA,Semarang)->" km" // perkiraan jarak FMIPA-Semaran ...
                               ^
\end{euleroutput}
\begin{eulercomment}
Ada fungsi untuk judul, dengan mempertimbangkan bentuk elips bumi.
Sekali lagi, kami mencetak dengan cara yang canggih.
\end{eulercomment}
\begin{eulerprompt}
>sdegprint(esdir(FMIPA,Solo))
\end{eulerprompt}
\begin{euleroutput}
       65.34°
\end{euleroutput}
\begin{eulercomment}
Sudut segitiga melebihi 180° pada bola.
\end{eulercomment}
\begin{eulerprompt}
>asum=sangle(Solo,FMIPA,Semarang)+sangle(FMIPA,Solo,Semarang)+sangle(FMIPA,Semarang,Solo); degprint(asum)
\end{eulerprompt}
\begin{euleroutput}
  180°0'10.77''
\end{euleroutput}
\begin{eulercomment}
Ini bisa digunakan untuk menghitung luas segitiga. Catatan: Untuk
segitiga kecil, cara ini tidak akurat karena kesalahan pengurangan
dalam asum-pi.
\end{eulercomment}
\begin{eulerprompt}
>(asum-pi)*rearth(48°)^2->" km^2" // perkiraan luas segitiga FMIPA-Solo-Semarang
\end{eulerprompt}
\begin{euleroutput}
  Commands must be separated by semicolon or comma!
  Found:  // perkiraan luas segitiga FMIPA-Solo-Semarang (character 32)
  You can disable this in the Options menu.
  Error in:
  (asum-pi)*rearth(48°)^2->" km^2" // perkiraan luas segitiga FM ...
                                  ^
\end{euleroutput}
\begin{eulercomment}
Ada sebuah fungsi untuk hal ini, yang menggunakan garis lintang
rata-rata segitiga untuk menghitung radius bumi, dan menangani
kesalahan pembulatan untuk segitiga yang sangat kecil.
\end{eulercomment}
\begin{eulerprompt}
>esarea(Solo,FMIPA,Semarang)->" km^2", //perkiraan yang sama dengan fungsi esarea()
\end{eulerprompt}
\begin{euleroutput}
  2123.64310526 km^2
\end{euleroutput}
\begin{eulercomment}
Kita juga dapat menambahkan vektor ke posisi. Sebuah vektor berisi
arah dan jarak, keduanya dalam radian. Untuk mendapatkan sebuah
vektor, kita menggunakan svector. Untuk menambahkan sebuah vektor ke
sebuah posisi, kita menggunakan saddvector.
\end{eulercomment}
\begin{eulerprompt}
>v=svector(FMIPA,Solo); sposprint(saddvector(FMIPA,v)), sposprint(Solo),
\end{eulerprompt}
\begin{euleroutput}
  S 7°34.333' E 110°49.683'
  S 7°34.333' E 110°49.683'
\end{euleroutput}
\begin{eulercomment}
Fungsi-fungsi ini mengasumsikan bola yang ideal. Hal yang sama di
bumi.
\end{eulercomment}
\begin{eulerprompt}
>sposprint(esadd(FMIPA,esdir(FMIPA,Solo),esdist(FMIPA,Solo))), sposprint(Solo),
\end{eulerprompt}
\begin{euleroutput}
  S 7°34.333' E 110°49.683'
  S 7°34.333' E 110°49.683'
\end{euleroutput}
\begin{eulercomment}
Mari kita beralih ke contoh yang lebih besar, Tugu Jogja dan Monas
Jakarta (menggunakan Google Earth untuk menemukan koordinatnya).
\end{eulercomment}
\begin{eulerprompt}
>Tugu=[-7.7833°,110.3661°]; Monas=[-6.175°,106.811944°];
>sposprint(Tugu), sposprint(Monas)
\end{eulerprompt}
\begin{euleroutput}
  S 7°46.998' E 110°21.966'
  S 6°10.500' E 106°48.717'
\end{euleroutput}
\begin{eulercomment}
Menurut Google Earth, jaraknya adalah 429,66 km. Kami mendapatkan
perkiraan yang bagus.
\end{eulercomment}
\begin{eulerprompt}
>esdist(Tugu,Monas)->" km" // perkiraan jarak Tugu Jogja - Monas Jakarta
\end{eulerprompt}
\begin{euleroutput}
  Commands must be separated by semicolon or comma!
  Found:  // perkiraan jarak Tugu Jogja - Monas Jakarta (character 32)
  You can disable this in the Options menu.
  Error in:
  esdist(Tugu,Monas)->" km" // perkiraan jarak Tugu Jogja - Mona ...
                           ^
\end{euleroutput}
\begin{eulercomment}
Judulnya sama dengan yang dihitung di Google Earth.
\end{eulercomment}
\begin{eulerprompt}
>degprint(esdir(Tugu,Monas))
\end{eulerprompt}
\begin{euleroutput}
  294°17'2.85''
\end{euleroutput}
\begin{eulercomment}
Namun demikian, kita tidak lagi mendapatkan posisi target yang tepat,
jika kita menambahkan arah dan jarak ke posisi semula. Hal ini
terjadi, karena kita tidak menghitung fungsi inversi secara tepat,
tetapi mengambil perkiraan radius bumi di sepanjang jalur.
\end{eulercomment}
\begin{eulerprompt}
>sposprint(esadd(Tugu,esdir(Tugu,Monas),esdist(Tugu,Monas)))
\end{eulerprompt}
\begin{euleroutput}
  S 6°10.500' E 106°48.717'
\end{euleroutput}
\begin{eulercomment}
Namun demikian, kesalahannya tidak besar.
\end{eulercomment}
\begin{eulerprompt}
>sposprint(Monas),
\end{eulerprompt}
\begin{euleroutput}
  S 6°10.500' E 106°48.717'
\end{euleroutput}
\begin{eulercomment}
Tentu saja, kita tidak bisa berlayar dengan arah yang sama dari satu
tujuan ke tujuan lainnya, jika kita ingin mengambil jalur terpendek.
Bayangkan, Anda terbang ke arah NE mulai dari titik mana pun di bumi.
Kemudian Anda akan berputar ke kutub utara. Lingkaran besar tidak
mengikuti arah yang konstan!


Perhitungan berikut ini menunjukkan bahwa kita akan melenceng dari
tujuan yang benar, jika kita menggunakan arah yang sama selama
perjalanan.
\end{eulercomment}
\begin{eulerprompt}
>dist=esdist(Tugu,Monas); hd=esdir(Tugu,Monas);
\end{eulerprompt}
\begin{eulercomment}
Sekarang kita tambahkan 10 kali sepersepuluh dari jarak tersebut,
dengan menggunakan arah menuju Monas, kita akan sampai di Tugu.
\end{eulercomment}
\begin{eulerprompt}
>p=Tugu; loop 1 to 10; p=esadd(p,hd,dist/10); end;
\end{eulerprompt}
\begin{eulercomment}
Hasilnya jauh berbeda.
\end{eulercomment}
\begin{eulerprompt}
>sposprint(p), skmprint(esdist(p,Monas))
\end{eulerprompt}
\begin{euleroutput}
  S 6°11.250' E 106°48.372'
       1.529km
\end{euleroutput}
\begin{eulercomment}
Sebagai contoh lain, mari kita ambil dua titik di bumi pada garis
lintang yang sama.
\end{eulercomment}
\begin{eulerprompt}
>P1=[30°,10°]; P2=[30°,50°];
\end{eulerprompt}
\begin{eulercomment}
Jalur terpendek dari P1 ke P2 bukanlah lingkaran lintang 30°, tetapi
jalur yang lebih pendek yang dimulai 10° lebih jauh ke utara di P1.
\end{eulercomment}
\begin{eulerprompt}
>sdegprint(esdir(P1,P2))
\end{eulerprompt}
\begin{euleroutput}
       79.69°
\end{euleroutput}
\begin{eulercomment}
Namun, jika kita mengikuti pembacaan kompas ini, kita akan berputar ke
kutub utara! Jadi, kita harus menyesuaikan arah kita di sepanjang
jalan. Untuk tujuan kasar, kita menyesuaikannya pada 1/10 dari jarak
total.
\end{eulercomment}
\begin{eulerprompt}
>p=P1;  dist=esdist(P1,P2); ...
>  loop 1 to 10; dir=esdir(p,P2); sdegprint(dir), p=esadd(p,dir,dist/10); end;
\end{eulerprompt}
\begin{euleroutput}
       79.69°
       81.67°
       83.71°
       85.78°
       87.89°
       90.00°
       92.12°
       94.22°
       96.29°
       98.33°
\end{euleroutput}
\begin{eulercomment}
Jaraknya tidak tepat, karena kita akan menambahkan sedikit kesalahan,
jika kita mengikuti judul yang sama terlalu lama.
\end{eulercomment}
\begin{eulerprompt}
>skmprint(esdist(p,P2))
\end{eulerprompt}
\begin{euleroutput}
       0.203km
\end{euleroutput}
\begin{eulercomment}
Kita akan mendapatkan perkiraan yang baik, jika kita menyesuaikan arah
setiap 1/100 dari total jarak dari Tugu ke Monas.
\end{eulercomment}
\begin{eulerprompt}
>p=Tugu; dist=esdist(Tugu,Monas); ...
>  loop 1 to 100; p=esadd(p,esdir(p,Monas),dist/100); end;
>skmprint(esdist(p,Monas))
\end{eulerprompt}
\begin{euleroutput}
       0.000km
\end{euleroutput}
\begin{eulercomment}
Untuk keperluan navigasi, kita bisa mendapatkan urutan posisi GPS di
sepanjang Bundaran Hotel Indonesia menuju Monas dengan fungsi
navigate.
\end{eulercomment}
\begin{eulerprompt}
>load spherical; v=navigate(Tugu,Monas,10); ...
>  loop 1 to rows(v); sposprint(v[#]), end;
\end{eulerprompt}
\begin{euleroutput}
  S 7°46.998' E 110°21.966'
  S 7°37.422' E 110°0.573'
  S 7°27.829' E 109°39.196'
  S 7°18.219' E 109°17.834'
  S 7°8.592' E 108°56.488'
  S 6°58.948' E 108°35.157'
  S 6°49.289' E 108°13.841'
  S 6°39.614' E 107°52.539'
  S 6°29.924' E 107°31.251'
  S 6°20.219' E 107°9.977'
  S 6°10.500' E 106°48.717'
\end{euleroutput}
\begin{eulercomment}
Kami menulis sebuah fungsi, yang memplot bumi, dua posisi, dan posisi
di antaranya.
\end{eulercomment}
\begin{eulerprompt}
>function testplot ...
\end{eulerprompt}
\begin{eulerudf}
  useglobal;
  plotearth;
  plotpos(Tugu,"Tugu Jogja"); plotpos(Monas,"Tugu Monas");
  plotposline(v);
  endfunction
\end{eulerudf}
\begin{eulercomment}
Now plot everything.
\end{eulercomment}
\begin{eulerprompt}
>plot3d("testplot",angle=25, height=6,>own,>user,zoom=4):
\end{eulerprompt}
\begin{eulercomment}
Atau gunakan plot3d untuk mendapatkan tampilan anaglyph. Ini terlihat
sangat bagus dengan kacamata merah/cyan.
\end{eulercomment}
\begin{eulerprompt}
>plot3d("testplot",angle=25,height=6,distance=5,own=1,anaglyph=1,zoom=4):
\end{eulerprompt}
\eulerheading{Latihan}
\begin{eulercomment}
1. Gambarlah segi-n beraturan jika diketahui titik pusat O, n, dan
jarak titik pusat ke titik-titik sudut segi-n tersebut (jari-jari
lingkaran luar segi-n), r.

Petunjuk:

- Besar sudut pusat yang menghadap masing-masing sisi segi-n adalah
(360/n).\\
- Titik-titik sudut segi-n merupakan perpotongan lingkaran luar segi-n
dan garis-garis yang melalui pusat dan saling membentuk sudut sebesar
kelipatan (360/n).\\
- Untuk n ganjil, pilih salah satu titik sudut adalah di atas.\\
- Untuk n genap, pilih 2 titik di kanan dan kiri lurus dengan titik
pusat.\\
- Anda dapat menggambar segi-3, 4, 5, 6, 7, dst beraturan.
\end{eulercomment}
\begin{eulerprompt}
>load geometry;
>setPlotRange(-3.5,3.5,-3.5,3.5);
>A=[-2,-2]; plotPoint(A,"A");
>B=[2,-2]; plotPoint(B,"B");
>C=[0,3]; plotPoint(C,"C");
>plotSegment(A,B,"c");
>plotSegment(B,C,"a");
>plotSegment(A,C,"b");
>aspect(1):
>c=circleThrough(A,B,C);
>R=getCircleRadius(c);
>O=getCircleCenter(c);
>plotPoint(O,"O");
>l=angleBisector(A,C,B);
>color(2); plotLine(l); color(1);
>plotCircle(c,"Lingkaran luar segitiga ABC"):
\end{eulerprompt}
\begin{eulercomment}
2. Gambarlah suatu parabola yang melalui 3 titik yang diketahui.\\
Petunjuk:\\
- Misalkan persamaan parabolanya y= ax\textasciicircum{}2+bx+c.\\
- Substitusikan koordinat titik-titik yang diketahui ke persamaan
tersebut.\\
- Selesaikan SPL yang terbentuk untuk mendapatkan nilai-nilai a, b, c.
\end{eulercomment}
\begin{eulerprompt}
>load geometry;
>setPlotRange(5); P=[2,0]; Q=[4,0]; R=[0,-4];
>plotPoint(P,"P"); plotPoint(Q,"Q"); plotPoint(R,"R"):
>sol &= solve([a+b=-c,16*a+4*b=-c,c=-4],[a,b,c])
\end{eulerprompt}
\begin{euleroutput}
  
                       [[a = - 1, b = 5, c = - 4]]
  
\end{euleroutput}
\begin{eulerprompt}
>function y&=-x^2+5*x-4
\end{eulerprompt}
\begin{euleroutput}
  
                                 2
                              - x  + 5 x - 4
  
\end{euleroutput}
\begin{eulerprompt}
>plot2d("-x^2+5*x-4",-5,5,-5,5):
\end{eulerprompt}
\begin{eulercomment}
3. Gambarlah suatu segi-4 yang diketahui keempat titik sudutnya,
misalnya A, B, C, D.\\
\end{eulercomment}
\begin{eulerttcomment}
   - Tentukan apakah segi-4 tersebut merupakan segi-4 garis singgung
\end{eulerttcomment}
\begin{eulercomment}
(sisinya-sisintya merupakan garis singgung lingkaran yang sama yakni
lingkaran dalam segi-4 tersebut).\\
\end{eulercomment}
\begin{eulerttcomment}
   - Suatu segi-4 merupakan segi-4 garis singgung apabila keempat
\end{eulerttcomment}
\begin{eulercomment}
garis bagi sudutnya bertemu di satu titik.\\
\end{eulercomment}
\begin{eulerttcomment}
   - Jika segi-4 tersebut merupakan segi-4 garis singgung, gambar
\end{eulerttcomment}
\begin{eulercomment}
lingkaran dalamnya.\\
\end{eulercomment}
\begin{eulerttcomment}
   - Tunjukkan bahwa syarat suatu segi-4 merupakan segi-4 garis
\end{eulerttcomment}
\begin{eulercomment}
singgung apabila hasil kali panjang sisi-sisi yang berhadapan sama.
\end{eulercomment}
\begin{eulerprompt}
>load geometry;
>setPlotRange(-4.5,4.5,-4.5,4.5);
>A=[-3,-3]; plotPoint(A,"A");
>B=[3,-3]; plotPoint(B,"B");
>C=[3,3]; plotPoint(C,"C");
>D=[-3,3]; plotPoint(D,"D");
>plotSegment(A,B,"");
>plotSegment(B,C,"");
>plotSegment(C,D,"");
>plotSegment(A,D,"");
>aspect(1):
>l=angleBisector(A,B,C);
>m=angleBisector(B,C,D);
>P=lineIntersection(l,m);
>color(5); plotLine(l); plotLine(m); color(1);
>plotPoint(P,"P"):
>r=norm(P-projectToLine(P,lineThrough(A,B)));
>plotCircle(circleWithCenter(P,r),"Lingkaran dalam segiempat ABCD"):
>AB=norm(A-B) //panjang sisi AB
\end{eulerprompt}
\begin{euleroutput}
  6
\end{euleroutput}
\begin{eulerprompt}
>CD=norm(C-D) //panjang sisi CD
\end{eulerprompt}
\begin{euleroutput}
  6
\end{euleroutput}
\begin{eulerprompt}
>AD=norm(A-D) //panjang sisi AD
\end{eulerprompt}
\begin{euleroutput}
  6
\end{euleroutput}
\begin{eulerprompt}
>BC=norm(B-C) //panjang sisi BC
\end{eulerprompt}
\begin{euleroutput}
  6
\end{euleroutput}
\begin{eulerprompt}
>AB.CD
\end{eulerprompt}
\begin{euleroutput}
  36
\end{euleroutput}
\begin{eulerprompt}
>AD.BC
\end{eulerprompt}
\begin{euleroutput}
  36
\end{euleroutput}
\begin{eulercomment}
4. Gambarlah suatu ellips jika diketahui kedua titik fokusnya,
misalnya P dan Q. Ingat ellips dengan fokus P dan Q adalah tempat
kedudukan titik-titik yang jumlah jarak ke P dan ke Q selalu sama
(konstan).
\end{eulercomment}
\begin{eulerprompt}
>P=[-3,-1]; Q=[3,-1];
>function d1(x,y):=sqrt((x-P[1])^2+(y-P[2])^2)
>Q=[3,-1]; function d2(x,y):=sqrt((x-P[1])^2+(y-P[2])^2)+sqrt((x-Q[1])^2+(y-Q[2])^2)
>fcontour("d2",xmin=-2,xmax=2,ymin=-3,ymax=1,hue=1):
>plot3d("d2",xmin=-2,xmax=2,ymin=-3,ymax=1):
>plot2d("abs(x+1)+abs(x-1)",xmin=-3,xmax=3):
\end{eulerprompt}
\begin{eulercomment}
5. Gambarlah suatu hiperbola jika diketahui kedua titik fokusnya,
misalnya P dan Q. Ingat ellips dengan fokus P dan Q adalah tempat
kedudukan titik-titik yang selisih jarak ke P dan ke Q selalu sama
(konstan).
\end{eulercomment}
\begin{eulerprompt}
>P=[-1,-1]; Q=[1,-1];
>function d1(x,y):=sqrt((x-p[1])^2+(y-p[2])^2)
>Q=[1,-1]; function d2(x,y):=sqrt((x-P[1])^2+(y-P[2])^2)+sqrt((x+Q[1])^2+(y+Q[2])^2)
>fcontour("d2",xmin=-2,xmax=2,ymin=-3,ymax=1,hue=1):
>plot3d("d2",xmin=-2,xmax=2,ymin=-3,ymax=1):
>plot2d("abs(x+1)+abs(x-1)",xmin=-3,xmax=3):
\end{eulerprompt}
\begin{euleroutput}
  
\end{euleroutput}
\eulersubheading{}
\begin{eulercomment}
\begin{eulercomment}
\eulerheading{EMT untuk Statistika}
\begin{eulercomment}
Dalam buku catatan ini, kami mendemonstrasikan plot statistik utama,
tes dan distribusi dalam Euler.


Mari kita mulai dengan beberapa statistik deskriptif. Ini bukanlah
sebuah pengantar statistik. Jadi, Anda mungkin memerlukan beberapa
latar belakang untuk memahami detailnya.


Asumsikan pengukuran berikut ini. Kita ingin menghitung nilai
rata-rata dan deviasi standar yang diukur.
\end{eulercomment}
\begin{eulerprompt}
>M=[1000,1004,998,997,1002,1001,998,1004,998,997]; ...
>median(M), mean(M), dev(M),
\end{eulerprompt}
\begin{euleroutput}
  999
  999.9
  2.72641400622
\end{euleroutput}
\begin{eulercomment}
Kita dapat memplot plot kotak dan kumis untuk data tersebut. Dalam
kasus kami, tidak ada pencilan.
\end{eulercomment}
\begin{eulerprompt}
>aspect(1.75); boxplot(M):
\end{eulerprompt}
\eulerimg{15}{images/EMT 1-6_Mifta Dilla Aninditya_23030630029-578.png}
\begin{eulercomment}
Kami menghitung probabilitas bahwa suatu nilai lebih besar dari 1005,
dengan mengasumsikan nilai yang diukur dari distribusi normal.


Semua fungsi untuk distribusi dalam Euler diakhiri dengan ...dis dan
menghitung distribusi probabilitas kumulatif (CPF).

\end{eulercomment}
\begin{eulerformula}
\[
\text{normaldis(x,m,d)}=\int_{-\infty}^x \frac{1}{d\sqrt{2\pi}}e^{-\frac{1}{2}(\frac{t-m}{d})^2}\ dt.
\]
\end{eulerformula}
\begin{eulercomment}
Kami mencetak hasilnya dalam \% dengan akurasi 2 digit menggunakan
fungsi cetak.
\end{eulercomment}
\begin{eulerprompt}
>print((1-normaldis(1005,mean(M),dev(M)))*100,2,unit=" %")
\end{eulerprompt}
\begin{euleroutput}
        3.07 %
\end{euleroutput}
\begin{eulercomment}
Untuk contoh berikutnya, kami mengasumsikan jumlah pria berikut ini
dalam rentang ukuran tertentu.
\end{eulercomment}
\begin{eulerprompt}
>r=155.5:4:187.5; v=[22,71,136,169,139,71,32,8];
\end{eulerprompt}
\begin{eulercomment}
Berikut ini adalah plot distribusinya.
\end{eulercomment}
\begin{eulerprompt}
>plot2d(r,v,a=150,b=200,c=0,d=190,bar=1,style="\(\backslash\)/"):
\end{eulerprompt}
\eulerimg{15}{images/EMT 1-6_Mifta Dilla Aninditya_23030630029-579.png}
\begin{eulercomment}
Kita dapat memasukkan data mentah tersebut ke dalam tabel.


Tabel adalah sebuah metode untuk menyimpan data statistik. Tabel kita
harus berisi tiga kolom: Awal rentang, akhir rentang, jumlah orang
dalam rentang.


Tabel dapat dicetak dengan header. Kami menggunakan vektor string
untuk mengatur header.
\end{eulercomment}
\begin{eulerprompt}
>T:=r[1:8]' | r[2:9]' | v'; writetable(T,labc=["BB","BA","Frek"])
\end{eulerprompt}
\begin{euleroutput}
          BB        BA      Frek
       155.5     159.5        22
       159.5     163.5        71
       163.5     167.5       136
       167.5     171.5       169
       171.5     175.5       139
       175.5     179.5        71
       179.5     183.5        32
       183.5     187.5         8
\end{euleroutput}
\begin{eulercomment}
Jika kita membutuhkan nilai rata-rata dan statistik lain dari ukuran,
kita perlu menghitung titik tengah rentang. Kita dapat menggunakan dua
kolom pertama dari tabel kita untuk hal ini.


Sumbol “\textbar{}” digunakan untuk memisahkan kolom, fungsi “writetable”
digunakan untuk menulis tabel, dengan opsi “labc” untuk menentukan
judul kolom.
\end{eulercomment}
\begin{eulerprompt}
>(T[,1]+T[,2])/2 // the midpoint of each interval
\end{eulerprompt}
\begin{euleroutput}
          157.5 
          161.5 
          165.5 
          169.5 
          173.5 
          177.5 
          181.5 
          185.5 
\end{euleroutput}
\begin{eulercomment}
Tetapi akan lebih mudah, untuk melipat rentang dengan vektor
[1/2,1/2].
\end{eulercomment}
\begin{eulerprompt}
>M=fold(r,[0.5,0.5])
\end{eulerprompt}
\begin{euleroutput}
  [157.5,  161.5,  165.5,  169.5,  173.5,  177.5,  181.5,  185.5]
\end{euleroutput}
\begin{eulercomment}
Sekarang kita dapat menghitung rata-rata dan deviasi sampel dengan
frekuensi yang diberikan.
\end{eulercomment}
\begin{eulerprompt}
>\{m,d\}=meandev(M,v); m, d,
\end{eulerprompt}
\begin{euleroutput}
  169.901234568
  5.98912964449
\end{euleroutput}
\begin{eulercomment}
Mari kita tambahkan distribusi normal dari nilai-nilai tersebut ke
dalam diagram batang di atas. Rumus untuk distribusi normal dengan
rata-rata m dan deviasi standar d adalah:

\end{eulercomment}
\begin{eulerformula}
\[
y=\frac{1}{d\sqrt{2\pi}}e^{\frac{-(x-m)^2}{2d^2}}.
\]
\end{eulerformula}
\begin{eulercomment}
Karena nilainya antara 0 dan 1, untuk memplotnya pada diagram batang,
nilai tersebut harus dikalikan dengan 4 kali jumlah data.
\end{eulercomment}
\begin{eulerprompt}
>plot2d("qnormal(x,m,d)*sum(v)*4", ...
>  xmin=min(r),xmax=max(r),thickness=3,add=1):
\end{eulerprompt}
\eulerimg{15}{images/EMT 1-6_Mifta Dilla Aninditya_23030630029-580.png}
\eulerheading{Tabel}
\begin{eulercomment}
Dalam direktori buku catatan ini, Anda dapat menemukan file dengan
tabel. Data tersebut mewakili hasil survei. Berikut adalah empat baris
pertama dari file tersebut. Data berasal dari buku online berbahasa
Jerman “Einführung in die Statistik mit R” oleh A. Handl.
\end{eulercomment}
\begin{eulerprompt}
>printfile("table.dat",4);
\end{eulerprompt}
\begin{euleroutput}
  Could not open the file
  table.dat
  for reading!
  Try "trace errors" to inspect local variables after errors.
  printfile:
      open(filename,"r");
\end{euleroutput}
\begin{eulercomment}
Tabel berisi 7 kolom angka atau token (string). Kita ingin membaca
tabel tersebut dari file. Pertama, kita menggunakan terjemahan kita
sendiri untuk token-token tersebut.


Untuk itu, kita mendefinisikan set token. Fungsi strtokens()
mendapatkan vektor string token dari string yang diberikan.
\end{eulercomment}
\begin{eulerprompt}
>mf:=["m","f"]; yn:=["y","n"]; ev:=strtokens("g vg m b vb");
\end{eulerprompt}
\begin{eulercomment}
Sekarang kita membaca tabel dengan terjemahan ini.


Argumen tok2, tok4, dan lain-lain adalah terjemahan dari kolom-kolom
tabel. Argumen-argumen ini tidak ada dalam daftar parameter
readtable(), jadi Anda perlu memberikannya dengan “:=”.
\end{eulercomment}
\begin{eulerprompt}
>\{MT,hd\}=readtable("table.dat",tok2:=mf,tok4:=yn,tok5:=ev,tok7:=yn);
\end{eulerprompt}
\begin{euleroutput}
  Could not open the file
  table.dat
  for reading!
  Try "trace errors" to inspect local variables after errors.
  readtable:
      if filename!=none then open(filename,"r"); endif;
\end{euleroutput}
\begin{eulerprompt}
>load over statistics;
\end{eulerprompt}
\begin{eulercomment}
Untuk mencetak, kita perlu menentukan set token yang sama. Kami
mencetak empat baris pertama saja.
\end{eulercomment}
\begin{eulerprompt}
>writetable(MT[1:10],labc=hd,wc=5,tok2:=mf,tok4:=yn,tok5:=ev,tok7:=yn);
\end{eulerprompt}
\begin{euleroutput}
  MT is not a variable!
  Error in:
  writetable(MT[1:10],labc=hd,wc=5,tok2:=mf,tok4:=yn,tok5:=ev,to ...
                     ^
\end{euleroutput}
\begin{eulercomment}
Tanda titik “.” mewakili nilai yang tidak tersedia.


Jika kita tidak ingin menentukan token untuk terjemahan sebelumnya,
kita hanya perlu menentukan kolom mana yang berisi token dan bukan
angka.
\end{eulercomment}
\begin{eulerprompt}
>ctok=[2,4,5,7]; \{MT,hd,tok\}=readtable("table.dat",ctok=ctok);
\end{eulerprompt}
\begin{euleroutput}
  Could not open the file
  table.dat
  for reading!
  Try "trace errors" to inspect local variables after errors.
  readtable:
      if filename!=none then open(filename,"r"); endif;
\end{euleroutput}
\begin{eulercomment}
The function readtable() now returns a set of tokens.
\end{eulercomment}
\begin{eulerprompt}
>tok
\end{eulerprompt}
\begin{euleroutput}
  Variable tok not found!
  Error in:
  tok ...
     ^
\end{euleroutput}
\begin{eulercomment}
Tabel berisi entri dari file dengan token yang diterjemahkan menjadi
angka.


String khusus NA=“.” ditafsirkan sebagai “Tidak Tersedia”, dan
mendapatkan NAN (bukan angka) dalam tabel. Terjemahan ini dapat diubah
dengan parameter NA, dan NAval.
\end{eulercomment}
\begin{eulerprompt}
>MT[1]
\end{eulerprompt}
\begin{euleroutput}
  MT is not a variable!
  Error in:
  MT[1] ...
       ^
\end{euleroutput}
\begin{eulercomment}
Here is the content of the table with untranslated numbers.
\end{eulercomment}
\begin{eulerprompt}
>writetable(MT,wc=5)
\end{eulerprompt}
\begin{euleroutput}
  Variable or function MT not found.
  Error in:
  writetable(MT,wc=5) ...
               ^
\end{euleroutput}
\begin{eulercomment}
For convenience, you can put the output of readtable() into a list.
\end{eulercomment}
\begin{eulerprompt}
>Table=\{\{readtable("table.dat",ctok=ctok)\}\};
\end{eulerprompt}
\begin{euleroutput}
  Could not open the file
  table.dat
  for reading!
  Try "trace errors" to inspect local variables after errors.
  readtable:
      if filename!=none then open(filename,"r"); endif;
\end{euleroutput}
\begin{eulercomment}
Dengan menggunakan kolom token yang sama dan token yang dibaca dari
file, kita dapat mencetak tabel. Kita dapat menentukan ctok, tok, dll.
atau menggunakan daftar Tabel.
\end{eulercomment}
\begin{eulerprompt}
>writetable(Table,ctok=ctok,wc=5);
\end{eulerprompt}
\begin{euleroutput}
  Variable or function Table not found.
  Error in:
  writetable(Table,ctok=ctok,wc=5); ...
                  ^
\end{euleroutput}
\begin{eulercomment}
Fungsi tablecol() mengembalikan nilai kolom dari tabel, melewatkan
setiap baris dengan nilai NAN (“.” dalam file), dan indeks kolom, yang
berisi nilai-nilai ini.
\end{eulercomment}
\begin{eulerprompt}
>\{c,i\}=tablecol(MT,[5,6]);
\end{eulerprompt}
\begin{euleroutput}
  Variable or function MT not found.
  Error in:
  \{c,i\}=tablecol(MT,[5,6]); ...
                   ^
\end{euleroutput}
\begin{eulercomment}
Kita dapat menggunakan ini untuk mengekstrak kolom dari tabel untuk
tabel baru.
\end{eulercomment}
\begin{eulerprompt}
>j=[1,5,6]; writetable(MT[i,j],labc=hd[j],ctok=[2],tok=tok)
\end{eulerprompt}
\begin{euleroutput}
  MT is not a variable!
  Error in:
  j=[1,5,6]; writetable(MT[i,j],labc=hd[j],ctok=[2],tok=tok) ...
                               ^
\end{euleroutput}
\begin{eulercomment}
Tentu saja, kita perlu mengekstrak tabel itu sendiri dari daftar Tabel
dalam kasus ini.
\end{eulercomment}
\begin{eulerprompt}
>MT=Table[1];
\end{eulerprompt}
\begin{euleroutput}
  Table is not a variable!
  Error in:
  MT=Table[1]; ...
             ^
\end{euleroutput}
\begin{eulercomment}
Tentu saja, kita juga dapat menggunakannya untuk menentukan nilai
rata-rata kolom atau nilai statistik lainnya.
\end{eulercomment}
\begin{eulerprompt}
>mean(tablecol(MT,6))
\end{eulerprompt}
\begin{euleroutput}
  Variable or function MT not found.
  Error in:
  mean(tablecol(MT,6)) ...
                  ^
\end{euleroutput}
\begin{eulercomment}
Fungsi getstatistics() mengembalikan elemen-elemen dalam sebuah
vektor, dan jumlahnya. Kita menerapkannya pada nilai “m” dan “f” pada
kolom kedua tabel kita.
\end{eulercomment}
\begin{eulerprompt}
>\{xu,count\}=getstatistics(tablecol(MT,2)); xu, count,
\end{eulerprompt}
\begin{euleroutput}
  Variable or function MT not found.
  Error in:
  \{xu,count\}=getstatistics(tablecol(MT,2)); xu, count, ...
                                      ^
\end{euleroutput}
\begin{eulercomment}
Kita bisa mencetak hasilnya dalam tabel baru.
\end{eulercomment}
\begin{eulerprompt}
>writetable(count',labr=tok[xu])
\end{eulerprompt}
\begin{euleroutput}
  Variable count not found!
  Error in:
  writetable(count',labr=tok[xu]) ...
                   ^
\end{euleroutput}
\begin{eulercomment}
Fungsi selecttable() mengembalikan sebuah tabel baru dengan nilai
dalam satu kolom yang dipilih dari vektor indeks. Pertama, kita
mencari indeks dari dua nilai kita dalam tabel token.
\end{eulercomment}
\begin{eulerprompt}
>v:=indexof(tok,["g","vg"])
\end{eulerprompt}
\begin{euleroutput}
  Variable or function tok not found.
  Error in:
  v:=indexof(tok,["g","vg"]) ...
                ^
\end{euleroutput}
\begin{eulercomment}
Sekarang kita dapat memilih baris-baris dari tabel, yang memiliki
salah satu nilai dalam v di baris ke-5.
\end{eulercomment}
\begin{eulerprompt}
>MT1:=MT[selectrows(MT,5,v)]; i:=sortedrows(MT1,5);
\end{eulerprompt}
\begin{euleroutput}
  Variable or function MT not found.
  Error in:
  MT1:=MT[selectrows(MT,5,v)]; i:=sortedrows(MT1,5); ...
                       ^
\end{euleroutput}
\begin{eulercomment}
Sekarang kita dapat mencetak tabel, dengan nilai yang diekstrak dan
diurutkan di kolom ke-5.
\end{eulercomment}
\begin{eulerprompt}
>writetable(MT1[i],labc=hd,ctok=ctok,tok=tok,wc=7);
\end{eulerprompt}
\begin{euleroutput}
   Person    Sex    Age Titanic Evaluation    Tip Problem
        2      f     23       y          g    1.8       n
        3      f     26       y          g    1.8       y
        6      m     28       y          g    2.8       y
       18      m     38       y          g      .       n
       16      m     26       y          g    2.8       n
       15      f     31       y          g    0.8       n
       12      m     32       y          g    1.8       n
       23      f     38       y          g    2.8       n
       14      f     25       y          g    1.8       y
        9      f     24       y         vg    1.8       y
        7      f     31       y         vg    2.8       n
       20      f     28       y         vg    1.8       n
       22      f     28       y         vg    1.8       y
       13      m     29       y         vg    1.8       y
       11      f     23       y         vg    1.8       y
\end{euleroutput}
\begin{eulercomment}
Untuk statistik berikutnya, kita ingin menghubungkan dua kolom tabel.
Jadi kita mengekstrak kolom 2 dan 4 dan mengurutkan tabel.
\end{eulercomment}
\begin{eulerprompt}
>i=sortedrows(MT,[2,4]);  ...
>  writetable(tablecol(MT[i],[2,4])',ctok=[1,2],tok=tok)
\end{eulerprompt}
\begin{euleroutput}
           m         n
           m         n
           m         n
           m         n
           m         n
           m         n
           m         n
           m         y
           m         y
           m         y
           m         y
           m         y
           f         n
           f         y
           f         y
           f         y
           f         y
           f         y
           f         y
           f         y
           f         y
           f         y
           f         y
           f         y
           f         y
\end{euleroutput}
\begin{eulercomment}
Dengan getstatistics(), kita juga dapat menghubungkan hitungan dalam
dua kolom tabel satu sama lain.
\end{eulercomment}
\begin{eulerprompt}
>MT24=tablecol(MT,[2,4]); ...
>\{xu1,xu2,count\}=getstatistics(MT24[1],MT24[2]); ...
>writetable(count,labr=tok[xu1],labc=tok[xu2])
\end{eulerprompt}
\begin{euleroutput}
                     n         y
           m         7         5
           f         1        12
\end{euleroutput}
\begin{eulercomment}
Tabel dapat ditulis ke sebuah file.
\end{eulercomment}
\begin{eulerprompt}
>filename="test.dat"; ...
>writetable(count,labr=tok[xu1],labc=tok[xu2],file=filename);
\end{eulerprompt}
\begin{eulercomment}
Then we can read the table from the file.
\end{eulercomment}
\begin{eulerprompt}
>\{MT2,hd,tok2,hdr\}=readtable(filename,>clabs,>rlabs); ...
>writetable(MT2,labr=hdr,labc=hd)
\end{eulerprompt}
\begin{euleroutput}
                     n         y
           m         7         5
           f         1        12
\end{euleroutput}
\begin{eulercomment}
And delete the file.
\end{eulercomment}
\begin{eulerprompt}
>fileremove(filename);
\end{eulerprompt}
\eulerheading{Distribusi}
\begin{eulercomment}
Dengan plot2d, ada metode yang sangat mudah untuk memplot distribusi
data eksperimen.
\end{eulercomment}
\begin{eulerprompt}
>p=normal(1,1000); //1000 random normal-distributed sample p
>plot2d(p,distribution=20,style="\(\backslash\)/"); // plot the random sample p
>plot2d("qnormal(x,0,1)",add=1): // add the standard normal distribution plot
\end{eulerprompt}
\eulerimg{15}{images/EMT 1-6_Mifta Dilla Aninditya_23030630029-581.png}
\begin{eulercomment}
Please note the different between the bar plot (sample) and the normal curve (the real
distribution) . Reenter the three commands to see another sampling result.
\end{eulercomment}
\begin{eulercomment}
Here is a comparison of 10 simulations of 1000 normal distributed values using a so-called box
plot. This plot shows the median, the 25\% and 75\% quartiles, the minimal and maximal values,
and the outliers.
\end{eulercomment}
\begin{eulerprompt}
>p=normal(10,1000); boxplot(p):
\end{eulerprompt}
\eulerimg{15}{images/EMT 1-6_Mifta Dilla Aninditya_23030630029-582.png}
\begin{eulercomment}
To generate random integers, Euler has intrandom. Let us simulate dice throws and plot the
distribution.

We use the getmultiplicities(v,x) function, which counts how often the elements of v appear in
x. Then we plot the the result using columnsplot().
\end{eulercomment}
\begin{eulerprompt}
>k=intrandom(1,6000,6);  ...
>columnsplot(getmultiplicities(1:6,k));  ...
>ygrid(1000,color=red):
\end{eulerprompt}
\eulerimg{15}{images/EMT 1-6_Mifta Dilla Aninditya_23030630029-583.png}
\begin{eulercomment}
While intrandom(n,m,k) returns uniformly distributed integers from 1 to k, it is possible to
use any other given distribution of integers with randpint().

In the following example, the probabilities for 1,2,3 are 0.4,0.1,0.5 respectively.
\end{eulercomment}
\begin{eulerprompt}
>randpint(1,1000,[0.4,0.1,0.5]); getmultiplicities(1:3,%)
\end{eulerprompt}
\begin{euleroutput}
  [404,  93,  503]
\end{euleroutput}
\begin{eulercomment}
Euler can produce random values from more distributions. Have a look
into the reference.

E.g., we try the exponential distribution. A continuous random
variable X is said to have an exponential distribution, if its PDF is
given by\\
\end{eulercomment}
\begin{eulerformula}
\[
f_X(x)=\lambda e^{-\lambda x},\quad x>0,\quad \lambda>0,
\]
\end{eulerformula}
\begin{eulercomment}
with parameter\\
\end{eulercomment}
\begin{eulerformula}
\[
\lambda=\frac{1}{\mu},\quad \mu \text{ is the mean, and denoted by } X \sim \text{Exponential}(\lambda).
\]
\end{eulerformula}
\begin{eulerprompt}
>plot2d(randexponential(1,1000,2),>distribution):
\end{eulerprompt}
\eulerimg{15}{images/EMT 1-6_Mifta Dilla Aninditya_23030630029-584.png}
\begin{eulercomment}
For many distributions, Euler can compute the distribution function and the inverse.
\end{eulercomment}
\begin{eulerprompt}
>plot2d("normaldis",-4,4): 
\end{eulerprompt}
\eulerimg{15}{images/EMT 1-6_Mifta Dilla Aninditya_23030630029-585.png}
\begin{eulercomment}
The following is one way to plot a quantile.
\end{eulercomment}
\begin{eulerprompt}
>plot2d("qnormal(x,1,1.5)",-4,6);  ...
>plot2d("qnormal(x,1,1.5)",a=2,b=5,>add,>filled):
\end{eulerprompt}
\eulerimg{15}{images/EMT 1-6_Mifta Dilla Aninditya_23030630029-586.png}
\begin{eulerformula}
\[
\text{normaldis(x,m,d)}=\int_{-\infty}^x \frac{1}{d\sqrt{2\pi}}e^{-\frac{1}{2}(\frac{t-m}{d})^2}\ dt.
\]
\end{eulerformula}
\begin{eulercomment}
The probability to be in the green area is the following.
\end{eulercomment}
\begin{eulerprompt}
>normaldis(5,1,1.5)-normaldis(2,1,1.5)
\end{eulerprompt}
\begin{euleroutput}
  0.248662156979
\end{euleroutput}
\begin{eulercomment}
This can be computed numerically with the following integral.\\
\end{eulercomment}
\begin{eulerformula}
\[
\int_2^5 \frac{1}{1.5\sqrt{2\pi}}e^{-\frac{1}{2}(\frac{x-1}{1.5})^2}\ dx.
\]
\end{eulerformula}
\begin{eulerprompt}
>gauss("qnormal(x,1,1.5)",2,5)
\end{eulerprompt}
\begin{euleroutput}
  0.248662156979
\end{euleroutput}
\begin{eulercomment}
Let us compare the binomial distribution with the normal distribution of same mean and
deviation. The function invbindis() solves a linear interpolation between integer values.
\end{eulercomment}
\begin{eulerprompt}
>invbindis(0.95,1000,0.5), invnormaldis(0.95,500,0.5*sqrt(1000))
\end{eulerprompt}
\begin{euleroutput}
  525.516721219
  526.007419394
\end{euleroutput}
\begin{eulercomment}
The function qdis() is the density of the chi-square distribution. As usual, Euler maps
vectors to this function. Thus we get  a plot of all chi-square distributions with degrees 5
to 30 easily in the following way.
\end{eulercomment}
\begin{eulerprompt}
>plot2d("qchidis(x,(5:5:50)')",0,50):
\end{eulerprompt}
\eulerimg{15}{images/EMT 1-6_Mifta Dilla Aninditya_23030630029-587.png}
\begin{eulercomment}
Euler has accurate functions to evaluate distributions. Let us check chidis() with an
integral.

The naming tries to be consistent. E.g.,

- the chi-square distribution is chidis(),\\
- the inverse function is invchidis(),\\
- the density is qchidis().

The complements of the distribution (upper tail) is chicdis().
\end{eulercomment}
\begin{eulerprompt}
>chidis(1.5,2), integrate("qchidis(x,2)",0,1.5)
\end{eulerprompt}
\begin{euleroutput}
  0.527633447259
  0.527633447259
\end{euleroutput}
\eulerheading{Distribusi Diskrit}
\begin{eulercomment}
Untuk menentukan distribusi diskrit Anda sendiri, Anda dapat
menggunakan metode berikut.


Pertama, kita tetapkan fungsi distribusinya.
\end{eulercomment}
\begin{eulerprompt}
>wd = 0|((1:6)+[-0.01,0.01,0,0,0,0])/6
\end{eulerprompt}
\begin{euleroutput}
  [0,  0.165,  0.335,  0.5,  0.666667,  0.833333,  1]
\end{euleroutput}
\begin{eulercomment}
The meaning is that with probability wd[i+1]-wd[i] we produce the random value i.

This is almost a uniform distribution. Let us define a random number generator for this. The
find(v,x) function finds the value x in the vector v. It works for vectors x too.
\end{eulercomment}
\begin{eulerprompt}
>function wrongdice (n,m) := find(wd,random(n,m))
\end{eulerprompt}
\begin{eulercomment}
The error is so subtle that we see it only with very many iterations.
\end{eulercomment}
\begin{eulerprompt}
>columnsplot(getmultiplicities(1:6,wrongdice(1,1000000))):
\end{eulerprompt}
\eulerimg{15}{images/EMT 1-6_Mifta Dilla Aninditya_23030630029-588.png}
\begin{eulercomment}
Here is a simple function to check for uniform distribution of the
values 1...K in v. We accept the result, if for all frequencies

\end{eulercomment}
\begin{eulerformula}
\[
\left|f_i-\frac{1}{K}\right| < \frac{\delta}{\sqrt{n}}.
\]
\end{eulerformula}
\begin{eulerprompt}
>function checkrandom (v, delta=1) ...
\end{eulerprompt}
\begin{eulerudf}
    K=max(v); n=cols(v);
    fr=getfrequencies(v,1:K);
    return max(fr/n-1/K)<delta/sqrt(n);
    endfunction
\end{eulerudf}
\begin{eulercomment}
Indeed the function rejects the uniform distribution.
\end{eulercomment}
\begin{eulerprompt}
>checkrandom(wrongdice(1,1000000))
\end{eulerprompt}
\begin{euleroutput}
  0
\end{euleroutput}
\begin{eulercomment}
And it accepts the built-in random generator.
\end{eulercomment}
\begin{eulerprompt}
>checkrandom(intrandom(1,1000000,6))
\end{eulerprompt}
\begin{euleroutput}
  1
\end{euleroutput}
\begin{eulercomment}
We can compute the binomial distribution. First there is binomialsum(), which returns the
probability of i or less hits out of n trials.
\end{eulercomment}
\begin{eulerprompt}
>bindis(410,1000,0.4)
\end{eulerprompt}
\begin{euleroutput}
  0.751401349654
\end{euleroutput}
\begin{eulercomment}
The inverse Beta function is used to compute a Clopper-Pearson confidence interval for the
parameter p. The default level is alpha.

The meaning of this interval is that if p is outside the interval, the observed result of 410
in 1000 is rare.
\end{eulercomment}
\begin{eulerprompt}
>clopperpearson(410,1000)
\end{eulerprompt}
\begin{euleroutput}
  [0.37932,  0.441212]
\end{euleroutput}
\begin{eulercomment}
The following commands are the direct way to get the above result. But for large n, the direct
summation is not accurate and slow.
\end{eulercomment}
\begin{eulerprompt}
>p=0.4; i=0:410; n=1000; sum(bin(n,i)*p^i*(1-p)^(n-i))
\end{eulerprompt}
\begin{euleroutput}
  0.751401349655
\end{euleroutput}
\begin{eulercomment}
By the way, invbinsum() computes the inverse of binomialsum().
\end{eulercomment}
\begin{eulerprompt}
>invbindis(0.75,1000,0.4)
\end{eulerprompt}
\begin{euleroutput}
  409.932733047
\end{euleroutput}
\begin{eulercomment}
In Bridge, we assume 5 outstanding cards (out of 52) in two hands (26 cards). Let us compute
the probability of a distribution worse than 3:2 (e.g. 0:5, 1:4, 4:1 or 5:0).
\end{eulercomment}
\begin{eulerprompt}
>2*hypergeomsum(1,5,13,26)
\end{eulerprompt}
\begin{euleroutput}
  0.321739130435
\end{euleroutput}
\begin{eulercomment}
There is also a simulation of multinomial distributions.
\end{eulercomment}
\begin{eulerprompt}
>randmultinomial(10,1000,[0.4,0.1,0.5])
\end{eulerprompt}
\begin{euleroutput}
            399           110           491 
            419           100           481 
            412            97           491 
            409           110           481 
            413           100           487 
            393            92           515 
            406            73           521 
            432            89           479 
            407            98           495 
            393            98           509 
\end{euleroutput}
\eulerheading{Memplot Data}
\begin{eulercomment}
Untuk memplot data, kami mencoba hasil pemilihan umum Jerman sejak
tahun 1990, yang diukur dalam kursi.
\end{eulercomment}
\begin{eulerprompt}
>BW := [ ...
>1990,662,319,239,79,8,17; ...
>1994,672,294,252,47,49,30; ...
>1998,669,245,298,43,47,36; ...
>2002,603,248,251,47,55,2; ...
>2005,614,226,222,61,51,54; ...
>2009,622,239,146,93,68,76; ...
>2013,631,311,193,0,63,64];
\end{eulerprompt}
\begin{eulercomment}
For the parties, we use a string of names.
\end{eulercomment}
\begin{eulerprompt}
>P:=["CDU/CSU","SPD","FDP","Gr","Li"];
\end{eulerprompt}
\begin{eulercomment}
Let us print the percentages nicely.

First we extract the necessary columns. Columns 3 to 7 are the seats of each party, and column
2 is the total number of seats. column is the year of the election.
\end{eulercomment}
\begin{eulerprompt}
>BT:=BW[,3:7]; BT:=BT/sum(BT); YT:=BW[,1]';
\end{eulerprompt}
\begin{eulercomment}
Then we print the statistics in table form. We use the names as column headers, and the years
as headers for the rows. The default width for the columns is wc=10, but we prefer a denser
output. The columns will be expanded for the labels of the columns, if necessary.
\end{eulercomment}
\begin{eulerprompt}
>writetable(BT*100,wc=6,dc=0,>fixed,labc=P,labr=YT)
\end{eulerprompt}
\begin{euleroutput}
         CDU/CSU   SPD   FDP    Gr    Li
    1990      48    36    12     1     3
    1994      44    38     7     7     4
    1998      37    45     6     7     5
    2002      41    42     8     9     0
    2005      37    36    10     8     9
    2009      38    23    15    11    12
    2013      49    31     0    10    10
\end{euleroutput}
\begin{eulercomment}
The following matrix multiplication extracts the sum of the percentages of the two big parties
showing that the small parties have gained footage in the parliament until 2009.
\end{eulercomment}
\begin{eulerprompt}
>BT1:=(BT.[1;1;0;0;0])'*100
\end{eulerprompt}
\begin{euleroutput}
  [84.29,  81.25,  81.1659,  82.7529,  72.9642,  61.8971,  79.8732]
\end{euleroutput}
\begin{eulercomment}
There is also a simple statistical plot. We use it to display lines and points simultaneously.
The alternative is to call plot2d twice with \textgreater{}add.
\end{eulercomment}
\begin{eulerprompt}
>statplot(YT,BT1,"b"):
\end{eulerprompt}
\eulerimg{15}{images/EMT 1-6_Mifta Dilla Aninditya_23030630029-589.png}
\begin{eulercomment}
Define some colors for each party.
\end{eulercomment}
\begin{eulerprompt}
>CP:=[rgb(0.5,0.5,0.5),red,yellow,green,rgb(0.8,0,0)];
\end{eulerprompt}
\begin{eulercomment}
Now we can plot the results of the 2009 election and the changes into one plot using figure.
We can add a vector of columns to each plot.
\end{eulercomment}
\begin{eulerprompt}
>figure(2,1);  ...
>figure(1); columnsplot(BW[6,3:7],P,color=CP); ...
>figure(2); columnsplot(BW[6,3:7]-BW[5,3:7],P,color=CP);  ...
>figure(0):
\end{eulerprompt}
\eulerimg{15}{images/EMT 1-6_Mifta Dilla Aninditya_23030630029-590.png}
\begin{eulercomment}
Data plots combine rows of statistical data in one plot.
\end{eulercomment}
\begin{eulerprompt}
>J:=BW[,1]'; DP:=BW[,3:7]'; ...
>dataplot(YT,BT',color=CP);  ...
>labelbox(P,colors=CP,styles="[]",>points,w=0.2,x=0.3,y=0.4):
\end{eulerprompt}
\eulerimg{15}{images/EMT 1-6_Mifta Dilla Aninditya_23030630029-591.png}
\begin{eulercomment}
A 3D columns plot shows rows of statistical data in form of columns. We provide labels for the
rows and the columns. angle is the viewing angle.
\end{eulercomment}
\begin{eulerprompt}
>columnsplot3d(BT,scols=P,srows=YT, ...
>  angle=30°,ccols=CP):
\end{eulerprompt}
\eulerimg{15}{images/EMT 1-6_Mifta Dilla Aninditya_23030630029-592.png}
\begin{eulercomment}
Another representation is the mosaic plot. Note that the columns of the plot represent the
columns of the matrix here. Because of the length of the label CDU/CSU, we take a smaller
window than usual.
\end{eulercomment}
\begin{eulerprompt}
>shrinkwindow(>smaller);  ...
>mosaicplot(BT',srows=YT,scols=P,color=CP,style="#"); ...
>shrinkwindow():
\end{eulerprompt}
\eulerimg{15}{images/EMT 1-6_Mifta Dilla Aninditya_23030630029-593.png}
\begin{eulercomment}
We can also do a pie chart. Since black and yellow form a coalition, we reorder the elements.
\end{eulercomment}
\begin{eulerprompt}
>i=[1,3,5,4,2]; piechart(BW[6,3:7][i],color=CP[i],lab=P[i]):
\end{eulerprompt}
\eulerimg{15}{images/EMT 1-6_Mifta Dilla Aninditya_23030630029-594.png}
\begin{eulercomment}
Here is another kind of plot.
\end{eulercomment}
\begin{eulerprompt}
>starplot(normal(1,10)+4,lab=1:10,>rays):
\end{eulerprompt}
\eulerimg{15}{images/EMT 1-6_Mifta Dilla Aninditya_23030630029-595.png}
\begin{eulercomment}
Some plots in plot2d are good for statics. Here is an impulse plot of random data, uniformly
distributed in [0,1].
\end{eulercomment}
\begin{eulerprompt}
>plot2d(makeimpulse(1:10,random(1,10)),>bar):
\end{eulerprompt}
\eulerimg{15}{images/EMT 1-6_Mifta Dilla Aninditya_23030630029-596.png}
\begin{eulercomment}
But for exponentially distributed data, we may need a logarithmic plot.
\end{eulercomment}
\begin{eulerprompt}
>logimpulseplot(1:10,-log(random(1,10))*10):
\end{eulerprompt}
\eulerimg{15}{images/EMT 1-6_Mifta Dilla Aninditya_23030630029-597.png}
\begin{eulercomment}
The function columnsplot() is easier to use, since it needs just a vector of values. Moreover,
it can set its labels to anything we want, we demonstrated this already in this tutorial.

Here is another application, where we count characters in a sentence and plot a statistics.
\end{eulercomment}
\begin{eulerprompt}
>v=strtochar("the quick brown fox jumps over the lazy dog"); ...
>w=ascii("a"):ascii("z"); x=getmultiplicities(w,v); ...
>cw=[]; for k=w; cw=cw|char(k); end; ...
>columnsplot(x,lab=cw,width=0.05):
\end{eulerprompt}
\eulerimg{15}{images/EMT 1-6_Mifta Dilla Aninditya_23030630029-598.png}
\begin{eulercomment}
It is also possible to manually set axes.
\end{eulercomment}
\begin{eulerprompt}
>n=10; p=0.4; i=0:n; x=bin(n,i)*p^i*(1-p)^(n-i); ...
>columnsplot(x,lab=i,width=0.05,<frame,<grid); ...
>yaxis(0,0:0.1:1,style="->",>left); xaxis(0,style="."); ...
>label("p",0,0.25), label("i",11,0); ...
>textbox(["Binomial distribution","with p=0.4"]):
\end{eulerprompt}
\eulerimg{15}{images/EMT 1-6_Mifta Dilla Aninditya_23030630029-599.png}
\begin{eulercomment}
The following is a way to plot the frequencies of numbers in a vector.

We create a vector of integer random numbers 1 to 6.
\end{eulercomment}
\begin{eulerprompt}
>v:=intrandom(1,10,10)
\end{eulerprompt}
\begin{euleroutput}
  [7,  9,  1,  10,  4,  2,  9,  3,  9,  1]
\end{euleroutput}
\begin{eulercomment}
Then extract the unique numbers in v.
\end{eulercomment}
\begin{eulerprompt}
>vu:=unique(v)
\end{eulerprompt}
\begin{euleroutput}
  [1,  2,  3,  4,  7,  9,  10]
\end{euleroutput}
\begin{eulercomment}
And plot the frequencies in a columns plot.
\end{eulercomment}
\begin{eulerprompt}
>columnsplot(getmultiplicities(vu,v),lab=vu,style="/"):
\end{eulerprompt}
\eulerimg{15}{images/EMT 1-6_Mifta Dilla Aninditya_23030630029-600.png}
\begin{eulercomment}
We want to demonstrate functions for the empirical distribution of values.
\end{eulercomment}
\begin{eulerprompt}
>x=normal(1,20);
\end{eulerprompt}
\begin{eulercomment}
The function empdist(x,vs) needs a sorted array of values. So we have to sort x before we can
use it.
\end{eulercomment}
\begin{eulerprompt}
>xs=sort(x);
\end{eulerprompt}
\begin{eulercomment}
Then we plot the empirical distribution and some density bars into one plot. Instead of a bar
plot for the distribution we use a sawtooth plot this time.
\end{eulercomment}
\begin{eulerprompt}
>figure(2,1); ...
>figure(1); plot2d("empdist",-4,4;xs); ...
>figure(2); plot2d(histo(x,v=-4:0.2:4,<bar));  ...
>figure(0):
\end{eulerprompt}
\eulerimg{15}{images/EMT 1-6_Mifta Dilla Aninditya_23030630029-601.png}
\begin{eulercomment}
A scatter plot is easy to do in Euler with the usual point plot. The
following graph shows that the X and X+Y are clearly positively
correlated.
\end{eulercomment}
\begin{eulerprompt}
>x=normal(1,100); plot2d(x,x+rotright(x),>points,style=".."):
\end{eulerprompt}
\eulerimg{15}{images/EMT 1-6_Mifta Dilla Aninditya_23030630029-602.png}
\begin{eulercomment}
Often, we wish to compare two samples of different distributions. This can be done with a
quantile-quantile-plot.

For a test, we try the student-t distribution and exponential distribution.
\end{eulercomment}
\begin{eulerprompt}
>x=randt(1,1000,5); y=randnormal(1,1000,mean(x),dev(x)); ...
>plot2d("x",r=6,style="--",yl="normal",xl="student-t",>vertical); ...
>plot2d(sort(x),sort(y),>points,color=red,style="x",>add):
\end{eulerprompt}
\eulerimg{15}{images/EMT 1-6_Mifta Dilla Aninditya_23030630029-603.png}
\begin{eulercomment}
The plot clearly shows that the normal distributed values tend to be smaller at the extreme
ends.

If we have two distributions of different size, we can expand the smaller one or shrink the
larger one. The following function is good for both. It takes the median values with
percentages between 0 and 1.
\end{eulercomment}
\begin{eulerprompt}
>function medianexpand (x,n) := median(x,p=linspace(0,1,n-1));
\end{eulerprompt}
\begin{eulercomment}
Let us compare two equal distributions.
\end{eulercomment}
\begin{eulerprompt}
>x=random(1000); y=random(400); ...
>plot2d("x",0,1,style="--"); ...
>plot2d(sort(medianexpand(x,400)),sort(y),>points,color=red,style="x",>add):
\end{eulerprompt}
\eulerimg{15}{images/EMT 1-6_Mifta Dilla Aninditya_23030630029-604.png}
\eulerheading{Regresi dan Korelasi}
\begin{eulercomment}
Regresi linier dapat dilakukan dengan fungsi polyfit() atau berbagai
fungsi kecocokan.


Sebagai permulaan, kita mencari garis regresi untuk data univariat
dengan polyfit(x,y,1).
\end{eulercomment}
\begin{eulerprompt}
>x=1:10; y=[2,3,1,5,6,3,7,8,9,8]; writetable(x'|y',labc=["x","y"])
\end{eulerprompt}
\begin{euleroutput}
           x         y
           1         2
           2         3
           3         1
           4         5
           5         6
           6         3
           7         7
           8         8
           9         9
          10         8
\end{euleroutput}
\begin{eulercomment}
We want to compare non-weighted and weighted fits. First the coefficients of the linear fit.
\end{eulercomment}
\begin{eulerprompt}
>p=polyfit(x,y,1)
\end{eulerprompt}
\begin{euleroutput}
  [0.733333,  0.812121]
\end{euleroutput}
\begin{eulercomment}
Now the coefficients with a weight that emphasizes the last values.
\end{eulercomment}
\begin{eulerprompt}
>w &= "exp(-(x-10)^2/10)"; pw=polyfit(x,y,1,w=w(x))
\end{eulerprompt}
\begin{euleroutput}
  [4.71566,  0.38319]
\end{euleroutput}
\begin{eulercomment}
We put everything into one plot for the points and the regression lines, and for the weights
used.
\end{eulercomment}
\begin{eulerprompt}
>figure(2,1);  ...
>figure(1); statplot(x,y,"b",xl="Regression"); ...
>  plot2d("evalpoly(x,p)",>add,color=blue,style="--"); ...
>  plot2d("evalpoly(x,pw)",5,10,>add,color=red,style="--"); ...
>figure(2); plot2d(w,1,10,>filled,style="/",fillcolor=red,xl=w); ...
>figure(0):
\end{eulerprompt}
\eulerimg{15}{images/EMT 1-6_Mifta Dilla Aninditya_23030630029-605.png}
\begin{eulercomment}
For another example we read a survey of students, their ages, the ages of their parents and
the number of siblings from a file.

This table contains "m" and "f" in the second column. We use the variable tok2 to set the
proper translations instead of letting readtable() collect the translations.
\end{eulercomment}
\begin{eulerprompt}
>\{MS,hd\}:=readtable("table1.dat",tok2:=["m","f"]);  ...
>writetable(MS,labc=hd,tok2:=["m","f"]);
\end{eulerprompt}
\begin{euleroutput}
  Could not open the file
  table1.dat
  for reading!
  Try "trace errors" to inspect local variables after errors.
  readtable:
      if filename!=none then open(filename,"r"); endif;
\end{euleroutput}
\begin{eulercomment}
How do the ages depend on each other? A first impression comes from a pairwise scatterplot.
\end{eulercomment}
\begin{eulerprompt}
>scatterplots(tablecol(MS,3:5),hd[3:5]):
\end{eulerprompt}
\begin{euleroutput}
  Variable or function MS not found.
  Error in:
  scatterplots(tablecol(MS,3:5),hd[3:5]): ...
                          ^
\end{euleroutput}
\begin{eulercomment}
It is clear that the age of the father and mother depend on each other. Let us determine and
plot the regression line.
\end{eulercomment}
\begin{eulerprompt}
>cs:=MS[,4:5]'; ps:=polyfit(cs[1],cs[2],1)
\end{eulerprompt}
\begin{euleroutput}
  MS is not a variable!
  Error in:
  cs:=MS[,4:5]'; ps:=polyfit(cs[1],cs[2],1) ...
              ^
\end{euleroutput}
\begin{eulercomment}
This is obviously the wrong model. The regression line would be s=17+0.74t, where t is the age
of the mother and s the age of the father. The age difference may depend a little bit on the
age, but not that much.

Rather, we suspect a function like s=a+t. Then a is the mean of the s-t. It is the average age
difference between fathers and mothers.
\end{eulercomment}
\begin{eulerprompt}
>da:=mean(cs[2]-cs[1])
\end{eulerprompt}
\begin{euleroutput}
  cs is not a variable!
  Error in:
  da:=mean(cs[2]-cs[1]) ...
                ^
\end{euleroutput}
\begin{eulercomment}
Let us plot this into one scatter plot.
\end{eulercomment}
\begin{eulerprompt}
>plot2d(cs[1],cs[2],>points);  ...
>plot2d("evalpoly(x,ps)",color=red,style=".",>add);  ...
>plot2d("x+da",color=blue,>add):
\end{eulerprompt}
\begin{euleroutput}
  cs is not a variable!
  Error in:
  plot2d(cs[1],cs[2],>points);  plot2d("evalpoly(x,ps)",color=re ...
              ^
\end{euleroutput}
\begin{eulercomment}
Here is a box plot of the two ages. This only shows, that the ages are different.
\end{eulercomment}
\begin{eulerprompt}
>boxplot(cs,["mothers","fathers"]):
\end{eulerprompt}
\begin{euleroutput}
  Variable or function cs not found.
  Error in:
  boxplot(cs,["mothers","fathers"]): ...
            ^
\end{euleroutput}
\begin{eulercomment}
It is interesting that the difference in medians is not as large as the difference in means.
\end{eulercomment}
\begin{eulerprompt}
>median(cs[2])-median(cs[1])
\end{eulerprompt}
\begin{euleroutput}
  cs is not a variable!
  Error in:
  median(cs[2])-median(cs[1]) ...
              ^
\end{euleroutput}
\begin{eulercomment}
The correlation coefficient suggests a positive correlation.
\end{eulercomment}
\begin{eulerprompt}
>correl(cs[1],cs[2])
\end{eulerprompt}
\begin{euleroutput}
  cs is not a variable!
  Error in:
  correl(cs[1],cs[2]) ...
              ^
\end{euleroutput}
\begin{eulercomment}
The correlation of the ranks is a measure for the same order in both vectors. It is also quite
positive.
\end{eulercomment}
\begin{eulerprompt}
>rankcorrel(cs[1],cs[2])
\end{eulerprompt}
\begin{euleroutput}
  cs is not a variable!
  Error in:
  rankcorrel(cs[1],cs[2]) ...
                  ^
\end{euleroutput}
\eulerheading{Membuat Fungsi baru}
\begin{eulercomment}
Tentu saja, bahasa EMT dapat digunakan untuk memprogram fungsi baru.
Misalnya, kita mendefinisikan fungsi kemiringan.


\end{eulercomment}
\begin{eulerformula}
\[
\text{sk}(x) = \dfrac{\sqrt{n} \sum_i (x_i-m)^3}{\left(\sum_i (x_i-m)^2\right)^{3/2}}
\]
\end{eulerformula}
\begin{eulercomment}
di mana m adalah rata-rata dari x.
\end{eulercomment}
\begin{eulerprompt}
>function skew (x:vector) ...
\end{eulerprompt}
\begin{eulerudf}
  m=mean(x);
  return sqrt(cols(x))*sum((x-m)^3)/(sum((x-m)^2))^(3/2);
  endfunction
\end{eulerudf}
\begin{eulercomment}
As you see, we can easily use the matrix language to get a very short
and efficient implementation. Let us try this function.
\end{eulercomment}
\begin{eulerprompt}
>data=normal(20); skew(normal(10))
\end{eulerprompt}
\begin{euleroutput}
  0.360073860591
\end{euleroutput}
\begin{eulercomment}
Here is another function, called the Pearson skewness coefficient.
\end{eulercomment}
\begin{eulerprompt}
>function skew1 (x) := 3*(mean(x)-median(x))/dev(x)
>skew1(data)
\end{eulerprompt}
\begin{euleroutput}
  -0.277267811621
\end{euleroutput}
\eulerheading{Monte Carlo Simulation}
\begin{eulercomment}
Euler can be used to simulate random events. We have already seen simple examples above. Here
is another one, which simulates 1000 times 3 dice throws, and asks for the distribution of the
sums.
\end{eulercomment}
\begin{eulerprompt}
>ds:=sum(intrandom(1000,3,6))';  fs=getmultiplicities(3:18,ds)
\end{eulerprompt}
\begin{euleroutput}
  [5,  17,  35,  44,  75,  97,  114,  116,  143,  116,  104,  53,  40,
  22,  13,  6]
\end{euleroutput}
\begin{eulercomment}
We can plot this now.
\end{eulercomment}
\begin{eulerprompt}
>columnsplot(fs,lab=3:18):
\end{eulerprompt}
\begin{eulercomment}
To determine the expected distribution is not so easy. We use an advanced recursion for this.

The following function counts the number of ways the number k can be represented as the sum of
n numbers in the range of 1 to m. It works recursively in an obvious way.
\end{eulercomment}
\begin{eulerprompt}
>function map countways (k; n, m) ...
\end{eulerprompt}
\begin{eulerudf}
    if n==1 then return k>=1 && k<=m
    else
      sum=0; 
      loop 1 to m; sum=sum+countways(k-#,n-1,m); end;
      return sum;
    end;
  endfunction
\end{eulerudf}
\begin{eulercomment}
Here is the result for three throws of dices.
\end{eulercomment}
\begin{eulerprompt}
>countways(5:25,5,5)
\end{eulerprompt}
\begin{euleroutput}
  [1,  5,  15,  35,  70,  121,  185,  255,  320,  365,  381,  365,  320,
  255,  185,  121,  70,  35,  15,  5,  1]
\end{euleroutput}
\begin{eulerprompt}
>cw=countways(3:18,3,6)
\end{eulerprompt}
\begin{euleroutput}
  [1,  3,  6,  10,  15,  21,  25,  27,  27,  25,  21,  15,  10,  6,  3,
  1]
\end{euleroutput}
\begin{eulercomment}
We add the expected values to the plot.
\end{eulercomment}
\begin{eulerprompt}
>plot2d(cw/6^3*1000,>add); plot2d(cw/6^3*1000,>points,>add):
\end{eulerprompt}
\begin{eulercomment}
For another simulation, the deviation of the mean value of n 0-1-normal distributed random
variables is 1/sqrt(n).
\end{eulercomment}
\begin{eulerprompt}
>longformat; 1/sqrt(10)
\end{eulerprompt}
\begin{euleroutput}
  0.316227766017
\end{euleroutput}
\begin{eulercomment}
Let us check this with a simulation. We produce 10000 times 10 random vectors.
\end{eulercomment}
\begin{eulerprompt}
>M=normal(10000,10); dev(mean(M)')
\end{eulerprompt}
\begin{euleroutput}
  0.319493614817
\end{euleroutput}
\begin{eulerprompt}
>plot2d(mean(M)',>distribution):
\end{eulerprompt}
\begin{eulercomment}
The median of 10 0-1-normal distributed random numbers has a larger deviation.
\end{eulercomment}
\begin{eulerprompt}
>dev(median(M)')
\end{eulerprompt}
\begin{euleroutput}
  0.374460271535
\end{euleroutput}
\begin{eulercomment}
Since we can easily generate random walks, we can simulate the Wiener process. We take 1000
steps of 1000 processes. We then plot the standard deviation and the mean of the n-th step of
these processes together with the expected values in red.
\end{eulercomment}
\begin{eulerprompt}
>n=1000; m=1000; M=cumsum(normal(n,m)/sqrt(m)); ...
>t=(1:n)/n; figure(2,1); ...
>figure(1); plot2d(t,mean(M')'); plot2d(t,0,color=red,>add); ...
>figure(2); plot2d(t,dev(M')'); plot2d(t,sqrt(t),color=red,>add); ...
>figure(0):
\end{eulerprompt}
\eulerheading{Tes}
\begin{eulercomment}
Tes adalah alat yang penting dalam statistik. Dalam Euler, banyak tes
yang diterapkan. Semua tes ini mengembalikan kesalahan yang kita
terima jika kita menolak hipotesis nol.


Sebagai contoh, kita menguji lemparan dadu untuk distribusi yang
seragam. Pada 600 lemparan, kita mendapatkan nilai berikut, yang kita
masukkan ke dalam uji chi-kuadrat.
\end{eulercomment}
\begin{eulerprompt}
>chitest([90,103,114,101,103,89],dup(100,6)')
\end{eulerprompt}
\begin{euleroutput}
  0.498830517952
\end{euleroutput}
\begin{eulercomment}
The chi-square test also has a mode, which uses a Monte Carlo simulation to test the
statistics. The result should be almost the same. The parameter \textgreater{}p interprets the y-vector as
a vector of probabilities.
\end{eulercomment}
\begin{eulerprompt}
>chitest([90,103,114,101,103,89],dup(1/6,6)',>p,>montecarlo)
\end{eulerprompt}
\begin{euleroutput}
  0.526
\end{euleroutput}
\begin{eulercomment}
This error is much too large. So we cannot reject uniform distribution. This does not prove
that our dice was fair. But we cannot reject our hypothesis.

Next we generate 1000 dice throws using the random number generator, and do the same test.
\end{eulercomment}
\begin{eulerprompt}
>n=1000; t=random([1,n*6]); chitest(count(t*6,6),dup(n,6)')
\end{eulerprompt}
\begin{euleroutput}
  0.528028118442
\end{euleroutput}
\begin{eulercomment}
Let us test for the mean value 100 with the t-test.
\end{eulercomment}
\begin{eulerprompt}
>s=200+normal([1,100])*10; ...
>ttest(mean(s),dev(s),100,200)
\end{eulerprompt}
\begin{euleroutput}
  0.0218365848476
\end{euleroutput}
\begin{eulercomment}
The function ttest() needs the mean value, the deviation, the number of data, and the mean
value to test for.

Now let us check two measurements for the same mean. We reject the hypothesis that they have
the same mean, if the result is \textless{}0.05.
\end{eulercomment}
\begin{eulerprompt}
>tcomparedata(normal(1,10),normal(1,10))
\end{eulerprompt}
\begin{euleroutput}
  0.38722000942
\end{euleroutput}
\begin{eulercomment}
If we add a bias to one distribution, we get more rejections. Repeat this simulation several
times to see the effect.
\end{eulercomment}
\begin{eulerprompt}
>tcomparedata(normal(1,10),normal(1,10)+2)
\end{eulerprompt}
\begin{euleroutput}
  5.60009101758e-07
\end{euleroutput}
\begin{eulercomment}
In the next example, we generate 20 random dice throws 100 times and count the ones in it.
There must be 20/6=3.3 ones on average.
\end{eulercomment}
\begin{eulerprompt}
>R=random(100,20); R=sum(R*6<=1)'; mean(R)
\end{eulerprompt}
\begin{euleroutput}
  3.28
\end{euleroutput}
\begin{eulercomment}
We now compare the number of ones with the binomial distribution. First we plot the
distribution of ones.
\end{eulercomment}
\begin{eulerprompt}
>plot2d(R,distribution=max(R)+1,even=1,style="\(\backslash\)/"):
>t=count(R,21);
\end{eulerprompt}
\begin{eulercomment}
Then we compute the expected values.
\end{eulercomment}
\begin{eulerprompt}
>n=0:20; b=bin(20,n)*(1/6)^n*(5/6)^(20-n)*100;
\end{eulerprompt}
\begin{eulercomment}
We have to collect several numbers to get categories, which are big enough.
\end{eulercomment}
\begin{eulerprompt}
>t1=sum(t[1:2])|t[3:7]|sum(t[8:21]); ...
>b1=sum(b[1:2])|b[3:7]|sum(b[8:21]);
\end{eulerprompt}
\begin{eulercomment}
The chi-square test rejects the hypothesis that our distribution is a binomial distribution,
if its result is \textless{}0.05.
\end{eulercomment}
\begin{eulerprompt}
>chitest(t1,b1)
\end{eulerprompt}
\begin{euleroutput}
  0.53921579764
\end{euleroutput}
\begin{eulercomment}
The following example contains results of two groups of persons (male and female, say) voting
for one out of six parties.
\end{eulercomment}
\begin{eulerprompt}
>A=[23,37,43,52,64,74;27,39,41,49,63,76];  ...
>  writetable(A,wc=6,labr=["m","f"],labc=1:6)
\end{eulerprompt}
\begin{euleroutput}
             1     2     3     4     5     6
       m    23    37    43    52    64    74
       f    27    39    41    49    63    76
\end{euleroutput}
\begin{eulercomment}
We wish to test for independence of the votes from the sex. The chi\textasciicircum{}2 table test does this.
The result is way to large to reject independence. So we cannot say, if the voting depends on
the sex from these data.
\end{eulercomment}
\begin{eulerprompt}
>tabletest(A)
\end{eulerprompt}
\begin{euleroutput}
  0.990701632326
\end{euleroutput}
\begin{eulercomment}
The following is the expected table, if we assume the observed frequencies of voting.
\end{eulercomment}
\begin{eulerprompt}
>writetable(expectedtable(A),wc=6,dc=1,labr=["m","f"],labc=1:6)
\end{eulerprompt}
\begin{euleroutput}
             1     2     3     4     5     6
       m  24.9  37.9  41.9  50.3  63.3  74.7
       f  25.1  38.1  42.1  50.7  63.7  75.3
\end{euleroutput}
\begin{eulercomment}
We can compute the corrected contingency coefficient. Since it very close to 0, we conclude
that the voting does not depend on the sex.
\end{eulercomment}
\begin{eulerprompt}
>contingency(A)
\end{eulerprompt}
\begin{euleroutput}
  0.0427225484717
\end{euleroutput}
\begin{eulercomment}
\begin{eulercomment}
\eulerheading{Some More Tests}
\begin{eulercomment}
Next we use a variance analysis (F-test) to test three samples of normally distributed data
for same mean value. The method is called ANOVA (analysis of variance). In Euler, the function
varanalysis() is used.
\end{eulercomment}
\begin{eulerprompt}
>x1=[109,111,98,119,91,118,109,99,115,109,94]; mean(x1),
\end{eulerprompt}
\begin{euleroutput}
  106.545454545
\end{euleroutput}
\begin{eulerprompt}
>x2=[120,124,115,139,114,110,113,120,117]; mean(x2),
\end{eulerprompt}
\begin{euleroutput}
  119.111111111
\end{euleroutput}
\begin{eulerprompt}
>x3=[120,112,115,110,105,134,105,130,121,111]; mean(x3)
\end{eulerprompt}
\begin{euleroutput}
  116.3
\end{euleroutput}
\begin{eulerprompt}
>varanalysis(x1,x2,x3)
\end{eulerprompt}
\begin{euleroutput}
  0.0138048221371
\end{euleroutput}
\begin{eulercomment}
This means, we reject the hypothesis of same mean value. We do this with an error probability
of 1.3\%.

There is also the median test, which rejects data samples with different mean distribution
testing the median of the united sample.
\end{eulercomment}
\begin{eulerprompt}
>a=[56,66,68,49,61,53,45,58,54];
>b=[72,81,51,73,69,78,59,67,65,71,68,71];
>mediantest(a,b)
\end{eulerprompt}
\begin{euleroutput}
  0.0241724220052
\end{euleroutput}
\begin{eulercomment}
Another test on equality is the rank test. It is much sharper than the median test.
\end{eulercomment}
\begin{eulerprompt}
>ranktest(a,b)
\end{eulerprompt}
\begin{euleroutput}
  0.00199969612469
\end{euleroutput}
\begin{eulercomment}
In the following example, both distributions have the same mean.
\end{eulercomment}
\begin{eulerprompt}
>ranktest(random(1,100),random(1,50)*3-1)
\end{eulerprompt}
\begin{euleroutput}
  0.129608141484
\end{euleroutput}
\begin{eulercomment}
Let us now try to simulate two treatments a and b applied to different
persons.
\end{eulercomment}
\begin{eulerprompt}
>a=[8.0,7.4,5.9,9.4,8.6,8.2,7.6,8.1,6.2,8.9];
>b=[6.8,7.1,6.8,8.3,7.9,7.2,7.4,6.8,6.8,8.1];
\end{eulerprompt}
\begin{eulercomment}
The signum test decides, if a is better than b.
\end{eulercomment}
\begin{eulerprompt}
>signtest(a,b)
\end{eulerprompt}
\begin{euleroutput}
  0.0546875
\end{euleroutput}
\begin{eulercomment}
This is too much of an error. We cannot reject that a is as good as b.

The Wilcoxon test is sharper than this test, but relies on the quantitative value of the
differences.
\end{eulercomment}
\begin{eulerprompt}
>wilcoxon(a,b)
\end{eulerprompt}
\begin{euleroutput}
  0.0296680599405
\end{euleroutput}
\begin{eulercomment}
Let us try two more tests using generated series.
\end{eulercomment}
\begin{eulerprompt}
>wilcoxon(normal(1,20),normal(1,20)-1)
\end{eulerprompt}
\begin{euleroutput}
  0.0068706451766
\end{euleroutput}
\begin{eulerprompt}
>wilcoxon(normal(1,20),normal(1,20))
\end{eulerprompt}
\begin{euleroutput}
  0.275145971064
\end{euleroutput}
\eulerheading{Random Numbers}
\begin{eulercomment}
The following is a test for the random number generator. Euler uses a very good generator, so
we need not expect any problems.

First we generate ten millions of random numbers in [0,1].
\end{eulercomment}
\begin{eulerprompt}
>n:=10000000; r:=random(1,n);
\end{eulerprompt}
\begin{eulercomment}
Next we count the distances between two numbers less than 0.05.
\end{eulercomment}
\begin{eulerprompt}
>a:=0.05; d:=differences(nonzeros(r<a));
\end{eulerprompt}
\begin{eulercomment}
Finally, we plot the number of times, each distance occurred, and compare with the expected
value.
\end{eulercomment}
\begin{eulerprompt}
>m=getmultiplicities(1:100,d); plot2d(m); ...
>  plot2d("n*(1-a)^(x-1)*a^2",color=red,>add):
\end{eulerprompt}
\begin{eulercomment}
Clear the data.
\end{eulercomment}
\begin{eulerprompt}
>remvalue n;
\end{eulerprompt}
\begin{eulercomment}
\end{eulercomment}
\begin{eulerttcomment}
 Pengantar untuk Pengguna Proyek R
\end{eulerttcomment}
\begin{eulercomment}
Jelas, EMT tidak bersaing dengan R sebagai sebuah paket statistik.
Namun, ada banyak prosedur dan fungsi statistik yang tersedia di EMT
juga. Jadi EMT dapat memenuhi kebutuhan dasar. Bagaimanapun, EMT hadir
dengan paket numerik dan sistem aljabar komputer.


Buku ini diperuntukkan bagi Anda yang sudah terbiasa dengan R, tetapi
perlu mengetahui perbedaan sintaks EMT dan R. Kami mencoba memberikan
gambaran umum mengenai hal-hal yang jelas dan kurang jelas yang perlu
Anda ketahui.


Selain itu, kami juga membahas cara-cara untuk bertukar data di antara
kedua sistem tersebut.
\end{eulercomment}
\begin{eulercomment}
Note that this is a work in progress.
\end{eulercomment}
\eulerheading{Sintaks Dasar}
\begin{eulercomment}
Hal pertama yang Anda pelajari dalam R adalah membuat sebuah vektor.
Dalam EMT, perbedaan utamanya adalah bahwa operator : dapat mengambil
ukuran langkah. Selain itu, operator ini memiliki daya ikat yang
rendah.
\end{eulercomment}
\begin{eulerprompt}
>n=10; 0:n/20:n-1
\end{eulerprompt}
\begin{euleroutput}
  [0,  0.5,  1,  1.5,  2,  2.5,  3,  3.5,  4,  4.5,  5,  5.5,  6,  6.5,
  7,  7.5,  8,  8.5,  9]
\end{euleroutput}
\begin{eulercomment}
The c() function does not exist. It is possible to use vectors to concatenate things.

The following example is, like many others, from the "Interoduction to R" that comes with the
R project. If you read this PDF, you will find that I follow its path in this tutorial.
\end{eulercomment}
\begin{eulerprompt}
>x=[10.4, 5.6, 3.1, 6.4, 21.7]; [x,0,x]
\end{eulerprompt}
\begin{euleroutput}
  [10.4,  5.6,  3.1,  6.4,  21.7,  0,  10.4,  5.6,  3.1,  6.4,  21.7]
\end{euleroutput}
\begin{eulercomment}
The colon operator with step size of EMT is replaced by the function seq() in R. We can write
this function in EMT.
\end{eulercomment}
\begin{eulerprompt}
>function seq(a,b,c) := a:b:c; ...
>seq(0,-0.1,-1)
\end{eulerprompt}
\begin{euleroutput}
  [0,  -0.1,  -0.2,  -0.3,  -0.4,  -0.5,  -0.6,  -0.7,  -0.8,  -0.9,  -1]
\end{euleroutput}
\begin{eulercomment}
The function rep() of R is not present in EMT. For vector input, it could be written as
follows.
\end{eulercomment}
\begin{eulerprompt}
>function rep(x:vector,n:index) := flatten(dup(x,n)); ...
>rep(x,2)
\end{eulerprompt}
\begin{euleroutput}
  [10.4,  5.6,  3.1,  6.4,  21.7,  10.4,  5.6,  3.1,  6.4,  21.7]
\end{euleroutput}
\begin{eulercomment}
Note that "=" or ":=" is used for assignments. The "-\textgreater{}" operator is used for units in EMT.
\end{eulercomment}
\begin{eulerprompt}
>125km -> " miles"
\end{eulerprompt}
\begin{euleroutput}
  77.6713990297 miles
\end{euleroutput}
\begin{eulercomment}
The "\textless{}-" operator for assignment is misleading anyway, and not a good idea of R. The following
will compare a and -4 in EMT.
\end{eulercomment}
\begin{eulerprompt}
>a=2; a<-4
\end{eulerprompt}
\begin{euleroutput}
  0
\end{euleroutput}
\begin{eulercomment}
In R, "a\textless{}-4\textless{}3" works, but "a\textless{}-4\textless{}-3" does not. I had similar ambiguities in EMT too, but tried
to eliminate them by and by.

EMT and R have vectors of boolean type. But in EMT, the numbers 0 and 1 are used to represent
false and true. In R, the values true and false can nevertheless used in ordinary arithmetic
just like in EMT.
\end{eulercomment}
\begin{eulerprompt}
>x<5, %*x
\end{eulerprompt}
\begin{euleroutput}
  [0,  0,  1,  0,  0]
  [0,  0,  3.1,  0,  0]
\end{euleroutput}
\begin{eulercomment}
EMT throws errors or yields NAN depending on the flag "errors".
\end{eulercomment}
\begin{eulerprompt}
>errors off; 0/0, isNAN(sqrt(-1)), errors on;
\end{eulerprompt}
\begin{euleroutput}
  NAN
  1
\end{euleroutput}
\begin{eulercomment}
Strings are the same in R and EMT. Both are in the current locale, not in Unicode.

In R there are packages for Unicode. In EMT, a string can be Unicode string. A unicode string
can be translated to the local encoding and vice versa. Moreover, u"..." can contain HTML
entities.
\end{eulercomment}
\begin{eulerprompt}
>u"&#169; Ren&eacut; Grothmann"
\end{eulerprompt}
\begin{euleroutput}
  © René Grothmann
\end{euleroutput}
\begin{eulercomment}
The following may or may not display correctly on your system as A with dot and dash above it.
It depends on the font you are using.
\end{eulercomment}
\begin{eulerprompt}
>chartoutf([480])
\end{eulerprompt}
\begin{euleroutput}
  Ǡ
\end{euleroutput}
\begin{eulercomment}
The string concatenation is done with "+" or "\textbar{}". It can include
numbers, which will print in the current format.
\end{eulercomment}
\begin{eulerprompt}
>"pi = "+pi
\end{eulerprompt}
\begin{euleroutput}
  pi = 3.14159265359
\end{euleroutput}
\eulerheading{Pengindeksan}
\begin{eulercomment}
Sebagian besar waktu, ini akan bekerja seperti pada R.


Tetapi EMT akan menginterpretasikan indeks negatif dari bagian
belakang vektor, sementara R menginterpretasikan x[n] sebagai x tanpa
elemen ke-n.
\end{eulercomment}
\begin{eulerprompt}
>x, x[1:3], x[-2]
\end{eulerprompt}
\begin{euleroutput}
  [1,  2,  3,  4,  5,  6,  7,  8,  9,  10]
  [1,  2,  3]
  9
\end{euleroutput}
\begin{eulercomment}
The behavior of R can be achieved in EMT with drop().
\end{eulercomment}
\begin{eulerprompt}
>drop(x,2)
\end{eulerprompt}
\begin{euleroutput}
  [1,  3,  4,  5,  6,  7,  8,  9,  10]
\end{euleroutput}
\begin{eulercomment}
Logical vectors are not treated differently as index in EMT, in contrast to R. You need to
extract the non-zero elements first in EMT.
\end{eulercomment}
\begin{eulerprompt}
>x, x>5, x[nonzeros(x>5)]
\end{eulerprompt}
\begin{euleroutput}
  [1,  2,  3,  4,  5,  6,  7,  8,  9,  10]
  [0,  0,  0,  0,  0,  1,  1,  1,  1,  1]
  [6,  7,  8,  9,  10]
\end{euleroutput}
\begin{eulercomment}
Just as in R, the index vector can contain repetitions.
\end{eulercomment}
\begin{eulerprompt}
>x[[1,2,2,1]]
\end{eulerprompt}
\begin{euleroutput}
  [1,  2,  2,  1]
\end{euleroutput}
\begin{eulercomment}
But names for indices are not possible in EMT. For a statistical package, this may often be
necessary to ease access to elements of vectors.

To mimic this behavior, we can define a function as the following.
\end{eulercomment}
\begin{eulerprompt}
>function sel (v,i,s) := v[indexof(s,i)]; ...
>s=["first","second","third","fourth"]; sel(x,["first","third"],s)
\end{eulerprompt}
\begin{euleroutput}
  
  Trying to overwrite protected function sel!
  Error in:
  function sel (v,i,s) := v[indexof(s,i)]; ... ...
               ^
  
  Trying to overwrite protected function sel!
  Error in:
  function sel (v,i,s) := v[indexof(s,i)]; ... ...
               ^
  
  Trying to overwrite protected function sel!
  Error in:
  function sel (v,i,s) := v[indexof(s,i)]; ... ...
               ^
  [10.4,  3.1]
\end{euleroutput}
\eulerheading{Tipe Data}
\begin{eulercomment}
EMT memiliki lebih banyak tipe data yang tetap dibandingkan R. Jelas,
dalam R terdapat vektor yang berkembang. Anda bisa mengatur sebuah
vektor numerik kosong v dan memberikan sebuah nilai pada elemen v[17].
Hal ini tidak mungkin dilakukan dalam EMT.


Hal berikut ini sedikit tidak efisien.
\end{eulercomment}
\begin{eulerprompt}
>v=[]; for i=1 to 10000; v=v|i; end;
\end{eulerprompt}
\begin{eulercomment}
EMT will now construct a vector with v and i appended on the stack and copy that vector back
to the global variable v.

The more efficient pre-defines the vector.
\end{eulercomment}
\begin{eulerprompt}
>v=zeros(10000); for i=1 to 10000; v[i]=i; end;
\end{eulerprompt}
\begin{eulercomment}
To change date types in EMT, you can use functions like complex().
\end{eulercomment}
\begin{eulerprompt}
>complex(1:4)
\end{eulerprompt}
\begin{euleroutput}
  [ 1+0i ,  2+0i ,  3+0i ,  4+0i  ]
\end{euleroutput}
\begin{eulercomment}
Conversions to strings is possible for elementary data types only. The current format is used
for simple string concatenation. But there are functions like print() or frac().

For vectors, you can easily write your own function.
\end{eulercomment}
\begin{eulerprompt}
>function tostr (v) ...
\end{eulerprompt}
\begin{eulerudf}
  s="[";
  loop 1 to length(v);
     s=s+print(v[#],2,0);
     if #<length(v) then s=s+","; endif;
  end;
  return s+"]";
  endfunction
\end{eulerudf}
\begin{eulerprompt}
>tostr(linspace(0,1,10))
\end{eulerprompt}
\begin{euleroutput}
  [0.00,0.10,0.20,0.30,0.40,0.50,0.60,0.70,0.80,0.90,1.00]
\end{euleroutput}
\begin{eulercomment}
For communication with Maxima, there exists a function convertmxm(), which can also be used to
format a vector for output.
\end{eulercomment}
\begin{eulerprompt}
>convertmxm(1:10)
\end{eulerprompt}
\begin{euleroutput}
  [1,2,3,4,5,6,7,8,9,10]
\end{euleroutput}
\begin{eulercomment}
For Latex the tex command can be used to obtain the Latex command.
\end{eulercomment}
\begin{eulerprompt}
>tex(&[1,2,3])
\end{eulerprompt}
\begin{euleroutput}
  \(\backslash\)left[ 1 , 2 , 3 \(\backslash\)right] 
\end{euleroutput}
\eulerheading{Factors and Tables}
\begin{eulercomment}
In the introduction to R there is an example with so called factors.

The following is a list of the territories of 30 states.
\end{eulercomment}
\begin{eulerprompt}
>austates = ["tas", "sa", "qld", "nsw", "nsw", "nt", "wa", "wa", ...
>"qld", "vic", "nsw", "vic", "qld", "qld", "sa", "tas", ...
>"sa", "nt", "wa", "vic", "qld", "nsw", "nsw", "wa", ...
>"sa", "act", "nsw", "vic", "vic", "act"];
\end{eulerprompt}
\begin{eulercomment}
Assume, we have corresponding incomes in each state.
\end{eulercomment}
\begin{eulerprompt}
>incomes = [60, 49, 40, 61, 64, 60, 59, 54, 62, 69, 70, 42, 56, ...
>61, 61, 61, 58, 51, 48, 65, 49, 49, 41, 48, 52, 46, ...
>59, 46, 58, 43];
\end{eulerprompt}
\begin{eulercomment}
Now, we want to compute the mean of incomes in the territories. Being a statistical program, R
has factor() and tappy() for this.

EMT can make this by finding the index of territories in the unique list of territories.
\end{eulercomment}
\begin{eulerprompt}
>auterr=sort(unique(austates)); f=indexofsorted(auterr,austates)
\end{eulerprompt}
\begin{euleroutput}
  [6,  5,  4,  2,  2,  3,  8,  8,  4,  7,  2,  7,  4,  4,  5,  6,  5,  3,
  8,  7,  4,  2,  2,  8,  5,  1,  2,  7,  7,  1]
\end{euleroutput}
\begin{eulercomment}
At that point, we can write our own loop function to do things for one factor only.

Or we can mimic the tapply() function in the following way.
\end{eulercomment}
\begin{eulerprompt}
>function map tappl (i; f$:call, cat, x) ...
\end{eulerprompt}
\begin{eulerudf}
  u=sort(unique(cat));
  f=indexof(u,cat);
  return f$(x[nonzeros(f==indexof(u,i))]);
  endfunction
\end{eulerudf}
\begin{eulercomment}
It is a bit inefficient, since it computes The unique territories for each i, but it works.
\end{eulercomment}
\begin{eulerprompt}
>tappl(auterr,"mean",austates,incomes)
\end{eulerprompt}
\begin{euleroutput}
  [44.5,  57.3333,  55.5,  53.6,  55,  60.5,  56,  52.25]
\end{euleroutput}
\begin{eulercomment}
Note that it works for each vector of territories.
\end{eulercomment}
\begin{eulerprompt}
>tappl(["act","nsw"],"mean",austates,incomes)
\end{eulerprompt}
\begin{euleroutput}
  [44.5,  57.3333]
\end{euleroutput}
\begin{eulercomment}
Now, the statistical package of EMT defines tables just as in R. The functions readtable() and
writetable() can be used for input and output.

So we can print the average state income in the territories in a friendly way.
\end{eulercomment}
\begin{eulerprompt}
>writetable(tappl(auterr,"mean",austates,incomes),labc=auterr,wc=7)
\end{eulerprompt}
\begin{euleroutput}
      act    nsw     nt    qld     sa    tas    vic     wa
     44.5  57.33   55.5   53.6     55   60.5     56  52.25
\end{euleroutput}
\begin{eulercomment}
We can also try to mimic the behavior of R completely.

The factors should clearly be kept in a collection with the types and the categories (states
and territories in our example). For EMT, we add the pre-computed indices.
\end{eulercomment}
\begin{eulerprompt}
>function makef (t) ...
\end{eulerprompt}
\begin{eulerudf}
  ## Factor data
  ## Returns a collection with data t, unique data, indices.
  ## See: tapply
  u=sort(unique(t));
  return \{\{t,u,indexofsorted(u,t)\}\};
  endfunction
\end{eulerudf}
\begin{eulerprompt}
>statef=makef(austates);
\end{eulerprompt}
\begin{eulercomment}
Now the third element of the collection will contain the indices.
\end{eulercomment}
\begin{eulerprompt}
>statef[3]
\end{eulerprompt}
\begin{euleroutput}
  [6,  5,  4,  2,  2,  3,  8,  8,  4,  7,  2,  7,  4,  4,  5,  6,  5,  3,
  8,  7,  4,  2,  2,  8,  5,  1,  2,  7,  7,  1]
\end{euleroutput}
\begin{eulercomment}
Now we can mimic tapply() in the following way. It will return a table as a collection of
table data and column headings.
\end{eulercomment}
\begin{eulerprompt}
>function tapply (t:vector,tf,f$:call) ...
\end{eulerprompt}
\begin{eulerudf}
  ## Makes a table of data and factors
  ## tf : output of makef()
  ## See: makef
  uf=tf[2]; f=tf[3]; x=zeros(length(uf));
  for i=1 to length(uf);
     ind=nonzeros(f==i);
     if length(ind)==0 then x[i]=NAN;
     else x[i]=f$(t[ind]);
     endif;
  end;
  return \{\{x,uf\}\};
  endfunction
\end{eulerudf}
\begin{eulercomment}
We did not add much type checking here. The only precaution concerns categories (factors) with
no data. But one should check for the correct length of t and for the correctness of the
collection tf.

This table can be printed as a table with writetable().
\end{eulercomment}
\begin{eulerprompt}
>writetable(tapply(incomes,statef,"mean"),wc=7)
\end{eulerprompt}
\begin{euleroutput}
      act    nsw     nt    qld     sa    tas    vic     wa
     44.5  57.33   55.5   53.6     55   60.5     56  52.25
\end{euleroutput}
\eulerheading{Larik}
\begin{eulercomment}
EMT hanya memiliki dua dimensi untuk array. Tipe datanya disebut
matriks. Akan lebih mudah untuk menulis fungsi untuk dimensi yang
lebih tinggi atau pustaka C untuk ini.


R memiliki lebih dari dua dimensi. Dalam R, larik adalah sebuah vektor
dengan sebuah bidang dimensi.


Dalam EMT, sebuah vektor adalah sebuah matriks dengan satu baris. Ini
bisa dibuat menjadi sebuah matriks dengan redim().
\end{eulercomment}
\begin{eulerprompt}
>shortformat; X=redim(1:20,4,5)
\end{eulerprompt}
\begin{euleroutput}
          1         2         3         4         5 
          6         7         8         9        10 
         11        12        13        14        15 
         16        17        18        19        20 
\end{euleroutput}
\begin{eulercomment}
Extraction of rows and columns, or sub-matrices, is much like in R.
\end{eulercomment}
\begin{eulerprompt}
>X[,2:3]
\end{eulerprompt}
\begin{euleroutput}
          2         3 
          7         8 
         12        13 
         17        18 
\end{euleroutput}
\begin{eulercomment}
However, in R it is possible to set a list of specific indices of the vector to a value. The
same is possible in EMT only with a loop.
\end{eulercomment}
\begin{eulerprompt}
>function setmatrixvalue (M, i, j, v) ...
\end{eulerprompt}
\begin{eulerudf}
  loop 1 to max(length(i),length(j),length(v))
     M[i\{#\},j\{#\}] = v\{#\};
  end;
  endfunction
\end{eulerudf}
\begin{eulercomment}
We demonstrate this to show that matrices are passed by reference in EMT. If you do not want
to change the original matrix M, you need to copy it in the function.
\end{eulercomment}
\begin{eulerprompt}
>setmatrixvalue(X,1:3,3:-1:1,0); X,
\end{eulerprompt}
\begin{euleroutput}
          1         2         0         4         5 
          6         0         8         9        10 
          0        12        13        14        15 
         16        17        18        19        20 
\end{euleroutput}
\begin{eulercomment}
The outer product in EMT can only be done between vectors. It is automatic due to the matrix
language. One vector needs to be a column vector and the other a row vector.
\end{eulercomment}
\begin{eulerprompt}
>(1:5)*(1:5)'
\end{eulerprompt}
\begin{euleroutput}
          1         2         3         4         5 
          2         4         6         8        10 
          3         6         9        12        15 
          4         8        12        16        20 
          5        10        15        20        25 
\end{euleroutput}
\begin{eulercomment}
In the introduction PDF for R there is an example, which computes the distribution of ab-cd
for a,b,c,d chosen from 0 to n randomly. The solution in R is form a 4-dimensional matrix and
run table() over it.

Of course, this can be achieved with a loop. But loops are not effective in EMT or R. In EMT,
we could write the loop in C and that would be the quickest solution.

But we want to mimic the behavior of R. For this, we need to flatten the multiplications ab
and make a matrix of ab-cd.
\end{eulercomment}
\begin{eulerprompt}
>a=0:6; b=a'; p=flatten(a*b); q=flatten(p-p'); ...
>u=sort(unique(q)); f=getmultiplicities(u,q); ...
>statplot(u,f,"h"):
\end{eulerprompt}
\eulerimg{15}{images/EMT 1-6_Mifta Dilla Aninditya_23030630029-606.png}
\begin{eulercomment}
Besides the exact multiplicities, EMT can compute frequencies in
vectors.
\end{eulercomment}
\begin{eulerprompt}
>getfrequencies(q,-50:10:50)
\end{eulerprompt}
\begin{euleroutput}
  [0,  23,  132,  316,  602,  801,  333,  141,  53,  0]
\end{euleroutput}
\begin{eulercomment}
The most easy way to plot this as a distribution is the following.
\end{eulercomment}
\begin{eulerprompt}
>plot2d(q,distribution=11):
\end{eulerprompt}
\eulerimg{15}{images/EMT 1-6_Mifta Dilla Aninditya_23030630029-607.png}
\begin{eulercomment}
But it is also possible to pre-compute the count in chosen intervals beforehand. Of course,
the following uses getfrequencies() internally.

Since the histo() function returns frequencies, we need to scale these so that the integral
under the bar graph is 1.
\end{eulercomment}
\begin{eulerprompt}
>\{x,y\}=histo(q,v=-55:10:55); y=y/sum(y)/differences(x); ...
>plot2d(x,y,>bar,style="/"):
\end{eulerprompt}
\eulerimg{15}{images/EMT 1-6_Mifta Dilla Aninditya_23030630029-608.png}
\eulerheading{Daftar}
\begin{eulercomment}
EMT memiliki dua jenis daftar. Pertama adalah daftar global yang dapat
diubah, dan yang kedua adalah jenis daftar yang tidak dapat diubah.
Kita tidak peduli dengan daftar global di sini.


Tipe daftar yang tidak dapat diubah disebut koleksi dalam EMT. Ia
berperilaku seperti struktur dalam C, tetapi elemen-elemennya hanya
diberi nomor dan tidak diberi nama.
\end{eulercomment}
\begin{eulerprompt}
>L=\{\{"Fred","Flintstone",40,[1990,1992]\}\}
\end{eulerprompt}
\begin{euleroutput}
  Fred
  Flintstone
  40
  [1990,  1992]
\end{euleroutput}
\begin{eulercomment}
Currently the elements do not have names, though names can be set for special purposes. They
are accessed by numbers.
\end{eulercomment}
\begin{eulerprompt}
>(L[4])[2]
\end{eulerprompt}
\begin{euleroutput}
  1992
\end{euleroutput}
\eulerheading{Input dan Output File (Membaca dan Menulis Data)}
\begin{eulercomment}
Anda mungkin sering ingin mengimpor matriks data dari sumber lain ke
EMT. Tutorial ini akan menjelaskan kepada Anda tentang berbagai cara
untuk melakukan hal tersebut. Fungsi yang sederhana adalah
writematrix() dan readmatrix().

Mari kita tunjukkan bagaimana cara membaca dan menulis sebuah vektor
real ke sebuah file.
\end{eulercomment}
\begin{eulerprompt}
>a=random(1,100); mean(a), dev(a),
\end{eulerprompt}
\begin{euleroutput}
  0.49597
  0.31551
\end{euleroutput}
\begin{eulercomment}
To write the data to a file, we use the function writematrix().

Since this introduction is most likely in a directory, where the user has no write access, we
write the data to the user home directory. For own notebooks, this is not necessary, since the
data file will be written into the same directory.
\end{eulercomment}
\begin{eulerprompt}
>filename="test.dat";
\end{eulerprompt}
\begin{eulercomment}
Now we write the column vector a' to the file. This yields one number in each line of the
file.
\end{eulercomment}
\begin{eulerprompt}
>writematrix(a',filename);
\end{eulerprompt}
\begin{eulercomment}
To read the data, we use readmatrix().
\end{eulercomment}
\begin{eulerprompt}
>a=readmatrix(filename)';
\end{eulerprompt}
\begin{eulercomment}
And remove the file.
\end{eulercomment}
\begin{eulerprompt}
>fileremove(filename);
>mean(a), dev(a),
\end{eulerprompt}
\begin{euleroutput}
  0.49597
  0.31551
\end{euleroutput}
\begin{eulercomment}
The functions writematrix() or writetable() can be configured for other languages.

E.g., if you have an Indonesian system (decimal point with comma), your Excel needs values
with decimal commas separated by semicolons in a csv (the default is comma separated values)
file. The following file "test.csv" should appear on your cuurent folder.
\end{eulercomment}
\begin{eulerprompt}
>filename="test.csv"; ...
>writematrix(random(5,3),file=filename,separator=",");
\end{eulerprompt}
\begin{eulercomment}
You can now open this file with an Indonesian Excel directly.
\end{eulercomment}
\begin{eulerprompt}
>fileremove(filename);
\end{eulerprompt}
\begin{eulercomment}
Sometimes we have strings with tokens like the following.
\end{eulercomment}
\begin{eulerprompt}
>s1:="f m m f m m m f f f m m f";  ...
>s2:="f f f m m f f";
\end{eulerprompt}
\begin{eulercomment}
To tokenize this, we define a vector of tokens.
\end{eulercomment}
\begin{eulerprompt}
>tok:=["f","m"]
\end{eulerprompt}
\begin{euleroutput}
  f
  m
\end{euleroutput}
\begin{eulercomment}
Then we can count the number of times each token appears in the string, and put the result
into a table.
\end{eulercomment}
\begin{eulerprompt}
>M:=getmultiplicities(tok,strtokens(s1))_ ...
>  getmultiplicities(tok,strtokens(s2));
\end{eulerprompt}
\begin{eulercomment}
Write the table with the token headers.
\end{eulercomment}
\begin{eulerprompt}
>writetable(M,labc=tok,labr=1:2,wc=8)
\end{eulerprompt}
\begin{euleroutput}
                 f       m
         1       6       7
         2       5       2
\end{euleroutput}
\begin{eulercomment}
For statics, EMT can read and write tables.
\end{eulercomment}
\begin{eulerprompt}
>file="test.dat"; open(file,"w"); ...
>writeln("A,B,C"); writematrix(random(3,3)); ...
>close();
\end{eulerprompt}
\begin{eulercomment}
The file looks like this.
\end{eulercomment}
\begin{eulerprompt}
>printfile(file)
\end{eulerprompt}
\begin{euleroutput}
  A,B,C
  0.03024293229479773,0.3407346359279212,0.9981222547778821
  0.8841420654915625,0.7880733073624927,0.1065167957236653
  0.5809058688473883,0.2131757983565064,0.194329579136341
  
\end{euleroutput}
\begin{eulercomment}
The function readtable() in its simplest form can read this and return a collection of values
and heading lines.
\end{eulercomment}
\begin{eulerprompt}
>L=readtable(file,>list);
\end{eulerprompt}
\begin{eulercomment}
This collection can be printed with writetable() to the notebook, or to a file.
\end{eulercomment}
\begin{eulerprompt}
>writetable(L,wc=10,dc=5)
\end{eulerprompt}
\begin{euleroutput}
           A         B         C
     0.03024   0.34073   0.99812
     0.88414   0.78807   0.10652
     0.58091   0.21318   0.19433
\end{euleroutput}
\begin{eulercomment}
The matrix of values is the first element of L. Note that mean() in EMT computes the mean
values of the rows of a matrix.
\end{eulercomment}
\begin{eulerprompt}
>mean(L[1])
\end{eulerprompt}
\begin{euleroutput}
    0.45637 
    0.59291 
    0.32947 
\end{euleroutput}
\eulerheading{CSV Files}
\begin{eulercomment}
First, let us write a matrix into a file. For the output, we generate a file in the current
working directory. 
\end{eulercomment}
\begin{eulerprompt}
>file="test.csv";  ...
>M=random(3,3); writematrix(M,file);
\end{eulerprompt}
\begin{eulercomment}
Here is the content of this file.
\end{eulercomment}
\begin{eulerprompt}
>printfile(file)
\end{eulerprompt}
\begin{euleroutput}
  0.8892627081774781,0.2992928980104143,0.05547892009639564
  0.4927337020087822,0.657741026910016,0.1630322504126193
  0.3611695308701182,0.1468642523900892,0.6154432857002501
  
\end{euleroutput}
\begin{eulercomment}
This CVS can be opened on English systems into Excel by a double click. If you get such a file
on a German system, you need to import the data into Excel taking care of the decimal dot.

But the decimal dot is the default format for EMT too. You can read a matrix from a file with
readmatrix().
\end{eulercomment}
\begin{eulerprompt}
>readmatrix(file)
\end{eulerprompt}
\begin{euleroutput}
    0.88926   0.29929  0.055479 
    0.49273   0.65774   0.16303 
    0.36117   0.14686   0.61544 
\end{euleroutput}
\begin{eulercomment}
It is possible to write several matrices to one file. The open() command can open a file for
writing with the "w" parameter. The default is "r" for reading.
\end{eulercomment}
\begin{eulerprompt}
>open(file,"w"); writematrix(M); writematrix(M'); close();
\end{eulerprompt}
\begin{eulercomment}
The matrices are separated by a blank line. To read the matrices, open the file and call
readmatrix() several times.
\end{eulercomment}
\begin{eulerprompt}
>open(file); A=readmatrix(); B=readmatrix(); A==B, close();
\end{eulerprompt}
\begin{euleroutput}
          1         0         0 
          0         1         0 
          0         0         1 
\end{euleroutput}
\begin{eulercomment}
In Excel or similar spreadsheets, you can export a matrix as CSV (comma separated values). In
Excel 2007, use "save as" and "other formats", then select "CSV". Make sure, the current table
contains only data you wish to export.

Here is an example.
\end{eulercomment}
\begin{eulerprompt}
>printfile("excel-data.csv")
\end{eulerprompt}
\begin{euleroutput}
  Could not open the file
  excel-data.csv
  for reading!
  Try "trace errors" to inspect local variables after errors.
  printfile:
      open(filename,"r");
\end{euleroutput}
\begin{eulercomment}
As you can see, my German system has used a semicolon as separator and a decimal comma. You
can can change this in the system settings or in Excel, but it is not necessary for reading
the matrix into EMT.

The easiest way to read this into Euler is readmatrix(). All commas are replaced by dots with
the parameter \textgreater{}comma. For English CSV, simply omit this parameter.
\end{eulercomment}
\begin{eulerprompt}
>M=readmatrix("excel-data.csv",>comma)
\end{eulerprompt}
\begin{euleroutput}
  Could not open the file
  excel-data.csv
  for reading!
  Try "trace errors" to inspect local variables after errors.
  readmatrix:
      if filename<>"" then open(filename,"r"); endif;
\end{euleroutput}
\begin{eulercomment}
Let us plot this.
\end{eulercomment}
\begin{eulerprompt}
>plot2d(M'[1],M'[2:3],>points,color=[red,green]'):
\end{eulerprompt}
\eulerimg{15}{images/EMT 1-6_Mifta Dilla Aninditya_23030630029-609.png}
\begin{eulercomment}
There are more elementary ways to read data from a file. You can open the file and read the
numbers line by line. The function getvectorline() will read numbers from a line of data. By
default, it expects a decimal dot. But it can also use a decimal comma, if you call
setdecimaldot(",") before you use this function.

The following function is an example for this. It will stop at the end of the file or an empty
line.
\end{eulercomment}
\begin{eulerprompt}
>function myload (file) ...
\end{eulerprompt}
\begin{eulerudf}
  open(file);
  M=[];
  repeat
     until eof();
     v=getvectorline(3);
     if length(v)>0 then M=M_v; else break; endif;
  end;
  return M;
  close(file);
  endfunction
\end{eulerudf}
\begin{eulerprompt}
>myload(file)
\end{eulerprompt}
\begin{euleroutput}
    0.88926         0   0.29929         0  0.055479 
    0.49273         0   0.65774         0   0.16303 
    0.36117         0   0.14686         0   0.61544 
\end{euleroutput}
\begin{eulercomment}
It would also be possible to read all numbers in that file with getvector().
\end{eulercomment}
\begin{eulerprompt}
>open(file); v=getvector(10000); close(); redim(v[1:9],3,3)
\end{eulerprompt}
\begin{euleroutput}
    0.88926         0   0.29929 
          0  0.055479   0.49273 
          0   0.65774         0 
\end{euleroutput}
\begin{eulercomment}
Thus it is very easy to save a vector of values, one value in each line and read back this
vector.
\end{eulercomment}
\begin{eulerprompt}
>v=random(1000); mean(v)
\end{eulerprompt}
\begin{euleroutput}
  0.50857
\end{euleroutput}
\begin{eulerprompt}
>writematrix(v',file); mean(readmatrix(file)')
\end{eulerprompt}
\begin{euleroutput}
  0.50857
\end{euleroutput}
\eulerheading{Menggunakan Tabel}
\begin{eulercomment}
Tabel dapat digunakan untuk membaca atau menulis data numerik. Sebagai
contoh, kita menulis tabel dengan judul baris dan kolom ke file.
\end{eulercomment}
\begin{eulerprompt}
>file="test.tab"; M=random(3,3);  ...
>open(file,"w");  ...
>writetable(M,separator=",",labc=["one","two","three"]);  ...
>close(); ...
>printfile(file)
\end{eulerprompt}
\begin{euleroutput}
  one,two,three
        0.45,      0.79,      0.63
        0.58,      0.87,      0.76
        0.47,      0.42,       0.5
\end{euleroutput}
\begin{eulercomment}
This can be imported into Excel.

To read the file in EMT, we use readtable().
\end{eulercomment}
\begin{eulerprompt}
>\{M,headings\}=readtable(file,>clabs); ...
>writetable(M,labc=headings)
\end{eulerprompt}
\begin{euleroutput}
         one       two     three
        0.45      0.79      0.63
        0.58      0.87      0.76
        0.47      0.42       0.5
\end{euleroutput}
\eulerheading{Menganalisis Garis}
\begin{eulercomment}
Anda bahkan dapat mengevaluasi setiap baris dengan tangan. Misalkan,
kita memiliki baris dengan format berikut.
\end{eulercomment}
\begin{eulerprompt}
>line="2020-11-03,Tue,1'114.05"
\end{eulerprompt}
\begin{euleroutput}
  2020-11-03,Tue,1'114.05
\end{euleroutput}
\begin{eulercomment}
First we can tokenize the line.
\end{eulercomment}
\begin{eulerprompt}
>vt=strtokens(line)
\end{eulerprompt}
\begin{euleroutput}
  2020-11-03
  Tue
  1'114.05
\end{euleroutput}
\begin{eulercomment}
Then we can evaluate each element of the line using appropriate evaluations.
\end{eulercomment}
\begin{eulerprompt}
>day(vt[1]),  ...
>indexof(["mon","tue","wed","thu","fri","sat","sun"],tolower(vt[2])),  ...
>strrepl(vt[3],"'","")()
\end{eulerprompt}
\begin{euleroutput}
  7.3816e+05
  2
  1114
\end{euleroutput}
\begin{eulercomment}
Using regular expressions, it is possible to extract almost any information from a line of
data.

Assume we have the following line an HTML document.
\end{eulercomment}
\begin{eulerprompt}
>line="<tr><td>1145.45</td><td>5.6</td><td>-4.5</td><tr>"
\end{eulerprompt}
\begin{euleroutput}
  <tr><td>1145.45</td><td>5.6</td><td>-4.5</td><tr>
\end{euleroutput}
\begin{eulercomment}
To extract this, we use a regular expression, which searches for

\end{eulercomment}
\begin{eulerttcomment}
 - a closing bracket >,
 - any string not containing brackets with a sub-match "(...)",
 - an opening and a closing bracket using the shortest solution,
 - again any string not containing brackets,
 - and an opening bracket <.
\end{eulerttcomment}
\begin{eulercomment}

Regular expressions are somewhat difficult to learn but very powerful.
\end{eulercomment}
\begin{eulerprompt}
>\{pos,s,vt\}=strxfind(line,">([^<>]+)<.+?>([^<>]+)<");
\end{eulerprompt}
\begin{eulercomment}
The result is the position of the match, the matched string, and a vector of strings for
sub-matches.
\end{eulercomment}
\begin{eulerprompt}
>for k=1:length(vt); vt[k](), end;
\end{eulerprompt}
\begin{euleroutput}
  1145.5
  5.6
\end{euleroutput}
\begin{eulercomment}
Here is a function, which reads all numerical items between \textless{}td\textgreater{} and
\textless{}/td\textgreater{}.
\end{eulercomment}
\begin{eulerprompt}
>function readtd (line) ...
\end{eulerprompt}
\begin{eulerudf}
  v=[]; cp=0;
  repeat
     \{pos,s,vt\}=strxfind(line,"<td.*?>(.+?)</td>",cp);
     until pos==0;
     if length(vt)>0 then v=v|vt[1]; endif;
     cp=pos+strlen(s);
  end;
  return v;
  endfunction
\end{eulerudf}
\begin{eulerprompt}
>readtd(line+"<td>non-numerical</td>")
\end{eulerprompt}
\begin{euleroutput}
  1145.45
  5.6
  -4.5
  non-numerical
\end{euleroutput}
\eulerheading{Membaca dari Web}
\begin{eulercomment}
Situs web atau file dengan URL dapat dibuka di EMT dan dapat dibaca
baris demi baris.


Dalam contoh, kita membaca versi saat ini dari situs EMT. Kami
menggunakan ekspresi reguler untuk memindai “Versi ...” dalam judul.
\end{eulercomment}
\begin{eulerprompt}
>function readversion () ...
\end{eulerprompt}
\begin{eulerudf}
  urlopen("http://www.euler-math-toolbox.de/Programs/Changes.html");
  repeat
    until urleof();
    s=urlgetline();
    k=strfind(s,"Version ",1);
    if k>0 then substring(s,k,strfind(s,"<",k)-1), break; endif;
  end;
  urlclose();
  endfunction
\end{eulerudf}
\begin{eulerprompt}
>readversion
\end{eulerprompt}
\begin{euleroutput}
  Version 2024-01-12
\end{euleroutput}
\eulerheading{Masukan dan Keluaran Variabel}
\begin{eulercomment}
Anda dapat menulis variabel dalam bentuk definisi Euler ke file atau
ke baris perintah.
\end{eulercomment}
\begin{eulerprompt}
>writevar(pi,"mypi");
\end{eulerprompt}
\begin{euleroutput}
  mypi = 3.141592653589793;
\end{euleroutput}
\begin{eulercomment}
For a test, we generate an Euler file in the work directory of EMT.
\end{eulercomment}
\begin{eulerprompt}
>file="test.e"; ...
>writevar(random(2,2),"M",file); ...
>printfile(file,3)
\end{eulerprompt}
\begin{euleroutput}
  M = [ ..
  0.1955673152688992, 0.008834065287790982;
  0.4658057454943137, 0.2384061400147723];
\end{euleroutput}
\begin{eulercomment}
We can now load the file. It will define the matrix M.
\end{eulercomment}
\begin{eulerprompt}
>load(file); show M,
\end{eulerprompt}
\begin{euleroutput}
  M = 
    0.19557 0.0088341 
    0.46581   0.23841 
\end{euleroutput}
\begin{eulercomment}
By the way, if writevar() is used on a variable, it will print the variable definition with
the name of this variable.
\end{eulercomment}
\begin{eulerprompt}
>writevar(M); writevar(inch$)
\end{eulerprompt}
\begin{euleroutput}
  M = [ ..
  0.1955673152688992, 0.008834065287790982;
  0.4658057454943137, 0.2384061400147723];
  inch$ = 0.0254;
\end{euleroutput}
\begin{eulercomment}
We can also open a new file or append to an existing file. In the example we append to the
previously generated file.
\end{eulercomment}
\begin{eulerprompt}
>open(file,"a"); ...
>writevar(random(2,2),"M1"); ...
>writevar(random(3,1),"M2"); ...
>close();
>load(file); show M1; show M2;
\end{eulerprompt}
\begin{euleroutput}
  M1 = 
    0.88193  0.015331 
    0.48977   0.64017 
  M2 = 
    0.47897 
    0.35946 
   0.062931 
\end{euleroutput}
\begin{eulercomment}
To remove any files use fileremove().
\end{eulercomment}
\begin{eulerprompt}
>fileremove(file);
\end{eulerprompt}
\begin{eulercomment}
A row vector in a file does not need commas, if each number is in a new line. Let us generate
such a file, writing every line one by one with writeln().
\end{eulercomment}
\begin{eulerprompt}
>open(file,"w"); writeln("M = ["); ...
>for i=1 to 5; writeln(""+random()); end; ...
>writeln("];"); close(); ...
>printfile(file)
\end{eulerprompt}
\begin{euleroutput}
  M = [
  0.792172462835
  0.462782239698
  0.551827188537
  0.555990804573
  0.0575841506453
  ];
\end{euleroutput}
\begin{eulerprompt}
>load(file); M
\end{eulerprompt}
\begin{euleroutput}
  [0.79217,  0.46278,  0.55183,  0.55599,  0.057584]
\end{euleroutput}
\begin{euleroutput}
  
\end{euleroutput}
\eulersubheading{}
\end{eulernotebook}
\end{document}
